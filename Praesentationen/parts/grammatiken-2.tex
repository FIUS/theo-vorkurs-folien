\subsubsection{formale Notation}
\begin{frame}[fragile]{Formale Notation}
    Wir beschreiben eine \alert{\emph{Grammatik}} durch ein geordentes \alert{\emph{Tupel}} $G = (V, \Sigma, P, S)$
    \begin{itemize}
        \item V ist die Menge der verwendeten Nichtterminale
        \item $\Sigma$ die Menge der Terminale bzw. unser Alphabet
        \item P ist die Menge der Produktionsregeln
        \item S ist die Startvariable
    \end{itemize}
    \metroset{block=fill}
    \begin{exampleblock}{Beispiel für  L = \{$ww^R\;|\;w^R\text{ ist w rückwärts, }w \in \{a, b\}^n, \; n \geq 1$\}}
        $G = (V,\Sigma,P,S)$, mit\\
        $V = \{S\}$\\
        $\Sigma = \{a,b\}$\\
        $P = \{S \rightarrow aSa, S \rightarrow bSb, S \rightarrow aa, S \rightarrow bb$\}\\
        \qquad bzw. kurz: $P = \{S \rightarrow aSa\;|\;bSb\;|\;aa\;|\;bb$\}
    \end{exampleblock}
\end{frame}

{\setbeamercolor{palette primary}{bg=ExColor}
\begin{frame}{Denkpause}
    \begin{columns}
\column{0.5\textwidth}
    \begin{alertblock}{knifflige Aufgabe}
    Bob will durch das Labyrinth laufen. Er hat folgende Möglichkeiten:\\
    $\Sigma$ = \{\text{\Rewind, \MoveUp, \Forward, \MoveDown}\}
    \begin{itemize}
        \item Bob kann nicht auf ein Feld zurücktreten von dem er gerade kam
        \item Bob geht bei jedem Schritt ein Feld in die angegebene Richtung
    \end{itemize}
    \end{alertblock}
\column{0.5\textwidth}
\begin{figure}
        \centering
        \includegraphics[width=0.7\textwidth]{../figures/GBeispiel.png}
        \caption{Bob's Problem}
        
    \end{figure}
\end{columns}
\alert{Geben Sie eine Grammatik an, welche die Sprache beschreibt, die Bob durch alle ihm möglichen Wege des Labyrinths führt.}
\end{frame}
}

{\setbeamercolor{palette primary}{bg=ExColor}
\begin{frame}{Denkpause}
    \begin{alertblock}{Beispiel}
    \begin{figure}
        \centering
        \includegraphics[width=0.7\textwidth]{../figures/GBeispiel1.png}
        \caption{der direkte Weg ist repräsentiert durch das Wort \alert{\MoveDown\Forward\Forward\Forward\Forward\Forward\Forward}}
    \end{figure}
    \end{alertblock}
\end{frame}
}  

{\setbeamercolor{palette primary}{bg=ExColor}
\begin{frame}{Lösung}
    \only<1>{
    \begin{figure}
        \centering
        \includegraphics[width=0.7\textwidth]{../figures/GBeispiel2.png}
        \caption{Indirekter Weg}
        
    \end{figure}
    }
    \only<2>{
    \begin{figure}
        \centering
        \includegraphics[width=0.7\textwidth]{../figures/GBeispiel3.png}
        \caption{Schlaufe Uhrzeigersinn}
        
    \end{figure}\textbf{}
    }
    \only<3>{
    \begin{figure}
        \centering
        \includegraphics[width=0.7\textwidth]{../figures/GBeispiel4.png}
        \caption{Schlaufe gegen Uhrzeigersinn}
        
    \end{figure}
    }
\end{frame}
}

{\setbeamercolor{palette primary}{bg=ExColor}
\begin{frame}{Lösung}
    \begin{columns}
    \column{0.45\textwidth}
    \begin{alertblock}{Eine Möglichkeit:}
     %Wir nehmen uns zwei Variablen um zwischen den Einstiegsrichtungen zu unterscheiden für jeden Entscheidungspunkt und konstruieren damit  unsere Grammatik:\\
         $G = (V, \Sigma, P, S)$, wobei \\
         $V = \{S, A_u, A_r, B_u, B_l\}$ \\
         $\Sigma = \{\text{\Rewind, \MoveUp, \Forward, \MoveDown}\}$ \\
         $P = \{S \rightarrow \text\MoveDown A_u \mid \text\MoveDown A_r,$\\
         \qquad\; $A_u \rightarrow \text{\Forward\Forward\Forward\Forward} B_l$\\
         \qquad\; $A_r \rightarrow \text{\MoveDown\MoveDown\Forward\Forward\MoveUp\Forward\Forward\MoveUp} B_u,$\\
         \qquad\; $B_l \rightarrow \text{\MoveDown\Rewind\Rewind\MoveDown\Rewind\Rewind\MoveUp\MoveUp} A_u \mid \text{\Forward\Forward},$\\
         \qquad\; $B_u \rightarrow \text{\Rewind\Rewind\Rewind\Rewind} A_r \mid \text{\Forward\Forward}\}$
    \end{alertblock}
    \column{0.55\textwidth}
    \begin{figure}
        \centering
        \includegraphics[width=0.9\textwidth]{../figures/GBeispielHowTo.png}
        \caption{Es muss unterschieden werden, ob Bob von links, rechts oder unten kam}
        
    \end{figure}
    \end{columns}
    \small\emph{Erinnerung:} Bob kann nicht auf ein Feld zurücktreten von dem er gerade kam
\end{frame}
} 

\subsubsection{Ableiten}
\begin{frame}[fragile]{Ableiten}
    Wir können durch das Ableiten formal zeigen, dass ein Wort von einer Grammatik erzeugt wird.\\
    \small{Wir betrachten L = \{$ww^R\;|\;w^R\text{ ist w rückwärts, }w \in \{a, b\}^n, n>0, n\in \mathbb{N}$\}\\
    mit der Grammatik $G=(V,\Sigma,P,S)$, wobei\\
    $V=\{S\}$, $\Sigma=\{a,b\}$, $P = \{S \rightarrow aSa \mid bSb \mid aa \mid bb$\}}
    \metroset{block=fill}
    \begin{exampleblock}{Beipspiel}
        Wir zeigen $ww^R = ababbbbaba \in$ L.\\
        \small{$S\Longrightarrow_G aSa \Longrightarrow_G abSba \Longrightarrow_G  abaSaba \Longrightarrow_G ababSbaba$ \\ $\Longrightarrow_G ababbbbaba$}\\\qed
    \end{exampleblock}
\end{frame}

{\setbeamercolor{palette primary}{bg=ExColor}
\begin{frame}{Denkpause}
    \begin{alertblock}{Aufgaben}
    Zeige.
    \end{alertblock}
    \metroset{block=fill}
    \begin{block}{Normal}
    \begin{itemize}
        \item $P_1=\{S\rightarrow aaS\;|\;\emptyWord\}$ erzeugt $aaaa$
        \item $P_2=\{S\rightarrow AB$, $A\rightarrow aAb \;|\; ab\; |\;\emptyWord$, $B\rightarrow cB \;|\; \emptyWord\}$ erzeugt $aabbc$
        \item $P_3=\{S\rightarrow UV$, $U\rightarrow aU \;|\; bU \; |\; \emptyWord$, $V\rightarrow c \;|\; d\}$ erzeugt $abac$
        \item $P_4=\{S\rightarrow XXX$, $X\rightarrow a \;|\; b \;|\; c\}$ erzeugt $aac$
    \end{itemize}
    \end{block}
    \begin{block}{Etwas Schwerer}
    \begin{itemize}
        \item $P_5=\{S\rightarrow a \;|\; aaaS\}$ erzeugt $aaaa$
        \item $P_6=\{S\rightarrow AAAB$, $AB\rightarrow BA, 
        A\rightarrow cA \;|\; Ac \;|\; a, 
        B\rightarrow cB \;|\; Bc \;|\; b\}$ erzeugt $cabcacca$
        \item $P_7=\{S\rightarrow U\text{\Stopsign} \;|\; \text{\Stopsign}$, $U\rightarrow \text{\Rewind} U \;|\; \text{\MoveUp} U \;|\; \text{\Forward} U \;|\; \text{\MoveDown} U \;|\;\emptyWord\}$ erzeugt \Forward\Stopsign
    \end{itemize}
    \end{block}
\end{frame}
}

{\setbeamercolor{palette primary}{bg=ExColor}
\begin{frame}{Lösungen}
Alle Lösungen sind Beispiellösungen, es sind auch andere möglich.
    \begin{itemize}[<+- | alert@+>]
        \item $S\Longrightarrow_G aaS \Longrightarrow_G aaaaS \Longrightarrow_G aaaa$
        \item $S\Longrightarrow_G AB \Longrightarrow_G aAbB \Longrightarrow_G aabbB \Longrightarrow_G aabbcB \Longrightarrow_G aabbc$
        \item $S\Longrightarrow_G UV \Longrightarrow_G aUV \Longrightarrow_G abUB \Longrightarrow_G abaUB \Longrightarrow_G abaB \Longrightarrow_G abac$
        \item $S\Longrightarrow_G XXX \Longrightarrow_G aXX \Longrightarrow_G aaX \Longrightarrow_G aac$
        \item $S\Longrightarrow_G aaaS \Longrightarrow_G aaaa$
        \item $S\Longrightarrow_G AAAB \Longrightarrow_G AABA \Longrightarrow_G ABAA \Longrightarrow_G cABAA \Longrightarrow_G caBAA \Longrightarrow_G cabAA \Longrightarrow_G cabcAA \Longrightarrow_G cabcaA\Longrightarrow_G cabcacA \Longrightarrow_G cabcaccA \Longrightarrow_G cabcacca$
        \item $S\Longrightarrow_G U\text{\Stopsign} \Longrightarrow_G \text{\Forward}U\text{\Stopsign} \Longrightarrow_G \text{\Forward}\text{\Stopsign}$
    \end{itemize}
\end{frame}
}  
