\begin{frame}{Automaten: Formal}
    Ein \textbf{DEA M} lässt sich beschreiben durch ein geordnetes 5-Tupel\\
    \alert{$M=(Z, \Sigma, \delta, z_0, E)$} mit:
    \begin{itemize}
        \item $Z$: Die Menge der Zustände
        \item $\Sigma$: Das Alphabet
        \item $\delta$: Die Überführungsfunktion
        \item $z_0$: Der Startzustand
        \item $E$: Die Menge der Endzustände
    \end{itemize}
\end{frame}


\begin{frame}{Automaten: Formal}
\begin{columns}
    \column{0.5\textwidth}
        \alert{$L_1=\{a^{n}b^{m} \mid n,m \in \mathbb{N}\}$}\\
        \vspace{0.6cm}
        \begin{tikzpicture}[->,>=stealth',shorten >=1pt,auto,node distance=1.5cm,semithick]
        \node [initial,state,accepting]   (0)              {$q_0$};
        \node [state,accepting]           (1) [right of=0] {$q_1$};
        \node [state]                     (2) [below of=0] {$F$};
        
        \path   (0) edge                    node {b}    (1)
                    edge [loop above]       node {a}    (0)
                (1) edge                    node {a}    (2)
                    edge [loop above]       node {b}    (1)
                (2) edge [loop right]       node {a,b}  (2);
        \end{tikzpicture}

    \column{0.5\textwidth}
    \alert{$M=(Z, \Sigma, \delta, q_0, E)$} mit:
    \begin{itemize}
        \item<2-> \alert<2>{$Z=\{q_0, q_1, F\}$}
        \item<3-> \alert<3>{$\Sigma=\{a, b\}$}
        \item<4-> \alert<4>{$\delta$: 
        \begin{itemize}
            \item $\delta(q_0, a)=q_0$
            \item $\delta(q_0, b)=q_1$
            \item $\delta(q_1, a)=F$
            \item $\delta(q_1, b)=q_1$
            \item $\delta(F, a)=F$
            \item $\delta(F, b)=F$
        \end{itemize}}
        \item<5-> \alert<5>{$E=\{q_0, q_1\}$}
    \end{itemize}
\end{columns}
\end{frame}

\begin{frame}{Kurz selbst denken...}
\begin{columns}
    \column{0.5\textwidth}
        \alert{$L_2=\{w \in \{a, b\}^* \mid |w|=3\}$}\\
        %TÖDÖ Töröööööt Benjamin der kleiner Elefant fragt Janette: Ey machst du automat amk!
        
        \vspace{0.6cm}
        \begin{tikzpicture}[->,>=stealth',shorten >=1pt,auto,node distance=1.5cm,semithick]
        \node [initial,state]   (0)              {$q_0$};
        \node [state]           (1) [below of=0] {$q_1$};
        \node [state]           (2) [below of=1] {$q_2$};
        \node [state,accepting] (3) [below of=2] {$q_E$};
        \node [state]           (f) [right of=3] {$F$};
        
        \path   (0) edge                node {a,b}  (1)
                (1) edge                node {a,b}  (2)
                (2) edge                node {a,b}  (3)
                (3) edge                node {a,b}  (f)
                (f) edge [loop above]   node {a,b} (f);
        \end{tikzpicture}

    \column{0.5\textwidth}
    \alert{$M=(Z, \Sigma, \delta, q_0, E)$} mit:
    \begin{itemize}
        \item<2-> \alert<2>{$Z=\{q_0, q_1, q_2, q_E, F\}$}
        \item<3-> \alert<3>{$\Sigma=\{a, b\}$}
        \item<4-> \alert<4>{$\delta$: 
        \begin{itemize}
            \item $\delta(q_0, a)=q_1$
            \item $\delta(q_0, b)=q_1$
            \item $\delta(q_1, a)=q_2$
            \item $\delta(q_1, b)=q_2$
            \item $\delta(q_2, a)=q_3$
            \item $\delta(q_2, b)=q_3$
            \item $\delta(q_E, a)=F$
            \item $\delta(q_E, b)=F$
            \item $\delta(F, a)=F$
            \item $\delta(F, b)=F$
        \end{itemize}}
        \item<5-> \alert<5>{$E=\{q_E\}$}
    \end{itemize}
\end{columns}
\end{frame}

\begin{frame}{Automaten und Grammatiken}
    \metroset{block=fill}
    \begin{exampleblock}{Satz}
    Jede durch \only<6->{\alert<6>{deterministische }}endliche Automaten erkennbare Sprache ist auch regulär (also Typ 3).
    \end{exampleblock}
	\onslide<2-> %
    Sei \alert<2>{$A \subseteq \SigmaStern$} eine Sprache und \alert<2>{M ein \only<-5>{Automat}\only<6->{\alert<6>{DEA}}} mit \alert<2-3>{$\mathbf{T(M) = A}$}.\\
    \onslide<3-> %
    (d.h. $M$ erkennt die Sprache $A$) \\
    \vspace{.3cm} %
    
    \onslide<4-> %
    Wir suchen eine \alert<4>{Typ 3-Grammatik G} mit \alert<4-5>{$\mathbf{L(G) = A}$}.\\
    \onslide<5-> %
    (d.h. die Grammatik $G$ erzeugt die Sprache $A$)
    \vspace{.3cm} %
    
    \onslide<6-> %
    \begin{alertblock}{Anmerkung}
    	Wir beschränken uns auf DEAs; in der Vorlesung werdet ihr aber eine allgemeinere Äquivalenz zeigen.
    \end{alertblock}
\end{frame}

\begin{frame}{Automaten und Grammatiken: Ansatz}
	Wir stellen einige Parallelen fest:
	% note: \alert or \usebeamercolor[fg]{alerted text} inside p-columns breaks layout
	% default alert color of metropolis is mLightBrown
	\begin{center}\begin{tabular}{p{.46\textwidth}|p{.46\textwidth}}
		\textbf{Endliche Automaten} & \textbf{Reguläre Grammatiken}
		\onslide<2-> \\\hline
		Wörter werden \textcolor<2>{mLightBrown}{von links nach rechts} gelesen
		& Wörter werden \textcolor<2>{mLightBrown}{von links nach rechts} erzeugt
		\onslide<3-> \\\hline
		\textcolor<3>{mLightBrown}{Pro Schritt} wird \textcolor<3>{mLightBrown}{ein Buchstabe} gelesen
		& \textcolor<3>{mLightBrown}{Pro Schritt} wird \textcolor<3>{mLightBrown}{ein Buchstabe} erzeugt
		\onslide<4-> \\\hline
		\textcolor<4>{mLightBrown}{Ein Startzustand}
		& \textcolor<4>{mLightBrown}{Ein Startsymbol}
		\onslide<5-> \\\hline
		Beim Lesen des \textcolor<5>{mLightBrown}{letzten Buchstabens} wird in einen \textcolor<5>{mLightBrown}{Endzustand} übergegangen
		& Beim Erzeugen des \textcolor<5>{mLightBrown}{letzten Buchstabens} wird \textcolor<5>{mLightBrown}{keine neue Variable} erzeugt
		\onslide<6-> \\\hline
		Ansonsten wird in jedem Schritt \textcolor<6>{mLightBrown}{aus einem Zustand in genau einen Zustand} übergegangen
		& Ansonsten wird in jedem Schritt \textcolor<6>{mLightBrown}{aus einer Variable genau eine Variable} erzeugt
	\end{tabular}\end{center}
\end{frame}

\begin{frame}{DEA zu Grammatik: Konstruktion}
    Also: \alert<4-5>{Zustände $\hat{=}$ Variablen}; \alert<6>{Übergänge $\hat{=}$ Produktionen}
    
    \onslide<2-> %
    Sei also ein \alert<2>{DEA $M = (\alert<4>{Z}, \alert<3>{\Sigma}, \alert<6>{\delta}, \alert<5>{z_0}, E)$} gegeben. \\
    Wir definieren die \alert<2>{Grammatik $G = (\alert<4>{V}, \alert<3>{\Sigma}, \alert<6>{P}, \alert<5>{S})$} mit:
    \begin{itemize}
        \item<3- | alert@3> Alphabet $\Sigma$
        \item<4- | alert@4> Variablenmenge $V = Z$
        \item<5- | alert@5> Startsymbol $S = z_0$
        \item<6- | alert@6> Produktionsmenge $P$ erzeugen wir aus $\delta$
    \end{itemize}
\end{frame}

\begin{frame}{DEA zu Grammatik: Produktionsregeln}\begin{columns}
    \begin{column}{0.5\textwidth}
        Für die Produktionsmenge $P$ wandeln wir die Übergänge um. \\
        \vspace{.3cm}
        \onslide<2-> %
        Jedem $\delta$-Übergang \alert<2-3>{$\delta(z_1, a)=z_2$} ordnen wir folgende Regeln zu:
        \begin{itemize}
            \item<2- | alert@2> $z_1 \rightarrow a z_2$
            \item<3-> Und zusätzlich, falls \alert<3>{$z_2 \in E:$ $z_1 \rightarrow a$}
        \end{itemize}
        \onslide<4->{ %
            Falls $z_0 \in E$, brauchen wir außerdem $z_0 \to \emptyWord$.
        }
    \end{column}
    \begin{column}{0.5\textwidth}\centering\alert{\only<2-3>{ %
        \begin{tikzpicture}[->,>=stealth',shorten >=1pt,auto,node distance=2cm,semithick]
            \node [state] (1)  {$z_1$};
            \onslide<2>{
                \node [state] (2) [right of=1]  {$z_2$};
                \path (1) edge node {a} (2);
            }\onslide<3>{
                \node [state, accepting] (2') [right of=1]  {$z_2$};
                \path (1) edge node {a} (2');
            }
        \end{tikzpicture} \\
        \glqq$\delta (z_1,a) = z_2$\grqq \\
        wird zu
        \begin{align*}
            &z_1 \to az_2 \\
            &\onslide<3->{z_1 \to a}
        \end{align*}
    }\only<4>{
        \begin{tikzpicture}[->,>=stealth',shorten >=1pt,auto,node distance=2cm,semithick]
            \node [state,initial,accepting] (0)  {$z_0$};
        \end{tikzpicture} \\
        \glqq$z_0 \in E$\grqq \\
        wird zu $$z_0 \to \emptyWord$$
    }}\end{column}
\end{columns}\end{frame}

\begin{frame}{DEA zu Grammatik: Korrektheit (Beweisidee)}
    \begin{alertblock}{Zu zeigen: $x \in T(M)$ gdw. $x \in L(G)$}
    	\onslide<2-> %
        Sei $x = a_1 a_2 \ldots a_{n-1} a_n$ ein Wort, das von $M$ akzeptiert wird. \\
        \onslide<3-> %
        \begin{center}\centering\begin{tikzpicture}[->,>=stealth',shorten >=1pt,auto,node distance=1.75cm,semithick]
            \alert<5-5>{\node [initial,state]   (0)              {$z_0$};}
            \alert<5-6>{\node [state]           (1) [right of=0] {$z_1$};}
            \alert<6-8>{\node                   (5) [right of=1] {$\ldots$};}
            \alert<8-9>{\node [state]           (8) [right of=5] {$z_{n-1}$};}
            \alert<9-9>{\node [state,accepting] (9) [right of=8] {$z_n$};}
            
            \alert<5>{\path (0) edge node {$a_1$}     (1);}
            \alert<6>{\path (1) edge node {$a_2$}     (5);}
            \alert<8>{\path (5) edge node {$a_{n-1}$} (8);}
            \alert<9>{\path (8) edge node {$a_n$}     (9);}
        \end{tikzpicture}\end{center}
        \onslide<4-> %
        Mit den passenden Regeln \only<5-9>{(z.B. %
        	\only<5>{$z_0 \to a_1 z_1$}%
        	\only<6>{$z_1 \to a_2 z_2$}%
        	\only<7>{$z_2 \to a_3 z_3$}%
        	\only<8>{$z_{n-2} \to a_{n-1} z_{n-1}$}%
        	\only<9>{$z_{n-1} \to a_n$}%
        ) }lässt sich $x$ ableiten:
        $$\alert<5>{z_0} \alert<5>{\Rightarrow} \alert<5>{a_1} \alert<5-6>{z_1} \alert<6>{\Rightarrow} a_1 \alert<6>{a_2} \alert<6-7>{z_2} \alert<7>{\Rightarrow} \alert<7-8>{\ldots} \alert<8>{\Rightarrow} a_1 a_2 \ldots \alert<8>{a_{n-1}} \alert<8-9>{z_{n-1}} \alert<9>{\Rightarrow} a_1 a_2 \ldots a_{n-1} \alert<9>{a_n} = x$$
        \onslide<10-> %
        Also können wir das Wort in der Grammatik ableiten. \\
        \onslide<11-> %
        \alert<11>{Funktioniert die Argumentation auch andersrum?}
        \onslide<12-> %
        \alert<12>{Ja!}
    \end{alertblock}
\end{frame}

\begin{frame}{DEA zu Grammatik: Korrektheit}
    \begin{alertblock}{Zu zeigen: $x \in T(M)$ gdw. $x \in L(G)$}
        Dabei gilt immer noch: $x = a_1 a_2 \ldots a_{n-1} a_n$ \\
        \onslide<2-> %
        Die folgenden Aussagen sind äquivalent:
        \begin{itemize}
            \item<3- | alert@3> $x$ wird von Automat $M$ erkannt \emph{$(x\in T(M))$}
            \item<4- | alert@4> Es gibt eine Folge von Zuständen $z_0, z_1, \ldots, z_{n-1}, z_n$ mit:
            $z_0$ ist Startzustand, $z_n$ ist Endzustand \textbf{und}:
            $\forall i \in \{1, \ldots, n\}: \delta(z_{i-1}, a_i) = z_i$
            \item<5- | alert@5> Es gibt Folge an Variablen $z_0, z_1, \dots, z_{n-1}$ mit:
            $z_0$ ist Startvariable, $(z_{n-1} \to a_n) \in P$ \textbf{und}:
            $\forall i \in \{1, \ldots, n-1\}: (z_{i-1} \to a_i z_i) \in P$
            \item<6- | alert@6> Es gibt Folge an Variablen $z_0, z_1, \dots, z_{n-1}$ mit:
            $z_0$ ist Startvariable \textbf{und}: $z_0 \Rightarrow a_1 z_1 \Rightarrow \ldots \Rightarrow a_1 a_2 \ldots a_{n-1} a_n$
            \item<7- | alert@7> x wird von der Grammatik G produziert \emph{$(x \in L(G))$}
        \end{itemize}
    	\onslide<8-> %
        Also gilt die Äquivalenz. \qed
    \end{alertblock}
\end{frame}

% \begin{frame}[fragile]{Grammatik zu Automaten}
%     Eine reguläre Grammatik und eine endlicher Automat können die selben Sprachen beschreiben.
%     \begin{exampleblock}{\glqq$\implies$\grqq}
%     \only<1>{Aus $z_1 \to az_2 \in P$ wird:\\
%     \begin{center}
%     \vspace{0.3cm}
%         \begin{tikzpicture}[->,>=stealth',shorten >=1pt,auto,node distance=2cm,
%                         semithick]
%         %\tikzstyle{every state}=[fill=ExColor,draw=none,text=white]
        
%         \node       [state]                 (1) [right of=1]  {$z_1$};
%         \node       [state]                 (2) [right of=2]  {$z_2$};
        
%         \path
%                 (1) edge                node {a}  (2);
%         \end{tikzpicture}\\
%         \glqq$\delta (z_1,a) = z_2$\grqq
%     \end{center}}
    
%     \only<2>{
%     Falls $z_2 \in F$ ($z_2$ ein Endzustand)\\
%     Aus $z_1 \to a \in P$ wird:\\
%     \begin{center}
%     \vspace{0.3cm}
%         \begin{tikzpicture}[->,>=stealth',shorten >=1pt,auto,node distance=2cm,
%                         semithick]
%         %\tikzstyle{every state}=[fill=ExColor,draw=none,text=white]
        
%         \node       [state]                 (1) [right of=1]  {$z_1$};
%         \node       [state, accepting]      (2) [right of=2]  {$z_2$};
        
%         \path
%                 (1) edge                node {a}  (2);
%         \end{tikzpicture}\\
%         \glqq$\delta (z_1,a) = z_2$\grqq
%     \end{center}
%     }
%     \end{exampleblock}
% \end{frame}
