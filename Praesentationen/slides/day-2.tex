% Copyright 2018-2024 FIUS
%
% This file is part of theo-vorkurs-folien.
%
% theo-vorkurs-folien is free software: you can redistribute it and/or modify
% it under the terms of the GNU General Public License as published by
% the Free Software Foundation, either version 3 of the License, or
% (at your option) any later version.
%
% theo-vorkurs-folien is distributed in the hope that it will be useful,
% but WITHOUT ANY WARRANTY; without even the implied warranty of
% MERCHANTABILITY or FITNESS FOR A PARTICULAR PURPOSE.  See the
% GNU General Public License for more details.
%
% You should have received a copy of the GNU General Public License
% along with theo-vorkurs-folien.  If not, see <https://www.gnu.org/licenses/>.

\documentclass[aspectratio=43,10pt]{beamer}

\usetheme[progressbar=frametitle,subsectionpage=progressbar]{metropolis}
\usepackage{appendixnumberbeamer}
\usepackage[ngerman]{babel}
\usepackage[utf8]{inputenc}
%\usepackage{t1enc}
\usepackage{iftex}

\ifLuaTeX
    \usepackage{fontspec}
\else
    \ifxetex
        \usepackage{fontspec}
    \else
        \usepackage[T1]{fontenc}
    \fi
\fi

\usepackage[sfdefault,scaled=.85,lf]{FiraSans}
\usepackage{newtxsf}

\usepackage{booktabs}
\usepackage[scale=2]{ccicons}
\usepackage{hyperref}

\usepackage{pgf}
\makeatletter
\@ifclasswith{beamer}{notes}{
    \usepackage{pgfpages}
    \setbeameroption{show notes on second screen}
}{}
\makeatother
\usepackage{tikz}
\usetikzlibrary{arrows,automata,positioning,shapes,arrows.meta,shapes.geometric, through, calc}
\usepackage{pgfplots}
\usepgfplotslibrary{dateplot}

\usepackage{xspace}
\newcommand{\themename}{\textbf{\textsc{metropolis}}\xspace}

\usepackage{blindtext}
\usepackage{graphicx}
\usepackage{subcaption}
\usepackage{comment}
\usepackage{mathtools}
\usepackage{amsmath}
\usepackage{centernot}
\usepackage{amssymb}
\usepackage{proof}
\usepackage{tabularx}
\renewcommand{\figurename}{Abb.}
\usepackage{tikzsymbols}
\let\Coffeecup\relax
\let\Heart\relax
\let\Smiley\relax
\usepackage{marvosym}
\usepackage{mathtools}
\usepackage{qrcode}
\usepackage{advdate}
\usepackage{ifthen}
\usepackage{tikz-among-us}
\usepackage{multicol}
\usepackage{outlines}
\usepackage{ulem}

% \IfFontExistsTF{Segoe UI}{%
%     \usefonttheme{professionalfonts}
%     \defaultfontfeatures{Scale = MatchLowercase}
%     \setsansfont{Consolas}[
%         Scale = 1.0,
%         BoldFont = Consolas ,
%         BoldItalicFont = Consolas ]
%     \ifLuaTeX
% }{
% }
% \else
% \ifxelatex
%     \IfFontExistsTF{Segoe UI}{%
%         \usefonttheme{professionalfonts}
%         \defaultfontfeatures{Scale = MatchLowercase}
%         \setsansfont{Consolas}[
%             Scale = 1.0,
%             BoldFont = Segoe UI Semibold ,
%             BoldItalicFont = Segoe UI Semibold Italic ]
%     }{
%     }
% \fi

\definecolor{Bluecreen}{HTML}{0873aa}

\makeatletter
\setlength{\metropolis@progressonsectionpage@linewidth}{1.1em}
\setlength{\metropolis@progressinheadfoot@linewidth}{2pt}
\makeatother

\setbeamercolor{progress bar}{%
    fg=Bluecreen,
    bg=Bluecreen!70!black!45
}

\makeatletter
\newlength{\metropolis@progressonsectionpage@blockwidth}%
\newlength{\metropolis@progressonsectionpage@blockborder}%
\setlength{\metropolis@progressonsectionpage@blockborder}{1pt}%
\setbeamertemplate{progress bar in section page}{
    \vspace{0.5\metropolis@progressonsectionpage@linewidth}
    \setlength{\metropolis@progressonsectionpage}{%
        \textwidth * \ratio{\insertframenumber pt}{\inserttotalframenumber pt}%
    }%
    \setlength{\metropolis@progressonsectionpage@blockwidth}{%
        0.05\textwidth - 0.05\metropolis@progressonsectionpage@blockborder
    }%
    \begin{tikzpicture}
        \fill[bg] (0,-\metropolis@progressonsectionpage@blockborder) rectangle (\textwidth, \metropolis@progressonsectionpage@linewidth + \metropolis@progressonsectionpage@blockborder);
        %\fill[fg] (0,0) rectangle (\metropolis@progressonsectionpage, \metropolis@progressonsectionpage@linewidth);

        \foreach \i in {1,...,20} {%
                \pgfmathparse{\insertframenumber*100/\inserttotalframenumber >= \i*100/20 ? 1 : 0}
                \ifthenelse {\pgfmathresult>0}
                {%
                    \pgfmathparse{\i*0.05-0.005}
                    \fill[fg] (\i*\metropolis@progressonsectionpage@blockwidth, 0) rectangle ++ (-\metropolis@progressonsectionpage@blockwidth + \metropolis@progressonsectionpage@blockborder, \metropolis@progressonsectionpage@linewidth);
                }
                {\node at (\i,0) {\i};}% otherwise do nothing
            }

        \node[color=white] at (0.5\textwidth, 0.5\metropolis@progressonsectionpage@linewidth) {\textnormal{%
                \fontsize{0.85\metropolis@progressonsectionpage@linewidth}{\metropolis@progressonsectionpage@linewidth}\selectfont loading
                \pgfmathparse{\insertframenumber*100/\inserttotalframenumber}%
                \pgfmathprintnumber[fixed,precision=2]{\pgfmathresult}\,\% complete...%
            }%
        };

    \end{tikzpicture}%
}
\makeatother

\newcommand\daynr{0}


\definecolor{ExColor}{HTML}{17819b}

\newcommand{\emptyWord}{\varepsilon}
\let \emptyset\varnothing
\newcommand{\SigmaStern}{\Sigma^{*}}
\newcommand{\absval}[1]{|#1|}
\newcommand{\defeq}{\vcentcolon=}
\newcommand{\eqdef}{=\vcentcolon}
\newcommand{\nimplies}{\centernot\implies}

\newcommand{\naturals}{\ensuremath{\mathbb{N}}}
\newcommand{\integers}{\ensuremath{\mathbb{Z}}}
\newcommand{\rationals}{\ensuremath{\mathbb{Q}}}
\newcommand{\reals}{\ensuremath{\mathbb{R}}}
\newcommand{\iffspace}{\ensuremath{\iff\ }}

\newcommand{\sus}[1]{%
    \tikz{\node[scale=0.05] at (0,0) {\amongUsOriginal{#1}{cyan}}}
}

\setbeamertemplate{footline}[text line]
{\parbox{\linewidth}{Fachgruppe Informatik\hfill\insertpagenumber\hfill Vorkurs Theoretische Informatik\vspace{0.2in}}}

\newcommand{\Center}[1]{
    \begin{frame}<handout:0>[standout]
        #1
    \end{frame}
}

\newcommand{\cleft}[2][.]{%
    \begingroup\colorlet{savedleftcolor}{.}%
    \color{#1}\left#2\color{savedleftcolor}%
}
\newcommand{\cright}[2][.]{%
    \color{#1}\right#2\endgroup
}

% Make subsections in toc small
\makeatletter
\setbeamertemplate{subsection in toc}{\small\leftskip=2em\inserttocsubsection\par}
\makeatother

\AtBeginDocument{%

    % Fix section pages in appendix
    \apptocmd{\appendix}{%
        \setbeamertemplate{section page}[simple]%
    }{}{}
}

\addtobeamertemplate{block begin}{}{\vskip 0em}
\addtobeamertemplate{block alerted begin}{}{\vskip 0em}
\addtobeamertemplate{block example begin}{}{\vskip 0em}

\newsavebox\tikzBox
\newenvironment{includetikzpicture}[1]{%
    \def\sizeArgument{#1}\begin{lrbox}{\tikzBox}\begin{tikzpicture}
            }{%
        \end{tikzpicture}\end{lrbox}\resizebox{\sizeArgument}{!}{\usebox\tikzBox}%
}

\pgfkeys{
    /sevenseg/.is family, /sevenseg,
    shrink/.estore in     = \sevensegShrink,    % avoids overlapping of segments
    oncolor/.estore in    = \sevensegOncolor,   % color of an ON segment
    offcolor/.estore in   = \sevensegOffcolor,  % color of an OFF segment
    size/.estore in       = \sevensegSize,      % height
    onSize/.estore in     = \sevensegOnsize,     % line width if ON
    offSize/.estore in    = \sevensegOffsize    % line width if OFF
}

\pgfkeys{
    /sevenseg,
    default/.style = {
        shrink = 0.1, 
        size = 1em, 
        oncolor = red, 
        offcolor = blue!70!black!40,
        onSize = 2,
        offSize =1.5
        }
}

\newcommand{\sevenseg}[2][]% options, values
{%
  \pgfkeys{/sevenseg, default, #1}%
    \begin{tikzpicture}[x=\sevensegSize,y=\sevensegSize,baseline=.25*\sevensegSize]%
        % unten li
        \path (0,0) ++(0,0) coordinate (P1);
        % unten re
        \path (0,0) ++(0.5,0) coordinate (P2);
        % mitte li
        \path (0,0) ++(90:0.5) coordinate (P3);
        % mitte re
        \path (P2)  ++(90:0.5) coordinate (P4);
        % oben li
        \path (P3)  ++(90:0.5) coordinate (P5);
        % oben re
        \path (P4)  ++(90:0.5) coordinate (P6);
        % then step through the 1/0 values in the segment array


    \foreach \val [count=\i from 0] in {#2} {%
      \ifthenelse{\equal{\val}{1}}%
        {\let\mycolor=\sevensegOncolor \let\mysize=\sevensegOnsize}%
        {\let\mycolor=\sevensegOffcolor \let\mysize=\sevensegOffsize}%

      % then draw segment depending on \i
      \ifthenelse{\equal{\i}{0}}{\path[draw=\mycolor, line width=\mysize] (P5) -- (P6);}{}%
      \ifthenelse{\equal{\i}{1}}{\path[draw=\mycolor, line width=\mysize] (P6) -- (P4);}{}%
      \ifthenelse{\equal{\i}{2}}{\path[draw=\mycolor, line width=\mysize] (P4) -- (P2);}{}%
      \ifthenelse{\equal{\i}{3}}{\path[draw=\mycolor, line width=\mysize] (P1) -- (P2);}{}%
      \ifthenelse{\equal{\i}{4}}{\path[draw=\mycolor, line width=\mysize] (P1) -- (P3);}{}%
      \ifthenelse{\equal{\i}{5}}{\path[draw=\mycolor, line width=\mysize] (P3) -- (P5);}{}%
      \ifthenelse{\equal{\i}{6}}{\path[draw=\mycolor, line width=\mysize] (P3) -- (P4);}{}%
    }
  \end{tikzpicture}%
}

\newcommand{\sevensegnum}[2][]%
{%                                          
    \ifthenelse{\equal{#2}{0}}{\sevenseg[#1]{1,1,1,1,1,1,0,}}{%
        \ifthenelse{\equal{#2}{1}}{\sevenseg[#1]{0,1,1,0,0,0,0,}}{%
            \ifthenelse{\equal{#2}{2}}{\sevenseg[#1]{1,1,0,1,1,0,1,}}{%
                \ifthenelse{\equal{#2}{3}}{\sevenseg[#1]{1,1,1,1,0,0,1,}}{%
                    \ifthenelse{\equal{#2}{4}}{\sevenseg[#1]{0,1,1,0,0,1,1,}}{%
                        \ifthenelse{\equal{#2}{5}}{\sevenseg[#1]{1,0,1,1,0,1,1,}}{%
                            \ifthenelse{\equal{#2}{6}}{\sevenseg[#1]{1,0,1,1,1,1,1,}}{%
                                \ifthenelse{\equal{#2}{7}}{\sevenseg[#1]{1,1,1,0,0,0,0,}}{%
                                    \ifthenelse{\equal{#2}{8}}{\sevenseg[#1]{1,1,1,1,1,1,1,}}{%
                                        \ifthenelse{\equal{#2}{9}}{\sevenseg[#1]{1,1,1,1,0,1,1,}}{%
                                            {\sevenseg[#1]{0,0,0,0,0,0,0,}}
                                        }
                                    }
                                }
                            }
                        }
                    }
                }
            }
        }
    }%
}

% Choose engine primitive
\makeatletter
\@ifundefined{pdfuniformdeviate}{%
  \let\RandDeviate\uniformdeviate   % XeLaTeX / LuaLaTeX
}{%
  \let\RandDeviate\pdfuniformdeviate % pdfLaTeX
}
\makeatother

% Fully expandable random bit (0 or 1)
\newcommand{\randbit}{\the\numexpr\RandDeviate 2\relax}

% Convert one random draw to a clean literal 0/1
\newcommand{\randbitZ}{%
  \ifnum\numexpr\RandDeviate 2\relax>0 1\else 0\fi
}

% Use it to build the seven-seg list
\newcommand{\randsevenseg}{%
  \sevenseg[]{\randbitZ,\randbitZ,\randbitZ,\randbitZ,\randbitZ,\randbitZ,\randbitZ}%
}


% Copyright 2018-2022 FIUS
%
% This file is part of theo-vorkurs-folien.
%
% theo-vorkurs-folien is free software: you can redistribute it and/or modify
% it under the terms of the GNU General Public License as published by
% the Free Software Foundation, either version 3 of the License, or
% (at your option) any later version.
%
% theo-vorkurs-folien is distributed in the hope that it will be useful,
% but WITHOUT ANY WARRANTY; without even the implied warranty of
% MERCHANTABILITY or FITNESS FOR A PARTICULAR PURPOSE.  See the
% GNU General Public License for more details.
%
% You should have received a copy of the GNU General Public License
% along with theo-vorkurs-folien.  If not, see <https://www.gnu.org/licenses/>.



% Configuration for slides

% The date of the first day of the Theo-Vorkurs in Format dd/mm/yyyy
\SetDate[10/10/2022]

% Invite URL to the Ersti-Telegram-Group. Used for text on slide as well as QR-Code
\newcommand\telegramurl{https://t.me/+Q92w5biyY903NjEy}

\title{Vorkurs Theoretische Informatik}
\subtitle{Grundlagen der Beweise}
\date{Dienstag, 01.10.2019}
\author{Arbeitskreis Theoretische Informatik}
\institute{Fachgruppe Informatik}
% \titlegraphic{\hfill\includegraphics[height=1.5cm]{logo.pdf}}

\begin{document}

\maketitle

\begin{frame}[fragile]{Übersicht}
  \setbeamertemplate{section in toc}[sections numbered]
  \tableofcontents
  % [hideallsubsections]
\end{frame}

\section{Beweisen}

% Copyright 2018-2022 FIUS
%
% This file is part of theo-vorkurs-folien.
%
% theo-vorkurs-folien is free software: you can redistribute it and/or modify
% it under the terms of the GNU General Public License as published by
% the Free Software Foundation, either version 3 of the License, or
% (at your option) any later version.
%
% theo-vorkurs-folien is distributed in the hope that it will be useful,
% but WITHOUT ANY WARRANTY; without even the implied warranty of
% MERCHANTABILITY or FITNESS FOR A PARTICULAR PURPOSE.  See the
% GNU General Public License for more details.
%
% You should have received a copy of the GNU General Public License
% along with theo-vorkurs-folien.  If not, see <https://www.gnu.org/licenses/>.

\begin{frame}{Einführung}
    \begin{alertblock}{Was ist ein Beweis?}
        \begin{itemize}
            \item lückenlose Folge von logischen Schlüssen,\\welche zur zu beweisenden Behauptung führen
            \item nicht nur einleuchtend, sondern zweifelsfrei korrekt
        \end{itemize}
    \end{alertblock}
    \onslide<2|handout:1>{
        \begin{alertblock}{Warum beweisen?}
            \begin{itemize}
                \item Aussage basierend auf Fakten und nicht subjektiv belegen
                \item Bestätigung von Aussagen für weitere Nutzung
                \item Zeigen der absoluten Wahrheit
            \end{itemize}
        \end{alertblock}
    }
\end{frame}

\subsection{Beweisbeispiel: Transitivität der Teilmenge}
\begin{frame}[fragile]{Beispielbeweis}
    \begin{exampleblock}{Zu zeigen: Teilmengen sind transitiv.}
        \begin{enumerate}
            \item<1->\alert<1|handout:0>{
                      \only<1|handout:0>{zu zeigen: }\onslide<2->{z.z. }$A\subseteq B\wedge B\subseteq C \alert<3|handout:0>{\implies\text{}}A\subseteq C$
                  }
            \item<2->\alert<2|handout:0>{
                      \only<2|handout:0>{Umschreiben:\\}
                      $\iff $\alert<4,5|handout:0>{$($\alert<9|handout:0>{$($\alert<6|handout:0>{$\forall x$}$\ : x \in A \implies x \in B)$}$ \wedge $\alert<10|handout:0>{$($\alert<6|handout:0>{$\forall x$}$\ : x \in B \implies x \in C)$}$)$}\\
                  \qquad\alert<3|handout:0>{$\implies$}$\;($\alert<6|handout:0>{$\forall x$}$\ :\ $\alert<7|handout:0>{$x \in A$}$ \implies x \in C)$
                  }
            \item<3->\alert<3|handout:0>{
                      \only<3|handout:0>{\emph{Implikation}\\
                          linke Seite wahr $\implies$ rechte Seite muss wahr sein.\\
                          linke Seite falsch $\implies$ beliebiges kann folgen\\
                          $\implies$ uns interessiert also nur der Fall \emph{links ist wahr}}
                      \alert<4>{\only<4,5|handout:0>{Wir machen uns also \emph{\textquotedbl die linke Seite ist wahr\textquotedbl} zur Voraussetzung}\alert<5>{\only<5|handout:0>{:\\Angenommen, $A \subseteq B \wedge B \subseteq C$ gilt.}}}
                  \onslide<6->{Ang., $A \subseteq B \wedge B \subseteq C$.}
                  }
            \item<6->\alert<6|handout:0>{
                      \only<6|handout:0>{Jetzt geht der Beweis richtig los.\\Wähle beliebiges $x$, um Allgemeinheit zu wahren\dots\\}
                      \onslide<6->{Sei $x$ beliebig}\alert<7>{\onslide<7-|handout:0>{ mit \alert<9>{$x\in A$.}}}
                  }
            \item<8->\alert<8-9|handout:0>{
                      \only<8|handout:0>{Wir können jetzt unsere Voraussetzungen ausnutzen,\\um $x\in C$ zu folgern.}
                      \onslide<9->$\implies x\in B$
                      \alert<10|handout:0>{\onslide<10->$\implies x\in C$}
                      \onslide<11>\qed
                  }
        \end{enumerate}
    \end{exampleblock}
\end{frame}


\subsubsection{Beweistechnik: Direkter Beweis}
\begin{frame}[fragile]{Direkter Beweis}
    \begin{alertblock}{Zeige A$\implies$B direkt}
    Setze A voraus und folgere dann schrittweise B.\\
    Durch jede korrekte Folgerung, vergrößert sich die Menge der als wahr bekannten Aussagen (sog. Annahmen).
    \end{alertblock}
    \metroset{block=fill}
    \begin{exampleblock}{Beispiel}
    Z.z.: \alert<2>{$\forall a\in\mathbb{Z}$}: \alert<3>{a ist gerade} $\implies$ \alert<6>{$a^2$ gerade.}
    \begin{enumerate}
        \item \alert<2>{Sei $a\in\mathbb{Z}$ beliebig.}
        \item \alert<3>{Angenommen, a ist gerade.}
        \item \alert<4>{$\implies \exists n\in\mathbb{Z} : a = 2n$}
        \item \alert<5>{$\implies a^2 = (2n)^2 = 4n^2 = 2 \cdot 2n^2$}
        \item \alert<6>{$\implies a^2$ ist gerade}\qed\;
    \end{enumerate}
    \end{exampleblock}
    \small{\emph{\alert<4>{Anmerkung:}} Zahl $n\in\mathbb{Z}$ heißt gerade, wenn es ein $k\in\mathbb{Z}$ gibt mit $n=2k$.}
\end{frame}

\subsubsection{Beweistechnik: Kontraposition}
\begin{frame}[fragile]{Beweis durch Kontraposition}
    \begin{alertblock}{Zeige A$\implies$B, indem man stattdessen $\neg$B$\implies\neg$A zeigt.}
    \end{alertblock}
    \metroset{block=fill}
    \begin{exampleblock}{Beispiel}
    Z.z.: \alert<7>{\alert<1>{$\forall n\in\mathbb{N}$:} $n^2$ gerade $\implies$\alert<2>{ $n$ gerade}}
    \begin{enumerate}
        \item\alert<1>{Sei $n \in \mathbb{N}$ beliebig.}
        \item\alert<2>{Angenommen, $n$ ist \emph{nicht} gerade.}
        \item\alert<3>{$\implies n=2k+1$, für ein $k \in \mathbb{Z}$}
        \item $\only<4>{\overset{\alert{quadrieren}}}{\; \leadsto \;} n^2 = (2k+1)^2 = 4k^2+4k+1 = 2\alert<5>{(2k^2+2k)}+1$
        \item $\implies n^2= \alert<6>{2\alert<5>{m}+1}$, für $m=2k^2+2k$
        \item $\implies n^2$ ist \alert<6>{ungerade}.
        \item Da ($\forall n\in\mathbb{N}$: $n$ ungerade $\implies n^2$ ungerade) gilt, \\
        $\leadsto$ \alert<7>{($\forall n\in\mathbb{N}$: $n^2$ gerade $\implies n$ gerade)}, was zu beweisen war.\qed\;
    \end{enumerate}
    \end{exampleblock}
    \footnotesize{\alert<3,6>{\emph{Anmerkung:}} Zahl $n\in\mathbb{Z}$ heißt ungerade, wenn es ein $k\in\mathbb{Z}$ gibt mit $n=2k+1$.}
\end{frame}

\begin{frame}[fragile]{Beweis durch Kontraposition}
%Work in progress
\begin{alertblock}{Wieso dürfen wir das so machen?}.
\end{alertblock}
\metroset{block=fill}
\begin{exampleblock}{Beweis}
Z.z.: ($\neg$A$\implies\neg$B)$\iff$(B$\implies$A)
    \begin{flalign*}
        \;(\neg A\implies\neg B) \iff & (\neg (\neg A) \vee \neg B)&\\
        \iff & (A \vee \neg B)&\\
        \iff & (\neg B \vee A)&\\
        \iff & (B \implies A)&\qed\;
    \end{flalign*}
\end{exampleblock}
\small\emph{Erinnerung:} $A\implies B$ kann man auch $\neg A\vee B$ schreiben.
\end{frame}

\begin{frame}[standout]
    Verdauungspause
\end{frame}

\subsubsection{Beweistechnik: Widerspruch}
\begin{frame}[fragile]{Beweis durch Widerspruch}
\small{
    \begin{alertblock}{Zeige, dass A gilt, indem man zeigt dass $\neg$A falsch ist.}
    %Spezialfall der Kontraposition: $A\text{ ist wahr}\implies A \iff \neg A\implies \text{falsch}$
    \emph{Erinnerung:} Eine Aussage ist entweder wahr oder falsch.\\
    Wenn $\neg$A falsch ist, muss A wahr sein.
    \end{alertblock}
    \metroset{block=fill}
    \begin{exampleblock}{Beispiel}
        Z. z. $\sqrt{2}$ ist irrational.
        \begin{enumerate}
            \item<1-> \alert<1>{Ang. $\sqrt{2}$ ist rational.}
            \item<2-> \alert<2>{Dann $\exists p, q \in \mathbb{Z} : \sqrt{2} = \frac{p}{q}$} $\wedge$ \alert<3,11>{$p, q$ sind teilerfremd.}
            \only<2,3>{\alert<2>{{\\\emph{Anmerkung:}}} $r\in\mathbb{Q}\iff\exists p,q\in\mathbb{Z}:r=\frac{p}{q}$}.
            \only<2,3>{\alert<3>{{\\\emph{Anmerkung:}}} $\frac{p}{q}$ kann man immer soweit kürzen, dass $p,q$ teilerfremd sind.}
            \item<4-> \only<4>{Quadrieren und Umformen:\\}$\leadsto(\sqrt{2})^2 = (\frac{p}{q})^2 \iff 2 = \frac{p^2}{q^2} \iff \alert<5,8>{2q^2=p^2}$
            \item<5-> \alert<5>{$\leadsto p^2$ ist gerade.} \alert<6,10>{$\leadsto p$ ist gerade.}\only<5,6>{\alert<5>{{\\\emph{Erinnerung:}}} $\forall n\in\mathbb{Z}:n$ gerade $\iff\exists k\in\mathbb{Z}:2k=n$.}\only<5,6>{\alert<6>{{\\\emph{Erinnerung:}}} $\forall n\in\mathbb{N}$: $n^2$ gerade $\implies n$ gerade \\\qquad\qquad\quad(siehe Beispiel Kontraposition)}
            \item<7-> \alert<7>{Also ist $p^2$ durch 4 teilbar.} \alert<8>{$\leadsto 2q^2$ ist durch 4 teilbar.}\only<7>{\\\alert<7>{\emph{Herleitung:}} $p^2=p\cdot p\overset{\text{p gerade}}{=\joinrel=\joinrel=\joinrel=}(2k)\cdot(2k)=\alert<7>{4}k^2,$ mit $k\in\mathbb{Z}$}
            \item<9-> $\leadsto q^2$ ist gerade. \alert<10>{$\leadsto q$ ist gerade.}
            \item<10-> \alert<10>{$\leadsto p,q$ nicht teilerfremd.} \alert<11>{$\leadsto$ Widerspruch}\only<11>{\qed\;}
        \end{enumerate}
    \end{exampleblock}
}
\end{frame}

\begin{frame}[standout]
    Verdauungspause
\end{frame}


%\subsubsection{Tricks}
\begin{frame}[fragile]{Tricks: Fallunterscheidung}
    \begin{alertblock}{Hilfe! Der Beweis ist zu komplex! Was nun?}
        Manchmal lässt sich ein Beweis in kleinere Aussagen zerlegen. Wenn wir alle Teilaussagen beweisen, haben wir die Gesamtaussage gezeigt.
    \end{alertblock}
    \metroset{block=fill}
    \small\begin{exampleblock}{Beispiel}
        Z.z. für alle $n\in\mathbb{N}$ gilt, dass der Rest von $n^2 \div 4$ entweder 0 oder 1 ist.
        \footnotesize\begin{itemize}
            \item 
                \alert{Fall 1:} n ist gerade\\
                $n^2=n\cdot n\overset{\text{n gerade}}{=\joinrel=\joinrel=\joinrel=\joinrel=}(2k)\cdot(2k) = 4k^2$,  mit $k\in\mathbb{Z}$\\
                $\implies 4k^2 \div 4 = k^2$ Rest: 0
            \item \alert{Fall 2:} n ist ungerade\\
                $n^2=n\cdot n\overset{\text{n ungerade}}{=\joinrel=\joinrel=\joinrel=\joinrel=\joinrel=}(2k+1)\cdot(2k+1)=(2k)^2+2(2k)+1=4(k^2+k)+1$, \\mit $k\in\mathbb{Z}$\\
                $\implies (4(k^2+k)+1) \div 4= k^2+k$ Rest: 1
        \end{itemize}
        Da $n$ nur gerade oder ungerade sein kann, ist der Rest von $n^2\div4$ \\entweder 0 oder 1. \qed\;
    \end{exampleblock}
\end{frame}

\begin{frame}[fragile]{Tricks: Beispiele und Gegenbeispiele}
    \begin{alertblock}{Reicht nicht auch ein Beispiel als Beweis?}
      % Manchmal\dots
    \end{alertblock}
    \metroset{block=fill}
    \begin{block}{Wann ein Beispiel \emph{nicht} ausreicht:}
        Zeige allgemeine Aussagen, also Aussagen der Form:\\$\forall n\in\mathbb{N}$ gilt \dots, $\neg\exists n\in\mathbb{N}$\dots, $\exists!n\in\mathbb{N}$\dots, etc.\\
        \alert{Warum nicht?}\\
        Beispiele zeigen uns nur endlich viele Möglichkeiten.\\
        \glqq für Alle gilt\dots\grqq, \glqq es existiert kein\dots\grqq, \glqq es existiert genau ein\dots\grqq, etc. \\sind meist zu allgemeine Aussagen um sie mit endlich vielen Beispielen lückenlos zu beweisen.
    \end{block}
\end{frame}

\begin{frame}[fragile]{Tricks: Beispiele und Gegenbeispiele}
    \begin{alertblock}{Reicht nicht auch ein Beispiel als Beweis?}
      % Manchmal\dots
    \end{alertblock}
    \metroset{block=fill}
    \begin{block}{Wann ein Beispiel ausreichen kann:}
        Zeige nicht allgemeine Aussagen der Form:\\
        $\exists n\in\mathbb{N}$, $\neg\forall n\in\mathbb{N}$ gilt, \dots\\
        \alert{Warum?}\\
        \glqq es gibt ein Element, sodass\dots\grqq, \glqq für nicht alle Element gilt\dots\grqq\\wären durch Angabe eines solchen Elements gezeigt.
    \end{block}
        $\leadsto$ will man zeigen, dass eine Aussage falsch ist, sind die Formen entsprechend negiert.
\end{frame}


%\subsubsection{Aufgaben}
{\setbeamercolor{palette primary}{bg=ExColor}
\begin{frame}[fragile]{Denkpause}
    \begin{alertblock}{Aufgaben}
    Welche Beweistechnik könnte sich für die folgenden Aussagen eignen? Warum?
    \end{alertblock}
    
    \metroset{block=fill}
    \begin{block}{Normal}
        \begin{itemize}
            \item Für jede Primzahl $p$ ist $2^p-1$ eine Primzahl.
        \end{itemize}
    \end{block}
    \metroset{block=fill}
    \begin{block}{Etwas schwerer}
    \begin{itemize}
            \item Für jede ganze Zahl $x$ gilt $x\equiv 1\bmod 4 \implies x\equiv 1\bmod 2$
    \end{itemize}
        
    \end{block}
\end{frame}
}

{\setbeamercolor{palette primary}{bg=ExColor}
\begin{frame}[fragile]{Lösungen}
    \begin{alertblock}{Aufgaben}
    Welche Beweistechnik könnte sich für die folgenden Aussagen eignen? Warum?
    \end{alertblock}
    
    \metroset{block=fill}
    \begin{block}{Normal}
        \begin{itemize}
            \item Für jede Primzahl $p$ ist $2^p-1$ eine Primzahl.\\
            $\rightarrow$ Gegenbeispiel (sei $p\defeq11$)
        \end{itemize}
    \end{block}
    \metroset{block=fill}
    \begin{block}{Etwas schwerer}
        \begin{itemize}
            \item[] $\rightarrow$Direkter Beweis
            \item $x\equiv 1\bmod 4\iff\exists z\in\mathbb{Z} : 4 \cdot z + 1 = x$
            \item $\iff\exists z\in\mathbb{Z}: (2 \cdot 2) \cdot z + 1 = x$
            \item $\implies\exists u,z\in\mathbb{Z}: u = 2z\wedge 2u + 1 = x$
            \item $\implies x\equiv 1\bmod 2$
        \end{itemize}
    \end{block}
\end{frame}
}


\begin{frame}[standout]
  Induktion folgt morgen
\end{frame}

\begin{frame}[standout]
    Murmelpause
\end{frame}


\section{Mengenbeweise}

\begin{frame}{einfacher Einstieg}
        \onslide
            Zu zeigen: Schnitt ist Kommutativ, d.h. $A \cap B = B \cap A$
        \begin{columns}
        \column{0.5\textwidth}
        \onslide{
            \begin{align*}
                \text{\quotedblbase}\implies\text{\textquotedblright}:\\
                x\in A \cap B &\implies x\in A \wedge x \in B\\
                &\implies x\in B \wedge x \in A\\
                &\implies x\in B \cap A
            \end{align*}
            }
        \column{0.5\textwidth}
        \onslide{
            \begin{align*}
                \text{\quotedblbase}\impliedby\text{\textquotedblright}:\\
                x\in B \cap A &\implies x\in B \wedge x \in A\\
                &\implies x\in A \wedge x \in B\\
                &\implies x\in A \cap B
            \end{align*}
            }
        \end{columns}
        \qed\\
    \small{\emph{Anmerkung:} $\wedge$ ist kommutativ}
\end{frame}

\begin{frame}{einfacher Einstieg}
        \onslide
            Zu zeigen: $A \setminus (B \cup C) = (A \setminus B) \cap (A \setminus C)$
        \onslide{
            \begin{align*}
                x\in A \setminus (B \cup C) &\iff x\in A \wedge \neg (x \in B \cup C)\\
                &\iff x\in A \wedge \neg (x \in B \vee x \in C)\\
                &\iff x \in A \wedge \neg (x \in B) \wedge \neg (x \in C)\\
                &\iff x \in A \wedge \neg (x \in B) \wedge x \in A \wedge \neg (x \in C)\\
                &\iff (x \in A \wedge \neg (x \in B)) \wedge (x \in A \wedge \neg (x \in C))\\
                &\iff (x \in A \setminus B) \wedge (x \in A \setminus C)\\
                &\iff x \in (A \setminus B) \cap (A \setminus C)
            \end{align*}\qed
            }
        \\
    \small{\emph{Rechenregel:} $\neg (A \wedge B) \iff \neg A \vee \neg B$, \\ \hspace{1.9cm}$\neg (A \vee B) \iff \neg A \wedge \neg B$}
\end{frame}



\subsubsection{Aufgaben}
{\setbeamercolor{palette primary}{bg=ExColor}
\begin{frame}[fragile]{Denkpause}
    \begin{alertblock}{Aufgaben}
    Versuche dich an den folgenden Mengenbeweisen.
    \end{alertblock}
    
    \metroset{block=fill}
    \begin{block}{Normal}
        \begin{itemize}
            \item $\overline{\overline{A}} = A$
        \end{itemize}
    \end{block}
    \metroset{block=fill}
    \begin{block}{Etwas schwerer}
        \begin{itemize}
            \item $A\cap B=\overline{(\overline{A}\cup\overline{B})}$
        \end{itemize}
    \end{block}
\end{frame}
}

{\setbeamercolor{palette primary}{bg=ExColor}
\begin{frame}[fragile]{Lösungen}
\onslide Zu zeigen: $A=\overline{\overline{A}}$
    \onslide{
    \begin{align*}
        x\in\overline{\overline{A}}
        &\iff\neg(x\in\overline{A})\\
        &\iff\neg(\neg (x\in A))\\
        &\iff x\in A
    \end{align*}\qed
    }
\end{frame}
}

{\setbeamercolor{palette primary}{bg=ExColor}
\begin{frame}[fragile]{Lösungen}
    \onslide Zu zeigen: $B\cap A=\overline{(\overline{A}\cup\overline{B})}$
    \onslide{
    \begin{align*}
        x \in \overline{(\overline{A} \cup \overline{B})}
        &\iff \neg(x \in \overline{A} \cup \overline{B})
        \\&\iff \neg(x \in \overline{A} \vee x \in \overline{B})
        \\&\iff \neg(\neg(x \in A) \vee \neg(x \in B))
        \\&\iff x \in A \wedge x \in B
        \\&\iff x \in A \cap B
        \\&\iff x \in B \cap A
    \end{align*}\qed
    }
\end{frame}
}


%\section{Weitere Mengenbeweise}

% Copyright 2018-2022 FIUS
%
% This file is part of theo-vorkurs-folien.
%
% theo-vorkurs-folien is free software: you can redistribute it and/or modify
% it under the terms of the GNU General Public License as published by
% the Free Software Foundation, either version 3 of the License, or
% (at your option) any later version.
%
% theo-vorkurs-folien is distributed in the hope that it will be useful,
% but WITHOUT ANY WARRANTY; without even the implied warranty of
% MERCHANTABILITY or FITNESS FOR A PARTICULAR PURPOSE.  See the
% GNU General Public License for more details.
%
% You should have received a copy of the GNU General Public License
% along with theo-vorkurs-folien.  If not, see <https://www.gnu.org/licenses/>.

\begin{frame}{Weiterer Mengenbeweis}
    Ein weiterer Mengenbeweis...
    \metroset{block=fill}
    \begin{block}{\alert{Aufgabe}}
    
    $L_1=\{6m \mid m \in \mathbb{N}\}$\\
    $L_2=\{2n \mid n \in \mathbb{N}\}$\\
    Zu zeigen:\\
    $L_1 \subsetneq L_2$\\
    d.h. $(\forall x: x \in L_1 \implies x \in L_2) \wedge (\exists x: x \in L_2 \land x \notin L_1)$
    
   
   

\end{block}

\end{frame}

\begin{frame}{Weiterer Mengenbeweis}
    \metroset{block=fill}
    \only<1|handout:1>{
    \begin{block}{\alert{$\forall x: x \in L_1 \implies x \in L_2$}}
    Sei $x$ beliebig.\\
    Angenommen, $x \in L_1$.\\
    Es gilt: $x = 6m = 2 \cdot 3 \cdot m = 2 \cdot (3m)$\\
    Da $m \in \mathbb{N}$, setze $n=3m$\\
    Daraus folgt: $x = 2n$\\
    $\leadsto x \in L_2$
    \end{block}
    }
    \only<2|handout:1>{
    \begin{block}{\alert{$\exists x: x \in L_2 \land x \notin L_1$}}
    Beweis durch Angabe eines Gegenbeispiels: $x=2$\\
    $x \in L_2$, für $n=1$, da $x=2 \cdot 1 \in L_2$,\\
    \vspace{0.3cm}
    aber $6m=2$ mit $m=\frac{2}{6}=\frac{1}{3} \notin \mathbb{N}$\\
        $\leadsto x \notin L_1$
    \end{block}
    }
    \only<3|handout:2>{
    Da gezeigt wurde:\\
       \vspace{0.3cm}
    $\forall x: x \in L_1 \implies x \in L_2$\\
    \alert{und}\\
    $\exists x: x \in L_2 \land x \notin L_1$\\
    \vspace{0.3cm}
    \textbf{gilt $L_1 \subsetneq L_2$.}
    }
\end{frame}

{\setbeamercolor{palette primary}{bg=ExColor}
\begin{frame}[fragile]{Denkpause}
    \begin{alertblock}{Aufgabe}
    Versuche dich an folgendem Mengenbeweis.
    \end{alertblock}
    
    \metroset{block=fill}
    \begin{block}{Etwas Schwerer}
   
    $L_1=\{l \in \mathbb{Z}\mid$ l ist gerade\}\\
    $L_2=\{n \in \mathbb{Z}\mid$ n ist durch 3 teilbar\}\\
    $L_3=\{m \in \mathbb{Z}\mid$ m ist durch 6 teilbar\}\\
    Zu zeigen: $L_1 \cap L_2 = L_3$\\
    d.h. $(\forall x: x \in L_1 \cap L_2 \Rightarrow x \in L_3) \land (\forall x: x \in L_3 \Rightarrow x \in L_1 \cap L_2)$\\

    \end{block}

\end{frame}
}

{\setbeamercolor{palette primary}{bg=ExColor}
\begin{frame}<handout:0>[fragile]{Lösung}
    \begin{alertblock}{Aufgaben}
   Schritt 1: Zu zeigen: $\forall x: x \in L_1 \cap L_2 \Rightarrow x \in L_3$
\end{alertblock}
    
    Sei x beliebig.\\
    Angenommen, $x \in L_1 \cap L_2$\\
    Dann ist x gerade $\Rightarrow$ $x= 2k$ für $k \in \mathbb{Z}$\\
    Außerdem durch 3 teilbar $\Rightarrow$ $x= 3m$ für $m \in \mathbb{Z}$\\
    $\Rightarrow$ da x ein Vielfaches von 2 und 3 ist folgt: $\exists l \in \mathbb{Z}$, sodass $x=2\cdot 3 \cdot l = 6l$\\
    $\leadsto x \in L_3$


   
\end{frame}
}
{\setbeamercolor{palette primary}{bg=ExColor}
\begin{frame}<handout:0>[fragile]{Lösung}
    \begin{alertblock}{Aufgaben}
   Schritt 2: Zu zeigen: $\forall x: x \in L_3 \Rightarrow x \in L_1 \cap L_2$\\
\end{alertblock}
    
    Sei x beliebig.\\
    Angenommen, $x \in L_3$\\
    Dann ist $x = 6k$ für $k \in \mathbb{Z}$\\
    Da $x = 2 \cdot (3k)$, ist x gerade\\
    Da $x= 3 \cdot (2k)$, ist x durch 3 teilbar\\
    $\leadsto x \in L_1 \cap L_2$


    
\end{frame}
}

{\setbeamercolor{palette primary}{bg=ExColor}
\begin{frame}<handout:0>[fragile]{Lösung}
      
     
    Da gezeigt wurde:\\
       \vspace{0.3cm}
    $\forall x: x \in L_1 \cap L_2 \Rightarrow x \in L_3$\\
    \alert{und}\\
    $\forall x: x \in L_3 \Rightarrow x \in L_1 \cap L_2$\\
    \vspace{0.3cm}
    \textbf{gilt $L_1 \cap L_2 = L_3$.}
    
 
\end{frame}
}



%\section{Wiederholung} 
%\begin{frame}[fragile]{Das können wir jetzt beantworten}
%    \begin{alertblock}{Logik}
%    \begin{itemize}
%        \item bla
%    \end{itemize}
%    \end{alertblock}
%\end{frame}

\begin{frame}[standout]
  Noch Fragen?
\end{frame}

 \begin{frame}[fragile]{Glossar}
     \small
     \begin{tabular}{p{0.2\textwidth} p{0.25\textwidth} p{0.4\textwidth}}
     \toprule
     Abk.&Bedeutung&Was?!\\
     \midrule
         z.z. & zu zeigen & Was zu beweisen ist\\
         Sei&&bereits bekannte Objekte werden eingeführt und benannt\\
         $\exists$&es gibt ein&\\
         $\exists !$&es gibt genau ein&\\
         x ist genau y&x = y&\emph{genau} wird verwendet bei Äquivalenz\\
         x ist eindeutig&$\exists ! x$&\\
         der, die, das&&bestimmte Artikel weisen auf Eindeutigkeit hin\\
         gdw.&genau dann wenn&Äquivalenz zwischen Aussagen\\
     \bottomrule
     \end{tabular}
 \end{frame}
\begin{frame}[fragile]{Glossar}
    \small
    \begin{tabular}{p{0.33\textwidth} p{0.12\textwidth} p{0.4\textwidth}}
    \toprule
    Abk.&Bedeutung&Was?!\\
    \midrule
        A ist notwendig für B&B$\implies$A&A muss wahr sein,\\
        &&wenn B wahr ist\\
        A ist hinreichend für B&A$\implies$B&B muss wahr sein,\\
        &&wenn A wahr ist\\
        notwendig und hinreichend&A$\iff$B&genau dann wenn\\
    \bottomrule
    \end{tabular}
\end{frame}

\begin{frame}[fragile]{Glossar}
    \small
    \begin{tabular}{p{0.1\textwidth} p{0.33\textwidth} p{0.45\textwidth}}
    \toprule
    Abk.&Bedeutung&Was?!\\
    \midrule
        \OE & ohne Einschränkung & die Allgemeinheit der Aussage wird nicht durch getroffene Aussagen eingeschränkt\\
        o.B.d.A. & ohne Beschränkung der Allgemeinheit & wie \OE\\
        trivial&offensichtlich&Beweisschritte, welche keine weiter Begründung brauchen. (nicht verwenden!)\\
        $\qed$&Mic Drop&Kommt am Ende eines erfolgreichen Beweises\\
        q.e.d&quod erat demonstrandum&Was zu beweisen war\\
    \bottomrule
    \end{tabular}
\end{frame}

 \begin{frame}[fragile]{Cheatsheet}
     \small
     \begin{tabular}{p{0.2\textwidth} p{0.7\textwidth}}
     \toprule
     Gestalt&mögliches Vorgehen\\
     \midrule
         nicht F&Zeige, dass F nicht gilt.\\
         F und G&Zeige F und G in zwei getrennten Beweisen.\\
         F $\implies$ G&Füge F in die Menge der Annahmen hinzu und zeige G.\\
         F oder G&Zeige: nicht F $\implies$ G. \\&(Alternativ zeige: nicht G $\implies$ F.)\\
         F $\iff$ G&Zeige: F $\implies$ G und G $\implies$ F.\\
         $\forall x \in A : F$&Sei x ein beliebiges Element aus A. Zeige dann F.\\
         $\exists x \in A : F$&Sei x ein konkretes Element aus A. Zeige dann F.\\
     \bottomrule
     \end{tabular}
 \end{frame}

\end{document}