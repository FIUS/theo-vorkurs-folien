% Copyright 2018-2024 FIUS
%
% This file is part of theo-vorkurs-folien.
%
% theo-vorkurs-folien is free software: you can redistribute it and/or modify
% it under the terms of the GNU General Public License as published by
% the Free Software Foundation, either version 3 of the License, or
% (at your option) any later version.
%
% theo-vorkurs-folien is distributed in the hope that it will be useful,
% but WITHOUT ANY WARRANTY; without even the implied warranty of
% MERCHANTABILITY or FITNESS FOR A PARTICULAR PURPOSE.  See the
% GNU General Public License for more details.
%
% You should have received a copy of the GNU General Public License
% along with theo-vorkurs-folien.  If not, see <https://www.gnu.org/licenses/>.

\documentclass[aspectratio=43,10pt]{beamer}

\usetheme[progressbar=frametitle,subsectionpage=progressbar]{metropolis}
\usepackage{appendixnumberbeamer}
\usepackage[ngerman]{babel}
\usepackage[utf8]{inputenc}
%\usepackage{t1enc}
\usepackage{iftex}

\ifLuaTeX
    \usepackage{fontspec}
\else
    \ifxetex
        \usepackage{fontspec}
    \else
        \usepackage[T1]{fontenc}
    \fi
\fi

\usepackage[sfdefault,scaled=.85,lf]{FiraSans}
\usepackage{newtxsf}

\usepackage{booktabs}
\usepackage[scale=2]{ccicons}
\usepackage{hyperref}

\usepackage{pgf}
\makeatletter
\@ifclasswith{beamer}{notes}{
    \usepackage{pgfpages}
    \setbeameroption{show notes on second screen}
}{}
\makeatother
\usepackage{tikz}
\usetikzlibrary{arrows,automata,positioning,shapes,arrows.meta,shapes.geometric, through, calc}
\usepackage{pgfplots}
\usepgfplotslibrary{dateplot}

\usepackage{xspace}
\newcommand{\themename}{\textbf{\textsc{metropolis}}\xspace}

\usepackage{blindtext}
\usepackage{graphicx}
\usepackage{subcaption}
\usepackage{comment}
\usepackage{mathtools}
\usepackage{amsmath}
\usepackage{centernot}
\usepackage{amssymb}
\usepackage{proof}
\usepackage{tabularx}
\renewcommand{\figurename}{Abb.}
\usepackage{tikzsymbols}
\let\Coffeecup\relax
\let\Heart\relax
\let\Smiley\relax
\usepackage{marvosym}
\usepackage{mathtools}
\usepackage{qrcode}
\usepackage{advdate}
\usepackage{ifthen}
\usepackage{tikz-among-us}
\usepackage{multicol}
\usepackage{outlines}
\usepackage{ulem}

% \IfFontExistsTF{Segoe UI}{%
%     \usefonttheme{professionalfonts}
%     \defaultfontfeatures{Scale = MatchLowercase}
%     \setsansfont{Consolas}[
%         Scale = 1.0,
%         BoldFont = Consolas ,
%         BoldItalicFont = Consolas ]
%     \ifLuaTeX
% }{
% }
% \else
% \ifxelatex
%     \IfFontExistsTF{Segoe UI}{%
%         \usefonttheme{professionalfonts}
%         \defaultfontfeatures{Scale = MatchLowercase}
%         \setsansfont{Consolas}[
%             Scale = 1.0,
%             BoldFont = Segoe UI Semibold ,
%             BoldItalicFont = Segoe UI Semibold Italic ]
%     }{
%     }
% \fi

\definecolor{Bluecreen}{HTML}{0873aa}

\makeatletter
\setlength{\metropolis@progressonsectionpage@linewidth}{1.1em}
\setlength{\metropolis@progressinheadfoot@linewidth}{2pt}
\makeatother

\setbeamercolor{progress bar}{%
    fg=Bluecreen,
    bg=Bluecreen!70!black!45
}

\makeatletter
\newlength{\metropolis@progressonsectionpage@blockwidth}%
\newlength{\metropolis@progressonsectionpage@blockborder}%
\setlength{\metropolis@progressonsectionpage@blockborder}{1pt}%
\setbeamertemplate{progress bar in section page}{
    \vspace{0.5\metropolis@progressonsectionpage@linewidth}
    \setlength{\metropolis@progressonsectionpage}{%
        \textwidth * \ratio{\insertframenumber pt}{\inserttotalframenumber pt}%
    }%
    \setlength{\metropolis@progressonsectionpage@blockwidth}{%
        0.05\textwidth - 0.05\metropolis@progressonsectionpage@blockborder
    }%
    \begin{tikzpicture}
        \fill[bg] (0,-\metropolis@progressonsectionpage@blockborder) rectangle (\textwidth, \metropolis@progressonsectionpage@linewidth + \metropolis@progressonsectionpage@blockborder);
        %\fill[fg] (0,0) rectangle (\metropolis@progressonsectionpage, \metropolis@progressonsectionpage@linewidth);

        \foreach \i in {1,...,20} {%
                \pgfmathparse{\insertframenumber*100/\inserttotalframenumber >= \i*100/20 ? 1 : 0}
                \ifthenelse {\pgfmathresult>0}
                {%
                    \pgfmathparse{\i*0.05-0.005}
                    \fill[fg] (\i*\metropolis@progressonsectionpage@blockwidth, 0) rectangle ++ (-\metropolis@progressonsectionpage@blockwidth + \metropolis@progressonsectionpage@blockborder, \metropolis@progressonsectionpage@linewidth);
                }
                {\node at (\i,0) {\i};}% otherwise do nothing
            }

        \node[color=white] at (0.5\textwidth, 0.5\metropolis@progressonsectionpage@linewidth) {\textnormal{%
                \fontsize{0.85\metropolis@progressonsectionpage@linewidth}{\metropolis@progressonsectionpage@linewidth}\selectfont loading
                \pgfmathparse{\insertframenumber*100/\inserttotalframenumber}%
                \pgfmathprintnumber[fixed,precision=2]{\pgfmathresult}\,\% complete...%
            }%
        };

    \end{tikzpicture}%
}
\makeatother

\newcommand\daynr{0}


\definecolor{ExColor}{HTML}{17819b}

\newcommand{\emptyWord}{\varepsilon}
\let \emptyset\varnothing
\newcommand{\SigmaStern}{\Sigma^{*}}
\newcommand{\absval}[1]{|#1|}
\newcommand{\defeq}{\vcentcolon=}
\newcommand{\eqdef}{=\vcentcolon}
\newcommand{\nimplies}{\centernot\implies}

\newcommand{\naturals}{\ensuremath{\mathbb{N}}}
\newcommand{\integers}{\ensuremath{\mathbb{Z}}}
\newcommand{\rationals}{\ensuremath{\mathbb{Q}}}
\newcommand{\reals}{\ensuremath{\mathbb{R}}}
\newcommand{\iffspace}{\ensuremath{\iff\ }}

\newcommand{\sus}[1]{%
    \tikz{\node[scale=0.05] at (0,0) {\amongUsOriginal{#1}{cyan}}}
}

\setbeamertemplate{footline}[text line]
{\parbox{\linewidth}{Fachgruppe Informatik\hfill\insertpagenumber\hfill Vorkurs Theoretische Informatik\vspace{0.2in}}}

\newcommand{\Center}[1]{
    \begin{frame}<handout:0>[standout]
        #1
    \end{frame}
}

\newcommand{\cleft}[2][.]{%
    \begingroup\colorlet{savedleftcolor}{.}%
    \color{#1}\left#2\color{savedleftcolor}%
}
\newcommand{\cright}[2][.]{%
    \color{#1}\right#2\endgroup
}

% Make subsections in toc small
\makeatletter
\setbeamertemplate{subsection in toc}{\small\leftskip=2em\inserttocsubsection\par}
\makeatother

\AtBeginDocument{%

    % Fix section pages in appendix
    \apptocmd{\appendix}{%
        \setbeamertemplate{section page}[simple]%
    }{}{}
}

\addtobeamertemplate{block begin}{}{\vskip 0em}
\addtobeamertemplate{block alerted begin}{}{\vskip 0em}
\addtobeamertemplate{block example begin}{}{\vskip 0em}

\newsavebox\tikzBox
\newenvironment{includetikzpicture}[1]{%
    \def\sizeArgument{#1}\begin{lrbox}{\tikzBox}\begin{tikzpicture}
            }{%
        \end{tikzpicture}\end{lrbox}\resizebox{\sizeArgument}{!}{\usebox\tikzBox}%
}

\pgfkeys{
    /sevenseg/.is family, /sevenseg,
    shrink/.estore in     = \sevensegShrink,    % avoids overlapping of segments
    oncolor/.estore in    = \sevensegOncolor,   % color of an ON segment
    offcolor/.estore in   = \sevensegOffcolor,  % color of an OFF segment
    size/.estore in       = \sevensegSize,      % height
    onSize/.estore in     = \sevensegOnsize,     % line width if ON
    offSize/.estore in    = \sevensegOffsize    % line width if OFF
}

\pgfkeys{
    /sevenseg,
    default/.style = {
        shrink = 0.1, 
        size = 1em, 
        oncolor = red, 
        offcolor = blue!70!black!40,
        onSize = 2,
        offSize =1.5
        }
}

\newcommand{\sevenseg}[2][]% options, values
{%
  \pgfkeys{/sevenseg, default, #1}%
    \begin{tikzpicture}[x=\sevensegSize,y=\sevensegSize,baseline=.25*\sevensegSize]%
        % unten li
        \path (0,0) ++(0,0) coordinate (P1);
        % unten re
        \path (0,0) ++(0.5,0) coordinate (P2);
        % mitte li
        \path (0,0) ++(90:0.5) coordinate (P3);
        % mitte re
        \path (P2)  ++(90:0.5) coordinate (P4);
        % oben li
        \path (P3)  ++(90:0.5) coordinate (P5);
        % oben re
        \path (P4)  ++(90:0.5) coordinate (P6);
        % then step through the 1/0 values in the segment array


    \foreach \val [count=\i from 0] in {#2} {%
      \ifthenelse{\equal{\val}{1}}%
        {\let\mycolor=\sevensegOncolor \let\mysize=\sevensegOnsize}%
        {\let\mycolor=\sevensegOffcolor \let\mysize=\sevensegOffsize}%

      % then draw segment depending on \i
      \ifthenelse{\equal{\i}{0}}{\path[draw=\mycolor, line width=\mysize] (P5) -- (P6);}{}%
      \ifthenelse{\equal{\i}{1}}{\path[draw=\mycolor, line width=\mysize] (P6) -- (P4);}{}%
      \ifthenelse{\equal{\i}{2}}{\path[draw=\mycolor, line width=\mysize] (P4) -- (P2);}{}%
      \ifthenelse{\equal{\i}{3}}{\path[draw=\mycolor, line width=\mysize] (P1) -- (P2);}{}%
      \ifthenelse{\equal{\i}{4}}{\path[draw=\mycolor, line width=\mysize] (P1) -- (P3);}{}%
      \ifthenelse{\equal{\i}{5}}{\path[draw=\mycolor, line width=\mysize] (P3) -- (P5);}{}%
      \ifthenelse{\equal{\i}{6}}{\path[draw=\mycolor, line width=\mysize] (P3) -- (P4);}{}%
    }
  \end{tikzpicture}%
}

\newcommand{\sevensegnum}[2][]%
{%                                          
    \ifthenelse{\equal{#2}{0}}{\sevenseg[#1]{1,1,1,1,1,1,0,}}{%
        \ifthenelse{\equal{#2}{1}}{\sevenseg[#1]{0,1,1,0,0,0,0,}}{%
            \ifthenelse{\equal{#2}{2}}{\sevenseg[#1]{1,1,0,1,1,0,1,}}{%
                \ifthenelse{\equal{#2}{3}}{\sevenseg[#1]{1,1,1,1,0,0,1,}}{%
                    \ifthenelse{\equal{#2}{4}}{\sevenseg[#1]{0,1,1,0,0,1,1,}}{%
                        \ifthenelse{\equal{#2}{5}}{\sevenseg[#1]{1,0,1,1,0,1,1,}}{%
                            \ifthenelse{\equal{#2}{6}}{\sevenseg[#1]{1,0,1,1,1,1,1,}}{%
                                \ifthenelse{\equal{#2}{7}}{\sevenseg[#1]{1,1,1,0,0,0,0,}}{%
                                    \ifthenelse{\equal{#2}{8}}{\sevenseg[#1]{1,1,1,1,1,1,1,}}{%
                                        \ifthenelse{\equal{#2}{9}}{\sevenseg[#1]{1,1,1,1,0,1,1,}}{%
                                            {\sevenseg[#1]{0,0,0,0,0,0,0,}}
                                        }
                                    }
                                }
                            }
                        }
                    }
                }
            }
        }
    }%
}

% Choose engine primitive
\makeatletter
\@ifundefined{pdfuniformdeviate}{%
  \let\RandDeviate\uniformdeviate   % XeLaTeX / LuaLaTeX
}{%
  \let\RandDeviate\pdfuniformdeviate % pdfLaTeX
}
\makeatother

% Fully expandable random bit (0 or 1)
\newcommand{\randbit}{\the\numexpr\RandDeviate 2\relax}

% Convert one random draw to a clean literal 0/1
\newcommand{\randbitZ}{%
  \ifnum\numexpr\RandDeviate 2\relax>0 1\else 0\fi
}

% Use it to build the seven-seg list
\newcommand{\randsevenseg}{%
  \sevenseg[]{\randbitZ,\randbitZ,\randbitZ,\randbitZ,\randbitZ,\randbitZ,\randbitZ}%
}


% Copyright 2018-2022 FIUS
%
% This file is part of theo-vorkurs-folien.
%
% theo-vorkurs-folien is free software: you can redistribute it and/or modify
% it under the terms of the GNU General Public License as published by
% the Free Software Foundation, either version 3 of the License, or
% (at your option) any later version.
%
% theo-vorkurs-folien is distributed in the hope that it will be useful,
% but WITHOUT ANY WARRANTY; without even the implied warranty of
% MERCHANTABILITY or FITNESS FOR A PARTICULAR PURPOSE.  See the
% GNU General Public License for more details.
%
% You should have received a copy of the GNU General Public License
% along with theo-vorkurs-folien.  If not, see <https://www.gnu.org/licenses/>.



% Configuration for slides

% The date of the first day of the Theo-Vorkurs in Format dd/mm/yyyy
\SetDate[10/10/2022]

% Invite URL to the Ersti-Telegram-Group. Used for text on slide as well as QR-Code
\newcommand\telegramurl{https://t.me/+Q92w5biyY903NjEy}

\title{Vorkurs Theoretische Informatik}
\subtitle{Einführung in die Grundideen und in die Mengenlehre}
\date{Montag, 30.09.2019}
\author{Arbeitskreis Theoretische Informatik}
\institute{Fachgruppe Informatik}
% \titlegraphic{\hfill\includegraphics[height=1.5cm]{logo.pdf}}

\begin{document}

\maketitle

\begin{frame}[fragile]{Übersicht}
  \setbeamertemplate{section in toc}[sections numbered]
  \tableofcontents%[hideallsubsections]
\end{frame}

\section{Allgemeines} 

% Copyright 2018-2022 FIUS
%
% This file is part of theo-vorkurs-folien.
%
% theo-vorkurs-folien is free software: you can redistribute it and/or modify
% it under the terms of the GNU General Public License as published by
% the Free Software Foundation, either version 3 of the License, or
% (at your option) any later version.
%
% theo-vorkurs-folien is distributed in the hope that it will be useful,
% but WITHOUT ANY WARRANTY; without even the implied warranty of
% MERCHANTABILITY or FITNESS FOR A PARTICULAR PURPOSE.  See the
% GNU General Public License for more details.
%
% You should have received a copy of the GNU General Public License
% along with theo-vorkurs-folien.  If not, see <https://www.gnu.org/licenses/>.

\subsection{Organisatorisches}
\begin{frame}[fragile]{Wer sind wir?}
  \begin{itemize}
    \item
          Fachgruppe Informatik
          \begin{itemize}
            \item Unser Ziel: \\
                  Das Leben von uns Studis während des Studiums angenehmer zu gestalten
            \item organisieren Veranstaltungen (Grillen, Spieleabende, Vorkurse, ...)
            \item verleihen Prüfungen aus den früheren Semestern
            \item vertreten die studentische Sicht in offiziellen Gremien
            \item ...und vieles mehr (es gibt z.B. einen 3D-Drucker)
          \end{itemize}
    \item Arbeitskreis Theoretische Informatik
          \begin{itemize}
            \item Teilmenge der Fachgruppe Informatik
            \item haben diesen Vorkurs organisiert
          \end{itemize}
  \end{itemize}
\end{frame}
\note[itemize]{
  \item test
  \item test
}

\subsection{Tipps zum Studium}
\begin{frame}[fragile]{Tipps zum Studium}
  \begin{itemize}
    \item Nützliche Links:\\
          \begin{itemize}
            \item Fachgruppe Informatik:\\
                  \url{https://fius.de/}
            \item Handouts und Foliensätze:\\ \url{https://fius.de/index.php/studien-interessierte/vorkurs-theoretische-informatik/}
            \item Materialien Ergänzung Theoretische Informatik 1 (Wintersemester 19/20): \\
                  \url{https://fmi.uni-stuttgart.de/ti/teaching/w19/eti1/}
            \item Ersti Telegram-Gruppe:\\
                  \qrcode[hyperlink]{\telegramurl}
                  \url{\telegramurl}
          \end{itemize}
    \item E-Mail der Fachgruppe: fius@informatik.uni-stuttgart.de

  \end{itemize}
\end{frame}

\begin{frame}[fragile]{Infos zum Online-Ablauf}
  \begin{alertblock}{Ablauf und Notfallplan}
    \begin{itemize}
      \item Der Online-Vorkurs wird eine Übertragung aus/in einem Hörsaal sein
      \item Es wird zwischen Vorlesungs- und Aufgabephasen abgewechselt.
      \item Wir benutzen BigBlueButton - wenn ihr hier seid, wisst ihr das schon.
      \item Bei technischen Problemen, die sich nicht zügig beheben lassen, wechseln wir ggf. auf eine andere Plattform. Den Joinlink verschicken wir dann per Mail und stellen ihn auf \url{https://fius.de/index.php/studien-interessierte/vorkurs-theoretische-informatik/}.
    \end{itemize}
    \alert{Traut euch, Fragen zu stellen und mitzumachen.}
  \end{alertblock}
\end{frame}


\section{Theoretische Informatik}

% Copyright 2018-2022 FIUS
%
% This file is part of theo-vorkurs-folien.
%
% theo-vorkurs-folien is free software: you can redistribute it and/or modify
% it under the terms of the GNU General Public License as published by
% the Free Software Foundation, either version 3 of the License, or
% (at your option) any later version.
%
% theo-vorkurs-folien is distributed in the hope that it will be useful,
% but WITHOUT ANY WARRANTY; without even the implied warranty of
% MERCHANTABILITY or FITNESS FOR A PARTICULAR PURPOSE.  See the
% GNU General Public License for more details.
%
% You should have received a copy of the GNU General Public License
% along with theo-vorkurs-folien.  If not, see <https://www.gnu.org/licenses/>.

\subsection{Anwendung}
\begin{frame}[fragile]{Was ist eigentlich Theoretische Informatik?}
    \begin{itemize}
        \item Theoretische Informatik ist die \textbf{formale} Herangehensweise an Probleme.\\
        \item Grundlage für viele andere Gebiete der Informatik
    \end{itemize}
\end{frame}

\begin{frame}{Anwendung der theoretischen Informatik}
    \begin{itemize}
        \item Ist ein bestimmtes Problem lösbar, oder \textbf{können} wir gar keine Lösung finden?
        \item IT-Sicherheit / Kryptographie: Die Sicherheit bestimmter Algorithmen \textbf{beweisen}
        \item Reguläre Ausdrücke
        \item Künstliche Intelligenz
        \item Compilerbau
        \item ...und vieles mehr...
    \end{itemize}
\end{frame}

\subsection{Theoretische Informatik in deinem Studium}
\begin{frame}[fragile]{Theoretische Informatik in deinem Studium}
    Logik und diskrete Strukturen ist Orientierungsprüfung für Informatik, Medieninformatik, Softwaretechnik und Data Science.
    \begin{itemize}
        \item Du musst diese Prüfung spätestens zum Ende des dritten Semester bestanden haben.
        \item Du musst spätestens zum Ende des zweiten Semesters eine der beiden Orientierungsprüfungen angetreten haben.
        \item Du kannst die schriftliche Prüfung einmal nachschreiben und hast dann noch einen mündlichen Versuch im selben Semester.
    \end{itemize}
    \alert{Kennt eure \href{https://www.student.uni-stuttgart.de/pruefungsorganisation/pruefungsordnung/}{\underline{Prüfungsordnung}}!}
\end{frame}

\begin{frame}{Theoretische Informatik in deinem Studium}
    \begin{itemize}
        \item Theoretische Informatik I\\
              Logik und diskrete Strukturen\\
              \quad Dozent: Dr. Manfred Kufleitner
        \item Theoretische Informatik II\\
              Formale Sprachen und Berechnbarkeit\\
              \quad Dozent: Dr. Manfred Kufleitner
        \item Theoretische Informatik III\\
              Algorithmik und Komplexitätstheorie\\
              \quad Dozent: Prof. Stefan Funke
    \end{itemize}
    \alert{Altklausuren helfen bei der Prüfungsvorbereitung. \\Fragt auch nach den Klausuren des alten Fachs.}\\
    \textit{Ihr seid die ersten die die umstrukturierten Theo Fächer hören, die vohandenen Alklausuren passen vmtl. nicht auf die neue Strukturen.}
\end{frame}

%\begin{frame}{Literatur der Vorlesung}
%TODO: Rausfinden, welche Literatur in der VL empfohlen wird
%TODO: Ggf. gute YT-Kanäle finden
%\end{frame}


\section{Wörter, Sprachen und Mengen}

\begin{frame}[fragile]{Mengen}
    \begin{itemize}
        
        \item<1->
            Was ist eine \alert<1,2>{Menge}?
        \item<2->
            \only<2>{
            \vspace*{0.5cm}
                Eine Menge
                \begin{itemize}
                    \item ist eine \alert{Sammlung von Zeugs}
                    \item ist unsortiert
                    \item enthält keine Duplikate
                    \item wird mit geschweiften Klammern notiert
                \end{itemize}
                
                \metroset{block=fill}
                
                \begin{exampleblock}{Beispiel}
                    $\mathbb{N} = \{0, 1, 2, 3, \dots \}$ = Menge der Natürlichen Zahlen\\
                    Studenten = \{Janette, Julian, Joel, Fabian, $\dots$\}\\
                    $\{1,2\} = \{2,1\} = \{1,1,2,1,1,1\}$\\
                    $\emptyset = \{\} =$ leere Menge
                \end{exampleblock}}
            \uncover<3->{
            Was ist ein \alert<3,4>{Element}?}
        \item<4->
        \only<4>{
            \vspace*{0.5cm}
            Ein Element ist ein \alert{Ding aus einer Menge}.\\
            
            \metroset{block=fill}
                
            \begin{exampleblock}{Beispiel}
                $\mathbf{1}$ ist ein Element der \textbf{Natürlichen Zahlen}\\
                $\mathbf{1} \in \mathbb{N}$\\
                \vspace*{0.5cm}
                \textbf{Janette} ist ein Element aus der Menge der \textbf{Studenten}\\
                \textbf{Janette} $\in$ \textbf{Studenten}\\
                \vspace*{0.5cm}
                $\mathbf{a}$ ist in der Menge $\mathbf{\{u, v, w\}}$ nicht enthalten\\
                $\mathbf{a} \notin \mathbf{\{u, v, w\}}$
            \end{exampleblock}
        }
        \uncover<5->{
            Was ist eine \alert<5,6>{Teilmenge}?
        }
        \item<6>
            \vspace*{0.5cm}
            Eine Teilmenge ist eine \alert{spezielle Auswahl} von Elementen einer Menge.\\
            
            \metroset{block=fill}
            
            \begin{exampleblock}{Beispiel}
                $\{1, 2, 3\}$ ist eine Teilmenge der Natürlichen Zahlen\\
                $\{1,2,3\} \subseteq \mathbb{N}$\\
                \vspace*{0.5cm}
                \{\textbf{Janette}\} ist eine Teilmenge der \textbf{Studenten}\\
                \{\textbf{Janette}\} $\subseteq$ \textbf{Studenten}
            \end{exampleblock}
            
    \end{itemize}
\end{frame}

\begin{frame}{Mengen - Mal anders}
    \begin{alertblock}{Ein paar Definitionen}
    Eine nichtleere Menge einstelliger Symbole nennen wir \alert{Alphabet}.
    Es wird oft dargestellt durch den Bezeichner $\Sigma$.\\
    \end{alertblock}
    \metroset{block=fill}
    \begin{exampleblock}{Beispiele}
    \begin{itemize}
        \item $\Sigma = \{a,b\}$
        \item $\Sigma = \{0,1\}$
        \item $\Sigma = \{\text{Rechts, Links, Vorwärts, Rückwärts, Start, Stopp, Pause}\}$
    \end{itemize}
    \end{exampleblock}
\end{frame}

\begin{frame}[fragile]{Mengen - Mal anders}
    \begin{alertblock}{Ein paar Definitionen}
    Auf einem Alphabet können wir die Operation $\cdot$\;, genannt \alert{Konkatenation}, ausüben.\\
    $\rightarrow$ zum Beispiel ist dann $a \cdot b = ab$\\
    Eine beliebig lange Kette an Symbolen aus dem Alphabet nennen wir ein \alert{Wort}.
    \end{alertblock}
    \metroset{block=fill}
    \begin{exampleblock}{Beispiele}
    \begin{itemize}
        \item $abba$ ist ein \emph{Wort} über dem \emph{Alphabet} $\Sigma = \{a,b\}$
        \item $10011101$ ist ein \emph{Wort} über dem \emph{Alphabet} $\Sigma = \{0,1\}$
        \item StartVorwärtsRechtsVorwärtsStopp ist ein Wort über $\Sigma = \{\text{Rechts, Links, Vorwärts, Rückwärts, Start, Stopp, Pause}\}$
    \end{itemize}
    
    \end{exampleblock}
\end{frame}

\begin{frame}{Wortlängen}
    \begin{alertblock}{Wortlänge und das leere Wort}
        Eine endlich lange Kette an Symbolen aus dem Alphabet nennen wir ein Wort.
    \end{alertblock}
    \begin{itemize}
        \item Wort der Länge 3: z.B. $aaa, abc, \dots$
        \item Wort der Länge 2: z.B. $aa, ab, \dots$
        \item Wort der Länge 1: z.B. $a, b, c, \dots$
        \item Wort der Länge \alert<2>{0}: \alert<2>{$\emptyWord$}
    \end{itemize}
    \only<1>{Wir schreiben \alert<1>{$\absval{w}$} um \alert<1>{Länge des Wortes $w$} abzukürzen.\\}
    \only<2>{$\emptyWord$ \emph{(\glqq Epsilon\grqq)} nennen wir das \glqq leere Wort\grqq.
    \begin{itemize}
        \item \alert{Vergleich:} Es ist vergleichbar mit einem leerem String,\\ \qquad\qquad \,\, also: $\dq\dq=\emptyWord$
        \item \alert{Achtung:} Das leere Wort kann kein Teil eines Alphabets sein,\\\qquad\qquad da es nicht einstellig ist.
    \end{itemize}}
\end{frame}

\begin{frame}{neutrales Element der Konkatenation}
    \begin{alertblock}{Achtung}
        Wir können bei der Konkatenation auch das leere Wort anhängen. Es verhält sich hierbei als das neutrale Element.\\
        d.h. für ein beliebiges Wort $a$, ist $a\cdot\emptyWord = \emptyWord\cdot a=a$
        \begin{exampleblock}{Vergleich}
        \begin{itemize}
            \item Bei der Addition von Zahlen ist die 0 das neutrale Element\\
            $a+0=0+a=a$
            \item Bei der Multiplikation von Zahlen ist die 1 das neutrale Element\\
            $a*1=1*a=a$
        \end{itemize}
        
        \end{exampleblock}
    \end{alertblock}
\end{frame}


% Copyright 2018, 2019, 2020, 2021 FIUS
%
% This file is part of theo-vorkurs-folien.
%
% theo-vorkurs-folien is free software: you can redistribute it and/or modify
% it under the terms of the GNU General Public License as published by
% the Free Software Foundation, either version 3 of the License, or
% (at your option) any later version.
%
% theo-vorkurs-folien is distributed in the hope that it will be useful,
% but WITHOUT ANY WARRANTY; without even the implied warranty of
% MERCHANTABILITY or FITNESS FOR A PARTICULAR PURPOSE.  See the
% GNU General Public License for more details.
%
% You should have received a copy of the GNU General Public License
% along with theo-vorkurs-folien.  If not, see <https://www.gnu.org/licenses/>.

\begin{frame}[fragile]{$\Sigma^\ast$}
%\alert{Für \Sigma = \{a,b\}}:
\begin{figure}
    \centering
    \includegraphics[width=0.7\textwidth]{../figures/SigmaSternEpsilon.png}
    \caption{Menge von allen Kombinationen der Elemente von $\Sigma$ heißt $\Sigma^\ast$}
\end{figure}
\end{frame}

\begin{frame}[fragile]{$\Sigma^\ast$}
    %\metroset{block=fill}
    \begin{exampleblock}{Das heißt\dots}
    Sei $\Sigma = \{a,b\}$ unser \alert{Alphabet}.\\
    Wir beschreiben die Menge, die alle Möglichkeiten enthält Elemente aus $\Sigma$ zu \emph{konkatenieren} mit $\Sigma^\ast=\{\emptyWord,a,b,aa,ab,ba,bb,aba,\dots\}$.
    \end{exampleblock}\pause
    
    \begin{alertblock}{Achtung}
    $M^\ast$ über einer beliebigen Menge $M$ enthält immer das leere Wort $\emptyWord$!\\
    Sogar wenn $M = \{\} = \emptyset$.\\
    
    \end{alertblock}
\end{frame}

{\setbeamercolor{palette primary}{bg=ExColor}
\begin{frame}[fragile]{Denkpause}
    \begin{alertblock}{Aufgaben}
    Nenne jeweils 5 der kürzesten Elemente aus $\alert{\Sigma^{*}}$ für die folgenden Alphabete $\alert\Sigma$:
    \end{alertblock}
    
    \metroset{block=fill}
    \begin{block}{Normal}
        \begin{itemize}
            \item $\Sigma = \{a\}$
            \item $\Sigma = \{0, 1, 2, 3, 4, 5, 6, 7, 8, 9\}$
        \end{itemize}
    \end{block}
    \metroset{block=fill}
    \begin{block}{Etwas schwerer}
        \begin{itemize}
            \item $\Sigma$ = \{0, x, Biber\}
            \item $\Sigma$ = \{\Smiley, \Frowny\}
        \end{itemize}
    \end{block}
\end{frame}
}

{\setbeamercolor{palette primary}{bg=ExColor}
\begin{frame}<handout:0>{Lösungen}
    Mögliche Lösungen sind \dots
  \begin{itemize}[<+- | alert@+>]
        \item $\emptyWord, a, aa, aaa, aaaa \in \{a\}^{*}$
        \item $\emptyWord, 0, 1, 2, 3\in \{0,1,2,3,4,5,6,7,8,9\}^{*}$
        \item $\emptyWord, 0, x, Biber, xBiber \in \{0, x, Biber\}^{*}$
        \item $\emptyWord$, \Smiley, \Frowny, \Smiley\Smiley, \Smiley\Frowny$\; \in \{$\Smiley, \Frowny$\}^{*}$
    \end{itemize}
\end{frame}
}

\begin{frame}[fragile]{Alphabete und $\Sigma^{*}$}
\begin{alertblock}{Verständnisabfrage}
    Denke kurz über folgende Aufgabe nach...
    \end{alertblock}
    
    \metroset{block=fill}
    \begin{block}{Schwer}
        Welche Wörter sind in $M^{*}$ enthalten, wenn $M = \emptyset$ gilt?
    \end{block}
\end{frame}

{\setbeamercolor{palette primary}{bg=ExColor}
\begin{frame}<handout:0>{Lösungen}
  \begin{itemize}
        \item In $M^{*} = \{$ $ \}^{*}$ ist \textbf{nur} das leere Wort $\emptyWord$ enthalten.
    \end{itemize}
\end{frame}
}

\begin{frame}[fragile]{Sprachen}
\begin{figure}
    \centering
    \includegraphics[width=0.7\textwidth]{../figures/MysterySprache.png}
    \caption{Teilmengen unserer Obermenge (hier $\Sigma^*$) nennen wir Sprachen}
    \label{fig:my_label}
\end{figure}
\end{frame}

\begin{frame}[fragile]{Sprachen beschreiben}
\begin{figure}
    \centering
    \includegraphics[width=0.7\textwidth]{../figures/SprachReveal.png}
    \caption{Manche Sprachen können wir mit Regeln beschreiben}
    \label{fig:my_label}
\end{figure}
\end{frame}

\begin{frame}[fragile]{Sprachen beschreiben}
    \metroset{block=fill}
    \begin{exampleblock}{Beispiele für Sprachen in Mengenschreibweise}
    \begin{itemize}
        \item $L_1 = \{a^n\;|\;n\in\mathbb{N}\}$
        \item $L_2 = \{a^n\;|\;n \equiv 0 \pmod 2, n\in\mathbb{N}\}$
        \item $L_3 = \{uv\;|\;u\in\{a,b\}^\ast,\;v\in\{a\}\}$
        \item $L_4 = \{w\;|\;|w|_a = 3\}$
    \end{itemize}
    \end{exampleblock}
    Was soll das alles? Mehr dazu nach der Pause :)
\end{frame}


\begin{frame}[standout]
  Murmelpause
\end{frame}

\section{Mengenschreibweise} 

% Copyright 2018-2022 FIUS
%
% This file is part of theo-vorkurs-folien.
%
% theo-vorkurs-folien is free software: you can redistribute it and/or modify
% it under the terms of the GNU General Public License as published by
% the Free Software Foundation, either version 3 of the License, or
% (at your option) any later version.
%
% theo-vorkurs-folien is distributed in the hope that it will be useful,
% but WITHOUT ANY WARRANTY; without even the implied warranty of
% MERCHANTABILITY or FITNESS FOR A PARTICULAR PURPOSE.  See the
% GNU General Public License for more details.
%
% You should have received a copy of the GNU General Public License
% along with theo-vorkurs-folien.  If not, see <https://www.gnu.org/licenses/>.

\begin{frame}[fragile]{Wie sprechen wir das?}
$M = \{2n \alert{\mid} n \in \mathbb{N}\}$ \\

\emph{Die Menge $M$ enthält alle Elemente $2n$, \alert{für die gilt}: n stammt aus der Menge der natürlichen Zahlen.}
\vspace{5pt}
$$
\{0, 2, 4, 6, 8, 10, 12, \dots \}
$$

\metroset{block=fill}
\begin{alertblock}{Achtung}
    In der theoretischen Informatik enthält $\mathbb{N}$ ($\mathbb{N}$ ist die Menge der natürlichen Zahlen) die Zahl 0.
\end{alertblock}

\end{frame}


\begin{frame}[fragile]{Wie schreiben wir das?}
    \begin{itemize}[<+- | alert@+>]
        \item Mengenbeschreibungen können auch Sätze enthalten:
        $M_1 = \{0,2,4,6,8,\dots\} = \{x \mid \text{x ist gerade}\}$\\
        \hspace{4.5mm}$= \{x \mid$ Es gibt eine Zahl $k \in \mathbb{N} : 2k = x\}$\\
        
        \item Mengenbeschreibungen können Formeln sein:
        $M_2 = \{n^2+3 \mid n \in \mathbb{N}\} = \{3, 5, 7, 11, \dots\}$
        
        \item Diese formeln können im Beschreibungsteil sein:
        $M_3 = \{x \mid x = n^2 \text{ und } x\in \mathbb{N}\} = \{2, 4, 8, 16, \dots\}$
        
        \item Es können mehrere Beschreibungen gleichzeitig verwendet werden:
        $M_4 = \{n \mid n \in \mathbb{N}, n > 15, n < 20\} = \{16, 17, 18, 19\}$\\
    
    \end{itemize}
\end{frame}

{\setbeamercolor{palette primary}{bg=ExColor}
\begin{frame}[fragile]{Denkpause}
    \footnotesize
        \begin{alertblock}{Aufgaben}
            Findet Elemente aus den folgenden Mengen
        \end{alertblock}
        \metroset{block=fill}
        \begin{block}{Normal}
            \begin{itemize}
                \item $L_1 = \{a\}$
                \item $L_2 = \{u+v \mid u\in\{1,2\}^\ast,\;v\in\{7,9\}\}$
                \item $L_3 = \{w \mid w \in\{1,2,3\} |w| = 3\}$
            \end{itemize}
        \end{block}
        \begin{block}{Etwas Schwerer}
            \begin{itemize}
                \item $L_4 = \{a^n \mid n \equiv 1 \pmod 3, n\in\mathbb{N}\}$
                \item $L_5 = \{x \mid |x^2| = x^2\}$
                \item $L_7 = \{w \mid w \text{ ist prim}, w > 50\}$
            \end{itemize}
        \end{block}
        \emph{Anmerkung:} $x \equiv y \pmod n \iff n$ teilt $(x-y)$ ohne Rest $\iff x = m \cdot n + y$ mit $x,y,n,m \in \mathbb{Z}$
\end{frame}
}

{\setbeamercolor{palette primary}{bg=ExColor}
\begin{frame}<handout:0>{Lösungen}
  \begin{itemize}[<+- | alert@+>]
        \item 
            $L_1$: Enthält \textbf{nur} das einzelne Wort $a$!
        \item
            $L_2$: z.B. $7$, $9$, $17$, $27$, $1117$, $12121219$, \dots\\
            Wort besteht aus zwei Teilen: u z.B. $\emptyWord$, $1$, $2$, $12121$, ...\\ v ist entweder $7$ oder $9$!
        \item
            $L_3$: enthält alle Wörter der Länge 3, deren Zahlen nur $1$, $2$ oder $3$ sind.\\
            $\rightarrow$ $111$, $112$, $121$, $122$, $123$, $132$, $133$, $211$, $212$, $322$, ...
        \item
            $L_4 = \{a, aaaa, aaaaaaa, \dots\}$\\
            Wörter deren Länge durch 3 geteilt den Rest 1 ergeben.
        \item
            $L_5 = \{caaba, cccbaaa, abaca, aaab, \dots\}$\\
            genau 3 $a$'s, genau 1 $b$, beliebig viele $c$'s, keine Sortierung
        \item
            $L_6$ = \{\Stopsign, \Rewind \Stopsign, \MoveUp \Stopsign,\dots\;, \MoveDown \Rewind \MoveDown \Stopsign,\dots\}
        \item $L_7 = \emptyset$\\
            $L_7$ enthält \textbf{gar kein} Wort!
    \end{itemize}
\end{frame}
}


\section{Mengenoperationen}

\begin{frame}{Mengenoperationen - Schnitt}
\begin{columns}
\column{0.5\textwidth}
    \begin{alertblock}{Schnitt - $A\cap B$}
    Gegeben zwei Mengen A und B.\\
    In der Schnittmenge liegt alles, das in Menge A \textbf{und} in Menge B ist.
    \end{alertblock}
\column{0.5\textwidth}
\begin{figure}
    \centering
    \includegraphics[width=0.7\textwidth]{../figures/AundB.png}
    \caption{Veranschaulichung der Schnittmenge}
    \label{fig:my_label}
\end{figure}
\end{columns}
\end{frame}

\begin{frame}{Mengenoperationen - Vereinigung}
\begin{columns}
\column{0.5\textwidth}
    \begin{alertblock}{Vereinigung - $A\cup B$}
    Gegeben zwei Mengen A und B.\\
    In der Vereinigung liegt alles, das nur in A, nur in B \textbf{oder} in beiden Mengen liegt.
    \end{alertblock}
\column{0.5\textwidth}
\begin{figure}
    \centering
    \includegraphics[width=0.7\textwidth]{../figures/AoderB.png}
    \caption{Veranschaulichung der Vereinigung}
    \label{fig:my_label}
\end{figure}
\end{columns}
\end{frame}

\begin{frame}{Mengenoperationen - Komplement}
    \begin{columns}
\column{0.5\textwidth}
    \begin{alertblock}{Komplement - $\Bar{A}$}
    Gegeben sei eine Menge A.\\
    Im Komplement der Menge A liegen alle Elemente, die in $\Sigma^{*}$, aber nicht in der Menge A selbst liegen.
    \end{alertblock}
\column{0.5\textwidth}
\begin{figure}
    \centering
    \includegraphics[width=0.7\textwidth]{../figures/Akomp.png}
    \caption{Veranschaulichung des Komplements}
    \label{fig:my_label}
\end{figure}
\end{columns}
\onslide<2>{\alert{\emph{Anmerkung:}} Kann auch geschrieben werden als $\Sigma^{*}\setminus A$. \\
\hspace{2cm}(gesprochen $\Sigma^{*}$ \emph{\glqq ohne\grqq} $A$)}
\end{frame}

\begin{frame}{Mengenoperationen}
    Berechne folgende Mengen
    \metroset{block=fill}
    \begin{alertblock}{Normal}
        \begin{itemize}
            \item $M_1$: $\{a\}\cup \{b\}$
            \item $M_2$: $\{\} \cap \{u, v, w\}$
            \item $M_3$: $\mathbb{N} \cup \mathbb{Z}$
            \item $M_4$: $\overline{\{a^{n} | n \ \text{ist gerade}\} }$ , über dem Alphabet $\{a\}$
        \end{itemize}
    \end{alertblock}
        \begin{alertblock}{Schwer bis sehr schwer}
        \begin{itemize}
            \item $M_5$: $\{a, b, c\} \cap  \{a, \{b, c\}\}$
            \item $M_6$: $\{u \mid |u| \equiv 0 \mod 2, u \in \{a, b\}^{*}\}$\\\hspace{0.65cm}$\cup$ $\{v \mid |v| \equiv 0 \mod 4, v \in \{a, b\}^{*}\}$
            \item $M_7$: $\overline{\{a^{n} | n \ \text{ist gerade}\} }$ , über dem Alphabet $\{a,b\}$
        \end{itemize}
    \end{alertblock}
\end{frame}

{\setbeamercolor{palette primary}{bg=ExColor}
\begin{frame}{Lösungen}
  \begin{itemize}[<+- | alert@+>]
        \item 
            $M_1 = \{a, b\}$
        \item
            $M_2 = \emptyset$
        \item
            $M_3 = \mathbb{Z}$
        \item
            $M_4 = \{a^{n} \mid$ n ist ungerade$\}$, 
        \item
            $M_5 = \{a\}$
        \item
            $M_6 = \{u \mid |u| \equiv$ 0 mod 2$, u \in \{a, b\}^{*}\}$
        \item
            $M_7$: $\{a^{n} | n \ \text{ist ungerade}\} \cup\{w | w\in\{a,b\}^{*}, |w|_b \geq 1\}$
    \end{itemize}
\end{frame}
}


\section{Wiederholung} 
\begin{frame}[fragile]{Das können wir jetzt beantworten}
    \begin{alertblock}{Einführung}
    \begin{itemize}
        \item Theoretische Informatik ist ganz schön wichtig...
        \item ...für mein Studium.
    \end{itemize}
    \end{alertblock}
\end{frame}

\begin{frame}[fragile]{Das können wir jetzt beantworten}
    \begin{alertblock}{Mengen, Sprachen, Elemente}
    \begin{itemize}
        \item Was ist eine Menge?
        \item Was ist eine Sprache?
        \item Was sind Elemente einer Sprache/Menge?
    \end{itemize}
    \end{alertblock}
\end{frame}

\begin{frame}[fragile]{Das können wir jetzt beantworten}
    \begin{alertblock}{Alphabete, $\SigmaStern$}
    \begin{itemize}
        \item Was ist ein Alphabet?, Was ist ein Wort?
        \item Wie funktioniert die Konkatenation?
        \item Was ist der Unterschied zwischen $\Sigma$ und $\SigmaStern$?
        \item Das leere Wort: Welches ist das \textit{kleinste} Alphabet, mit $\emptyWord \in \SigmaStern$?
        \item Bilden von $\SigmaStern$ für gegebenes Alphabet $\Sigma$
    \end{itemize}
    \end{alertblock}
\end{frame}

\begin{frame}[fragile]{Das können wir jetzt beantworten}
    \begin{alertblock}{Operationen auf Mengen}
    \begin{itemize}
        \item Wie funktionieren Vereinigung, Schnitt und Komplement?
        \item Wie bilde ich Vereinigungen oder Schnittmengen zweier Mengen?
        \item Wie bilde ich das Komplement einer Menge?
        \item Wie kann ich Sprachen formal beschreiben?
        \item Hantieren mit verschiedenen seltsamen Mengen und den Verknüpfungen
    \end{itemize}
    \end{alertblock}
\end{frame}

\begin{frame}[standout]
  Noch Fragen?
\end{frame}

\begin{frame}[fragile]{Glossar}
    \small
    \begin{tabular}{p{0.05\textwidth} p{0.25\textwidth} p{0.5\textwidth}}
    \toprule
    Abk.&Bedeutung&Was?!\\
    \midrule
        $\mathbb{N}$&natürliche Zahlen (mit 0)&In der theoretischen Informatik enthält $\mathbb{N}$ die 0: $\mathbb{N}=\{0,1,2,3,\dots\}$\\
        $\mathbb{Z}$&ganze Zahlen&\\
        $\mathbb{Q}$&rationale Zahlen&können als Bruch dargestellt werden\\
        $\Sigma$ & Sigma& mit diesem Zeichen wird oft das Alphabet (die Menge an verwendbaren Symbolen) repräsentiert\\
        $\Sigma^\ast$&Sigma Stern&Menge aller Möglichkeiten Elemente aus $\Sigma$ hintereinander zu schreiben\\
        $\emptyset$&\{\}&leere Menge\\
        :&sodass&z.B. $\forall a,b\in\mathbb{Z}:$a teilt b\\
    \bottomrule
    \end{tabular}
\end{frame}

\end{document}