\subsection{Organisatorisches}
\begin{frame}[fragile]{Wer sind wir?}
    \begin{itemize}
        \item
            Fachgruppe Informatik
            \begin{itemize}
                \item Unser Ziel: \\
                Das Leben von uns Studis während des Studiums angenehmer zu gestalten
                \item organisieren Veranstaltungen (Grillen, Spieleabende, Vorkurse, ...)
                \item verleihen Prüfungen aus den früheren Semestern
                \item vertreten die studentische Sicht in offiziellen Gremien
                \item ...und vieles mehr (es gibt z.B. einen 3D-Drucker)
            \end{itemize}
        \item Arbeitskreis Theoretische Informatik
        \begin{itemize}
            \item Teilmenge der Fachgruppe Informatik
            \item haben diesen Vorkurs organisiert
        \end{itemize}
    \end{itemize}
\end{frame}

\subsection{Tipps zum Studium}
\begin{frame}[fragile]{Tipps zum Studium}
    \begin{itemize}
        \item Nützliche Links:\\
            \begin{itemize}
                \item Fachgruppe Informatik:\\
                \url{https://fius.informatik.uni-stuttgart.de/}
                \item Foliensätze:\\ \url{https://fius.informatik.uni-stuttgart.de/index.php/studien-interessierte/theo/}
                \item Ergänzung Theoretische Informatik 1 (Wintersemester 19/20): \\
                \url{https://fmi.uni-stuttgart.de/ti/teaching/w19/eti1/}
                \item Ersti Telegram-Gruppe:\\
                \qrcode[hyperlink]{https://t.me/joinchat/A6dEy07zU3wL55Bf-7h0zg}
                \url{https://t.me/joinchat/A6dEy07zU3wL55Bf-7h0zg}
        	\end{itemize}
        \item E-Mail der Fachgruppe: fius@informatik.uni-stuttgart.de

    \end{itemize}
\end{frame}

\begin{frame}[fragile]{Hygieneregeln}
	\begin{alertblock}{Die wichtigsten Regeln...}
		\begin{itemize}
			\item Überall im Gebäude, einschließlich auf dem Weg zum Sitzplatz oder zur Toilette muss Mund-Nasen-Schutz (MNS) getragen werden. Sobald man sich an seinem Sitzplatz befindet, darf der MNS abgenommen werden.
			\item Sollte der Mindestabstand (z.B. zwischen Tutor*in und Student*in) nicht eingehalten werden können, ist der Mund-Nasen-Schutz zu tragen.
			\item Die auf den Fluren eingezeichneten und mit Pfeilen markierten Laufwege sind einzuhalten. Es gilt das Rechtslaufgebot!
		\end{itemize}
	\end{alertblock}
\end{frame}

\begin{frame}[fragile]{Hygieneregeln}
	\begin{alertblock}{Warum das alles?}
		\begin{itemize}
			\item Stellt euch vor, jede*r hier - bis auf eine*r - hätte Corona. Wir wollen uns so verhalten, dass selbst in dem Fall diese eine Person gesund nach Hause gehen kann.
			\item Ihr könnt euch trotzdem untereinander kennen lernen und Aufgaben miteinander bearbeiten. Lasst euch von den Maßnamen nicht abschrecken - wir können sie einhalten und den Vorkurs damit gut meistern.
		\end{itemize}
	\alert{Inzwischen kennen wir es doch alle: Abstand, Hygiene, Maske auf.\\
	Auch hier im Raum: Sobald der Abstand unterschritten wird: Maske auf.}
	\end{alertblock}
\end{frame}

\begin{frame}[fragile]{Infos zum Online-Ablauf}
	\begin{alertblock}{Ablauf und Notfallplan}
		\begin{itemize}
			\item Es wird zwischen Vorlesungs- und Aufgabephasen abgewechselt.
			\item Wir benutzen BigBlueButton - wenn ihr hier seid, wisst ihr das schon. Sollte es zu technischen Problemen kommen, die sich nicht innerhalb von 10 Minuten beheben lassen, wechseln wir auf einen Discord-Server. Den Joinlink dazu findet ihr dann auf der Fachgruppenseite unter \hyperlink{https://fius.informatik.uni-stuttgart.de/index.php/mitteilungen/}{Mitteilungen}.
		\end{itemize}
		\alert{Traut euch, Fragen zu stellen und mitzumachen.}
	\end{alertblock}
\end{frame}
