\begin{frame}{Automaten: Formal}
  Ein \textbf{DEA M} lässt sich beschreiben durch ein geordnetes 5-Tupel\\
  \alert{$M=(Z, \Sigma, \delta, z_0, E)$} mit:
  \begin{itemize}
    \item $Z$: Die Menge der Zustände
    \item $\Sigma$: Das Alphabet
    \item $\delta$: Die Überführungsfunktion
    \item $z_0$: Der Startzustand
    \item $E$: Die Menge der Endzustände
  \end{itemize}
\end{frame}


\begin{frame}{Automaten: Formal - Beispiel}
  \begin{columns}
    \column{0.5\textwidth}
    \alert{$L_1=\{a^{n}b^{m} \mid n,m \in \mathbb{N}\}$}\\
    \vspace{0.6cm}
    \begin{tikzpicture}[->,>=stealth',shorten >=1pt,auto,node distance=1.5cm,semithick]
      \node [initial,state,accepting]   (0)              {$q_0$};
      \node [state,accepting]           (1) [right of=0] {$q_1$};
      \node [state]                     (2) [below of=1] {$q_f$};

      \path   (0) edge                    node {b}    (1)
      edge [loop above]       node {a}    (0)
      (1) edge                    node {a}    (2)
      edge [loop above]       node {b}    (1)
      (2) edge [loop right]       node {a,b}  (2);
    \end{tikzpicture}

    \column{0.5\textwidth}
    \alert{$M=(Z, \Sigma, \delta, q_0, E)$} mit:
    \begin{itemize}
      \item<2-|handout:1> \alert<2|handout:0>{$Z=\{q_0, q_1, q_f\}$}
      \item<3-|handout:1> \alert<3|handout:0>{$\Sigma=\{a, b\}$}
      \item<4-|handout:1> \alert<4|handout:0>{$\delta$:
              \begin{itemize}
                \item $\delta(q_0, a)=q_0$
                \item $\delta(q_0, b)=q_1$
                \item $\delta(q_1, a)=q_f$
                \item $\delta(q_1, b)=q_1$
                \item $\delta(q_f, a)=q_f$
                \item $\delta(q_f, b)=q_f$
              \end{itemize}}
      \item<5-|handout:1> \alert<5|handout:0>{$E=\{q_0, q_1\}$}
    \end{itemize}
  \end{columns}
\end{frame}

{\setbeamercolor{palette primary}{bg=ExColor}
\begin{frame}{Automaten: Formal - Aufgabe}
  \footnotesize
  \begin{columns}
    \column{0.5\textwidth}
    \begin{alertblock}{Aufgabe}
    \end{alertblock}
    \metroset{block=fill}
    \begin{block}{Normal}
      \alert{$L_2=\{w \in \{a, b\}^* \mid |w|=3\}$}
      %TÖDÖ Töröööööt Benjamin der kleiner Elefant fragt Janette: Ey machst du automat amk!

      \vspace{0.1cm}
      \begin{tikzpicture}[->,>=stealth',shorten >=1pt,auto,node distance=1.5cm,semithick]
        \node [initial,state]   (0)              {$q_0$};
        \node [state]           (1) [below of=0] {$q_1$};
        \node [state]           (2) [below of=1] {$q_2$};
        \node [state,accepting] (3) [below of=2] {$q_E$};
        \node [state]           (f) [right of=3] {$q_f$};

        \path   (0) edge                node {a,b}  (1)
        (1) edge                node {a,b}  (2)
        (2) edge                node {a,b}  (3)
        (3) edge                node {a,b}  (f)
        (f) edge [loop above]   node {a,b} (f);
      \end{tikzpicture}
      \vspace{0.1cm}
    \end{block}

    \column{0.5\textwidth}
    \only<2-|handout:0>{
      \begin{alertblock}{Lösung}
      \end{alertblock}
      \metroset{block=fill}
      \alert{$M=(Z, \Sigma, \delta, q_0, E)$} mit:
      \begin{itemize}
        \item<3-> \alert<2>{$Z=\{q_0, q_1, q_2, q_E, q_f\}$}
        \item<4-> \alert<3>{$\Sigma=\{a, b\}$}
        \item<5-> \alert<4>{$\delta$:
                \vspace{-0.2cm}
                \begin{itemize}
                  \item $\delta(q_0, a)=q_1$
                  \item $\delta(q_0, b)=q_1$
                  \item $\delta(q_1, a)=q_2$
                  \item $\delta(q_1, b)=q_2$
                  \item $\delta(q_2, a)=q_3$
                  \item $\delta(q_2, b)=q_3$
                  \item $\delta(q_E, a)=q_f$
                  \item $\delta(q_E, b)=q_f$
                  \item $\delta(q_F, a)=q_f$
                  \item $\delta(q_F, b)=q_f$
                \end{itemize}}
        \item<6-> \alert<5>{$E=\{q_E\}$}
      \end{itemize}
    }
  \end{columns}
\end{frame}}

\begin{frame}{Automaten und Grammatiken: Parallelen}
  Wir stellen einige Parallelen fest:
  % note: \alert or \usebeamercolor[fg]{alerted text} inside p-columns breaks layout
  % default alert color of metropolis is mLightBrown
  \begin{center}\begin{tabular}{p{.46\textwidth}|p{.46\textwidth}}
      \textbf{DEA} & \textbf{Reguläre Grammatiken}
      \onslide<2-|handout:1>                                                                                                                                            \\\hline
      Wörter werden \textcolor<2|handout:0>{mLightBrown}{von links nach rechts} gelesen
                   & Wörter werden \textcolor<2|handout:0>{mLightBrown}{von links nach rechts} erzeugt
      \onslide<3-|handout:1>                                                                                                                                            \\\hline
      \textcolor<3|handout:0>{mLightBrown}{Pro Schritt} wird \textcolor<3|handout:0>{mLightBrown}{ein Buchstabe} gelesen
                   & \textcolor<3|handout:0>{mLightBrown}{Pro Schritt} wird \textcolor<3|handout:0>{mLightBrown}{ein Buchstabe} erzeugt
      \onslide<4-|handout:1>                                                                                                                                            \\\hline
      \textcolor<4|handout:0>{mLightBrown}{Ein Startzustand}
                   & \textcolor<4|handout:0>{mLightBrown}{Ein Startsymbol}
      \onslide<5-|handout:1>                                                                                                                                            \\\hline
      Beim Lesen des \textcolor<5|handout:0>{mLightBrown}{letzten Buchstabens} wird in einen \textcolor<5|handout:0>{mLightBrown}{Endzustand} übergegangen
                   & Beim Erzeugen des \textcolor<5|handout:0>{mLightBrown}{letzten Buchstabens} wird \textcolor<5|handout:0>{mLightBrown}{keine neue Variable} erzeugt
      \onslide<6-|handout:1>                                                                                                                                            \\\hline
      In jedem Schritt wird \textcolor<6|handout:0>{mLightBrown}{aus einem Zustand in genau einen Zustand} übergegangen
                   & In jedem Schritt wird \textcolor<6|handout:0>{mLightBrown}{aus einer Variable genau eine Variable} erzeugt
    \end{tabular}\end{center}
\end{frame}

\begin{frame}{Automaten und Grammatiken}
  \metroset{block=fill}
  \begin{exampleblock}{Satz}
    Jede durch deterministische endliche Automaten erkennbare Sprache ist auch regulär (also Typ 3).
  \end{exampleblock}
  Der Beweis dieses Satzes findet sich im Anhang.\\
  Ihr werdet ihn auch in der Vorlesung zeigen.
\end{frame}
