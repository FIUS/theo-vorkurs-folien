{\setbeamercolor{palette primary}{bg=ExColor}
\begin{frame}[fragile]{Beweise}
	\begin{alertblock}{Aufgaben}
		Versuche dich an folgenden Beweisen:
	\end{alertblock}
	\metroset{block=fill}
	\begin{block}{Normal}
		Sei $L = \{a^nb^n \mid n \in \mathbb{N}\}$, dann gilt:
		\begin{align*}
			\forall w \in \{a, b\}^*: \left(w \in L \implies |w| \text{ ist gerade}\right)
		\end{align*}
	\end{block}
	\metroset{block=fill}
	\begin{block}{Wer noch nicht genug hat...}
		\begin{align*}
			\forall x \in \mathbb{Z}: x \equiv 0 \pmod{2} \vee x^2 \equiv 1 \pmod{4}
		\end{align*}
	\end{block}
\end{frame}
}

{\setbeamercolor{palette primary}{bg=ExColor}
\begin{frame}<handout:0>[fragile]{Lösungen}
	\begin{alertblock}{Aufgabe}
		z.Z.: Sei $L = \{a^nb^n \mid n \in \mathbb{N}\}$, dann gilt:\\
		$\forall w \in \{a, b\}^*: \left(w \in L \implies |w| \text{ ist gerade}\right)$
	\end{alertblock}
	\metroset{block=fill}
	\begin{alertblock}{Beweis}
		\begin{enumerate}
			\item<1-> Sei $w \in \{a,b\}^*$ beliebig.\\
			\item<2-> Angenommen,  $w \in L$.\\
			\item<3-> Dann existiert ein $n \in \mathbb{N}$ so, dass $w = a^nb^n$.\\
			\item<4-> Insbesondere gilt:
			      \begin{align*}
				      |w| = |a^nb^n| = |a^n| + |b^n| = n+n = 2n
			      \end{align*}
			\item<5-> Folglich ist $|w|$ gerade. \qed
		\end{enumerate}
	\end{alertblock}
\end{frame}
}

{\setbeamercolor{palette primary}{bg=ExColor}
\begin{frame}<handout:0>[fragile]{Lösungen}
	\begin{alertblock}{Aufgabe}
		z.Z.: $\forall x \in \mathbb{Z}: x \equiv 0 \pmod{2} \vee x^2 \equiv 1 \pmod{4}$
	\end{alertblock}
	\metroset{block=fill}
	\begin{alertblock}{Beweis}
		\begin{enumerate}
			\item<1-> Sei $x \in \mathbb{Z}$ beliebig.
			\item<2->
			      \begin{enumerate}
				      \item[i)]<2-> Sei $x \equiv 0 \pmod{2}$: Dann sind wir schon fertig.
				      \item[ii)]<3-> Sei $x \equiv 1 \pmod{2}$: Dann können wir $x$ schreiben als $x = 2k + 1$ für ein $k \in \mathbb{Z}$.
				            Es gilt:
				            \begin{align*}
					            x^2 = (2k+1)^2 = 4k^2 + 4k + 1 = 4(k^2 + k) + 1
				            \end{align*}
				            Also folgt unmittelbar laut Definition $x^2 \equiv 4(k^2 + k) + 1 \equiv 1 \pmod{4}$.
			      \end{enumerate}
			\item<4-> Somit ist die Behauptung bewiesen. \qed
		\end{enumerate}
	\end{alertblock}
\end{frame}
}