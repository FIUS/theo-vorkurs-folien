% Aufgaben zu grundlegenden Definitionen

{\setbeamercolor{palette primary}{bg=ExColor}
\begin{frame}[fragile]{Wiederholung Grundlegender Definitionen}
    \begin{alertblock}{Aufgaben}
        Beantworte die folgenden Fragen
    \end{alertblock}
    \metroset{block=fill}
    \begin{block}{Normal}
        \begin{itemize}
            \item Was ist der Unterschied zwischen $\Sigma$ und $\Sigma^*$?
            \item Welche besondere Eigenschaft besitzt das leere Wort $\emptyWord$? Insbesondere: Welchen Wert besitzt $|\emptyWord|$?
            \item Welchen Wert besitzt $|\emptyWord \cdot a^2 \cdot \emptyWord \cdot bab|_a$?
        \end{itemize}
    \end{block}
%    \begin{block}{Etwas Schwerer}
%        \begin{itemize}
%            \item Für welche $n \in \mathbb{N}$ gilt $a^n = a^3 \cdot \emptyWord \cdot a^{n-3}$?
%        \end{itemize}
%    \end{block}
\end{frame}
}

{\setbeamercolor{palette primary}{bg=ExColor}
\begin{frame}[fragile]{Lösungen}
    \only<1>{
        \metroset{block=fill}
        \begin{block}{Normal}
            Was ist der Unterschied zwischen $\Sigma$ und $\Sigma^*$?
        \end{block}
        \begin{enumerate}
            \item[i)] Das \alert{Alphabet} $\Sigma$ ist eine nichtleere Menge einstelliger Symbole.
            \item[ii)] $\Sigma^*$ ist die Menge aller möglichen Kombinationen (bzgl. der Konkatenation) der Elemente aus $\Sigma$.\\
            Insbesondere gilt hier $\emptyWord \in \Sigma^*$.
        \end{enumerate}
    }

    \only<2>{
        \metroset{block=fill}
        \begin{block}{Normal}
            Welche besondere Eigenschaft besitzt das leere Wort $\emptyWord$? Insbesondere: Welchen Wert besitzt $|\emptyWord|$?
        \end{block}
        $\emptyWord$ ist das neutrale Element bzgl. der Konkatenation. Das heißt:
        \begin{align*}
            \forall w \in \Sigma^*: w = w \cdot \emptyWord = \emptyWord \cdot w
        \end{align*}
        Insbesondere gilt also:
        \begin{align*}
            \forall w \in \Sigma^*: |w| = |w \cdot \emptyWord| = |w| + |\emptyWord|
        \end{align*}
        womit $|\emptyWord| = 0$ gelten muss.
    }

    \only<3>{
        \metroset{block=fill}
        \begin{block}{Normal}
            Welchen Wert besitzt $|\emptyWord \cdot a^2 \cdot \emptyWord \cdot bab|_a$?
        \end{block}
        Zunächst schreiben wir
        \begin{align*}
            \emptyWord \cdot a^2 \cdot \emptyWord \cdot bab = a^2bab = aabab
        \end{align*}
        und zählen anschließend die vorkommenden $a$'s. Also
        \begin{align*}
            |\emptyWord \cdot a^2 \cdot \emptyWord \cdot bab|_a = |aabab|_a = 3.
        \end{align*}
    }
\end{frame}
}

{\setbeamercolor{palette primary}{bg=ExColor}
\begin{frame}[fragile]{Aufgaben zu Mengen}
    \begin{alertblock}{Aufgaben}
        Welche der folgenden Aussagen ist korrekt?
    \end{alertblock}
    \metroset{block=fill}
    \begin{block}{Normal bis etwas Schwerer}
        \begin{minipage}[t]{0.45\textwidth}
            \begin{itemize}
                \item $a \in \{a, b, c\}$
                \item $a \subseteq \{a, b, c\}$
                \item $\{a,b\} \in \{a, b, \{a, b\}, c\}$
            \end{itemize}
        \end{minipage}
        \begin{minipage}[t]{0.45\textwidth}
            \begin{itemize}
                \item $\{a,b\} \subseteq \{a, b, \{a, b\}, c\}$
                \item $\{\{a,b\}\} \subseteq \{a, b, \{a, b\}, c\}$
            \end{itemize}
        \end{minipage}
    \end{block}
    Nenne jeweils 5 Wörter aus den folgenden Sprachen
    \metroset{block=fill}
    \begin{block}{Normal bis etwas Schwerer}
        \begin{itemize}
            \item $L_1 = \{a^nb^m \mid n, m \in \mathbb{N}\}$
            \item $L_2 = \{a^nb^n \mid n \in \mathbb{N}\}$
            \item $L_3 = L_2 \backslash L_1$
            \item $L_4 = \{w \in \{a,b\}^* \mid |w|_a \equiv 3 \pmod{4}\}$
        \end{itemize}
    \end{block}
\end{frame}
}

{\setbeamercolor{palette primary}{bg=ExColor}
\begin{frame}[fragile]{Lösungen}
    \only<1-5>{
        \begin{alertblock}{Aufgaben}
            Welche der folgenden Aussage ist korrekt?
        \end{alertblock}
        \metroset{block=fill}
        \begin{block}{Normal}
            \begin{align*}
                \only<1>{a \in \{a, b, c\}}
                \only<2>{a \subseteq \{a, b, c\}}
                \only<3>{\{a,b\} \in \{a, b, \{a, b\}, c\}}
                \only<4>{\{a,b\} \subseteq \{a, b, \{a, b\}, c\}}
                \only<5>{\{\{a,b\}\} \subseteq \{a, b, \{a, b\}, c\}}
            \end{align*}
        \end{block}
        \begin{center}
            \only<1, 3,4,5>{\alert{wahr}}
            \only<2>{\alert{falsch}}
        \end{center}
    }
    \only<6-9>{
        \begin{alertblock}{Aufgaben}
            Nenne jeweils 5 Wörter aus den folgenden Sprachen
        \end{alertblock}
        \metroset{block=fill}
        \begin{block}{Normal}
            \begin{align*}
                \only<6>{L_1 = \{a^nb^m \mid n, m \in \mathbb{N}\}}
                \only<7>{L_2 = \{a^nb^n \mid n \in \mathbb{N}\}}
                \only<8>{L_3 = L_1 \backslash L_2}
                \only<9>{L_4 = \{w \in \{a,b\}^* \mid |w|_a \equiv 3 \pmod{4}\}}
            \end{align*}
        \end{block}
        \begin{center}
            \only<6>{$\emptyWord, a, b, aa, ab, aaa, aab, aba, abb, ...$}
            \only<7>{$\emptyWord, ab, aabb, a^3b^3, a^4b^4, ...$}
            \only<8>{Keine Wörter; $\emptyset$}
            \only<9>{$aaa, baaa, abaa, aaba, aaab, a^7, ...$}
        \end{center}
    }
\end{frame}
}

% Aufgaben zu Aussagenlogik und Quantoren

{\setbeamercolor{palette primary}{bg=ExColor}
\begin{frame}[fragile]{Wiederholung Grundlegender Definitionen}
    \begin{alertblock}{Aufgaben}
        Bestimme den Wahrheitswert der folgenden Aussagen
    \end{alertblock}
    \metroset{block=fill}
    \begin{block}{Normal}
        \begin{itemize}
            \item $\neg (42 = 11) \wedge (|\emptyWord| > 0 \vee \emptyset^* = \{\emptyWord\})$
            \item $\forall w \in \{a, b\}^*: (|w|_a = 0 \implies w = \emptyWord)$
            \item $\forall w \in \{a, b\}^*: (|w|_a = 0 \implies ab = ba)$
            \item $(\forall w \in \{a, b\}^*: |w|_a = 0) \implies ab = ba$
        \end{itemize}
    \end{block}
    \metroset{block=fill}
    \begin{block}{Schwer}
        \begin{itemize}
            \item $\forall x \in \mathbb{Z}: x \equiv 0 \pmod{2} \vee x \equiv 1 \pmod{4}$
            \item $\forall w \in \{a, b\}^*: \left(\left(\exists u \in \{a,b\}^*: uw \neq u\right) \implies w \neq \emptyWord\right)$
        \end{itemize}
    \end{block}
\end{frame}
}

{\setbeamercolor{palette primary}{bg=ExColor}
\begin{frame}[fragile]{Lösungen}
    \only<1-4>{
        \metroset{block=fill}
        \begin{block}{Normal}
            \begin{align*}
                \only<1>{\neg (42 = 11) \wedge (|\emptyWord| > 0 \vee \emptyset^* = \{\emptyWord\})}
                \only<2>{\forall w \in \{a, b\}^*: (|w|_a = 0 \implies w = \emptyWord)}
                \only<3>{\forall w \in \{a, b\}^*: (|w|_a = 0 \implies ab = ba)}
                \only<4>{(\forall w \in \{a, b\}^*: |w|_a = 0) \implies ab = ba}
            \end{align*}
        \end{block}
        \begin{center}
            \only<1,4>{\alert{wahr}}
            \only<2,3>{\alert{falsch}}
        \end{center}
    }
    \only<5-6>{
        \metroset{block=fill}
        \begin{block}{Schwer}
            \begin{align*}
                \only<5>{\forall x \in \mathbb{Z}: x \equiv 0 \pmod{2} \vee x \equiv 1 \pmod{4}}
                \only<6>{\forall w \in \{a, b\}^*: \left(\left(\exists u \in \{a,b\}^*: uw \neq u\right) \implies w \neq \emptyWord\right)}
            \end{align*}
        \end{block}
        \begin{center}
            \only<5>{\alert{falsch}}
            \only<6>{\alert{wahr}}
        \end{center}
    }
\end{frame}
}
