\begin{frame}[fragile]{mehr Möglichkeiten reguläre Sprachen zu beschreiben}
    Graphen und \glqq Bilder\grqq\ sind oft nicht das optimale Mittel eine Sprache zu beschreiben.\\
    Die \alert{regulären Ausdrücke} bieten uns eine Möglichkeit Sprachen schnell und intuitiv zu beschreiben.
    \begin{alertblock}{Funktionsweise}
        \begin{enumerate}
            \item Wörter können mit einem angegebenen Muster abgeglichen werden.
            \item Lässt sich ein Wort durch das Muster beschreiben, ist es in der davon beschriebenen Sprache.
        \end{enumerate}
    \end{alertblock}
\end{frame}

\begin{frame}{Reguläre Ausdrücke verwenden}
    \begin{alertblock}{Induktive Definition der Syntax}
        \begin{itemize}
            \item \alert{$\emptyset$} und \alert{$\emptyWord$} sind reguläre Ausdrücke.
            \item \alert{a} ist ein regulärer Ausdruck (für alle a $\in \Sigma$).
            \item Wenn \alert{$\alpha$} und \alert{$\beta$} reguläre Ausdrücke sind,\\dann sind \alert{$\alpha \beta$}, \alert{($\alpha \mid \beta$)} und \alert{$(\alpha)^*$} auch reguläre Ausdrücke.
        \end{itemize}
    \end{alertblock}
    \metroset{block=fill}
    \begin{exampleblock}{Beispiel}
        $\gamma = ((a|b)^* \mid \emptyWord) \implies aba \in L(\gamma)$
    \end{exampleblock}
\end{frame}

\begin{frame}{Reguläre Ausdrücke verwenden}
    \begin{exampleblock}{Wie Sprachen und reguläre Ausdrücke zusammenhängen}
        \footnotesize
        \begin{itemize}[<+- | alert@+>]
            \item Wenn $\gamma = \emptyset$, beschreibt es die leere Sprache: $L(\gamma) = \{\}$
            \item Wenn $\gamma$ ein einzelnes Wort ist, ist genau dieses Wort in der Sprache enthalten.\\
                  $\gamma = \emptyWord$: $L(\gamma)=\{\emptyWord\}$,\quad$\gamma = a$: $L(\gamma)=\{a\}$
            \item Wenn $\gamma$ aus zwei hintereinandergeschrieben Ausdrücken besteht, repräsentiert Konkatenation.\\
                  $\gamma = (a)^*b: L(\gamma)=\{a^nb \mid n \in \mathbb{N}\}$
            \item Wenn $\gamma$ aus zwei mit \glqq oder\grqq\ verknüpften Ausdrücken besteht, sind beide Seiten in der Sprache enthalten.\\
                  $\gamma = (a \mid bc)$: $L(\gamma)=\{a, bc\}$
            \item Wenn $\gamma$ ein Ausdruck mit einem Stern ist, kann dieser innere Ausdruck beliebig oft wiederholt werden (auch null mal).\\
                  $\gamma = (a)^*$: $L(\gamma)=\{\emptyWord, a, aa, aaa, ...\}=\{a\}^*$
            \item Alles zusammen --- $\gamma = ((a)^* \mid (bc)^*)$: $L(\gamma)=\{\emptyWord, a, bc, aa, bcbc, aaa, ...\} = \{a\}^* \cup \{bc\}^*$
        \end{itemize}
    \end{exampleblock}
\end{frame}

{\setbeamercolor{palette primary}{bg=ExColor}
\begin{frame}{Reguläre Ausdrücke}
    \begin{alertblock}{Aufgaben}
        Finde einen regulären Ausdruck für die folgenden Sprachen
    \end{alertblock}
    \metroset{block=fill}
    \begin{block}{Normal}
        \begin{itemize}
            \item $L(\gamma_1) = \{a^{2n} \mid n\in\naturals\}$
            \item $L(\gamma_2) = \{a^nb^m \mid n, m\in\naturals\}$
            \item $L(\gamma_3) = \{uv \mid u\in\{a,b\}^\ast,\ v\in\{c,d\}\}$
            \item $L(\gamma_4) = \{w \mid |w| = 3, w\in \{a,b,c\}^*\}$
        \end{itemize}
    \end{block}
    \begin{block}{Etwas Schwerer}
        \begin{itemize}
            \item $L(\gamma_5) = \{a^n \mid n \equiv 1 \mod 3\}$
            \item $L(\gamma_6) = \{uv\mid u\in\{\text{\Rewind, \MoveUp, \Forward, \MoveDown}\}^\ast,\;v\in\{\text{\Stopsign}\}\}$
            \item $L(\gamma_7) = \{w \mid |w|_a = 3, |w|_b = 1, w\in \{a,b,c\}^*\}$
        \end{itemize}
    \end{block}
\end{frame}
}

{\setbeamercolor{palette primary}{bg=ExColor}
\begin{frame}<handout:0>{Lösung}
    Alle Lösungen sind Beispiellösungen, es sind auch andere möglich.\\
    Klammern die nicht zur Bedeutung beitragen, dürfen wir für die Kurzschreibweise weglassen.
    \begin{itemize}[<+- | alert@+>]
        \item $\gamma_1 = (aa)^*$
        \item $\gamma_2 = (a)^*(b)^*$
        \item $\gamma_3 = (a|b)^*\ (c|d)$
        \item $\gamma_4 = (a|b|c)(a|b|c)(a|b|c)$
        \item $\gamma_5 = a(aaa)^*$
        \item $\gamma_6 = ($\Rewind | \MoveUp | \Forward | \MoveDown$)^*$\Stopsign
        \item $\gamma_7 = c^*(ac^*ac^*ac^*b\mid ac^*ac^*bc^*a\mid ac^*bc^*ac^*a\mid bc^*ac^*ac^*a)c^*$
    \end{itemize}
\end{frame}
}

{\setbeamercolor{palette primary}{bg=ExColor}
\begin{frame}{Reguläre Ausdrücke}
    \begin{columns}
        \column{0.5\textwidth}
        \begin{alertblock} {Gegeben sei folgender DEA M:}
            \begin{tikzpicture}[->,>=stealth',shorten >=1pt,auto,node distance=2cm,
                    semithick]
                \node [initial,state]   (0)              {$q_0$};
                \node [state]           (1) [right of=0] {$q_1$};
                \node [state,accepting] (2) [below of=1] {$q_2$};
                \path   (0) edge               node {b} (1)
                edge               node {a} (2)
                (1) edge               node {b} (2)
                edge [loop above]  node {a} (1)
                (2) edge [loop below]  node {a,b} (2);
            \end{tikzpicture}
        \end{alertblock}
        \column{0.5\textwidth}
        \metroset{block=fill}
        \begin{block}{Welcher reguläre Ausdruck beschreibt $T(M)$?}
            \begin{enumerate}
                \item \alert<2>{$(a|b(a)^*b)(a|b)^*$}
                \item $a(ab)^*$
                \item $(a|b(a)^*b)(b)^*$
                \item $(a|b)^*$
            \end{enumerate}
        \end{block}
    \end{columns}
\end{frame}
}

{\setbeamercolor{palette primary}{bg=ExColor}
\begin{frame}{Reguläre Ausdrücke}
    \begin{alertblock}{Aufgaben}
        Begründe welche der folgenden Aussagen wahr/falsch sind:
    \end{alertblock}
    \metroset{block=fill}
    \begin{block}{Normal}
        \begin{itemize}
            \item $A_1:\ L\left(\left(a|a\right)^*\right) = L\left(\left(aa\right)^*\right)$
            \item $A_2:\ L\left(\left(a|b\right)^*\right) = L\left(\left(a\right)^*|\left(b\right)^*\right)$
            \item $A_3:\
                      L\left(\left(b\left(a|b\right)^*\right)|\left(b\left(a|b\right)^*\right)\right) = \left\{a,b\right\}^*$
        \end{itemize}
    \end{block}
    \begin{block}{Schwer}
        \begin{itemize}
            \item $A_4:\
                      L\left(\left(a|b\right)^*\right) = L\left(\left(\left(a\right)^*b\right)^*\left(a\right)^*\right)$
            \item $A_5:\ L\left(a\left(a|b\right)^*|b\left(a|b\right)^*\right) = \left\{a,b\right\}^*$
        \end{itemize}
    \end{block}
\end{frame}
}

{\setbeamercolor{palette primary}{bg=ExColor}
\begin{frame}<handout:0>{Lösung}
    \begin{itemize}[<+- | alert@+>]
        \item $A_1$: falsch
        \item $A_2$: falsch
        \item $A_3$: wahr
        \item $A_4$: falsch
        \item $A_5$: falsch
    \end{itemize}
\end{frame}
}

%%%%%%%%%%%%%%%%%%%%%%%%%%%%%%%%%%%%%%%%%%%%%%%%%%%
%
% "praktische" Aufgabe die noch eine Lösung braucht
%
%%%%%%%%%%%%%%%%%%%%%%%%%%%%%%%%%%%%%%%%%%%%%%%%%%%

% {\setbeamercolor{palette primary}{bg=ExColor}
% 	\begin{frame}{Reguläre Ausdrücke}
% 		\begin{alertblock}{Aufgaben}
% 			Schreibe einen regulären Ausdruck, der URLs (ungefähr) erkennen kann.
% 		\end{alertblock}
% 		\metroset{block=fill}
% 		\begin{block}{Bestandteile einer URL (vereinfacht für diese Aufgabe)}
% 			\begin{enumerate}
% 				\item \alert<2>{Protokoll}: http, https oder ftp \emph{(optional)}
% 				\item \alert<3>{Subdomain(s)}: www, oder andere Zeichenkette (Buchstaben, Zahlen, ausgewälte Symbole: -, ., \_, $\sim$) \emph{(optional)}
% 				\item \alert<4>{Domain}: Zeichenkette
% 				\item \alert<5>{top-level Domain(s)}: de, org, com, etc.
% 				\item \alert<6>{Pfad}: Zeichenketten getrennt von / oder einzelnes / \emph{(optional)}
% 				\item \alert<7>{Anfrage}: ? gefolgt von Zeichenkette = Zeichenkette \emph{(optional)}
% 				\item \alert<8>{Fragment}: \# gefolgt von Zeichenkette \emph{(optional)}
% 			\end{enumerate}
% 			\small{\alert<2>{https}://\alert<3>{fius}.\alert<3>{informatik}.\alert<4>{uni-stuttgart}.\alert<5>{de}\alert<6>{/index.php/easter-egg/}\alert<7>{?var=foo}\alert<8>{\#bar}}
% 		\end{block}
% 		\footnotesize{Abkürzung für diese Aufgabe: \emph{[Zeichen]} für erlaubtes Bestandteil einer Zeichenkette}
% 	\end{frame}
% }

% {\setbeamercolor{palette primary}{bg=ExColor}
% 	\begin{frame}<handout:0>{Lösung}
% 		Alle Lösungen sind Beispiellösungen, es sind auch andere möglich.\\
% 		Klammern die nicht zur Bedeutung beitragen, dürfen wir für die Kurzschreibweise weglassen.

% 		TODO

% 	\end{frame}
% }

%%%%%%%%%%%%%%%%%%%%%%%%%%%%%%%%%%%%%%%%%%%%%
%
% Das soll nach Regex zu NEA kommen
%
%%%%%%%%%%%%%%%%%%%%%%%%%%%%%%%%%%%%%%%%%%%%%

% {\setbeamercolor{palette primary}{bg=ExColor}
% \begin{frame}{Regulärer Ausdruck zu Automat}
%     \begin{alertblock}{Aufgaben}
%         Erinnert euch an euren Ausdruck zur Erkennung von URls zurück. Gebt einen äquivalenten Automaten an.
%     \end{alertblock}
%     \metroset{block=fill}
%     \begin{exampleblock}{Erinnerung}
%         Das war der Ausdruck aus der Musterlösung, ihr könnt aber auch euren eigenen benutzen.:

%         TODO

%     \end{exampleblock}
%     \footnotesize{Abkürzung für diese Aufgabe: \emph{[Zeichen]} für erlaubtes Bestandteil einer Zeichenkette}
% \end{frame}
% }

% {\setbeamercolor{palette primary}{bg=ExColor}
% \begin{frame}{Lösung}
%     Alle Lösungen sind Beispiellösungen, es sind auch andere möglich.

%     TODO

% \end{frame}
% }