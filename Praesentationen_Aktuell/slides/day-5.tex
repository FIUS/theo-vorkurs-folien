% !TeX program = lualatex
% !TeX spellcheck = de
% Copyright 2018-2022 FIUS
%
% This file is part of theo-vorkurs-folien.
%
% theo-vorkurs-folien is free software: you can redistribute it and/or modify
% it under the terms of the GNU General Public License as published by
% the Free Software Foundation, either version 3 of the License, or
% (at your option) any later version.
%
% theo-vorkurs-folien is distributed in the hope that it will be useful,
% but WITHOUT ANY WARRANTY; without even the implied warranty of
% MERCHANTABILITY or FITNESS FOR A PARTICULAR PURPOSE.  See the
% GNU General Public License for more details.
%
% You should have received a copy of the GNU General Public License
% along with theo-vorkurs-folien.  If not, see <https://www.gnu.org/licenses/>.

\documentclass[aspectratio=43,10pt]{beamer}

\usetheme[progressbar=frametitle]{metropolis}
\usepackage{appendixnumberbeamer}
\usepackage[ngerman]{babel}
\usepackage[utf8]{inputenc}
%\usepackage{t1enc}
\usepackage[T1]{fontenc}
\usepackage[sfdefault,scaled=.85,lf]{FiraSans}
\usepackage{newtxsf}

\usepackage{booktabs}
\usepackage[scale=2]{ccicons}
\usepackage{hyperref}

\usepackage{pgf}
\makeatletter
\@ifclasswith{beamer}{notes}{
  \usepackage{pgfpages}
  \setbeameroption{show notes on second screen}
}{}
\makeatother
\usepackage{tikz}
\usetikzlibrary{arrows,automata,positioning}
\usepackage{pgfplots}
\usepgfplotslibrary{dateplot}

\usepackage{xspace}
\newcommand{\themename}{\textbf{\textsc{metropolis}}\xspace}

\usepackage{blindtext}
\usepackage{graphicx}
\usepackage{subcaption}
\usepackage{comment}
\usepackage{mathtools}
\usepackage{amsmath}
\usepackage{centernot}
\usepackage{amssymb}
\usepackage{proof}
\usepackage{tabularx}
\renewcommand{\figurename}{Abb.}
\usepackage{marvosym}
\usepackage{mathtools}
\usepackage{qrcode}
\usepackage{advdate}

\newcommand\daynr{0}

\definecolor{ExColor}{HTML}{17819b}

\newcommand{\emptyWord}{\varepsilon}
\let \emptyset\varnothing
\newcommand{\SigmaStern}{\Sigma^{*}}
\newcommand{\absval}[1]{|#1|}
\newcommand{\defeq}{\vcentcolon=}
\newcommand{\eqdef}{=\vcentcolon}
\newcommand{\nimplies}{\centernot\implies}

\newcommand{\naturals}{\ensuremath{\mathbb{N}}}
\newcommand{\integers}{\ensuremath{\mathbb{Z}}}
\newcommand{\rationals}{\ensuremath{\mathbb{Q}}}
\newcommand{\reals}{\ensuremath{\mathbb{R}}}
\newcommand{\iffspace}{\ensuremath{\iff\;}}

\setbeamertemplate{footline}[text line]
{\parbox{\linewidth}{Fachgruppe Informatik\hfill\insertpagenumber\hfill Vorkurs Theoretische Informatik\vspace{0.2in}}}

\newcommand{\Center}[1]{
  \begin{frame}<handout:0>[standout]
    #1
  \end{frame}
}

% Fix section pages in appendix
\AtBeginDocument{%
  \apptocmd{\appendix}{%
    \setbeamertemplate{section page}[simple]%
  }{}{}
}

\addtobeamertemplate{block begin}{}{\vskip 0em}
\addtobeamertemplate{block alerted begin}{}{\vskip 0em}
\addtobeamertemplate{block example begin}{}{\vskip 0em}

% Copyright 2018-2022 FIUS
%
% This file is part of theo-vorkurs-folien.
%
% theo-vorkurs-folien is free software: you can redistribute it and/or modify
% it under the terms of the GNU General Public License as published by
% the Free Software Foundation, either version 3 of the License, or
% (at your option) any later version.
%
% theo-vorkurs-folien is distributed in the hope that it will be useful,
% but WITHOUT ANY WARRANTY; without even the implied warranty of
% MERCHANTABILITY or FITNESS FOR A PARTICULAR PURPOSE.  See the
% GNU General Public License for more details.
%
% You should have received a copy of the GNU General Public License
% along with theo-vorkurs-folien.  If not, see <https://www.gnu.org/licenses/>.



% Configuration for slides

% The date of the first day of the Theo-Vorkurs in Format dd/mm/yyyy
\SetDate[10/10/2022]

% Invite URL to the Ersti-Telegram-Group. Used for text on slide as well as QR-Code
\newcommand\telegramurl{https://t.me/+Q92w5biyY903NjEy}

% The url to the handout of the current day with the current day as argument. Used for the qr-code in the slides. 
\newcommand{\handouturl}[1]{https://fius.de/wp-content/uploads/2022/10/day-#1-handout.pdf}


\title{Vorkurs Theoretische Informatik}
\subtitle{Einführung in reguläre Sprachen}
\date{Freitag, 16.10.2020}
\author{Arbeitskreis  Theo Vorkurs}
\institute{Fachgruppe Informatik}
% \titlegraphic{\hfill\includegraphics[height=1.5cm]{logo.pdf}}

\begin{document}

\maketitle

\begin{frame}[fragile]{Übersicht}
  \setbeamertemplate{section in toc}[sections numbered]
  \tableofcontents%[hideallsubsections]
\end{frame}


\section{Grammatik und Automaten}

\begin{frame}{Automaten: Formal}
  Ein \textbf{DEA M} lässt sich beschreiben durch ein geordnetes 5-Tupel\\
  \alert{$M=(Z, \Sigma, \delta, z_0, E)$} mit:
  \begin{itemize}
    \item $Z$: Die Menge der Zustände
    \item $\Sigma$: Das Alphabet
    \item $\delta$: Die Überführungsfunktion
    \item $z_0$: Der Startzustand
    \item $E$: Die Menge der Endzustände
  \end{itemize}
\end{frame}


\begin{frame}{Automaten: Formal - Beispiel}
  \begin{columns}
    \column{0.5\textwidth}
    \alert{$L_1=\{a^{n}b^{m} \mid n,m \in \mathbb{N}\}$}\\
    \vspace{0.6cm}
    \begin{tikzpicture}[->,>=stealth',shorten >=1pt,auto,node distance=1.5cm,semithick]
      \node [initial,state,accepting]   (0)              {$q_0$};
      \node [state,accepting]           (1) [right of=0] {$q_1$};
      \node [state]                     (2) [below of=1] {$q_f$};

      \path   (0) edge                    node {b}    (1)
      edge [loop above]       node {a}    (0)
      (1) edge                    node {a}    (2)
      edge [loop above]       node {b}    (1)
      (2) edge [loop right]       node {a,b}  (2);
    \end{tikzpicture}

    \column{0.5\textwidth}
    \alert{$M=(Z, \Sigma, \delta, q_0, E)$} mit:
    \begin{itemize}
      \item<2-|handout:1> \alert<2|handout:0>{$Z=\{q_0, q_1, q_f\}$}
      \item<3-|handout:1> \alert<3|handout:0>{$\Sigma=\{a, b\}$}
      \item<4-|handout:1> \alert<4|handout:0>{$\delta$:
              \begin{itemize}
                \item $\delta(q_0, a)=q_0$
                \item $\delta(q_0, b)=q_1$
                \item $\delta(q_1, a)=q_f$
                \item $\delta(q_1, b)=q_1$
                \item $\delta(F, a)=q_f$
                \item $\delta(F, b)=q_f$
              \end{itemize}}
      \item<5-|handout:1> \alert<5|handout:0>{$E=\{q_0, q_1\}$}
    \end{itemize}
  \end{columns}
\end{frame}

{\setbeamercolor{palette primary}{bg=ExColor}
\begin{frame}{Automaten: Formal - Aufgabe}
  \footnotesize
  \begin{columns}
    \column{0.5\textwidth}
    \begin{alertblock}{Aufgabe}
    \end{alertblock}
    \metroset{block=fill}
    \begin{block}{Normal}
      \alert{$L_2=\{w \in \{a, b\}^* \mid |w|=3\}$}
      %TÖDÖ Töröööööt Benjamin der kleiner Elefant fragt Janette: Ey machst du automat amk!

      \vspace{0.1cm}
      \begin{tikzpicture}[->,>=stealth',shorten >=1pt,auto,node distance=1.5cm,semithick]
        \node [initial,state]   (0)              {$q_0$};
        \node [state]           (1) [below of=0] {$q_1$};
        \node [state]           (2) [below of=1] {$q_2$};
        \node [state,accepting] (3) [below of=2] {$q_E$};
        \node [state]           (f) [right of=3] {$q_f$};

        \path   (0) edge                node {a,b}  (1)
        (1) edge                node {a,b}  (2)
        (2) edge                node {a,b}  (3)
        (3) edge                node {a,b}  (f)
        (f) edge [loop above]   node {a,b} (f);
      \end{tikzpicture}
      \vspace{0.1cm}
    \end{block}

    \column{0.5\textwidth}
    \only<2-|handout:0>{
      \begin{alertblock}{Lösung}
      \end{alertblock}
      \metroset{block=fill}
      \alert{$M=(Z, \Sigma, \delta, q_0, E)$} mit:
      \begin{itemize}
        \item<3-> \alert<2>{$Z=\{q_0, q_1, q_2, q_E, q_f\}$}
        \item<4-> \alert<3>{$\Sigma=\{a, b\}$}
        \item<5-> \alert<4>{$\delta$:
                \vspace{-0.2cm}
                \begin{itemize}
                  \item $\delta(q_0, a)=q_1$
                  \item $\delta(q_0, b)=q_1$
                  \item $\delta(q_1, a)=q_2$
                  \item $\delta(q_1, b)=q_2$
                  \item $\delta(q_2, a)=q_3$
                  \item $\delta(q_2, b)=q_3$
                  \item $\delta(q_E, a)=q_f$
                  \item $\delta(q_E, b)=q_f$
                  \item $\delta(F, a)=q_f$
                  \item $\delta(F, b)=q_f$
                \end{itemize}}
        \item<6-> \alert<5>{$E=\{q_E\}$}
      \end{itemize}
    }
  \end{columns}
\end{frame}}

\begin{frame}{Automaten und Grammatiken: Parallelen}
  Wir stellen einige Parallelen fest:
  % note: \alert or \usebeamercolor[fg]{alerted text} inside p-columns breaks layout
  % default alert color of metropolis is mLightBrown
  \begin{center}\begin{tabular}{p{.46\textwidth}|p{.46\textwidth}}
      \textbf{DEA} & \textbf{Reguläre Grammatiken}
      \onslide<2-|handout:1>                                                                                                                        \\\hline
      Wörter werden \textcolor<2>{mLightBrown}{von links nach rechts} gelesen
                   & Wörter werden \textcolor<2|handout:0>{mLightBrown}{von links nach rechts} erzeugt
      \onslide<3-|handout:1>                                                                                                                        \\\hline
      \textcolor<3>{mLightBrown}{Pro Schritt} wird \textcolor<3>{mLightBrown}{ein Buchstabe} gelesen
                   & \textcolor<3|handout:0>{mLightBrown}{Pro Schritt} wird \textcolor<3>{mLightBrown}{ein Buchstabe} erzeugt
      \onslide<4-|handout:1>                                                                                                                        \\\hline
      \textcolor<4|handout:0>{mLightBrown}{Ein Startzustand}
                   & \textcolor<4|handout:0>{mLightBrown}{Ein Startsymbol}
      \onslide<5-|handout:1>                                                                                                                        \\\hline
      Beim Lesen des \textcolor<5|handout:0>{mLightBrown}{letzten Buchstabens} wird in einen \textcolor<5>{mLightBrown}{Endzustand} übergegangen
                   & Beim Erzeugen des \textcolor<5>{mLightBrown}{letzten Buchstabens} wird \textcolor<5>{mLightBrown}{keine neue Variable} erzeugt
      \onslide<6-|handout:1>                                                                                                                        \\\hline
      In jedem Schritt wird \textcolor<6|handout:0>{mLightBrown}{aus einem Zustand in genau einen Zustand} übergegangen
                   & In jedem Schritt wird \textcolor<6>{mLightBrown}{aus einer Variable genau eine Variable} erzeugt
    \end{tabular}\end{center}
\end{frame}

\begin{frame}{Automaten und Grammatiken}
  \metroset{block=fill}
  \begin{exampleblock}{Satz}
    Jede durch deterministische endliche Automaten erkennbare Sprache ist auch regulär (also Typ 3).
  \end{exampleblock}
  Der Beweis dieses Satzes findet sich im Anhang.
\end{frame}


\begin{frame}<handout:0>[standout]
  Verdauungspause
\end{frame}

\section{Reguläre Ausdrücke}

\begin{frame}[fragile]{mehr Möglichkeiten reguläre Sprachen zu beschreiben}
    Graphen und \glqq Bilder\grqq\ sind oft nicht das optimale Mittel eine Sprache zu beschreiben.\\
    Die \alert{regulären Ausdrücke} bieten uns eine Möglichkeit Sprachen schnell und intuitiv zu beschreiben.
    \begin{alertblock}{Funktionsweise}
        \begin{enumerate}
            \item Wörter können mit einem angegebenen Muster abgeglichen werden.
            \item Lässt sich ein Wort durch das Muster beschreiben, ist es in der davon beschriebenen Sprache.
        \end{enumerate}
    \end{alertblock}
\end{frame}

\begin{frame}{Reguläre Ausdrücke verwenden}
    \begin{alertblock}{Induktive Definition der Syntax}
        \begin{itemize}
            \item \alert{$\emptyset$} und \alert{$\emptyWord$} sind reguläre Ausdrücke.
            \item \alert{a} ist ein regulärer Ausdruck (für alle a $\in \Sigma$).
            \item Wenn \alert{$\alpha$} und \alert{$\beta$} reguläre Ausdrücke sind,\\dann sind \alert{$\alpha \beta$}, \alert{($\alpha \mid \beta$)} und \alert{$(\alpha)^*$} auch reguläre Ausdrücke.
        \end{itemize}
    \end{alertblock}
    \metroset{block=fill}
    \begin{exampleblock}{Beispiel}
        $\gamma = ((a|b)^* \mid \emptyWord) \implies aba \in L(\gamma)$
    \end{exampleblock}
\end{frame}

\begin{frame}{Reguläre Ausdrücke verwenden}
    \begin{exampleblock}{Wie Sprachen und reguläre Ausdrücke zusammenhängen}
        \footnotesize
        \begin{itemize}[<+- | alert@+>]
            \item Wenn $\gamma = \emptyset$, beschreibt es die leere Sprache: $L(\gamma) = \{\}$
            \item Wenn $\gamma$ ein einzelnes Wort ist, ist genau dieses Wort in der Sprache enthalten.\\
                  $\gamma = \emptyWord$: $L(\gamma)=\{\emptyWord\}$,\quad$\gamma = a$: $L(\gamma)=\{a\}$
            \item Wenn $\gamma$ aus zwei hintereinandergeschrieben Ausdrücken besteht, repräsentiert Konkatenation.\\
                  $\gamma = (a)^*b: L(\gamma)=\{a^nb \mid n \in \mathbb{N}\}$
            \item Wenn $\gamma$ aus zwei mit \glqq oder\grqq\ verknüpften Ausdrücken besteht, sind beide Seiten in der Sprache enthalten.\\
                  $\gamma = (a \mid bc)$: $L(\gamma)=\{a, bc\}$
            \item Wenn $\gamma$ ein Ausdruck mit einem Stern ist, kann dieser innere Ausdruck beliebig oft wiederholt werden (auch null mal).\\
                  $\gamma = (a)^*$: $L(\gamma)=\{\emptyWord, a, aa, aaa, ...\}=\{a\}^*$
            \item Alles zusammen --- $\gamma = ((a)^* \mid (bc)^*)$: $L(\gamma)=\{\emptyWord, a, bc, aa, bcbc, aaa, ...\} = \{a\}^* \cup \{bc\}^*$
        \end{itemize}
    \end{exampleblock}
\end{frame}

{\setbeamercolor{palette primary}{bg=ExColor}
\begin{frame}{Reguläre Ausdrücke}
    \begin{alertblock}{Aufgaben}
        Finde einen regulären Ausdruck für die folgenden Sprachen
    \end{alertblock}
    \metroset{block=fill}
    \begin{block}{Normal}
        \begin{itemize}
            \item $L(\gamma_1) = \{a^{2n} \mid n\in\naturals\}$
            \item $L(\gamma_2) = \{a^nb^m \mid n, m\in\naturals\}$
            \item $L(\gamma_3) = \{uv \mid u\in\{a,b\}^\ast,\ v\in\{c,d\}\}$
            \item $L(\gamma_4) = \{w \mid |w| = 3, w\in \{a,b,c\}^*\}$
        \end{itemize}
    \end{block}
    \begin{block}{Etwas Schwerer}
        \begin{itemize}
            \item $L(\gamma_5) = \{a^n \mid n \equiv 1 \mod 3\}$
            \item $L(\gamma_6) = \{uv\mid u\in\{\text{\Rewind, \MoveUp, \Forward, \MoveDown}\}^\ast,\;v\in\{\text{\Stopsign}\}\}$
            \item $L(\gamma_7) = \{w \mid |w|_a = 3, |w|_b = 1, w\in \{a,b,c\}^*\}$
        \end{itemize}
    \end{block}
\end{frame}
}

{\setbeamercolor{palette primary}{bg=ExColor}
\begin{frame}<handout:0>{Lösung}
    Alle Lösungen sind Beispiellösungen, es sind auch andere möglich.\\
    Klammern die nicht zur Bedeutung beitragen, dürfen wir für die Kurzschreibweise weglassen.
    \begin{itemize}[<+- | alert@+>]
        \item $\gamma_1 = (aa)^*$
        \item $\gamma_2 = (a)^*(b)^*$
        \item $\gamma_3 = (a|b)^*\ (c|d)$
        \item formal: $\gamma_4 = ((a|b)|c)((a|b)|c)((a|b)|c)$, kurz: $\gamma_4 = (a|b|c)(a|b|c)(a|b|c)$
        \item $\gamma_5 = a(aaa)^*$
        \item formal: $\gamma_6 = ($((\Rewind | \MoveUp) | \Forward) | \MoveDown$)^*$\Stopsign, kurz: $\gamma_6 = ($\Rewind | \MoveUp | \Forward | \MoveDown$)^*$\Stopsign
        \item formal: $\gamma_7 = (c)^*(a(c)^*a(c)^*a(c)^*b\mid a(c)^*a(c)^*b(c)^*a\mid a(c)^*b(c)^*a(c)^*a\mid b(c)^*a(c)^*a(c)^*a)(c)^*$,\\kurz: $\gamma_7 = c^*(ac^*ac^*ac^*b\mid ac^*ac^*bc^*a\mid ac^*bc^*ac^*a\mid bc^*ac^*ac^*a)c^*$
    \end{itemize}
\end{frame}
}

{\setbeamercolor{palette primary}{bg=ExColor}
\begin{frame}{Reguläre Ausdrücke}
    \begin{columns}
        \column{0.5\textwidth}
        \begin{alertblock} {Gegeben sei folgender DEA M:}
            \begin{tikzpicture}[->,>=stealth',shorten >=1pt,auto,node distance=2cm,
                    semithick]
                \node [initial,state]   (0)              {$q_0$};
                \node [state]           (1) [right of=0] {$q_1$};
                \node [state,accepting] (2) [below of=1] {$q_2$};
                \path   (0) edge               node {b} (1)
                edge               node {a} (2)
                (1) edge               node {b} (2)
                edge [loop above]  node {a} (1)
                (2) edge [loop below]  node {a,b} (2);
            \end{tikzpicture}
        \end{alertblock}
        \column{0.5\textwidth}
        \metroset{block=fill}
        \begin{block}{Welcher reguläre Ausdruck beschreibt $T(M)$?}
            \begin{enumerate}
                \item \alert<2>{$(a|b(a)^*b)(a|b)^*$}
                \item $a(ab)^*$
                \item $(a|b(a)^*b)(b)^*$
                \item $(a|b)^*$
            \end{enumerate}
        \end{block}
    \end{columns}
\end{frame}
}

{\setbeamercolor{palette primary}{bg=ExColor}
\begin{frame}{Reguläre Ausdrücke}
    \begin{alertblock}{Aufgaben}
        Begründe welche der folgenden Aussagen wahr/falsch sind:
    \end{alertblock}
    \metroset{block=fill}
    \begin{block}{Normal}
        \begin{itemize}
            \item $A_1:\ L\left(\left(a|a\right)^*\right) = L\left(\left(aa\right)^*\right)$
            \item $A_2:\ L\left(\left(a|b\right)^*\right) = L\left(\left(a\right)^*|\left(b\right)^*\right)$
            \item $A_3:\ L\left(\left(a|b\right)^*\right) = L\left(\left(\left(a\right)^*b\right)^*\left(a\right)^*\right)$
        \end{itemize}
    \end{block}
    \begin{block}{Etwas Schwerer}
        \begin{itemize}
            \item $A_4:\ L\left(b\left(a|b\right)^*|b\left(a|b\right)^*\right) = \left\{a,b\right\}^*$
            \item $A_5:\ L\left(a\left(a|b\right)^*|b\left(a|b\right)^*\right) = \left\{a,b\right\}^*$
        \end{itemize}
    \end{block}
\end{frame}
}

{\setbeamercolor{palette primary}{bg=ExColor}
\begin{frame}<handout:0>{Lösung}
    \begin{itemize}[<+- | alert@+>]
        \item $A_1$: falsch
        \item $A_2$: falsch
        \item $A_3$: wahr
        \item $A_4$: falsch
        \item $A_5$: falsch
    \end{itemize}
\end{frame}
}

%%%%%%%%%%%%%%%%%%%%%%%%%%%%%%%%%%%%%%%%%%%%%%%%%%%
%
% "praktische" Aufgabe die noch eine Lösung braucht
%
%%%%%%%%%%%%%%%%%%%%%%%%%%%%%%%%%%%%%%%%%%%%%%%%%%%

% {\setbeamercolor{palette primary}{bg=ExColor}
% 	\begin{frame}{Reguläre Ausdrücke}
% 		\begin{alertblock}{Aufgaben}
% 			Schreibe einen regulären Ausdruck, der URLs (ungefähr) erkennen kann.
% 		\end{alertblock}
% 		\metroset{block=fill}
% 		\begin{block}{Bestandteile einer URL (vereinfacht für diese Aufgabe)}
% 			\begin{enumerate}
% 				\item \alert<2>{Protokoll}: http, https oder ftp \emph{(optional)}
% 				\item \alert<3>{Subdomain(s)}: www, oder andere Zeichenkette (Buchstaben, Zahlen, ausgewälte Symbole: -, ., \_, $\sim$) \emph{(optional)}
% 				\item \alert<4>{Domain}: Zeichenkette
% 				\item \alert<5>{top-level Domain(s)}: de, org, com, etc.
% 				\item \alert<6>{Pfad}: Zeichenketten getrennt von / oder einzelnes / \emph{(optional)}
% 				\item \alert<7>{Anfrage}: ? gefolgt von Zeichenkette = Zeichenkette \emph{(optional)}
% 				\item \alert<8>{Fragment}: \# gefolgt von Zeichenkette \emph{(optional)}
% 			\end{enumerate}
% 			\small{\alert<2>{https}://\alert<3>{fius}.\alert<3>{informatik}.\alert<4>{uni-stuttgart}.\alert<5>{de}\alert<6>{/index.php/easter-egg/}\alert<7>{?var=foo}\alert<8>{\#bar}}
% 		\end{block}
% 		\footnotesize{Abkürzung für diese Aufgabe: \emph{[Zeichen]} für erlaubtes Bestandteil einer Zeichenkette}
% 	\end{frame}
% }

% {\setbeamercolor{palette primary}{bg=ExColor}
% 	\begin{frame}<handout:0>{Lösung}
% 		Alle Lösungen sind Beispiellösungen, es sind auch andere möglich.\\
% 		Klammern die nicht zur Bedeutung beitragen, dürfen wir für die Kurzschreibweise weglassen.

% 		TODO

% 	\end{frame}
% }

%%%%%%%%%%%%%%%%%%%%%%%%%%%%%%%%%%%%%%%%%%%%%
%
% Das soll nach Regex zu NEA kommen
%
%%%%%%%%%%%%%%%%%%%%%%%%%%%%%%%%%%%%%%%%%%%%%

{\setbeamercolor{palette primary}{bg=ExColor}
\begin{frame}{Regulärer Ausdruck zu Automat}
    \begin{alertblock}{Aufgaben}
        Erinnert euch an euren Ausdruck zur Erkennung von URls zurück. Gebt einen äquivalenten Automaten an.
    \end{alertblock}
    \metroset{block=fill}
    \begin{exampleblock}{Erinnerung}
        Das war der Ausdruck aus der Musterlösung, ihr könnt aber auch euren eigenen benutzen.:

        TODO

    \end{exampleblock}
    \footnotesize{Abkürzung für diese Aufgabe: \emph{[Zeichen]} für erlaubtes Bestandteil einer Zeichenkette}
\end{frame}
}

{\setbeamercolor{palette primary}{bg=ExColor}
\begin{frame}{Lösung}
    Alle Lösungen sind Beispiellösungen, es sind auch andere möglich.

    TODO

\end{frame}
}

\begin{frame}<handout:0>[standout]
  Murmelpause
\end{frame}

\section{Wiederholung}
\begin{frame}[fragile]{Das können wir jetzt beantworten}
  \begin{alertblock}{Tag 1: Mengen}
    \begin{itemize}
      \item Was ist eine Menge?
      \item Wie kann man zwei Mengen verknüpfen?
      \item Wie schreibt man formal Mengen auf?
    \end{itemize}
  \end{alertblock}
\end{frame}

\begin{frame}[fragile]{Das können wir jetzt beantworten}
  \begin{alertblock}{Tag 1: Formale Sprachen}
    \begin{itemize}
      \item Was ist eine Formale Sprache?
      \item Was ist ein Alphabet?
      \item Wie zeigt man, dass zwei Sprachen äquivalent sind?
    \end{itemize}
  \end{alertblock}
\end{frame}

% Aufgaben zu grundlegenden Definitionen

{\setbeamercolor{palette primary}{bg=ExColor}
\begin{frame}[fragile]{Wiederholung Grundlegender Definitionen}
    \begin{alertblock}{Aufgaben}
        Beantworte die folgenden Fragen
    \end{alertblock}
    \metroset{block=fill}
    \begin{block}{Normal}
        \begin{itemize}
            \item Was ist der Unterschied zwischen $\Sigma$ und $\Sigma^*$?
            \item Welche besondere Eigenschaft besitzt das leere Wort $\emptyWord$? Insbesondere: Welchen Wert besitzt $|\emptyWord|$?
            \item Welchen Wert besitzt $|\emptyWord \cdot a^2 \cdot \emptyWord \cdot bab|_a$?
        \end{itemize}
    \end{block}
%    \begin{block}{Etwas Schwerer}
%        \begin{itemize}
%            \item Für welche $n \in \mathbb{N}$ gilt $a^n = a^3 \cdot \emptyWord \cdot a^{n-3}$?
%        \end{itemize}
%    \end{block}
\end{frame}
}

{\setbeamercolor{palette primary}{bg=ExColor}
\begin{frame}[fragile]{Lösungen}
    \only<1>{
        \metroset{block=fill}
        \begin{block}{Normal}
            Was ist der Unterschied zwischen $\Sigma$ und $\Sigma^*$?
        \end{block}
        \begin{enumerate}
            \item[i)] Das \alert{Alphabet} $\Sigma$ ist eine nichtleere Menge einstelliger Symbole.
            \item[ii)] $\Sigma^*$ ist die Menge aller möglichen Kombinationen (bzgl. der Konkatenation) der Elemente aus $\Sigma$.\\
            Insbesondere gilt hier $\emptyWord \in \Sigma^*$.
        \end{enumerate}
    }

    \only<2>{
        \metroset{block=fill}
        \begin{block}{Normal}
            Welche besondere Eigenschaft besitzt das leere Wort $\emptyWord$? Insbesondere: Welchen Wert besitzt $|\emptyWord|$?
        \end{block}
        $\emptyWord$ ist das neutrale Element bzgl. der Konkatenation. Das heißt:
        \begin{align*}
            \forall w \in \Sigma^*: w = w \cdot \emptyWord = \emptyWord \cdot w
        \end{align*}
        Insbesondere gilt also:
        \begin{align*}
            \forall w \in \Sigma^*: |w| = |w \cdot \emptyWord| = |w| + |\emptyWord|
        \end{align*}
        womit $|\emptyWord| = 0$ gelten muss.
    }

    \only<3>{
        \metroset{block=fill}
        \begin{block}{Normal}
            Welchen Wert besitzt $|\emptyWord \cdot a^2 \cdot \emptyWord \cdot bab|_a$?
        \end{block}
        Zunächst schreiben wir
        \begin{align*}
            \emptyWord \cdot a^2 \cdot \emptyWord \cdot bab = a^2bab = aabab
        \end{align*}
        und zählen anschließend die vorkommenden $a$'s. Also
        \begin{align*}
            |\emptyWord \cdot a^2 \cdot \emptyWord \cdot bab|_a = |aabab|_a = 3.
        \end{align*}
    }
\end{frame}
}

{\setbeamercolor{palette primary}{bg=ExColor}
\begin{frame}[fragile]{Aufgaben zu Mengen}
    \begin{alertblock}{Aufgaben}
        Welche der folgenden Aussagen ist korrekt?
    \end{alertblock}
    \metroset{block=fill}
    \begin{block}{Normal bis etwas Schwerer}
        \begin{minipage}[t]{0.45\textwidth}
            \begin{itemize}
                \item $a \in \{a, b, c\}$
                \item $a \subseteq \{a, b, c\}$
                \item $\{a,b\} \in \{a, b, \{a, b\}, c\}$
            \end{itemize}
        \end{minipage}
        \begin{minipage}[t]{0.45\textwidth}
            \begin{itemize}
                \item $\{a,b\} \subseteq \{a, b, \{a, b\}, c\}$
                \item $\{\{a,b\}\} \subseteq \{a, b, \{a, b\}, c\}$
            \end{itemize}
        \end{minipage}
    \end{block}
    Nenne jeweils 5 Wörter aus den folgenden Sprachen
    \metroset{block=fill}
    \begin{block}{Normal bis etwas Schwerer}
        \begin{itemize}
            \item $L_1 = \{a^nb^m \mid n, m \in \mathbb{N}\}$
            \item $L_2 = \{a^nb^n \mid n \in \mathbb{N}\}$
            \item $L_3 = L_2 \backslash L_1$
            \item $L_4 = \{w \in \{a,b\}^* \mid |w|_a \equiv 3 \pmod{4}\}$
        \end{itemize}
    \end{block}
\end{frame}
}

{\setbeamercolor{palette primary}{bg=ExColor}
\begin{frame}[fragile]{Lösungen}
    \metroset{block=fill}
    \begin{block}{Normal}
        \begin{align*}
            \only<1>{a \in \{a, b, c\}}
            \only<2>{a \subseteq \{a, b, c\}}
            \only<3>{\{a,b\} \in \{a, b, \{a, b\}, c\}}
            \only<4>{\{a,b\} \subseteq \{a, b, \{a, b\}, c\}}
            \only<5>{\{\{a,b\}\} \subseteq \{a, b, \{a, b\}, c\}}
        \end{align*}
    \end{block}
    \begin{itemize}[<+- | alert@+>]
        \item wahr
        \item geht nicht
        \item wahr
        \item wahr
        \item wahr
    \end{itemize}
\end{frame}
}

{\setbeamercolor{palette primary}{bg=ExColor}
\begin{frame}[fragile]{Lösungen}
        \metroset{block=fill}
        \begin{block}{Normal}
            \begin{align*}
                \only<3>{&}\only<1,3>{L_1 = \{a^nb^m \mid n, m \in \mathbb{N}\}}\only<3>{\\}
                \only<3>{&}\only<2,3>{L_2 = \{a^nb^n \mid n \in \mathbb{N}\}}\only<3>{\\}
                \only<3>{&}\only<3>{L_3 = L_1 \backslash L_2}
                \only<4>{L_4 = \{w \in \{a,b\}^* \mid |w|_a \equiv 3 \pmod{4}\}}
            \end{align*}
        \end{block}
        \begin{itemize}[<+- | alert@+>]
            \item $\emptyWord, a, b, aa, ab, aaa, aab, aba, abb, ...$
            \item $\emptyWord, ab, aabb, a^3b^3, a^4b^4, ...$
            \item Keine Wörter; $\emptyset$
            \item $aaa, baaa, abaa, aaba, aaab, a^7, ...$
        \end{itemize}
\end{frame}
}

% Aufgaben zu Aussagenlogik und Quantoren

{\setbeamercolor{palette primary}{bg=ExColor}
\begin{frame}[fragile]{Wiederholung Grundlegender Definitionen}
    \begin{alertblock}{Aufgaben}
        Bestimme den Wahrheitswert der folgenden Aussagen
    \end{alertblock}
    \metroset{block=fill}
    \begin{block}{Normal}
        \begin{itemize}
            \item $\neg (42 = 11) \wedge (|\emptyWord| > 0 \vee \emptyset^* = \{\emptyWord\})$
            \item $\forall w \in \{a, b\}^*: (|w|_a = 0 \implies w = \emptyWord)$
            \item $\forall w \in \{a, b\}^*: (|w|_a = 0 \implies ab = ba)$
            \item $(\forall w \in \{a, b\}^*: |w|_a = 0) \implies ab = ba$
        \end{itemize}
    \end{block}
    \metroset{block=fill}
    \begin{block}{Schwer}
        \begin{itemize}
            \item $\forall x \in \mathbb{Z}: x \equiv 0 \pmod{2} \vee x \equiv 1 \pmod{4}$
        \end{itemize}
    \end{block}
\end{frame}
}

{\setbeamercolor{palette primary}{bg=ExColor}
\begin{frame}[fragile]{Lösungen}
    \only<1-4>{
        \metroset{block=fill}
        \begin{block}{Normal}
            \begin{align*}
                \only<1>{\neg (42 = 11) \wedge (|\emptyWord| > 0 \vee \emptyset^* = \{\emptyWord\})}
                \only<2>{\forall w \in \{a, b\}^*: (|w|_a = 0 \implies w = \emptyWord)}
                \only<3>{\forall w \in \{a, b\}^*: (|w|_a = 0 \implies ab = ba)}
                \only<4>{(\forall w \in \{a, b\}^*: |w|_a = 0) \implies ab = ba}
            \end{align*}
        \end{block}
        \begin{itemize}[<+- | alert@+>]
            \item richtig
            \item falsch
            \item falsch
            \item richtig
        \end{itemize}
    }
    \only<5>{
        \metroset{block=fill}
        \begin{block}{Schwer}
            \begin{align*}
                \only<5>{\forall x \in \mathbb{Z}: x \equiv 0 \pmod{2} \vee x \equiv 1 \pmod{4}}
            \end{align*}
        \end{block}
        \only<5>{
            Die Aussage ist falsch!\\
            Gegenbeispiel $x = 3$, dann ist $x \equiv 1 \pmod{2}$ und $x \equiv 3 \pmod{4}$.
        }
    }
\end{frame}
}


\begin{frame}[fragile]{Das können wir jetzt beantworten}
  \begin{alertblock}{Tag 2: Beweise}
    \begin{itemize}
      \item Was ist ein direkter Beweis?
      \item Wie funktioniert die Kontraposition?
      \item Wie funktioniert ein Widerspruchsbeweis?
    \end{itemize}
  \end{alertblock}
\end{frame}

\begin{frame}[fragile]{Das können wir jetzt beantworten}
  \begin{alertblock}{Tag 3: Grammatiken}
    \begin{itemize}
      \item Was sind Grammatiken?
      \item Was ist der Zusammenhang zwischen Grammatiken und Sprachen?
      \item Wie finde ich raus, ob ein Wort von einer Grammatik erzeugt wird?
    \end{itemize}
  \end{alertblock}
\end{frame}

\begin{frame}[fragile]{Das können wir jetzt beantworten}
  \begin{alertblock}{Tag 4: Reguläre Grammatiken}
    \begin{itemize}
      \item Wie sehen Produktionsregeln für reguläre Grammatiken aus?
      \item Bilden einer regulären Grammatik für gegebene reguläre Sprache
    \end{itemize}
  \end{alertblock}
\end{frame}

\begin{frame}[fragile]{Das können wir jetzt beantworten}
  \begin{alertblock}{Tag 4: Automaten}
    \begin{itemize}
      \item Was sind Automaten?
      \item Was macht einen deterministischen Automaten aus?
      \item Finden eines (deterministischen) Automaten für gegebene Sprache
    \end{itemize}
  \end{alertblock}
\end{frame}

\begin{frame}[fragile]{Das können wir jetzt beantworten}
  \begin{alertblock}{Heute: Repräsentationen regulärer Sprachen}
    \begin{itemize}
      \item Welche Möglichkeiten gibt es, reguläre Sprachen zu beschreiben?
      \item Wie wandelt man Automaten zu einer äquivalenten Grammatik um?
      \item Was ist ein regulärer Ausdruck?
    \end{itemize}
  \end{alertblock}
\end{frame}

\begin{frame}[fragile]{Das können wir jetzt beantworten}
  \begin{alertblock}{Heute: Reguläre Ausdrücke}
    \begin{itemize}
      \item Wie funktioniert die Konkatenation?
      \item Was bedeuten ($\alpha \mid \beta$) und $(\alpha)^*$?
      \item Finden eines regulären Ausdrucks für gegebene reguläre Sprache
    \end{itemize}
  \end{alertblock}
\end{frame}

\begin{frame}<handout:0>[standout]
  Noch Fragen?
\end{frame}

\begin{frame}[fragile]{Glossar}
  \small
  \begin{tabular}{p{0.2\textwidth} p{0.25\textwidth} p{0.4\textwidth}}
    \toprule
    Abk.                       & Bedeutung        & Was?!                                                                                                              \\
    \midrule
    \begin{tikzpicture}[->,>=stealth',shorten >=1pt,auto,node distance=1cm,semithick,baseline=(q0.base)]
      \node[initial,state](q0){$q_0$};
    \end{tikzpicture} & Startzustand     & Hier fängt der Automat beim Lesen eines Wortes an                                                                  \\
    \begin{tikzpicture}[->,>=stealth',shorten >=1pt,auto,node distance=1.4cm,semithick,baseline=(qi.base)]
      \node[state](qi){$q_i$};
      \node[state](qj)[right of=qi]{$q_j$};
      \path (qi) edge node {$a$} (qj);
    \end{tikzpicture} & Zustandsübergang & gibt an, welches Symbol eingelesen werden kann, um in den Folgezustand zu übergehen.                               \\
    \begin{tikzpicture}[->,>=stealth',shorten >=1pt,auto,node distance=1cm,semithick,baseline=(qe.base)]
      \node[accepting,state](qe){$q_E$};
    \end{tikzpicture} & Endzustand       & Hier kann ein fertig gelesenes Wort akzeptiert werden.                                                             \\
    \begin{tikzpicture}[->,>=stealth',shorten >=1pt,auto,node distance=2cm,semithick,baseline=(qi.base)]
      \node[state](qi){$\emptyset$};
      \path (qi) edge [loop right] node {$x \in \Sigma$} (B);
    \end{tikzpicture} & Fangzustand      & wird benötigt, um Determinismus zu gewährleisten. In Graphiken oft nicht eingezeichnet, ist aber da. Malt den hin. \\
    \bottomrule
  \end{tabular}
\end{frame}

\begin{frame}[fragile]{Glossar}
  \small
  \begin{tabular}{p{0.05\textwidth} p{0.37\textwidth} p{0.43\textwidth}}
    \toprule
    Abk.        & Bedeutung                          & Was?!                                                               \\
    \midrule
    T(M)        & Sprache von Automat M              & Die Sprache, die von einem Automat M erkannt wird                   \\
    L(G)        & Sprache von Grammatik G            & Die Sprache, die von einer Grammatik G erzeugt wird                 \\
    $\gamma$    & kleines Gamma                      & oft Bezeichner für regulären Ausdruck                               \\
    L($\gamma$) & Sprache von reg. Ausdruck $\gamma$ & Die Sprache. die von einem regulären Ausdruck $\gamma$ erkannt wird \\
    \bottomrule
  \end{tabular}
\end{frame}

\appendix
\begin{frame}[standout]
  Anhang
\end{frame}

\section{Beweis: DEA zu Grammatik}
% Copyright 2018-2022 FIUS
%
% This file is part of theo-vorkurs-folien.
%
% theo-vorkurs-folien is free software: you can redistribute it and/or modify
% it under the terms of the GNU General Public License as published by
% the Free Software Foundation, either version 3 of the License, or
% (at your option) any later version.
%
% theo-vorkurs-folien is distributed in the hope that it will be useful,
% but WITHOUT ANY WARRANTY; without even the implied warranty of
% MERCHANTABILITY or FITNESS FOR A PARTICULAR PURPOSE.  See the
% GNU General Public License for more details.
%
% You should have received a copy of the GNU General Public License
% along with theo-vorkurs-folien.  If not, see <https://www.gnu.org/licenses/>.

\begin{frame}{Automaten und Grammatiken}
  \metroset{block=fill}
  \begin{exampleblock}{Satz}
    Jede durch \only<6-|handout:0>{\alert<6>{deterministische}} endliche Automaten erkennbare Sprache ist auch regulär (also Typ 3).
  \end{exampleblock}
  \onslide<2-|handout:1> %
  Sei \alert<2|handout:0>{$A \subseteq \SigmaStern$} eine Sprache und \alert<2|handout:0>{M ein \only<-5|handout:0>{Automat}\only<6-|handout:1>{\alert<6|handout:0>{DEA}}} mit \alert<2-3|handout:0>{$\mathbf{T(M) = A}$}.\\
  \onslide<3-|handout:1> %
  (d.h. $M$ erkennt die Sprache $A$) \\
  \vspace{.3cm} %

  \onslide<4-|handout:1> %
  Wir suchen eine \alert<4|handout:0>{Typ 3-Grammatik G} mit \alert<4-5|handout:0>{$\mathbf{L(G) = A}$}.\\
  \onslide<5-|handout:1> %
  (d.h. die Grammatik $G$ erzeugt die Sprache $A$)
  \vspace{.3cm} %

  \onslide<6-|handout:1> %
  \begin{alertblock}{Anmerkung}
    Wir beschränken uns auf DEAs; in der Vorlesung werdet ihr aber eine allgemeinere Äquivalenz zeigen.
  \end{alertblock}
\end{frame}

\begin{frame}{DEA zu Grammatik: Konstruktion}
  Also: \alert<4-5|handout:0>{Zustände $\hat{=}$ Variablen}; \alert<6|handout:0>{Übergänge $\hat{=}$ Produktionen}

  \onslide<2-|handout:1> %
  Sei also ein \alert<2|handout:0>{DEA $M = (\alert<4|handout:0>{Z}, \alert<3|handout:0>{\Sigma}, \alert<6|handout:0>{\delta}, \alert<5|handout:0>{z_0}, E)$} gegeben. \\
    Wir definieren die \alert<2|handout:0>{Grammatik $G = (\alert<4|handout:0>{V}, \alert<3|handout:0>{\Sigma}, \alert<6|handout:0>{P}, \alert<5|handout:0>{S})$} mit:
  \begin{itemize}
    \item<3- | alert@3|handout:1> Alphabet $\Sigma$
    \item<4- | alert@4|handout:1> Variablenmenge $V = Z$
    \item<5- | alert@5|handout:1> Startsymbol $S = z_0$
    \item<6- | alert@6|handout:1> Produktionsmenge $P$ erzeugen wir aus $\delta$
  \end{itemize}
\end{frame}

\begin{frame}{DEA zu Grammatik: Produktionsregeln}\begin{columns}
    \begin{column}{0.5\textwidth}
      Für die Produktionsmenge $P$ wandeln wir die Übergänge um. \\
      \vspace{.3cm}
      \onslide<2-|handout:1> %
      Jedem $\delta$-Übergang \alert<2-3|handout:0>{$\delta(z_1, a)=z_2$} ordnen wir folgende Regeln zu:
      \begin{itemize}
        \item<2- | alert@2|handout:1> $z_1 \rightarrow a z_2$
        \item<3-|handout:1> Und zusätzlich, falls \alert<3|handout:0>{$z_2 \in E:$ $z_1 \rightarrow a$}
      \end{itemize}
      \onslide<4-|handout:1>{ %
        Falls $z_0 \in E$, brauchen wir außerdem $z_0 \to \emptyWord$.
      }
    \end{column}
    \begin{column}{0.5\textwidth}\centering\alert<handout:0>{\only<2-3|handout:1>{ %
        \begin{tikzpicture}[->,>=stealth',shorten >=1pt,auto,node distance=2cm,semithick]
          \node [state] (1)  {$z_1$};
          \onslide<2|handout:1>{
            \node [state] (2) [right of=1]  {$z_2$};
            \path (1) edge node {a} (2);
          }\onslide<3|handout:1>{
            \node [state, accepting] (2') [right of=1]  {$z_2$};
            \path (1) edge node {a} (2');
          }
        \end{tikzpicture} \\
        \glqq$\delta (z_1,a) = z_2$\grqq \\
        wird zu
        \begin{align*}
           & z_1 \to az_2            \\
           & \onslide<3->{z_1 \to a}
        \end{align*}
      }\only<4|handout:1>{
        \begin{tikzpicture}[->,>=stealth',shorten >=1pt,auto,node distance=2cm,semithick]
          \node [state,initial,accepting] (0)  {$z_0$};
        \end{tikzpicture} \\
        \glqq$z_0 \in E$\grqq \\
        wird zu $$z_0 \to \emptyWord$$
      }}\end{column}
  \end{columns}\end{frame}

\begin{frame}{DEA zu Grammatik: Korrektheit (Beweisidee)}
  \begin{alertblock}{Zu zeigen: $x \in T(M)$ gdw. $x \in L(G)$}
    \onslide<2-|handout:1> %
    Sei $x = a_1 a_2 \ldots a_{n-1} a_n$ ein Wort, das von $M$ akzeptiert wird. \\
    \onslide<3-|handout:1> %
    \begin{center}\centering\begin{tikzpicture}[->,>=stealth',shorten >=1pt,auto,node distance=1.75cm,semithick]
        \alert<5-5|handout:0>{\node [initial,state]   (0)              {$z_0$};}
        \alert<5-6|handout:0>{\node [state]           (1) [right of=0] {$z_1$};}
        \alert<6-8|handout:0>{\node                   (5) [right of=1] {$\ldots$};}
        \alert<8-9|handout:0>{\node [state]           (8) [right of=5] {$z_{n-1}$};}
        \alert<9-9|handout:0>{\node [state,accepting] (9) [right of=8] {$z_n$};}

        \alert<5|handout:0>{\path (0) edge node {$a_1$}     (1);}
        \alert<6|handout:0>{\path (1) edge node {$a_2$}     (5);}
        \alert<8|handout:0>{\path (5) edge node {$a_{n-1}$} (8);}
        \alert<9|handout:0>{\path (8) edge node {$a_n$}     (9);}
      \end{tikzpicture}\end{center}
    \onslide<4-|handout:1> %
    Mit den passenden Regeln \only<5-9|handout:1>{(z.B. %
      \only<5>{$z_0 \to a_1 z_1$}%
      \only<6>{$z_1 \to a_2 z_2$}%
      \only<7>{$z_2 \to a_3 z_3$}%
      \only<8>{$z_{n-2} \to a_{n-1} z_{n-1}$}%
      \only<9>{$z_{n-1} \to a_n$}%
      ) }lässt sich $x$ ableiten:
    $$\alert<5|handout:0>{z_0} \alert<5|handout:0>{\Rightarrow} \alert<5|handout:0>{a_1} \alert<5-6|handout:0>{z_1} \alert<6|handout:0>{\Rightarrow} a_1 \alert<6|handout:0>{a_2} \alert<6-7|handout:0>{z_2} \alert<7|handout:0>{\Rightarrow} \alert<7-8|handout:0>{\ldots} \alert<8|handout:0>{\Rightarrow} a_1 a_2 \ldots \alert<8|handout:0>{a_{n-1}} \alert<8-9|handout:0>{z_{n-1}} \alert<9|handout:0>{\Rightarrow} a_1 a_2 \ldots a_{n-1} \alert<9|handout:0>{a_n} = x$$
    \onslide<10-|handout:1> %
    Also können wir das Wort in der Grammatik ableiten. \\
    \onslide<11-|handout:1> %
    \alert<11|handout:0>{Funktioniert die Argumentation auch andersrum?}
    \onslide<12-|handout:1> %
    \alert<12|handout:0>{Ja!}
  \end{alertblock}
\end{frame}

\begin{frame}{DEA zu Grammatik: Korrektheit}
  \begin{alertblock}{Zu zeigen: $x \in T(M)$ gdw. $x \in L(G)$}
    Dabei gilt immer noch: $x = a_1 a_2 \ldots a_{n-1} a_n$ \\
    \onslide<2-|handout:1> %
    Die folgenden Aussagen sind äquivalent:
    \begin{itemize}
      \item<3- | alert@3|handout:1|handout:alert@0> $x$ wird von Automat $M$ erkannt \emph{$(x\in T(M))$}
      \item<4- | alert@4|handout:1|handout:alert@0> Es gibt eine Folge von Zuständen $z_0, z_1, \ldots, z_{n-1}, z_n$ mit:
            $z_0$ ist Startzustand, $z_n$ ist Endzustand \textbf{und}:
            $\forall i \in \{1, \ldots, n\}: \delta(z_{i-1}, a_i) = z_i$
      \item<5- | alert@5|handout:1|handout:alert@0> Es gibt Folge an Variablen $z_0, z_1, \dots, z_{n-1}$ mit:
            $z_0$ ist Startvariable, $(z_{n-1} \to a_n) \in P$ \textbf{und}:
            $\forall i \in \{1, \ldots, n-1\}: (z_{i-1} \to a_i z_i) \in P$
      \item<6- | alert@6|handout:1|handout:alert@0> Es gibt Folge an Variablen $z_0, z_1, \dots, z_{n-1}$ mit:
            $z_0$ ist Startvariable \textbf{und}: $z_0 \Rightarrow a_1 z_1 \Rightarrow \ldots \Rightarrow a_1 a_2 \ldots a_{n-1} a_n$
      \item<7- | alert@7|handout:1|handout:alert@0> x wird von der Grammatik G produziert \emph{$(x \in L(G))$}
    \end{itemize}
    \onslide<8-|handout:1> %
    Also gilt die Äquivalenz. \qed
  \end{alertblock}
\end{frame}

% \begin{frame}[fragile]{Grammatik zu Automaten}
%  Eine reguläre Grammatik und eine endlicher Automat können die selben Sprachen beschreiben.
%  \begin{exampleblock}{\glqq$\implies$\grqq}
%  \only<1>{Aus $z_1 \to az_2 \in P$ wird:\\
%  \begin{center}
%  \vspace{0.3cm}
%      \begin{tikzpicture}[->,>=stealth',shorten >=1pt,auto,node distance=2cm,
%                      semithick]
%      %\tikzstyle{every state}=[fill=ExColor,draw=none,text=white]

%      \node       [state]                 (1) [right of=1]  {$z_1$};
%      \node       [state]                 (2) [right of=2]  {$z_2$};

%      \path
%              (1) edge                node {a}  (2);
%      \end{tikzpicture}\\
%      \glqq$\delta (z_1,a) = z_2$\grqq
%  \end{center}}

%  \only<2>{
%  Falls $z_2 \in F$ ($z_2$ ein Endzustand)\\
%  Aus $z_1 \to a \in P$ wird:\\
%  \begin{center}
%  \vspace{0.3cm}
%      \begin{tikzpicture}[->,>=stealth',shorten >=1pt,auto,node distance=2cm,
%                      semithick]
%      %\tikzstyle{every state}=[fill=ExColor,draw=none,text=white]

%      \node       [state]                 (1) [right of=1]  {$z_1$};
%      \node       [state, accepting]      (2) [right of=2]  {$z_2$};

%      \path
%              (1) edge                node {a}  (2);
%      \end{tikzpicture}\\
%      \glqq$\delta (z_1,a) = z_2$\grqq
%  \end{center}
%  }
%  \end{exampleblock}
% \end{frame}


\begin{frame}<handout:0>[fragile]{Online-Whiteboard}
  \phantom{text}
\end{frame}

\end{document}
