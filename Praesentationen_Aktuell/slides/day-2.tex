% !TeX program = lualatex
% !TeX spellcheck = de
% Copyright 2018-2022 FIUS
%
% This file is part of theo-vorkurs-folien.
%
% theo-vorkurs-folien is free software: you can redistribute it and/or modify
% it under the terms of the GNU General Public License as published by
% the Free Software Foundation, either version 3 of the License, or
% (at your option) any later version.
%
% theo-vorkurs-folien is distributed in the hope that it will be useful,
% but WITHOUT ANY WARRANTY; without even the implied warranty of
% MERCHANTABILITY or FITNESS FOR A PARTICULAR PURPOSE.  See the
% GNU General Public License for more details.
%
% You should have received a copy of the GNU General Public License
% along with theo-vorkurs-folien.  If not, see <https://www.gnu.org/licenses/>.

\documentclass[aspectratio=43,10pt]{beamer}

\usetheme[progressbar=frametitle]{metropolis}
\usepackage{appendixnumberbeamer}
\usepackage[ngerman]{babel}
\usepackage[utf8]{inputenc}
%\usepackage{t1enc}
\usepackage[T1]{fontenc}
\usepackage[sfdefault,scaled=.85,lf]{FiraSans}
\usepackage{newtxsf}

\usepackage{booktabs}
\usepackage[scale=2]{ccicons}
\usepackage{hyperref}

\usepackage{pgf}
\makeatletter
\@ifclasswith{beamer}{notes}{
  \usepackage{pgfpages}
  \setbeameroption{show notes on second screen}
}{}
\makeatother
\usepackage{tikz}
\usetikzlibrary{arrows,automata,positioning}
\usepackage{pgfplots}
\usepgfplotslibrary{dateplot}

\usepackage{xspace}
\newcommand{\themename}{\textbf{\textsc{metropolis}}\xspace}

\usepackage{blindtext}
\usepackage{graphicx}
\usepackage{subcaption}
\usepackage{comment}
\usepackage{mathtools}
\usepackage{amsmath}
\usepackage{centernot}
\usepackage{amssymb}
\usepackage{proof}
\usepackage{tabularx}
\renewcommand{\figurename}{Abb.}
\usepackage{marvosym}
\usepackage{mathtools}
\usepackage{qrcode}
\usepackage{advdate}

\newcommand\daynr{0}

\definecolor{ExColor}{HTML}{17819b}

\newcommand{\emptyWord}{\varepsilon}
\let \emptyset\varnothing
\newcommand{\SigmaStern}{\Sigma^{*}}
\newcommand{\absval}[1]{|#1|}
\newcommand{\defeq}{\vcentcolon=}
\newcommand{\eqdef}{=\vcentcolon}
\newcommand{\nimplies}{\centernot\implies}

\newcommand{\naturals}{\ensuremath{\mathbb{N}}}
\newcommand{\integers}{\ensuremath{\mathbb{Z}}}
\newcommand{\rationals}{\ensuremath{\mathbb{Q}}}
\newcommand{\reals}{\ensuremath{\mathbb{R}}}
\newcommand{\iffspace}{\ensuremath{\iff\;}}

\setbeamertemplate{footline}[text line]
{\parbox{\linewidth}{Fachgruppe Informatik\hfill\insertpagenumber\hfill Vorkurs Theoretische Informatik\vspace{0.2in}}}

\newcommand{\Center}[1]{
  \begin{frame}<handout:0>[standout]
    #1
  \end{frame}
}

% Fix section pages in appendix
\AtBeginDocument{%
  \apptocmd{\appendix}{%
    \setbeamertemplate{section page}[simple]%
  }{}{}
}

\addtobeamertemplate{block begin}{}{\vskip 0em}
\addtobeamertemplate{block alerted begin}{}{\vskip 0em}
\addtobeamertemplate{block example begin}{}{\vskip 0em}

% Copyright 2018-2022 FIUS
%
% This file is part of theo-vorkurs-folien.
%
% theo-vorkurs-folien is free software: you can redistribute it and/or modify
% it under the terms of the GNU General Public License as published by
% the Free Software Foundation, either version 3 of the License, or
% (at your option) any later version.
%
% theo-vorkurs-folien is distributed in the hope that it will be useful,
% but WITHOUT ANY WARRANTY; without even the implied warranty of
% MERCHANTABILITY or FITNESS FOR A PARTICULAR PURPOSE.  See the
% GNU General Public License for more details.
%
% You should have received a copy of the GNU General Public License
% along with theo-vorkurs-folien.  If not, see <https://www.gnu.org/licenses/>.



% Configuration for slides

% The date of the first day of the Theo-Vorkurs in Format dd/mm/yyyy
\SetDate[10/10/2022]

% Invite URL to the Ersti-Telegram-Group. Used for text on slide as well as QR-Code
\newcommand\telegramurl{https://t.me/+Q92w5biyY903NjEy}

% The url to the handout of the current day with the current day as argument. Used for the qr-code in the slides. 
\newcommand{\handouturl}[1]{https://fius.de/wp-content/uploads/2022/10/day-#1-handout.pdf}


\title{Vorkurs Theoretische Informatik}
\subtitle{Grundlagen der Beweise}
\date{Dienstag, 13.10.2020}
\author{Arbeitskreis Theo Vorkurs}
\institute{Fachgruppe Informatik}
% \titlegraphic{\hfill\includegraphics[height=1.5cm]{logo.pdf}}

\begin{document}

\maketitle

\begin{frame}[fragile]{Übersicht}
  \setbeamertemplate{section in toc}[sections numbered]
  \tableofcontents
  % [hideallsubsections]
\end{frame}

\section{Quantoren}

\begin{frame}[fragile]{Quantoren}
    Oft wollen wir Aussagen nicht nur für ein Element, sondern für viele Elemente treffen.
    \metroset{block=fill}
    \begin{exampleblock}{Beispiel}
        $A_1$: Für die Zahl 5 gilt: Sie hat einen Nachfolger\\
        \emph{Allgemeiner:}\\
        $A_2$: Für jede natürliche Zahl n gilt: n hat einen Nachfolger
    \end{exampleblock}
    \begin{exampleblock}{Beispiel}
        $A_3$: Für die Zahl 5 gilt: Sie ist eine Primzahl\\
        \emph{Allgemeiner:}\\
        $A_4$: Es gibt eine natürliche Zahl n, so dass gilt: n ist eine Primzahl
    \end{exampleblock}
\end{frame}

\begin{frame}[fragile]{Quantoren}
    Mithilfe von \textbf{Quantoren} vereinfachen wir uns die Schreibweise dieser Aussagen.\\
    \vspace{0.5cm}
    Quantor \alert{$\forall$}: Die Aussage gilt für alle Elemente.\\
    \metroset{block=fill}
    \begin{exampleblock}{Beispiel}
        $A_1$: $\forall k \in \mathbb{N}:$ 2k ist gerade
    \end{exampleblock}
    Quantor \alert{$\exists$}: Die Aussage gilt für mindestens ein Element.\\
    \metroset{block=fill}
    \begin{exampleblock}{Beispiel}
        $A_2$: $\exists k \in \mathbb{N}:$ k ist Primzahl
    \end{exampleblock}
\end{frame}

\begin{frame}[fragile]{Quantoren}
    In einer Aussage können mehrere Quantoren vorkommen.\\
    Wir lesen dann von links nach rechts.
    \metroset{block=fill}
    \begin{exampleblock}{Beispiel}
        $A_1$: $\forall x,y \in \mathbb{N}: \exists z \in \mathbb{N}: x+y = z$\\
        Bedeutung: Für zwei beliebige Zahlen x und y aus $\mathbb{N}$ gibt es eine weitere natürliche Zahl z, so dass $x+y=z$ gilt.
    \end{exampleblock}
\end{frame}

\begin{frame}[fragile]{Quantoren}
    \alert{Achtung!}\\
    Die Reihenfolge von zwei Quantoren zu vertauschen, kann die Bedeutung einer Aussage deutlich verändern.
    \metroset{block=fill}
    \begin{exampleblock}{Beispiel}
        x,y $\in$ Studenten\\
        \textbf{$A_1$: $\forall x \exists y:$ x schlägt y\\
        $A_2$: $\exists x \forall y:$ x schlägt y\\ }
        Was ist der Unterschied zwischen beiden Aussagen?
    \end{exampleblock}
\end{frame}

\begin{frame}{Quantoren}
    \begin{alertblock}{Aufgabe}
      Wir formulieren folgende Aussage mithilfe von Quantoren und den Symbolen der Aussagenlogik (Junktoren).
    \end{alertblock}
    \metroset{block=fill}
    \begin{block}{Beispiel}
    \begin{itemize}
        \item $A_1$: Eine ganze Zahl ist eine natürliche Zahl, wenn sie positiv oder null ist.
    \end{itemize}
    \end{block}
    \begin{block}{Hinführung}
    \begin{itemize}
        \item $A_1$: Für alle ganzen Zahlen x gilt: Wenn x positiv oder null ist, ist x eine natürliche Zahl.
    \end{itemize}
    \end{block}
    \begin{block}{\alert{Lösung}}
    \begin{itemize}
        \item $A_1$: $\forall x \in \mathbb{Z}: x \geq 0 \implies x \in \mathbb{N}$
    \end{itemize}
    \end{block}
\end{frame}

{\setbeamercolor{palette primary}{bg=ExColor}
\begin{frame}{Denkpause}
    \begin{alertblock}{Aufgaben}
      Formuliere folgende Aussagen mithilfe von Quantoren und den Symbolen der Aussagenlogik (Junktoren). 
    \end{alertblock}
    \metroset{block=fill}
    \begin{block}{Normal}
    \begin{itemize}
        \item $A_1$: Die Differenz zweier ganzer Zahlen ist wieder eine ganze Zahl.
    \end{itemize}
    \end{block}
    \begin{block}{Schwer}
    \begin{itemize}
        \item $A_2$: Jede natürliche Zahl lässt sich als Summe von vier Quadratzahlen darstellen.
    \end{itemize}
    \end{block}
    \begin{block}{Da haben selbst wir keinen Bock drauf}
    \begin{itemize}
        \item $A_3$: Eine natürliche Zahl, die von einer von ihr verschiedenen natürlichen Zahl größer als 1 geteilt wird, ist nicht prim.
    \end{itemize}
    \end{block}
\end{frame}
}

{\setbeamercolor{palette primary}{bg=ExColor}
\begin{frame}{Lösungen}
  \begin{itemize}[<+- | alert@+>]
        \item 
            $A_1$: $\forall x,y \in \mathbb{Z}: x-y \in \mathbb{Z}$
        \item
            $A_2$: $\forall x \in \mathbb{N}: \exists a, b, c, d \in \mathbb{N}: x = a^2 + b^2 + c^2 + d^2$
        \item
            $A_3$: $\forall x \in \mathbb{N}: \left(\exists y \in \mathbb{N}: (y>1) \wedge (y \neq x) \wedge (y \mid x)\right) \implies x\ \text{ist keine Primzahl}$.
    \end{itemize}
\end{frame}
}

\begin{frame}[fragile]{Äquivalente Schreibweisen von Mengenoperationen}
	Oft benötigen wir eine Aussagenlogische Äquivalente Bedingung von Mengenoperationen.
	\metroset{block=fill}
	\begin{block}{Operationen}
		\begin{itemize}
			\item<1-> \textbf{Teilmenge}: A \alert<1>{$\subseteq$} B $\leadsto$ $\forall x \in A: x \in B$\\
			\item<2-> \textbf{Vereinigung}: C = A \alert<2>{$\cup$} B $\leadsto$ $\forall x \in C: x \in A \vee x \in B$\\
			\item<3-> \textbf{Schnitt}: C = A \alert<3>{$\cap$} B $\leadsto$ $\forall x \in C: x \in A \wedge x \in B$\\
			\item<4-> \textbf{Komplement}: \alert<4>{$\overline{A}$} $\leadsto$ $\forall x \in \overline{A}: x \notin A$
		\end{itemize}
	\end{block}
	
\end{frame}


\section{Beweisen}

% Copyright 2018, 2019, 2020, 2021 FIUS
%
% This file is part of theo-vorkurs-folien.
%
% theo-vorkurs-folien is free software: you can redistribute it and/or modify
% it under the terms of the GNU General Public License as published by
% the Free Software Foundation, either version 3 of the License, or
% (at your option) any later version.
%
% theo-vorkurs-folien is distributed in the hope that it will be useful,
% but WITHOUT ANY WARRANTY; without even the implied warranty of
% MERCHANTABILITY or FITNESS FOR A PARTICULAR PURPOSE.  See the
% GNU General Public License for more details.
%
% You should have received a copy of the GNU General Public License
% along with theo-vorkurs-folien.  If not, see <https://www.gnu.org/licenses/>.

\begin{frame}{Einführung}
\begin{alertblock}{Was ist ein Beweis?}
\begin{itemize}
        \item lückenlose Folge von logischen Schlüssen,\\welche zur zu beweisenden Behauptung führen
        \item nicht nur einleuchtend, sondern zweifelsfrei korrekt
    \end{itemize}
\end{alertblock}
\end{frame}

\subsubsection{Beweisbeispiel: Transitivität der Teilmenge}
\begin{frame}[fragile]{Beispielbeweis}
\begin{exampleblock}{Zu zeigen: Teilmengen sind transitiv.}
\begin{enumerate}
    \item<1->\alert<1|handout:0>{
        \only<1|handout:0>{zu zeigen: }\onslide<2->{z.z. }$A\subseteq B\wedge B\subseteq C \alert<3|handout:0>{\implies\text{}}A\subseteq C$
        }
    \item<2->\alert<2|handout:0>{
        \only<2|handout:0>{Umschreiben:\\}
        $\iff $\alert<4,5|handout:0>{$($\alert<9|handout:0>{$($\alert<6|handout:0>{$\forall x$}$\ : x \in A \implies x \in B)$}$ \wedge $\alert<10|handout:0>{$($\alert<6|handout:0>{$\forall x$}$\ : x \in B \implies x \in C)$}$)$}\\
        \qquad\alert<3|handout:0>{$\implies$}$\;($\alert<6|handout:0>{$\forall x$}$\ :\ $\alert<7|handout:0>{$x \in A$}$ \implies x \in C)$
        }
    \item<3->\alert<3|handout:0>{
        \only<3|handout:0>{\emph{Implikation}\\
        linke Seite wahr $\implies$ rechte Seite muss wahr sein.\\
        linke Seite falsch $\implies$ beliebiges kann folgen\\
        $\implies$ uns interessiert also nur der Fall \emph{links ist wahr}}
        \alert<4>{\only<4,5|handout:0>{Wir machen uns also \emph{\textquotedbl die linke Seite ist wahr\textquotedbl} zur Voraussetzung}\alert<5>{\only<5|handout:0>{:\\Angenommen, $A \subseteq B \wedge B \subseteq C$ gilt.}}}
        \onslide<6->{Ang., $A \subseteq B \wedge B \subseteq C$.}
        }
    \item<6->\alert<6|handout:0>{
        \only<6|handout:0>{Jetzt geht der Beweis richtig los.\\Wähle beliebiges $x$, um Allgemeinheit zu wahren\dots\\}
        \onslide<6->{Sei $x$ beliebig}\alert<7>{\onslide<7-|handout:0>{, mit \alert<9>{$x\in A$.}}}
        }
    \item<8->\alert<8-9|handout:0>{
        \only<8|handout:0>{Wir können jetzt unsere Voraussetzungen ausnutzen,\\um $x\in C$ zu folgern.}
        \onslide<9->$\implies x\in B$
        \alert<10|handout:0>{\onslide<10->$\implies x\in C$}
        \onslide<11>\qed
    }
  \end{enumerate}
\end{exampleblock}
\end{frame}


\subsubsection{Beweistechnik: Direkter Beweis}
\begin{frame}[fragile]{Direkter Beweis}
    \begin{alertblock}{Zeige $A\implies B$ direkt}
    Setze $A$ voraus und folgere dann schrittweise $B$.\\
    Durch jede korrekte Folgerung, vergrößert sich die Menge der als wahr bekannten Aussagen (sog. Annahmen).
    \end{alertblock}
    \metroset{block=fill}
    \begin{exampleblock}{Beispiel}
    Z.z.: \alert<2|handout:0>{$\forall a\in\mathbb{Z}$}: \alert<3|handout:0>{$a$ ist gerade} $\implies$ \alert<6|handout:0>{$a^2$ gerade.}
    \begin{enumerate}
        \item \alert<2|handout:0>{Sei $a\in\mathbb{Z}$ beliebig.}
        \item \alert<3|handout:0>{Angenommen, $a$ ist gerade.}
        \item \alert<4|handout:0>{$\implies \exists n\in\mathbb{Z} : a = 2n$}
        \item \alert<5|handout:0>{$\implies a^2 = (2n)^2 = 4n^2 = 2 \cdot 2n^2$}
        \item \alert<6|handout:0>{$\implies a^2$ ist gerade}\qed\;
    \end{enumerate}
    \end{exampleblock}
    \small{\emph{\alert<4|handout:0>{Anmerkung:}} Zahl $n\in\mathbb{Z}$ heißt gerade, wenn es ein $k\in\mathbb{Z}$ gibt mit $n=2k$.}
\end{frame}

\subsubsection{Beweistechnik: Kontraposition}
\begin{frame}[fragile]{Beweis durch Kontraposition}
    \begin{alertblock}{Zeige $A\implies B$, indem man stattdessen $\neg B \implies\neg A$ zeigt.}
    \end{alertblock}
    \metroset{block=fill}
    \begin{exampleblock}{Beispiel}
    Z.z.: \alert<7|handout:0>{\alert<1|handout:0>{$\forall n\in\mathbb{N}$:} $n^2$ gerade $\implies$\alert<2|handout:0>{ $n$ gerade}} \\
    \qquad bzw. \alert<1|handout:0>{$\forall n\in\mathbb{N}$:} \alert<2|handout:0>{$\neg(n \text{ gerade})$} $\implies \neg(n^2$ gerade$)$ 
    \begin{enumerate}
        \item\alert<1|handout:0>{Sei $n \in \mathbb{N}$ beliebig.}
        \item\alert<2|handout:0>{Angenommen, $n$ ist \emph{nicht} gerade.}
        \item\alert<3|handout:0>{$\implies n=2k+1$, für ein $k \in \mathbb{Z}$}
        \item $\only<4>{\overset{\alert{quadrieren}}}{\; \leadsto \;} n^2 = (2k+1)^2 = 4k^2+4k+1 = 2\alert<5|handout:0>{(2k^2+2k)}+1$
        \item $\implies n^2= \alert<6|handout:0>{2\alert<5|handout:0>{m}+1}$, für $m=2k^2+2k$
        \item $\implies n^2$ ist \alert<6|handout:0>{ungerade}.
        \item Da ($\forall n\in\mathbb{N}$: $n$ ungerade $\implies n^2$ ungerade) gilt, \\
        $\leadsto$ \alert<7|handout:0>{($\forall n\in\mathbb{N}$: $n^2$ gerade $\implies n$ gerade)}, was zu beweisen war.\qed\;
    \end{enumerate}
    \end{exampleblock}
    \footnotesize{\alert<3,6|handout:0>{\emph{Anmerkung:}} Zahl $n\in\mathbb{Z}$ heißt ungerade, wenn es ein $k\in\mathbb{Z}$ gibt mit $n=2k+1$.}
\end{frame}

\begin{frame}[fragile]{Beweis durch Kontraposition}
%Work in progress
\begin{alertblock}{Wieso dürfen wir das so machen?}
\end{alertblock}
\metroset{block=fill}
\begin{exampleblock}{Beweis}
Z.z.: $(\neg A \implies\neg B) \iff (B \implies A)$
    \begin{flalign*}
        \;(\neg A\implies\neg B) \iff & (\neg (\neg A) \vee \neg B)&\\
        \iff & (A \vee \neg B)&\\
        \iff & (\neg B \vee A)&\\
        \iff & (B \implies A)&\qed\;
    \end{flalign*}
\end{exampleblock}
\small\emph{Erinnerung:} $A\implies B$ kann man auch $\neg A\vee B$ schreiben.
\end{frame}

\Center{Verdauungspause}

\subsubsection{Beweistechnik: Widerspruch}
\begin{frame}[fragile]{Beweis durch Widerspruch}
\small{
    \begin{alertblock}{Zeige, dass $A$ gilt, indem man zeigt dass $\neg A$ falsch ist.}
    %Spezialfall der Kontraposition: $A\text{ ist wahr}\implies A \iff \neg A\implies \text{falsch}$
    \emph{Erinnerung:} Eine Aussage ist entweder wahr oder falsch.\\
    Wenn $\neg A$ falsch ist, muss $A$ wahr sein.
    \end{alertblock}
    \metroset{block=fill}
    \begin{exampleblock}{Beispiel}
        Z. z. $\sqrt{2}$ ist irrational.
        \begin{enumerate}
            \item<1-|handout:1> \alert<1|handout:0>{Ang. $\sqrt{2}$ ist rational.}
            \item<2-|handout:1> \alert<2|handout:0>{Dann $\exists p, q \in \mathbb{Z} : \sqrt{2} = \frac{p}{q}$} $\wedge$ \alert<3,11|handout:0>{$p, q$ sind teilerfremd.}
            \only<2,3|handout:0>{\alert<2>{{\\\emph{Anmerkung:}}} $r\in\mathbb{Q}\iff\exists p,q\in\mathbb{Z}:r=\frac{p}{q}$.}
            \only<2,3|handout:0>{\alert<3>{{\\\emph{Anmerkung:}}} $\frac{p}{q}$ kann man immer soweit kürzen, dass $p,q$ teilerfremd sind.}
            \item<4-|handout:1> \only<4|handout:0>{Quadrieren und Umformen:\\}$\leadsto(\sqrt{2})^2 = (\frac{p}{q})^2 \iff 2 = \frac{p^2}{q^2} \iff \alert<5,8|handout:0>{2q^2=p^2}$
            \item<5-|handout:1> \alert<5|handout:0>{$\leadsto p^2$ ist gerade.} \alert<6,10|handout:0>{$\leadsto p$ ist gerade.}\only<5,6|handout:0>{\alert<5>{{\\\emph{Erinnerung:}}} $\forall n\in\mathbb{Z}:n$ gerade $\iff\exists k\in\mathbb{Z}:2k=n$.}\only<5,6|handout:0>{\alert<6>{{\\\emph{Erinnerung:}}} $\forall n\in\mathbb{N}$: $n^2$ gerade $\implies n$ gerade \\\qquad\qquad\quad(siehe Beispiel Kontraposition)}
            \item<7-|handout:1> \alert<7|handout:0>{Also ist $p^2$ durch 4 teilbar.} \alert<8|handout:0>{$\leadsto 2q^2$ ist durch 4 teilbar.}\only<7|handout:0>{\\\alert<7|handout:0>{\emph{Herleitung:}} $p^2=p\cdot p\overset{\text{p gerade}}{=\joinrel=\joinrel=\joinrel=}(2k)\cdot(2k)=\alert<7>{4}k^2,$ mit $k\in\mathbb{Z}$}
            \item<9-|handout:1> $\leadsto q^2$ ist gerade. \alert<10|handout:0>{$\leadsto q$ ist gerade.}
            \item<10-|handout:1> \alert<10|handout:0>{$\leadsto p,q$ nicht teilerfremd.} \alert<11|handout:0>{$\leadsto$ Widerspruch}\only<11|handout:1>{\qed\;}
        \end{enumerate}
    \end{exampleblock}
}
\end{frame}

% Copyright 2018, 2019, 2020, 2021 FIUS
%
% This file is part of theo-vorkurs-folien.
%
% theo-vorkurs-folien is free software: you can redistribute it and/or modify
% it under the terms of the GNU General Public License as published by
% the Free Software Foundation, either version 3 of the License, or
% (at your option) any later version.
%
% theo-vorkurs-folien is distributed in the hope that it will be useful,
% but WITHOUT ANY WARRANTY; without even the implied warranty of
% MERCHANTABILITY or FITNESS FOR A PARTICULAR PURPOSE.  See the
% GNU General Public License for more details.
%
% You should have received a copy of the GNU General Public License
% along with theo-vorkurs-folien.  If not, see <https://www.gnu.org/licenses/>.

%\subsubsection{Tricks}
\begin{frame}[fragile]{Tricks: Fallunterscheidung}
    \begin{alertblock}{Hilfe! Der Beweis ist zu komplex! Was nun?}
        Manchmal lässt sich ein Beweis in kleinere Aussagen zerlegen. Wenn wir alle Teilaussagen beweisen, haben wir die Gesamtaussage gezeigt.
    \end{alertblock}
    \metroset{block=fill}
    \small\begin{exampleblock}{Beispiel}
        Z.z. für alle $n\in\mathbb{N}$ gilt, dass der Rest von $n^2 \div 4$ entweder 0 oder 1 ist.
        \footnotesize\begin{itemize}
            \item 
                \alert{Fall 1:} n ist gerade\\
                $n^2=n\cdot n\overset{\text{n gerade}}{=\joinrel=\joinrel=\joinrel=\joinrel=}(2k)\cdot(2k) = 4k^2$,  mit $k\in\mathbb{Z}$\\
                $\implies 4k^2 \div 4 = k^2$ Rest: 0
            \item \alert{Fall 2:} n ist ungerade\\
                $n^2=n\cdot n\overset{\text{n ungerade}}{=\joinrel=\joinrel=\joinrel=\joinrel=\joinrel=}(2k+1)\cdot(2k+1)=(2k)^2+2(2k)+1=4(k^2+k)+1$, \\mit $k\in\mathbb{Z}$\\
                $\implies (4(k^2+k)+1) \div 4= k^2+k$ Rest: 1
        \end{itemize}
        Da $n$ nur gerade oder ungerade sein kann, ist der Rest von $n^2\div4$ \\entweder 0 oder 1. \qed\;
    \end{exampleblock}
\end{frame}

\begin{frame}[fragile]{Tricks: Beispiele und Gegenbeispiele}
    \begin{alertblock}{Reicht nicht auch ein Beispiel als Beweis?}
      % Manchmal\dots
    \end{alertblock}
    \metroset{block=fill}
    \begin{block}{Wann ein Beispiel \emph{nicht} ausreicht:}
        Zeige allgemeine Aussagen, also Aussagen der Form:\\$\forall n\in\mathbb{N}$ gilt \dots, $\neg\exists n\in\mathbb{N}$\dots, $\exists!n\in\mathbb{N}$\dots, etc.\\
        \alert{Warum nicht?}\\
        Beispiele zeigen uns nur endlich viele Möglichkeiten.\\
        \glqq für Alle gilt\dots\grqq, \glqq es existiert kein\dots\grqq, \glqq es existiert genau ein\dots\grqq, etc. \\sind meist zu allgemeine Aussagen um sie mit endlich vielen Beispielen lückenlos zu beweisen.
    \end{block}
\end{frame}

\begin{frame}[fragile]{Tricks: Beispiele und Gegenbeispiele}
    \begin{alertblock}{Reicht nicht auch ein Beispiel als Beweis?}
      % Manchmal\dots
    \end{alertblock}
    \metroset{block=fill}
    \begin{block}{Wann ein Beispiel ausreichen kann:}
        Zeige nicht allgemeine Aussagen der Form:\\
        $\exists n\in\mathbb{N}$, $\neg\forall n\in\mathbb{N}$ gilt, \dots\\
        \alert{Warum?}\\
        \glqq es gibt ein Element, sodass\dots\grqq, \glqq für nicht alle Element gilt\dots\grqq\\wären durch Angabe eines solchen Elements gezeigt.
    \end{block}
        $\leadsto$ will man zeigen, dass eine Aussage falsch ist, sind die Formen entsprechend negiert.
\end{frame}


%\subsubsection{Aufgaben}
{\setbeamercolor{palette primary}{bg=ExColor}
\begin{frame}[fragile]{Denkpause}
    \begin{alertblock}{Aufgaben}
    Welche Beweistechnik könnte sich für die folgenden Aussagen eignen? Warum?
    \end{alertblock}
    
    \metroset{block=fill}
    \begin{block}{Normal}
        \begin{itemize}
            \item Für jede Primzahl $p$ ist $2^p-1$ eine Primzahl.
        \end{itemize}
    \end{block}
    \metroset{block=fill}
    \begin{block}{Etwas schwerer}
    \begin{itemize}
            \item Es gibt eine ganze Zahl x, sodass $x\equiv 1\bmod 4 \implies x\equiv 1\bmod 2$ 
    \end{itemize}
        
    \end{block}
\end{frame}
}

{\setbeamercolor{palette primary}{bg=ExColor}
\begin{frame}[fragile]{Lösungen}
    \begin{alertblock}{Aufgaben}
    Welche Beweistechnik könnte sich für die folgenden Aussagen eignen? Warum?
    \end{alertblock}
    
    \metroset{block=fill}
    \begin{block}{Normal}
        \begin{itemize}
            \item Für jede Primzahl $p$ ist $2^p-1$ eine Primzahl.\\
            $\rightarrow$ Gegenbeispiel (sei $p\defeq11$)
        \end{itemize}
    \end{block}
    \metroset{block=fill}
    \begin{block}{Etwas schwerer}
        \begin{itemize}
            \item[] $\rightarrow$Direkter Beweis
            \item $x\equiv 1\bmod 4\iff\exists z\in\mathbb{Z} : 4 \cdot z + 1 = x$
            \item $\iff\exists z\in\mathbb{Z}: (2 \cdot 2) \cdot z + 1 = x$
            \item $\implies\exists u,z\in\mathbb{Z}: u = 2z\wedge 2u + 1 = x$
            \item $\implies x\equiv 1\bmod 2$
        \end{itemize}
    \end{block}
\end{frame}
}


\begin{frame}<handout:0>[standout]
  Induktion folgt morgen
\end{frame}

\begin{frame}<handout:0>[standout]
    Murmelpause
\end{frame}


\section{Mengenbeweise}

\begin{frame}{einfacher Einstieg}
        \onslide
            Zu zeigen: Schnitt ist Kommutativ, d.h. $A \cap B = B \cap A$
        \begin{columns}
        \column{0.5\textwidth}
        \onslide{
            \begin{align*}
                \text{\quotedblbase}\implies\text{\textquotedblright}:\\
                x\in A \cap B &\implies x\in A \wedge x \in B\\
                &\implies x\in B \wedge x \in A\\
                &\implies x\in B \cap A
            \end{align*}
            }
        \column{0.5\textwidth}
        \onslide{
            \begin{align*}
                \text{\quotedblbase}\impliedby\text{\textquotedblright}:\\
                x\in B \cap A &\implies x\in B \wedge x \in A\\
                &\implies x\in A \wedge x \in B\\
                &\implies x\in A \cap B
            \end{align*}
            }
        \end{columns}
        \qed\\
    \small{\emph{Anmerkung:} $\wedge$ ist kommutativ}
\end{frame}

\begin{frame}{einfacher Einstieg}
        \onslide
            Zu zeigen: $A \setminus (B \cup C) = (A \setminus B) \cap (A \setminus C)$
        \onslide{
            \begin{align*}
                x\in A \setminus (B \cup C) &\iff x\in A \wedge \neg (x \in B \cup C)\\
                &\iff x\in A \wedge \neg (x \in B \vee x \in C)\\
                &\iff x \in A \wedge \neg (x \in B) \wedge \neg (x \in C)\\
                &\iff x \in A \wedge \neg (x \in B) \wedge x \in A \wedge \neg (x \in C)\\
                &\iff (x \in A \wedge \neg (x \in B)) \wedge (x \in A \wedge \neg (x \in C))\\
                &\iff (x \in A \setminus B) \wedge (x \in A \setminus C)\\
                &\iff x \in (A \setminus B) \cap (A \setminus C)
            \end{align*}\qed
            }
        \\
    \small{\emph{Rechenregel:} $\neg (A \wedge B) \iff \neg A \vee \neg B$, \\ \hspace{1.9cm}$\neg (A \vee B) \iff \neg A \wedge \neg B$}
\end{frame}



\subsubsection{Aufgaben}
{\setbeamercolor{palette primary}{bg=ExColor}
\begin{frame}[fragile]{Denkpause}
    \begin{alertblock}{Aufgaben}
    Versuche dich an den folgenden Mengenbeweisen.
    \end{alertblock}
    
    \metroset{block=fill}
    \begin{block}{Normal}
        \begin{itemize}
            \item $\overline{\overline{A}} = A$
        \end{itemize}
    \end{block}
    \metroset{block=fill}
    \begin{block}{Etwas schwerer}
        \begin{itemize}
            \item $A\cap B=\overline{(\overline{A}\cup\overline{B})}$
        \end{itemize}
    \end{block}
\end{frame}
}

{\setbeamercolor{palette primary}{bg=ExColor}
\begin{frame}<handout:0>[fragile]{Lösungen}
\onslide Zu zeigen: $A=\overline{\overline{A}}$
    \onslide{
    \begin{align*}
        x\in\overline{\overline{A}}
        &\iff\neg(x\in\overline{A})\\
        &\iff\neg(\neg (x\in A))\\
        &\iff x\in A
    \end{align*}\qed
    }
\end{frame}
}

{\setbeamercolor{palette primary}{bg=ExColor}
\begin{frame}<handout:0>[fragile]{Lösungen}
    \onslide Zu zeigen: $B\cap A=\overline{(\overline{A}\cup\overline{B})}$
    \onslide{
    \begin{align*}
        x \in \overline{(\overline{A} \cup \overline{B})}
        &\iff \neg(x \in \overline{A} \cup \overline{B})
        \\&\iff \neg(x \in \overline{A} \vee x \in \overline{B})
        \\&\iff \neg(\neg(x \in A) \vee \neg(x \in B))
        \\&\iff \neg(\neg(x\in A))\wedge\neg(\neg(x\in B))
        \\&\iff x \in A \wedge x \in B
        \\&\iff x \in A \cap B
        \\&\iff x \in B \cap A
    \end{align*}\qed
    }
\end{frame}
}


%\section{Weitere Mengenbeweise}

\begin{frame}{Weiterer Mengenbeweis}
    Ein weiterer Mengenbeweis...
    \metroset{block=fill}
    \begin{block}{\alert{Aufgabe}}
    $L_1=\{w^{n} \mid n \in \mathbb{N}, w \in \{aaaa\}\}$\\
    $L_2=\{w \mid |w| \equiv 0 \bmod 4, w \in \{a\}^*\}$
    \end{block}
    Zu zeigen:\\
    $L_1 = L_2$\\
    d.h. $(\forall x: x \in L_1 \implies x \in L_2) \wedge (\forall x: x \in L_2 \implies x \in L_1)$

\end{frame}

\begin{frame}{Weiterer Mengenbeweis}
    \metroset{block=fill}
    \only<1>{
    \begin{block}{\alert{$\forall x: x \in L_1 \implies x \in L_2$}}
    Sei $x$ beliebig.\\
    Angenommen, $x \in L_1$.\\
    Es gilt: $w \in \{aaaa\}$. Damit gilt $|w|=4$. Es folgt $|x| = |w^{n}| = |w| * n = 4 * n$ mit $n \in \mathbb{N}$. Daraus folgt $|x| \equiv 0 \bmod 4$. Weiterhin gilt $(aaaa)^n \in \{a\}^*$.\\
    $\leadsto x \in L_2$
    \end{block}
    }
    \only<2>{
    \begin{block}{\alert{$\forall x: x \in L_2 \implies x \in L_1$}}
    Sei $x$ beliebig.\\
    Angenommen, $x \in L_2$.\\
    Es gilt: $|x| \equiv 0 \bmod 4$. \\ Damit gilt $|x| = 4 * n = |w| * n = |w^n|$ mit $w \in\{aaaa\}$ und $n \in \mathbb{N}$. \\ Weiterhin gilt $w \in \{a\}^*$. \\
    $\leadsto x \in L_1$
    \end{block}
    }
    \only<3>{
    Da gezeigt wurde:\\
    \vspace{0.3cm}
    $\forall x: x \in L_1 \implies x \in L_2$\\
    \alert{und}\\
    $\forall x: x \in L_2 \implies x \in L_1$\\
    \vspace{0.3cm}
    \textbf{gilt $L_1 = L_2$.}
    }
\end{frame}

{\setbeamercolor{palette primary}{bg=ExColor}
\begin{frame}[fragile]{Denkpause}
    \begin{alertblock}{Aufgaben}
    Versuche dich an folgenden Mengenbeweisen.
    \end{alertblock}
    
    \metroset{block=fill}
    \begin{block}{Etwas Schwerer}
    $L_1=\{a^{n}b^{m} \mid n<m ,mit \; n,m\in \mathbb{N}\}$\\
    $L_2=\{w \mid |w|_a < |w|_b, w \in \{a,b\}^*\}$\\
    \vspace{0.3cm}
    Zu zeigen: $L_1 \subsetneq L_2$
    \end{block}
    \metroset{block=fill}
    \begin{block}{Schwer}
    $L_1$: $\{w\mid |w|\equiv 0 \bmod 6\}$\\
    $L_2$: $\{w\mid |w|\equiv 0 \bmod 2\}$\\
    $L_3$: $\{w\mid |w|\equiv 0 \bmod 3\}$\\
    \vspace{0.3cm}
    Zu zeigen: $L_1 = L_2 \cap L_3$
    \end{block}
\end{frame}
}

{\setbeamercolor{palette primary}{bg=ExColor}
\begin{frame}[fragile]{Lösung}
    \begin{alertblock}{Aufgaben}
    z.Z. $L_1 \subsetneq L_2$\\
    d.h. $(\forall x: x \in L_1 \implies x \in L_2) \wedge (L_1 \neq L_2)$
    \end{alertblock}
    
    \begin{columns}
    \column{0.6\textwidth}
    \metroset{block=fill}
    \begin{alertblock}{$\forall x: x \in L_1 \implies x \in L_2$}
        Sei x beliebig. Ang. $x \in L_1$.\\
        Es gilt: $x=a^{n}b^{m}$,mit $n,m\in\mathbb{N}$.\\
        Damit gilt $|x|_a=n, |x|_b=m$ mit $n<m$.\\
        Also auch $|x|_a < |x|_b$.\\
        $\leadsto x \in L_2$.
    \end{alertblock}
    
    \column{0.4\textwidth}
    \metroset{block=fill}
    \begin{alertblock}{$L_1 \neq L_2$}
        Beweis durch Angabe eines Gegenbeispiels:\\
        $bba \in L_2$, aber $bba \notin L_1$\\
        Also sind $L_1$ und $L_2$ nicht gleich.
    \end{alertblock}
    \end{columns}
    \qed
\end{frame}
}

{\setbeamercolor{palette primary}{bg=ExColor}
\begin{frame}[fragile]{Lösung}
    \begin{alertblock}{Aufgaben}
    Z.z. $L_1 = L_2 \cap L_3$\\
    d.h. $\forall x: x \in L_1 \iff x \in L_2 \wedge x \in L_3$
    \end{alertblock}
    \only<1>{
    \metroset{block=fill}
    \begin{alertblock}{\glqq$\implies$\grqq}
        Sei x beliebig. Ang. $x \in L_1$.\\
        Dann gilt $|x|\equiv 0 \bmod 6$. Also $\exists k \in \mathbb{N}: |x| = 6k = 3 * 2 * k$.\\
        Somit gilt auch $\exists l \in \mathbb{N}: |x| = 3l$ mit $l=3*k$ und $\exists m \in \mathbb{N}: |x| = 2m$ mit $m=2*k$.\\
        $\leadsto x \in L_2 \wedge x \in L_3$.
    \end{alertblock}
    }
    \only<2>{
    \metroset{block=fill}
    \begin{alertblock}{\glqq$\impliedby$\grqq}
        Sei x beliebig. Ang. $x \in L_2 \wedge x \in L_3$.\\
        Dann $\exists k, l \in \mathbb{N}: |x| = 2 * k, |x| = 3 * l$.\\
        Somit sind 2 und 3 Teil der Primfaktorzerlegung von $|x|$.\\
        Dann gilt $\exists m \in \mathbb{N}: |x| = 6 * m$.\\
        $\leadsto x \in L_1$.
    \end{alertblock}
    }
    \only<3>{
    \begin{alertblock}{\glqq$\iff$\grqq}
     Da $\forall x: x \in L_1 \implies x \in L_2 \wedge x \in L_3$ und $\forall x: x \in L_2 \wedge x \in L_3 \implies x \in L_1$ gilt:\\
    $\forall x: x \in L_1 \iff x \in L_2 \wedge x \in L_3$ \qed
    \end{alertblock}
    }
\end{frame}
}


\section{Wiederholung}
\begin{frame}[fragile]{Das können wir jetzt beantworten}
    \begin{alertblock}{Grundlagen Beweise}
    \begin{itemize}
        \item Was ist ein Beweis?
        \item Was ist die Idee der Kontraposition?
        \item Was ist die Idee des Widerspruchsbeweis?
        \item Wann reicht ein Beispiel als Beweis?
    \end{itemize}
    \end{alertblock}
\end{frame}

\begin{frame}<handout:0>[standout]
  Noch Fragen?
\end{frame}

 \begin{frame}[fragile]{Glossar}
     \small
     \begin{tabular}{p{0.2\textwidth} p{0.25\textwidth} p{0.4\textwidth}}
     \toprule
     Abk.&Bedeutung&Was?!\\
     \midrule
         z.z. & zu zeigen & Was zu beweisen ist\\
         Sei&&bereits bekannte Objekte werden eingeführt und benannt\\
         $\exists$&es gibt ein&\\
         $\exists !$&es gibt genau ein&\\
         x ist genau y&x = y&\emph{genau} wird verwendet bei Äquivalenz\\
         x ist eindeutig&$\exists ! x$&\\
         der, die, das&&bestimmte Artikel weisen auf Eindeutigkeit hin\\
         gdw.&genau dann wenn&Äquivalenz zwischen Aussagen\\
     \bottomrule
     \end{tabular}
 \end{frame}
\begin{frame}[fragile]{Glossar}
    \small
    \begin{tabular}{p{0.33\textwidth} p{0.12\textwidth} p{0.4\textwidth}}
    \toprule
    Abk.&Bedeutung&Was?!\\
    \midrule
        A ist notwendig für B&B$\implies$A&A muss wahr sein,\\
        &&wenn B wahr ist\\
        A ist hinreichend für B&A$\implies$B&B muss wahr sein,\\
        &&wenn A wahr ist\\
        notwendig und hinreichend&A$\iff$B&genau dann wenn\\
    \bottomrule
    \end{tabular}
\end{frame}

\begin{frame}[fragile]{Glossar}
    \small
    \begin{tabular}{p{0.1\textwidth} p{0.33\textwidth} p{0.45\textwidth}}
    \toprule
    Abk.&Bedeutung&Was?!\\
    \midrule
        \OE & ohne Einschränkung & die Allgemeinheit der Aussage wird nicht durch getroffene Aussagen eingeschränkt\\
        o.B.d.A. & ohne Beschränkung der Allgemeinheit & wie \OE\\
        trivial&offensichtlich&Beweisschritte, welche keine weiter Begründung brauchen. (nicht verwenden!)\\
        $\qed$&Mic Drop&Kommt am Ende eines erfolgreichen Beweises\\
        q.e.d&quod erat demonstrandum&Was zu beweisen war\\
    \bottomrule
    \end{tabular}
\end{frame}

 \begin{frame}[fragile]{Cheatsheet}
     \small
     \begin{tabular}{p{0.2\textwidth} p{0.7\textwidth}}
     \toprule
     Gestalt&mögliches Vorgehen\\
     \midrule
         nicht F&Zeige, dass F nicht gilt.\\
         F und G&Zeige F und G in zwei getrennten Beweisen.\\
         F $\implies$ G&Füge F in die Menge der Annahmen hinzu und zeige G.\\
         F oder G&Zeige: nicht F $\implies$ G. \\&(Alternativ zeige: nicht G $\implies$ F.)\\
         F $\iff$ G&Zeige: F $\implies$ G und G $\implies$ F.\\
         $\forall x \in A : F$&Sei x ein beliebiges Element aus A. Zeige dann F.\\
         $\exists x \in A : F$&Sei x ein konkretes Element aus A. Zeige dann F.\\
     \bottomrule
     \end{tabular}
 \end{frame}

\appendix
\begin{frame}<handout:0>[fragile]{Online-Whiteboard}
	\phantom{text}
\end{frame}

\end{document}