% Copyright 2018-2022 FIUS
%
% This file is part of theo-vorkurs-folien.
%
% theo-vorkurs-folien is free software: you can redistribute it and/or modify
% it under the terms of the GNU General Public License as published by
% the Free Software Foundation, either version 3 of the License, or
% (at your option) any later version.
%
% theo-vorkurs-folien is distributed in the hope that it will be useful,
% but WITHOUT ANY WARRANTY; without even the implied warranty of
% MERCHANTABILITY or FITNESS FOR A PARTICULAR PURPOSE.  See the
% GNU General Public License for more details.
%
% You should have received a copy of the GNU General Public License
% along with theo-vorkurs-folien.  If not, see <https://www.gnu.org/licenses/>.

\documentclass[a4paper, 11pt]{scrreprt}
\usepackage{paperandpencil}
\usepackage[ngerman]{babel}
\usepackage[top=2.5cm,bottom=2.5cm,left=2cm,right=2cm]{geometry}
\usepackage{graphicx}
\pagestyle{plain}
\begin{document}
\pagenumbering{arabic}
\setcounter{page}{1}

\begin{figure}[h]
  \centering
  % \includegraphics{school_logo.png} Optional -uncomment to use
\end{figure}
\section*{\LARGE Feedback - Vorkurs Theoretische Informatik 2022}
\vspace{1cm}

\begin{tabular}{p{5.6cm}p{1.8cm}p{1.8cm}p{2.3cm}p{1.8cm}p{1.8cm}p{0.001cm}}
  Ich war überwiegend in Hörsaal... & $\Box$ 38.01 & $\Box$ 38.02 & $\Box$ 38.03/7.03 & $\Box$ 38.04 & $\Box$ 7.02
\end{tabular}

\question*{Welchen Studiengang und im wievielten Semester studierst du?}
\fbox{\parbox[c][1cm]{\textwidth}{\phantom{.}}}

\question*{Wie hast du vom Vorkurs erfahren?}
\fbox{\parbox[c][2.5cm]{\textwidth}{\phantom{.}}}

%\section*{Your job}
%\question{FRAGE}
\vertikalblocktwo{ja}{nein}{
  \blocktexttwo{Ich war auch im Mathe-Vorkurs des MINT-Kolleg (26.09.-07.10.)}
  \blocktexttwo{Ich war auch im Informatik-Vorkurs des MINT-Kolleg (12.09.-16.09. oder 19.09.-23.09.)}
  \blocktexttwo{Ich war auch im Java-Vorkurs von FIUS (04.10.-07.10.)}
}\\

\vertikalblockfour{stimme zu}{stimme eher zu}{stimme eher nicht zu}{stimme nicht zu}{
  \blocktextfour{Der Vorkurs hat mir weiter geholfen.}
  \blocktextfour{Der Vorkurs hat mir Spaß gemacht.}
  \blocktextfour{Der Vorkurs war gut organisiert.}
  \blocktextfour{Ich konnte im Vorkurs Kontakte knüpfen.}
}

\subsection*{Vortrag}
\vertikalblockfour{stimme zu}{stimme eher zu}{stimme eher nicht zu}{stimme nicht zu}{
  \blocktextfour{Die Vortragenden haben den Inhalt verständlich erklärt.}
  \blocktextfour{Der Vortrag wurde professionell durchgeführt.}
  \blocktextfour{Die zusätzlichen Ausführungen der Vortragenden waren hilfreich.}
  \blocktextfour{Die Vortragenden wirkten sympathisch.}
}

\subsection*{Übungsaufgaben}
\vertikalblockfour{stimme zu}{stimme eher zu}{stimme eher nicht zu}{stimme nicht zu}{
  \blocktextfour{Die Übungsaufgaben hatten einen Mehrwert für den Vorkurs.}
  \blocktextfour{Ich habe die Übungsaufgaben verstanden.}
  \blocktextfour{Bei Fragen habe ich schnell Hilfe bekommen.}
  \blocktextfour{Bei Fragen konnte mir gut weitergeholfen werden.}
  \blocktextfour{Ich konnte mich mit den anderen Teilnehmern austauschen.}
}

\subsection*{Zeiteinteilung}
\vertikalblockthree{zu schnell}{angemessen}{zu langsam}{
  \blocktextthree{Das Tempo des Vorkurs war...}
}
\vertikalblockthree{zu kurz}{angemessen}{zu lang}{
  \blocktextthree{Die Länge der Vorträge war...}
  \blocktextthree{Die Länge der Übungsphasen war...}
}

\subsection*{Foliensatz}
\vertikalblockfour{stimme zu}{stimme eher zu}{stimme eher nicht zu}{stimme nicht zu}{
  \blocktextfour{Die Folien waren verständlich.}
  \blocktextfour{Ich werde mir die Folien nach dem Vortrag nochmal anschauen.}
  \blocktextfour{Die zusätzlichen Ausführungen der Vortragenden waren hilfreich.}
}
\vertikalblockfour{täglich}{fast immer}{selten}{nie}{
  \blocktextfour{Ich habe mir das Handout vorher angeschaut.}
}

\subsection*{Themen}

\vertikalblockthree{teilweise neu}{größtenteils neu}{alle\\neu}{
  \blocktextthree{Die Themen waren mir...}
}

\question*{Wie gut hast du folgende Themenfelder verstanden?}
\vertikalblockfour{vollständig}{eher ja}{eher nein}{gar nicht}{
  \blocktextfour{Formale Sprachen}
  \blocktextfour{Aussagenlogik}
  \blocktextfour{Beweistechniken}
  \blocktextfour{Grammatiken}
  \blocktextfour{Automaten}
  \blocktextfour{Reguläre Ausdrücke}
}


\question*{Für welche Themen hättest du dir mehr Erklärung/Zeit gewünscht und für welche weniger?}
\fbox{\parbox[c][4cm]{\textwidth}{\phantom{.}}}

% \subsection*{Hygiene}
% \vertikalblockfour{stimme zu}{stimme eher zu}{stimme eher nicht zu}{stimme nicht zu}{
%   \blocktextfour{Die geltenden Hygieneregeln wurden verständlich erklärt.}
%   \blocktextfour{Für die Einhalung der Regeln wurde gesorgt.}
% }

\question*{	Warum hast du dich für die Variante in Präsenz entschieden?}
\fbox{\parbox[c][2cm]{\textwidth}{\phantom{.}}}

\question*{Was hat dir besonders gut gefallen?}
\fbox{\parbox[c][4cm]{\textwidth}{\phantom{.}}}

\question*{Was hat dir nicht so gut gefallen?}
\fbox{\parbox[c][4cm]{\textwidth}{\phantom{.}}}

\question*{Welche Verbesserungsvorschläge hast du für uns?}
\fbox{\parbox[c][4cm]{\textwidth}{\phantom{.}}}

\question*{Was ich sonst noch sagen wollte...}
\fbox{\parbox[c][4cm]{\textwidth}{\phantom{.}}}

\end{document}
