% Copyright 2018-2022 FIUS
%
% This file is part of theo-vorkurs-folien.
%
% theo-vorkurs-folien is free software: you can redistribute it and/or modify
% it under the terms of the GNU General Public License as published by
% the Free Software Foundation, either version 3 of the License, or
% (at your option) any later version.
%
% theo-vorkurs-folien is distributed in the hope that it will be useful,
% but WITHOUT ANY WARRANTY; without even the implied warranty of
% MERCHANTABILITY or FITNESS FOR A PARTICULAR PURPOSE.  See the
% GNU General Public License for more details.
%
% You should have received a copy of the GNU General Public License
% along with theo-vorkurs-folien.  If not, see <https://www.gnu.org/licenses/>.

\begin{frame}[fragile]{Wörter in Sprachen}
Wir können inzwischen Sprachen in Mengenschreibweise darstellen.\\Es gibt aber auch weitere Möglichkeiten Sprachen zu definieren.\\
\vspace{0.3cm}
Wir können Regeln formulieren mit denen wir alle Wörter einer Sprache schrittweise erzeugen können.
\end{frame}

\begin{frame}[fragile]{Beispiel Worterzeugung}
    \small{Wir betrachten L = \{$ww^R \mid w \in \{a, b\}^n, n>0, n\in \mathbb{N}$\}\\
    Hier ist z.B. $\alert<1>{w}w^R$ = \alert<1>{$ababb$}$bbaba\in L$.}\\
    \begin{enumerate}
    \item <2-> 
            \alert<2,5>{Wir beginnen mit einer Variablen $S$.}
    \item <3-> 
            \alert<3>{Wir formulieren Regeln um $S$ umzuwandeln:}
            \alert<4>{\onslide<4->{
            \begin{align*}\alert<6,8>{S \rightarrow aSa}&\text{ oder }\alert<7,9>{S \rightarrow bSb}\\\text{oder }S \rightarrow aa &\text{ oder }\alert<10>{S \rightarrow bb}\end{align*}}}\vspace{-0.3in}
    \item <5->
            \alert<5>{Damit können wir jetzt Wörter aus der Sprache beschreiben:}\\
            z.B.: \alert<6>{$a$}\alert<7>{$b$}\alert<8>{$a$}\alert<9>{$b$}\alert<10>{$bb$}\alert<9>{$b$}\alert<8>{$a$}\alert<7>{$b$}\alert<6>{$a$} 
            $\leadsto$ 
            \only<5>{\alert<5>{$S$}}\only<6>{\alert<6>{$aSa$}}\only<7>{a\alert<7>{$bSb$}$a$}\only<8>{$ab$\alert<8>{$aSa$}$ba$}\only<9>{$aba$\alert<9>{$bSb$}$aba$}\only<10->{$abab$\alert<10>{$bb$}$baba$}
    \item <11> \alert<11>{Wir nennen diese Umformungsregeln Produktionsregeln.}
    \end{enumerate}
    \vspace{1cm}
    \footnotesize{$w^R$ ist w rückwärts}
\end{frame}

\subsubsection{Produktionsregeln}
\begin{frame}{Produktionsregeln}
    \begin{alertblock}{Einschränkungen}
    \begin{itemize}
        \item \alert{\emph{Nichtterminale}} werden meist durch Großbuchstaben repräsentiert und müssen durch Produktionsregeln abgeändert werden.
        \item \alert{\emph{Terminale}} werden meist durch Kleinbuchstaben repräsentiert und sollten \emph{nicht} durch weitere Produktionsregeln abgeändert werden.
        \item Mehrere Symbole können auf einen Schlag überführt werden. Dabei sollten die Terminale nicht entfernt oder umsortiert werden.\\
        z.B. $AB \rightarrow CD$ ist erlaubt.\\
        Auch $abAB \rightarrow BbAa$, aber das gehört sich nicht.
    \end{itemize}
    \end{alertblock}
\end{frame}

\begin{frame}{Weitere Beispiele für Produktionen}
    \begin{alertblock}{Aufgaben}
    Gesucht: Produktionsregeln für die folgenden Sprachen.
    \end{alertblock}
    \metroset{block=fill}
    \begin{exampleblock}{$L_1 = \{a\}^*$}
    $\onslide<2->{P=\{S \rightarrow aS \only<3->{\mid \emptyWord\}}}$
    \end{exampleblock}
    \only<4->{
    \begin{exampleblock}{$L_2 = \{a, b\}^*$}
    $\onslide<5->{P=\{S \rightarrow  aS \only<6->{\mid bS}\only<7->{\mid \emptyWord\}}}$
    \end{exampleblock}
	}
    \only<8->{
	\begin{exampleblock}{$L_3 = \{a^nbbc^m \mid n, m\in\mathbb{N}\}$}
	$\begin{aligned}
		\onslide<9->{P=\{ S & \rightarrow  ABC,}\\
		\onslide<10->{A & \rightarrow aA \mid \emptyWord,} \\
		\onslide<11->{B & \rightarrow bb,} \\
		\onslide<12->{C & \rightarrow cC \mid \emptyWord \} }			
	\end{aligned}$
	\end{exampleblock}
	}
\end{frame}

{\setbeamercolor{palette primary}{bg=ExColor}
\begin{frame}{Denkpause}
    \begin{alertblock}{Aufgaben}
    Findet Produktionsregeln für die folgenden Sprachen.
    \end{alertblock}
    \metroset{block=fill}
    \begin{block}{Normal}
    \begin{itemize}
        \item $L_1 = \{a^{2n} \mid n\in\mathbb{N}\}$
        \item $L_2 = \{a^nb^nc^m \mid n, m\in\mathbb{N}\}$
        \item $L_3 = \{uv \mid u\in\{a,b\}^\ast, v\in\{c,d\}\}$
        \item $L_4 = \{w \mid |w| = 3, w\in \{a,b,c\}^*\}$
    \end{itemize}
    \end{block}
    \begin{block}{Etwas Schwerer}
    \begin{itemize}
        \item $L_5 = \{a^n \mid n \equiv 1 \mod 3\}$
        \item $L_6 = \{w \mid |w|_a = 3, |w|_b = 1, w\in \{a,b,c\}^*\}$
        \item $L_7 = \{uv \mid u\in\{\text{\Rewind, \MoveUp, \Forward, \MoveDown}\}^\ast, v\in\{\text{\Stopsign}\}\}$
        \item $L_8 = \{w\mid |w|=2, w \in \{a, b\}\}$
    \end{itemize}
    \end{block}
\end{frame}
}

{\setbeamercolor{palette primary}{bg=ExColor}
\begin{frame}<handout:0>{Lösungen}
Alle Lösungen sind Beispiellösungen, es sind auch andere möglich.
    \begin{itemize}
        \item<1-> \alert<1>{$P_1 = \{S\rightarrow aaS\;|\;\emptyWord$\}}
        \item<2-> \alert<2>{$P_2 = \{S\rightarrow AB$, $A\rightarrow aAb \;|\; ab\; |\;\emptyWord$, $B\rightarrow cB \;|\; \emptyWord$\}}
        \item<3-> \alert<3>{$P_3 = \{S\rightarrow UV$, $U\rightarrow aU \;|\; bU \; |\; \emptyWord$, $V\rightarrow c \;|\; d$\}}
        \item<4-> \alert<4>{$P_4 = \{S\rightarrow XXX$, $X\rightarrow a \;|\; b \;|\; c$\}}
        \item<5-> \alert<5>{$P_5 = \{S\rightarrow a \;|\; aaaS\}$}
        \item<6-> \alert<6>{$P_6 = \{S\rightarrow AAAB$, $AB\rightarrow BA$, 
        $A\rightarrow cA \;|\; Ac \;|\; a$, 
        $B\rightarrow cB \;|\; Bc \;|\; b$\}}
        \item<7-> \alert<7>{$P_7 = \{S\rightarrow U\text{\Stopsign} \;|\; \text{\Stopsign}$, $U\rightarrow \text{\Rewind} U \;|\; \text{\MoveUp} U \;|\; \text{\Forward} U \;|\; \text{\MoveDown} U \;|\;\emptyWord$\}}
        \item<8-> \alert<8>{$P_8 = \{\} \leadsto$ Wir brauchen keine Produktionsregeln!}
    \end{itemize}
\end{frame}
}  
