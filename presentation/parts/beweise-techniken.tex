% Copyright 2018-2022 FIUS
%
% This file is part of theo-vorkurs-folien.
%
% theo-vorkurs-folien is free software: you can redistribute it and/or modify
% it under the terms of the GNU General Public License as published by
% the Free Software Foundation, either version 3 of the License, or
% (at your option) any later version.
%
% theo-vorkurs-folien is distributed in the hope that it will be useful,
% but WITHOUT ANY WARRANTY; without even the implied warranty of
% MERCHANTABILITY or FITNESS FOR A PARTICULAR PURPOSE.  See the
% GNU General Public License for more details.
%
% You should have received a copy of the GNU General Public License
% along with theo-vorkurs-folien.  If not, see <https://www.gnu.org/licenses/>.

\subsubsection{Beweistechnik: Direkter Beweis}
\begin{frame}[fragile]{Direkter Beweis}
    \begin{alertblock}{Zeige $A\implies B$ direkt}
    Setze $A$ voraus und folgere dann schrittweise $B$.\\
    Durch jede korrekte Folgerung, vergrößert sich die Menge der Aussagen, die wir weiterverwenden können.
    \end{alertblock}
    \metroset{block=fill}
    \begin{exampleblock}{Beispiel}
    Z.z.: \alert<2|handout:0>{$\forall a\in\mathbb{Z}$}$\ :$ \alert<3|handout:0>{$a$ ist gerade} $\implies$ \alert<6|handout:0>{$a^2$ gerade.}
    \begin{enumerate}
        \item \alert<2|handout:0>{Sei $a\in\mathbb{Z}$ beliebig.}
        \item \alert<3|handout:0>{Angenommen, $a$ ist gerade.}
        \item \alert<4|handout:0>{$\implies \exists n\in\mathbb{Z} : a = 2n$}
        \item \alert<5|handout:0>{$\implies a^2 = (2n)^2 = 4n^2 = 2 \cdot 2n^2$}
        \item \alert<6|handout:0>{$\implies a^2$ ist gerade}\qed\;
    \end{enumerate}
    \end{exampleblock}
    \small{\emph{\alert<4|handout:0>{Anmerkung:}} Zahl $n\in\mathbb{Z}$ heißt gerade, wenn es ein $k\in\mathbb{Z}$ gibt mit $n=2k$.}
\end{frame}

\subsubsection{Beweistechnik: Kontraposition}
\begin{frame}[fragile]{Beweis durch Kontraposition}
    \begin{alertblock}{Zeige $A\implies B$, indem man stattdessen $\neg B \implies\neg A$ zeigt.}
    \end{alertblock}
    \metroset{block=fill}
    \begin{exampleblock}{Beispiel}
    Z.z.: \alert<7|handout:0>{\alert<1|handout:0>{$\forall n\in\mathbb{N}$:} $n^2$ gerade $\implies$\alert<2|handout:0>{ $n$ gerade}} \\
    \qquad bzw. \alert<1|handout:0>{$\forall n\in\mathbb{N}$:} \alert<2|handout:0>{$\neg(n \text{ gerade})$} $\implies \neg(n^2$ gerade$)$ 
    \begin{enumerate}
        \item\alert<1|handout:0>{Sei $n \in \mathbb{N}$ beliebig.}
        \item\alert<2|handout:0>{Angenommen, $n$ ist \emph{nicht} gerade.}
        \item\alert<3|handout:0>{$\implies n=2k+1$, für ein $k \in \mathbb{Z}$}
        \item $\overset{\alert<4|handout:0>{quadrieren}}{\; \leadsto \;} n^2 = (2k+1)^2 = 4k^2+4k+1 = 2\alert<5|handout:0>{(2k^2+2k)}+1$
        \item $\implies n^2= \alert<6|handout:0>{2\alert<5|handout:0>{m}+1}$, für $m=2k^2+2k$
        \item $\implies n^2$ ist \alert<6|handout:0>{ungerade}.
        \item Da ($\forall n\in\mathbb{N}$: $n$ ungerade $\implies n^2$ ungerade) gilt, \\
        folgt \alert<7|handout:0>{($\forall n\in\mathbb{N}$: $n^2$ gerade $\implies n$ gerade)}, was zu beweisen war.\qed\;
    \end{enumerate}
    \end{exampleblock}
    \footnotesize{\alert<3,6|handout:0>{\emph{Anmerkung:}} Zahl $n\in\mathbb{Z}$ heißt ungerade, wenn es ein $k\in\mathbb{Z}$ gibt mit $n=2k+1$.}
\end{frame}

\begin{frame}[fragile]{Beweis durch Kontraposition}
%Work in progress
\begin{alertblock}{Wieso dürfen wir das so machen?}
\end{alertblock}
\metroset{block=fill}
\begin{exampleblock}{Beweis}
Z.z.: $(\neg A \implies\neg B) \iff (B \implies A)$
    \begin{flalign*}
        \;(\neg A\implies\neg B) \iff & (\neg (\neg A) \vee \neg B)&\\
        \iff & (A \vee \neg B)&\\
        \iff & (\neg B \vee A)&\\
        \iff & (B \implies A)&\qed\;
    \end{flalign*}
\end{exampleblock}
\small\emph{Erinnerung:} $A\implies B$ kann man auch $\neg A\vee B$ schreiben.
\end{frame}

\Center{Verdauungspause}

\subsubsection{Beweistechnik: Widerspruch}
\begin{frame}[fragile]{Beweis durch Widerspruch}
\small{
    \begin{alertblock}{Zeige, dass $A$ gilt, indem man zeigt dass $\neg A$ falsch ist.}
    %Spezialfall der Kontraposition: $A\text{ ist wahr}\implies A \iff \neg A\implies \text{falsch}$
    \emph{Erinnerung:} Eine Aussage ist entweder wahr oder falsch.\\
    Wenn $\neg A$ falsch ist, muss $A$ wahr sein.
    \end{alertblock}
    \metroset{block=fill}
    \begin{exampleblock}{Beispiel}
        Z. z. $\sqrt{2}$ ist irrational.
        \begin{enumerate}
            \item<1-|handout:1> \alert<1|handout:0>{Ang. $\sqrt{2}$ ist rational.}
            \item<2-|handout:1> \alert<2|handout:0>{Dann $\exists p, q \in \mathbb{Z} : \sqrt{2} = \frac{p}{q}$} $\wedge$ \alert<3,11|handout:0>{$p, q$ sind teilerfremd}
            \only<2,3|handout:0>{\alert<2>{{\\\emph{Anmerkung:}}} $r\in\mathbb{Q}\iff\exists p,q\in\mathbb{Z}:r=\frac{p}{q}$.}
            \only<2,3|handout:0>{\alert<3>{{\\\emph{Anmerkung:}}} $\frac{p}{q}$ kann man immer soweit kürzen, dass $p,q$ teilerfremd sind.}
            \item<4-|handout:1> \only<4|handout:0>{Quadrieren und Umformen:\\}$\leadsto(\sqrt{2})^2 = (\frac{p}{q})^2 \iff 2 = \frac{p^2}{q^2} \iff \alert<5,8|handout:0>{2q^2=p^2}$
            \item<5-|handout:1> \alert<5|handout:0>{$\leadsto p^2$ ist gerade} \alert<6,10|handout:0>{$\leadsto p$ ist gerade}\only<5,6|handout:0>{\alert<5>{{\\\emph{Erinnerung:}}} $\forall n\in\mathbb{Z}:n$ gerade $\iff\exists k\in\mathbb{Z}:2k=n$.}\only<5,6|handout:0>{\alert<6>{{\\\emph{Erinnerung:}}} $\forall n\in\mathbb{N}$: $n^2$ gerade $\implies n$ gerade \\\qquad\qquad\quad(siehe Beispiel Kontraposition)}
            \item<7-|handout:1> \alert<7|handout:0>{Also ist $p^2$ durch 4 teilbar} \alert<8|handout:0>{$\leadsto 2q^2$ ist durch 4 teilbar}\only<7|handout:0>{\\\alert<7|handout:0>{\emph{Herleitung:}} $p^2=p\cdot p\overset{\text{p gerade}}{=\joinrel=\joinrel=\joinrel=}(2k)\cdot(2k)=\alert<7>{4}k^2,$ mit $k\in\mathbb{Z}$}
            \item<9-|handout:1> $\leadsto q^2$ ist gerade \alert<10|handout:0>{$\leadsto q$ ist gerade}
            \item<10-|handout:1> \alert<10|handout:0>{$\leadsto p,q$ nicht teilerfremd} \alert<11|handout:0>{$\leadsto$ Widerspruch}\only<11|handout:1>{\qed\;}
        \end{enumerate}
    \end{exampleblock}
}
\end{frame}