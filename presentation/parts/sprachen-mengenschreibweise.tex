% Copyright 2018-2022 FIUS
%
% This file is part of theo-vorkurs-folien.
%
% theo-vorkurs-folien is free software: you can redistribute it and/or modify
% it under the terms of the GNU General Public License as published by
% the Free Software Foundation, either version 3 of the License, or
% (at your option) any later version.
%
% theo-vorkurs-folien is distributed in the hope that it will be useful,
% but WITHOUT ANY WARRANTY; without even the implied warranty of
% MERCHANTABILITY or FITNESS FOR A PARTICULAR PURPOSE.  See the
% GNU General Public License for more details.
%
% You should have received a copy of the GNU General Public License
% along with theo-vorkurs-folien.  If not, see <https://www.gnu.org/licenses/>.

\begin{frame}[fragile]{Wie sprechen wir das?}
$M = \{2n \alert{\mid} n \in \mathbb{N}\}$ \\

\emph{Die Menge $M$ enthält alle Elemente $2n$, \alert{für die gilt}: n stammt aus der Menge der natürlichen Zahlen.}
\vspace{5pt}
$$
\{0, 2, 4, 6, 8, 10, 12, \dots \}
$$

\metroset{block=fill}
\begin{alertblock}{Achtung}
    In der theoretischen Informatik enthält $\mathbb{N}$ ($\mathbb{N}$ ist die Menge der natürlichen Zahlen) die Zahl 0.
\end{alertblock}

\end{frame}


\begin{frame}[fragile]{Wie schreiben wir das?}
    \begin{itemize}[<+- | alert@+>]
        \item Mengenbeschreibungen können auch Sätze enthalten:
        $M_1 = \{0,2,4,6,8,\dots\} = \{x \mid \text{x ist gerade}\}$\\
        \hspace{4.5mm}$= \{x \mid$ Es gibt eine Zahl $k \in \mathbb{N} : 2k = x\}$\\
        
        \item Mengenbeschreibungen können Formeln sein:
        $M_2 = \{n^2+3 \mid n \in \mathbb{N}\} = \{3, 5, 7, 11, \dots\}$
        
        \item Auch im Beschribungteil können Formeln stehen:
        $M_3 = \{x \mid x = n^2 \text{ und } x\in \mathbb{N}\} = \{2, 4, 8, 16, \dots\}$
        
        \item Es können mehrere Beschreibungen gleichzeitig verwendet werden:
        $M_4 = \{n \mid n \in \mathbb{N}, n > 15, n < 20\} = \{16, 17, 18, 19\}$\\
    
    \end{itemize}
\end{frame}

{\setbeamercolor{palette primary}{bg=ExColor}
\begin{frame}[fragile]{Denkpause}
    \footnotesize
        \begin{alertblock}{Aufgaben}
            Findet Elemente aus den folgenden Mengen
        \end{alertblock}
        \metroset{block=fill}
        \begin{block}{Normal}
            \begin{itemize}
                \item $L_1 = \{a\}$
                \item $L_2 = \{u+v \mid u\in\{1,2\},\;v\in\{7,9\}\}$
                \item $L_3 = \{w \mid |w| = 3\}$
                \item $L_4 = \{2x \mid x\in \mathbb N\}$
            \end{itemize}
        \end{block}
        \begin{block}{Etwas Schwerer}
            \begin{itemize}
                \item $L_5 = \{3^n \mid n \equiv 1 \pmod 3, n\in\mathbb{N}\}$
                \item $L_6 = \{x \mid |x^2| = x^2\}$
                \item $L_7 = \{w \mid w \text{ ist prim}, w > 50\}$
            \end{itemize}
        \end{block}
        \emph{Anmerkung:} $x \equiv y \pmod n \iff n$ teilt $(x-y)$ ohne Rest $\iff x = m \cdot n + y$ mit $x,y,n,m \in \mathbb{Z}$
\end{frame}
}

{\setbeamercolor{palette primary}{bg=ExColor}
\begin{frame}<handout:0>{Lösungen}
  \begin{itemize}[<+- | alert@+>]
        \item 
            $L_1$: Enthält \textbf{nur} das einzelne Wort $a$!
        \item
            $L_2 = \{8, 10, 9, 11\}$\\
            Wort besteht aus zwei Teilen, die addiert werden:\\ $u$ ist entweder $1$ oder $2$, $v$ ist entweder $7$ oder $9$!
        \item
            $L_3 = \{3, -3\}$\\
            Enthält alle Zahlen mit dem Betrag $3$.
        \item $L_4 = \{0, 2, 4, 6, 8, \dots \}$
        \item $L_5 = \{3, 3^4, 3^7, \dots\}$\\
            Enthält Zahlen $3^n$, wobei n durch 3 geteilt den Rest 1 ergibt.
        \item
            $L_6 = \mathbb{R}$\\
            $x^2$ ist immer positiv, also macht der Betrag hier keinen Unterschied.
        \item
            $L_7 = \{53, 59, 61, 67, \dots\}$
    \end{itemize}
\end{frame}
}
