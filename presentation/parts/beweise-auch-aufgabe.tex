% Copyright 2018-2022 FIUS
%
% This file is part of theo-vorkurs-folien.
%
% theo-vorkurs-folien is free software: you can redistribute it and/or modify
% it under the terms of the GNU General Public License as published by
% the Free Software Foundation, either version 3 of the License, or
% (at your option) any later version.
%
% theo-vorkurs-folien is distributed in the hope that it will be useful,
% but WITHOUT ANY WARRANTY; without even the implied warranty of
% MERCHANTABILITY or FITNESS FOR A PARTICULAR PURPOSE.  See the
% GNU General Public License for more details.
%
% You should have received a copy of the GNU General Public License
% along with theo-vorkurs-folien.  If not, see <https://www.gnu.org/licenses/>.

%%%%%%%%%%%%%%%%%%%%%%%%%%%%%%%%%%%%%%%%%%%%%%
%                                         %
% Diese Aufgabe sucht noch ein gutes Zuhause %
% Wo soll sie hin?                           %
%                                         %
%%%%%%%%%%%%%%%%%%%%%%%%%%%%%%%%%%%%%%%%%%%%%%

{\setbeamercolor{palette primary}{bg=ExColor}
\begin{frame}[fragile]{Denkpause}
	\begin{alertblock}{Aufgaben}
		\small{
			\metroset{block=fill}
			\begin{block}{Hier wird $3=0$ gefolgert. Was ist schief gelaufen?}
				$\text{Sei }x\text{ aus }\mathbb{R}$
				\begin{align*}
					\alert{x^2 + x + 1}       & \alert{= 0} \tag{es muss $x\neq0$} \\
					x(x^2 + x + 1)            & = x \cdot 0 \tag{$\cdot x$}        \\
					x^3 + x^2 + x             & = 0                                \\
					x^3 + \alert{x^2 + x + 1} & = 0 + 1\tag{$+ 1$}                 \\
					x^3                       & = 1\tag{$\sqrt[3]{\phantom{x}}$}   \\
					x                         & = 1
				\end{align*}
				Wir setzen unser Ergebnis oben ein und erhalten
				\begin{align*}
					1^2 + 1 + 1= 3 = 0\text{.}
				\end{align*}
			\end{block}
		}
	\end{alertblock}
\end{frame}
}

{\setbeamercolor{palette primary}{bg=ExColor}
\begin{frame}<handout:0>[fragile]{Lösungen}
	\begin{alertblock}{Aufgaben}
		\small{
			\metroset{block=fill}
			\begin{block}{Hier wird $3=0$ gefolgert. Was ist schief gelaufen?}
				Das Polynom $x^2 + x + 1 = 0$ hat keine Nullstellen in den reellen Zahlen.

				Aus der falschen Annahme, dass $x^2 + x + 1 = 0$ kann also nichts Aussagekräftiges mehr folgen.
			\end{block}
			\metroset{block=fill}
			\begin{alertblock}{Erinnerung an Aussagenlogik}
				Aus Falschem folgt Beliebiges.\\
				\begin{table}
					\begin{tabular}{ccc}
						\toprule
						A         & B         & A $\implies$ B \\
						\midrule
						1         & 1         & 1              \\
						1         & 0         & 0              \\
						\alert{0} & \alert{1} & \alert{1}      \\
						\alert{0} & \alert{0} & \alert{1}      \\
						\bottomrule
					\end{tabular}
				\end{table}
			\end{alertblock}
		}
	\end{alertblock}
\end{frame}
}