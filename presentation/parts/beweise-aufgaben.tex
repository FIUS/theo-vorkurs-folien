% Copyright 2018-2022 FIUS
%
% This file is part of theo-vorkurs-folien.
%
% theo-vorkurs-folien is free software: you can redistribute it and/or modify
% it under the terms of the GNU General Public License as published by
% the Free Software Foundation, either version 3 of the License, or
% (at your option) any later version.
%
% theo-vorkurs-folien is distributed in the hope that it will be useful,
% but WITHOUT ANY WARRANTY; without even the implied warranty of
% MERCHANTABILITY or FITNESS FOR A PARTICULAR PURPOSE.  See the
% GNU General Public License for more details.
%
% You should have received a copy of the GNU General Public License
% along with theo-vorkurs-folien.  If not, see <https://www.gnu.org/licenses/>.

%\subsubsection{Aufgaben}

\begin{frame}[fragile]{Modulo}
    \begin{center}
        \Large
        $$
            3+4 \equiv 1 \mod 6
        $$

        \pause
        Das bedeutet so viel wie:

        $$
            (3+4)\% 6 = 1 \% 6
        $$
    \end{center}
\end{frame}

{\setbeamercolor{palette primary}{bg=ExColor}
\begin{frame}[fragile]{Denkpause}
    \begin{alertblock}{Aufgaben}
    Welche Beweistechnik könnte sich für die folgenden Aussagen eignen? Warum?
    \end{alertblock}
    
    \metroset{block=fill}
    \begin{block}{Einfach}
        \begin{itemize}
            \item Alle $6n+1$ für ungerade $n \in \mathbb{N}$ und $n > 0$ sind Primzahlen.
        \end{itemize}
    \end{block}
    \metroset{block=fill}
    \begin{block}{Normal}
    \begin{itemize}
            \item Für jede ganze Zahl $x$ gilt $x\equiv 1\pmod 4 \implies x\equiv 1\pmod 2$
    \end{itemize}
    \end{block}
    \metroset{block=fill}
    \begin{block}{Etwas schwerer}
        \begin{itemize}
            \item Für alle $n \in \mathbb{N}$ mit $n>0$ gilt $n$ teilt $5n + 22!$.
        \end{itemize}
    \end{block}
\end{frame}
}

{\setbeamercolor{palette primary}{bg=ExColor}
\begin{frame}<handout:0>[fragile]{Lösungen}   
    \metroset{block=fill}
    \begin{block}{Einfach}
        Aussage: Alle $6n+1$ für ungerade $n \in \mathbb{N}$ und $n > 0$ sind Primzahlen.
        \begin{enumerate}
            \item<1-> Diese Aussage ist falsch!
            \item<2-> Wir widerlegen die Aussage durch ein Gegenbeispiel:
            \item<3-> Betrachte $n = 9$
            \item<4-> Dann gilt $6n+1 = 55$ mit Faktorisierung $55 = 5 \cdot 11$
            \item<5-> $55$ ist also insbesondere keine Primzahl \qed
        \end{enumerate}
    \end{block}
\end{frame}
}

{\setbeamercolor{palette primary}{bg=ExColor}
\begin{frame}<handout:0>[fragile]{Lösungen}   
    \metroset{block=fill}
    \begin{block}{Normal}
        Aussage: Für jede ganze Zahl $x$ gilt $x\equiv 1\pmod 4 \implies x\equiv 1\pmod 2$.
        \begin{enumerate}
            \item<2-> Wir zeigen die Aussage durch einen direkten Beweis 
            \item<3-> Sei $x \equiv 1 \pmod{4}$
            \item<4-> Dann existiert ein $k \in \mathbb{Z}$ mit $x = 4k + 1$
            \item<5-> Insbesondere gilt also $x = 2(2k)+1$
            \item<6-> Somit ist also $x \equiv 1 \pmod{2}$
        \end{enumerate}
    \end{block}
\end{frame}
}

{\setbeamercolor{palette primary}{bg=ExColor}
\begin{frame}<handout:0>[fragile]{Lösungen}   
    \metroset{block=fill}
    \begin{block}{Etwas schwerer}
        Aussage: Für alle $n \in \mathbb{N}$ gilt $n$ teilt $5n + 22!$.
        \begin{enumerate}
            \item<2-> Diese Aussage ist ebenso falsch!
            \item<3-> Hierfür kombinieren wir ein Gegenbeispiel mit einem Widerspruchsbeweis:
            \item<4-> Betrachte zunächst $n = 23$
            \item<5-> Zunächst ist klar, dass $23 \mid 5\cdot23$
            \item<6-> Angenommen, $23 \mid (5\cdot23 + 22!)$, dann teilt $23$ auch $(5\cdot23+22!)-5\cdot23 = 22!$
            \item<7-> $22!$ besitzt jedoch nur Primfaktoren kleiner $22$
            \item<8-> Durch die Eindeutigkeit der Primfaktorzerlegung ist das ein Widerspruch
        \end{enumerate}
    \end{block}
\end{frame}
}