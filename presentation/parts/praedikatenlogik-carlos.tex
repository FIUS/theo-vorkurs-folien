\subsection{Generisch}

\begin{frame}{Ganz Generisch}
    \begin{itemize}[<+->]
        \item Eigentlich haben wir kein Universum, keine Prädikate und keine Funktionen vorgegeben
        \item Gegeben ist nur eine Prädikatenlogische Formel
        \item Wir müssen bestimmen ob diese irgendwie erfüllbar ist
    \end{itemize}
\end{frame}

\begin{frame}{Beispiel}
    \note{Dieses Beispiel erweitert das Universum immer um ein Element und geht ein paar Möglichkeiten durch das Prädikat anzuwenden.
        Ein Pfeil von Knoten $X$ auf $Y$ im Graphen bedeutet, dass das $P(X, Y)$ wahr ist, ein fehlender Pfeil, dasss es falsch ist.
        \par
        Dieses Beispiel wurde aus der Ergänzung für Theo II 2018 von Carlos Carmino übernommen
        https://fmi.uni-stuttgart.de/ti/teaching/s18/eti2/ (Termin 1)
    }
    \begin{center}
        \begin{tikzpicture}[overlay]
            \only<1>{\node[] at (0cm,-5cm){\tiny Dieses Beispiel wurde aus der Ergänzung für Theo II 2018 von Carlos Carmino übernommen
                    \url{https://fmi.uni-stuttgart.de/ti/teaching/s18/eti2/}}};

            \only<4,8>{
                \node[] (a) at (-2cm, 0){\textcolor{green}{JA}};
                \draw[->] (a) to (-2.4cm, -1cm);
            }
            \only<3,5,6,7>{
                \node[] (a) at (-2cm, 0){\textcolor{red}{NEIN}};
                \draw[->] (a) to (-2.4cm, -1cm);
            }

            \only<3,5,6,7,8>{
                \node[] (b) at (0,0) {\textcolor{green}{JA}};
                \draw[->] (b) to (0, -1cm);
            }
            \only<4>{
                \node[] (b) at (0,0) {\textcolor{red}{NEIN}};
                \draw[->] (b) to (0, -1cm);
            }

            \only<3,5,6,7,8>{
                \node[] (c) at (2cm, 0) {\textcolor{green}{JA}};
                \draw[->] (c) to (2.4cm, -1cm);
            }
            \only<4>{
                \node[] (c) at (2cm, 0) {\textcolor{red}{NEIN}};
                \draw[->] (c) to (2.4cm, -1cm);
            }
        \end{tikzpicture}
    \end{center}

    $$
        (\forall X \exists Y: P(X,Y)) \wedge (\forall X: \lnot P(X,X)) \wedge (\exists X \forall Y : \lnot P(Y,X))
    $$

    \begin{center}
        \begin{tikzpicture}[
                base/.style={draw, circle, minimum size = .3cm}
            ]
            \only<2->{\node[base] (n0) at (0,0){0};}
            \only<4>{\draw[thick,->] (n0.90) arc (0:270:3mm);}

            \only<5->{\node[base, below left = of n0] (n1) {1};}

            \only<7->{\node[base, below right = of n0] (n2) {2};}


            \only<6,7,8>{\draw[thick,->] (n0) to (n1);}
            \only<7,8>{\draw[thick,->, bend right] (n1) to (n0);}
            \only<8>{\draw[thick,->, bend right] (n2) to (n0);}
        \end{tikzpicture}

        %

        \only<1>{$U= \emptyset$ müssen wir nicht testen, da das Universum nicht leer sein darf}
        \only<2-4>{Was ist mit dem Universum, das ein Element hat. Nennen wir das Element mal $0$}
        \only<5-6>{Lasst es uns mal mit $U= \{0,1\}$ probieren}
        \only<7->{Lasst es uns mal mit $U= \{0,1,2\}$ probieren}
    \end{center}

\end{frame}

\begin{frame}{Formal aufschreiben}
    \alert{Achtung:} Wir müssen das Ergebnis formal aufschreiben!

    \begin{align*}
        U                                   & = \{0,1,2\}                                                      \\
        I(P)                                & = \{(0,1), (1,0), (2,1)\}                                        \\
        \onslide<3->{\textcolor{gray}{I(f)} & = \textcolor{gray}{\{0 \mapsto 1, 1 \mapsto 2, 2 \mapsto 0\} } } \\
        \onslide<4->{I(a)                   & = 0}
    \end{align*}

    \begin{tikzpicture}[overlay]
        \only<2>{
            \node[] (a) at (7cm,3.5cm) {Das Universum};
            \node[] (b) at (5,1.1) {alle Kombinationen für die das Prädikat zu einer wahren Aussage wird};
            \draw[->] (a) to (5.62cm, 2.9cm);
            \draw[->] (b) to (5.3cm, 2cm);
        }
        \only{
            \node[] (b) at (5,-.4) {Wenn man eine Funktion hat, wird diese so angegeben};
            \node[] at (5,-.9) {(Haben wir im Beispiel nicht)};
            \draw[->] (b) to (5.3cm, 1.5cm);
        }<3>
        \only{
            \node[] (b) at (5,-.4) {Wenn man eine Konstante hat, wird sie so angegeben};
            \node[] at (5,-.9) {(Haben wir im Beispiel nicht)};
            \draw[->] (b) to (4.3cm, .8cm);
        }<4>
    \end{tikzpicture}


\end{frame}

{\setbeamercolor{palette primary}{bg=ExColor}
\begin{frame}{Denkpause}
    Finde Interpretationen, für die die Aussagen stimmen
    \metroset{block=fill}
    \begin{block}{Normal}
        \begin{itemize}
            \item $\forall x: {P_1(x,f_1(x))}$
            \item $\forall x\ \forall y : P_2(x,y)\land \forall x : \lnot P_2(a_2,x)$ mit $a_2$ Konstante
            \item $\forall x : \left(P_{3a}(x,f_3(x))\land P_{3b}(f_3(x),x)\right)$ mit $U=\mathbb{N}$ %Lösung: f: Nachfolge, P: <, Q: >
        \end{itemize}
    \end{block}
    \begin{block}{Schwer}
        \begin{itemize}
            \item $\left( \forall x\ \forall y\ \forall z : \left(P_4(x,y) \land P_4(y,z)\right)\implies P_4(x,z)\right) \land
                      \left( \forall x\ \forall y : P_4(x,y) \implies P_4(y,x) \right) \land
                      \left( \forall x : \lnot P_4(a_4, y) \land \lnot P_4(y, a_4) \right)$
                  mit $a_4$ Konstante
            \item $\left(\forall x\ \exists\ y \forall\ z \in U\setminus\{y\} : P_5(x,y)\land\lnot P_5(x,z)\right) \land
                      \left(\forall x,y\ \exists z_0,\dots,z_n : P_5(x,z_0) \land \left(\bigwedge_{i=0}^{n-1} P_5(z_i,z_{i+1})\right) \land P_5(z_n,y)\right)$
        \end{itemize}
    \end{block}
\end{frame}

\begin{frame}<handout:0>{Lösungen}
    \textit{Je Beispiele, es gibt weitere Lösungen}
    \begin{itemize}[<+- | alert@+>]
        \item $U=\{0\}$ \only<1>{\\}\only<2->{; }
              $I_1(P_1) = \{(0,0)\}$ \only<1>{\\}
              $I_2(f_1) = \{0\mapsto 0\}$
        \item Es gibt keine Interpretation, die die Aussage erfüllt, da $P_2$ für jede $x,y$ Kombination wahr sein soll aber trotzdem Kombinationen mit einem $a_2$ geben soll für die $P_2$ falsch ist.
        \item $U=\mathbb{N}$ (geg.) \only<3>{\\}\only<4->{; }
              $I_3(P_{3a}) = \{(a,b)\mid a < b\}$ \only<3>{\\}\only<4->{; }
              $I_3(P_{3b}) = \{(a,b)\mid a > b\}$ \only<3>{\\}\only<4->{; }
              $I_3(f_3) = \{n\mapsto n+1 \mid n\in \mathbb{N}\}$
        \item $U=\text{Menge aller Aussagen}$ \only<4>{\\}\only<5->{; }
              $I(P_4)=\{(a,b)\mid a \land b ist wahr\}$ \only<4>{\\}\only<5->{; }
              $I(a_4): \text{falsch}$
    \end{itemize}

\end{frame}

\begin{frame}<handout:0>{Lösungen}
    \textit{Je Beispiele, es gibt weitere Lösungen}
    \par
    \begin{tikzpicture}[
            base/.style={draw, circle, minimum size = .3cm}
        ]
        \node[base] (n0) at (0,0){0};
        \node[base, right = of n0] (n1) {1};
        \only<1>{
            \draw[thick,->,bend right] (n0) to (n1);
            \draw[thick,->,bend right] (n1) to (n0);
        }

        \only<2>{
            \node[base, below = of n1] (n2) {2};
            \node[base, below = of n0] (n3) {3};


            \draw[thick,->] (n0) to (n1);
            \draw[thick,->] (n1) to (n2);
            \draw[thick,->] (n2) to (n3);
            \draw[thick,->] (n3) to (n0);
        }
    \end{tikzpicture}
    \par
    \only<1>{
        $U=\{0,1\}$\\
        $I_3(P_5) = \{(0,1),(1,0)\}$
    }
    \only<2>{
        $U=\{0,1,2,3\}$\\
        $I_3(P_5) = \{(0,1),(1,2),(2,3),(3,0)\}$
    }
\end{frame}

\begin{frame}{Zum Knobeln}
    \small
    \metroset{block=fill}
    \begin{block}{}
        \begin{multicols}{2}
            \begin{itemize}
                \item $F_1 = \forall X \exists Y: P_0(X,Y)$
                \item $F_2 = \forall X: \overline{P_0(X,X)}$
                \item $F_3 = \exists Y \forall X: \overline{P_0(X,Y)}$
                \item $F_4 = \forall X: \overline{P_0(X,f(x))}$
                \item {\scriptsize$F_5 = \forall X \forall Y: (P_0(X,f(Y)) \implies P_0(Y,X))$}
                \item $F_6 = P(a, f(f(a)))$
                \item $F_7 = \forall X (\exists Y: P_0(Y,X)\implies P_1(X))$
                \item $F_8 = \overline{P_1(f(f(b)))}$
                \item $F_9 = P(f(c),c)$
            \end{itemize}
        \end{multicols}
        Gib eine Interpretation für $F = \bigwedge_{i=1}^9 F_i$ an
    \end{block}
\end{frame}
}