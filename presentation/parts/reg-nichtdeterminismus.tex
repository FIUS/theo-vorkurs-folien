% Copyright 2018-2022 FIUS
%
% This file is part of theo-vorkurs-folien.
%
% theo-vorkurs-folien is free software: you can redistribute it and/or modify
% it under the terms of the GNU General Public License as published by
% the Free Software Foundation, either version 3 of the License, or
% (at your option) any later version.
%
% theo-vorkurs-folien is distributed in the hope that it will be useful,
% but WITHOUT ANY WARRANTY; without even the implied warranty of
% MERCHANTABILITY or FITNESS FOR A PARTICULAR PURPOSE.  See the
% GNU General Public License for more details.
%
% You should have received a copy of the GNU General Public License
% along with theo-vorkurs-folien.  If not, see <https://www.gnu.org/licenses/>.

\begin{frame}{Beispiel NEA vs DEA}
    $L=\{uabaav \;|\; u, v \in \{a,b\}^* \}$
    \onslide<2->{
        \begin{alertblock}{NEA:}
            \begin{center}
                \begin{tikzpicture}[->,>=stealth',shorten >=1pt,auto,node distance=2cm,
                        semithick]
                    \node [initial,state]     (0)              {$q_0$};
                    \node [state]             (1) [right of=0] {$q_1$};
                    \node [state]             (2) [right of=1] {$q_2$};
                    \node [state]             (3) [right of=2] {$q_3$};
                    \node [state, accepting]  (4) [right of=3] {$q_4$};

                    \path   (0) edge[loop above]    node[above]     {$a,b$}   (0)
                    edge                node[above]     {$a$}     (1)
                    (1) edge                node[above]     {$b$}     (2)
                    (2) edge                node[above]     {$a$}     (3)
                    (3) edge                node[above]     {$a$}     (4)
                    (4) edge[loop above]    node[above]     {$a,b$}   (4);
                \end{tikzpicture}
            \end{center}
        \end{alertblock}
    }
    \onslide<3->{
        \begin{alertblock}{DEA:}
            \begin{center}
                \begin{tikzpicture}[->,>=stealth',shorten >=1pt,auto,node distance=2cm,
                        semithick]
                    \node [initial,state]     (0)              {$q_0$};
                    \node [state]             (1) [right of=0] {$q_1$};
                    \node [state]             (2) [right of=1] {$q_2$};
                    \node [state]             (3) [right of=2] {$q_3$};
                    \node [state, accepting]  (4) [right of=3] {$q_4$};

                    \path   (0) edge[loop above]    node[above]     {$b$}     (0)
                    edge                node[above]     {$a$}     (1)
                    (1) edge                node[above]     {$b$}     (2)
                    edge[loop above]    node[above]     {$a$}     (1)
                    (2) edge                node[above]     {$a$}     (3)
                    edge[bend left]     node[below]     {$b$}     (0)
                    (3) edge                node[above]     {$a$}     (4)
                    edge[bend left]     node[below]     {$b$}     (2)
                    (4) edge[loop above]    node[above]     {$a,b$}   (4);
                \end{tikzpicture}
            \end{center}
        \end{alertblock}
    }
\end{frame}

%%%%%%%%%%%%%%%%%%%%%%%%%%%%%%%%%%%%%%%%%%%%%%%%%%%%
% Ist das Kunst oder kann das weg/viel weiter vor? %
%%%%%%%%%%%%%%%%%%%%%%%%%%%%%%%%%%%%%%%%%%%%%%%%%%%%
%
%{\setbeamercolor{palette primary}{bg=ExColor}
%\begin{frame}{Scheinklausuraufgabe WS17/18}
% \begin{alertblock} {Gegeben sei folgender NEA M:}
%     Nenne Wörter die erkannt werden.\\
%     \begin{tikzpicture}[->,>=stealth',shorten >=1pt,auto,node distance=2cm,
%             semithick]
%         \node [initial,state]   (0)              {$q_0$};
%         \node [state,accepting]           (1) [right of=0] {$q_1$};
%         \node [state] (2) [below of=0] {$q_2$};
%         \path   (0) edge               node {$b$} (1)
%         edge               node {$a,b$} (2)
%         (1) edge               node {$a$} (2)
%         edge [loop above]  node {$b$} (1)
%         (2) edge [loop below]  node {$a,b$} (2);
%     \end{tikzpicture}
%     \onslide<2|handout:0>{\alert{\textbf{$\leadsto b, bb, bbb, bbbb, bbbbb,\dots$}}}
% \end{alertblock}
%\end{frame}
%}

{\setbeamercolor{palette primary}{bg=ExColor}
\begin{frame}{Denkpause}
    \footnotesize
    \begin{alertblock}{Aufgaben}
    \end{alertblock}
    \metroset{block=fill}
    \begin{block}{Normal}
        Finde einen passenden DEA oder NEA für die folgenden Sprachen:
        \begin{itemize}
            \item $L_1 = \{w \in \{a,b\}^* \mid |w|_a \equiv 1 \pmod{3}\}$
            \item $L_2 = \{ua\mid u \in \{a,b\}^*\}$
        \end{itemize}
    \end{block}
    \begin{block}{Etwas Schwerer}
        Finde einen passenden NEA mit \alert{einem} Startzustand für die folgende Sprache
        \begin{itemize}
            \item $L_3 = L_1 \cup L_2$
        \end{itemize}
    \end{block}
    \begin{block}{Sehr Schwer}
        Finde für $L_3$ einen passenden DEA
    \end{block}
\end{frame}
}

{\setbeamercolor{palette primary}{bg=ExColor}
\begin{frame}<handout:0>{Lösung: Normal}
    \only<1>{
        \begin{alertblock}{$L_1 = \{w \in \{a,b\}^* \mid |w|_a \equiv 1 \pmod{3}\}$}
            \begin{center}
                \begin{tikzpicture}[->,>=stealth',shorten >=1pt,auto,node distance=2cm,
                        semithick]
                    \node [initial,state]       (0)                 {$q_0$};
                    \node [state, accepting]    (1) [right of=0]    {$q_1$};
                    \node [state]               (2) [right of=1]    {$q_2$};

                    \path   (0) edge                node {$a$}    (1)
                    edge[loop above]    node {$b$}    (0)
                    (1) edge                node {$a$}    (2)
                    edge[loop above]    node {$b$}    (1)
                    (2) edge[bend left]     node[below] {$a$}    (0)
                    edge[loop above]    node {$b$}    (2);
                \end{tikzpicture}
            \end{center}
        \end{alertblock}}
    \only<2->{
        \begin{alertblock}{$L_2 = \{ua \mid u \in \{a,b\}^*\}$}
            \vspace*{5mm}
            \only<2>{NEA:}
            \only<3>{DEA:}
            \begin{center}
                \begin{tikzpicture}[->,>=stealth',shorten >=1pt,auto,node distance=2cm,
                        semithick]
                    \node [initial,state]   (0)              {$p_0$};
                    \node [state,accepting]           (1) [right of=0] {$p_1$};

                    \path<2>    (0) edge                node {$a$}    (1)
                    edge[loop above]    node {$a,b$}  (0);
                    \path<3>    (0) edge[bend right]    node[above] {$a$}     (1)
                    edge[loop above]    node        {$b$}     (0)
                    (1) edge[loop above]     node        {$a$}     (1)
                    edge[bend right]    node[above] {$b$}     (0);
                \end{tikzpicture}
            \end{center}
        \end{alertblock}}
\end{frame}
}

{\setbeamercolor{palette primary}{bg=ExColor}
\begin{frame}<handout:0>{Lösung: Etwas Schwieriger}
    \begin{alertblock}{$L_3 = L_1 \cup L_2$}
        \begin{tikzpicture}[->,>=stealth',shorten >=1pt,auto,node distance=2cm,
                semithick]
            \node [initial,state]   (s)                     {$s$};
            \node [state, accepting](q0) [above right= 1cm and 2cm of s] {$q_0$};
            \node [state]           (q1) [right of=q0]      {$q_1$};
            \node [state]           (q2) [right of=q1]      {$q_2$};

            \node [state]           (p0) [below right= 1cm and 2cm of s] {$p_0$};
            \node [state, accepting]           (p1) [right of=p0]      {$p_1$};

            \path       (s) edge                node[above] {$a$}     (q0)
            edge                node[below] {$a,b$}   (p0)
            edge[bend left=10]     node[above,pos=0.75] {$a$}     (p1)
            (q0)edge                node {$a$}    (q1)
            edge[loop above]    node {$b$}    (q0)
            (q1)edge                node {$a$}    (q2)
            edge[loop above]    node {$b$}    (q1)
            (q2)edge[bend left=23]     node[below,pos=0.75] {$a$}    (q0)
            edge[loop above]    node {$b$}    (q2)
            (p0)edge                node {$a$}    (p1)
            edge[loop below]    node {$a,b$}  (p0)
            (s) edge [bend right=17] node [below right] {$b$} (q2);

        \end{tikzpicture}
    \end{alertblock}
\end{frame}
}

%
% TODO: Notizen: wie ist die Animation zu verstehen?
%
{\setbeamercolor{palette primary}{bg=ExColor}
\begin{frame}<handout:0>{Lösung: Schwer}
    \begin{alertblock}{$L_3 = L_1 \cup L_2$}
        \begin{center}
            \begin{tikzpicture}[->,>=stealth',shorten >=1pt,auto,node distance=2cm,
                    semithick]
                \node [initial,state]       (00)                  {$q_0, p_0$};
                \node [state, accepting]    (01) [right of=00]    {$q_1, p_0$};
                \node [state]               (02) [right of=01]    {$q_2, p_0$};
                \node<1> [state]               (10) [below = 2cm of 00]    {$q_0, p_1$};
                \node<1> [state, accepting]    (11) [right of=10]    {$q_1, p_1$};
                \node<1> [state]               (12) [right of=11]    {$q_2, p_1$};
                \node<2> [state, accepting, orange]                 (10) [below = 2cm of 00]    {$q_0, p_1$};
                \node<2> [state, accepting, orange]                 (11) [right of=10]    {$q_1, p_1$};
                \node<2> [state, accepting, orange]                 (12) [right of=11]    {$q_2, p_1$};
                \node<3-> [state, accepting]                 (10) [below = 2cm of 00]    {$q_0, p_1$};
                \node<3-> [state, accepting]                 (11) [right of=10]    {$q_1, p_1$};
                \node<3-> [state, accepting]                 (12) [right of=11]    {$q_2, p_1$};

                \path<1>   (00)    edge                node        {$a$}    (01)
                edge[loop above]    node        {$b$}    (00)
                (01)    edge                node        {$a$}    (02)
                edge[loop above]    node        {$b$}    (01)
                (02)    edge[bend left]     node[below] {$a$}    (00)
                edge[loop above]    node        {$b$}    (02)
                (10)    edge                node        {$a$}    (11)
                edge[loop above]    node        {$b$}    (10)
                (11)    edge                node        {$a$}    (12)
                edge[loop above]    node        {$b$}    (11)
                (12)    edge[bend left]     node[below] {$a$}    (10)
                edge[loop above]    node        {$b$}    (12);

                \path<2>    (00)    edge[loop above]    node        {$b$}    (00)
                (01)    edge[loop above]    node        {$b$}    (01)
                (02)    edge[loop above]    node        {$b$}    (02)
                (10)    edge                node        {$a$}    (11)
                (11)    edge                node        {$a$}    (12)
                (12)    edge[bend left]     node[below] {$a$}    (10);

                \path<3>    (00)    edge[orange]        node[right] {$a$}    (11)
                edge[loop above]    node        {$b$}    (00)
                (01)    edge[orange]        node[right] {$a$}    (12)
                edge[loop above]    node        {$b$}    (01)
                (02)    edge[orange]        node[right] {$a$}    (10)
                edge[loop above]    node        {$b$}    (02)
                (10)    edge                node        {$a$}    (11)
                (11)    edge                node        {$a$}    (12)
                (12)    edge[bend left]     node[below] {$a$}    (10);

                \path<4>    (00)    edge                node[right] {$a$}    (11)
                edge[loop above]    node        {$b$}    (00)
                (01)    edge                node[right] {$a$}    (12)
                edge[loop above]    node        {$b$}    (01)
                (02)    edge                node[right] {$a$}    (10)
                edge[loop above]    node        {$b$}    (02)
                (10)    edge                node        {$a$}    (11)
                edge[orange]        node        {$b$}    (00)
                (11)    edge                node        {$a$}    (12)
                edge[orange]        node        {$b$}    (01)
                (12)    edge[bend left]     node[below] {$a$}    (10)
                edge[orange]        node        {$b$}    (02);
            \end{tikzpicture}
        \end{center}
    \end{alertblock}
\end{frame}
}
