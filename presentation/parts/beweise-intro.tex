% Copyright 2018-2022 FIUS
%
% This file is part of theo-vorkurs-folien.
%
% theo-vorkurs-folien is free software: you can redistribute it and/or modify
% it under the terms of the GNU General Public License as published by
% the Free Software Foundation, either version 3 of the License, or
% (at your option) any later version.
%
% theo-vorkurs-folien is distributed in the hope that it will be useful,
% but WITHOUT ANY WARRANTY; without even the implied warranty of
% MERCHANTABILITY or FITNESS FOR A PARTICULAR PURPOSE.  See the
% GNU General Public License for more details.
%
% You should have received a copy of the GNU General Public License
% along with theo-vorkurs-folien.  If not, see <https://www.gnu.org/licenses/>.

\begin{frame}{Einführung}
\begin{alertblock}{Was ist ein Beweis?}
\begin{itemize}
        \item lückenlose Folge von logischen Schlüssen,\\welche zur zu beweisenden Behauptung führen
        \item nicht nur einleuchtend, sondern zweifelsfrei korrekt
    \end{itemize}
\end{alertblock}
\onslide<2>{
\begin{alertblock}{Warum Beweisen?}
    \begin{itemize}
            \item Aussage basierend auf Fakten und nicht subjektiv belegen
            \item Bestätigung von Aussagen für weitere Nutzung
        \end{itemize}
    \end{alertblock}
}
\end{frame}

\subsubsection{Beweisbeispiel: Transitivität der Teilmenge}
\begin{frame}[fragile]{Beispielbeweis}
\begin{exampleblock}{Zu zeigen: Teilmengen sind transitiv.}
\begin{enumerate}
    \item<1->\alert<1|handout:0>{
        \only<1|handout:0>{zu zeigen: }\onslide<2->{z.z. }$A\subseteq B\wedge B\subseteq C \alert<3|handout:0>{\implies\text{}}A\subseteq C$
        }
    \item<2->\alert<2|handout:0>{
        \only<2|handout:0>{Umschreiben:\\}
        $\iff $\alert<4,5|handout:0>{$($\alert<9|handout:0>{$($\alert<6|handout:0>{$\forall x$}$\ : x \in A \implies x \in B)$}$ \wedge $\alert<10|handout:0>{$($\alert<6|handout:0>{$\forall x$}$\ : x \in B \implies x \in C)$}$)$}\\
        \qquad\alert<3|handout:0>{$\implies$}$\;($\alert<6|handout:0>{$\forall x$}$\ :\ $\alert<7|handout:0>{$x \in A$}$ \implies x \in C)$
        }
    \item<3->\alert<3|handout:0>{
        \only<3|handout:0>{\emph{Implikation}\\
        linke Seite wahr $\implies$ rechte Seite muss wahr sein.\\
        linke Seite falsch $\implies$ beliebiges kann folgen\\
        $\implies$ uns interessiert also nur der Fall \emph{links ist wahr}}
        \alert<4>{\only<4,5|handout:0>{Wir machen uns also \emph{\textquotedbl die linke Seite ist wahr\textquotedbl} zur Voraussetzung}\alert<5>{\only<5|handout:0>{:\\Angenommen, $A \subseteq B \wedge B \subseteq C$ gilt.}}}
        \onslide<6->{Ang., $A \subseteq B \wedge B \subseteq C$.}
        }
    \item<6->\alert<6|handout:0>{
        \only<6|handout:0>{Jetzt geht der Beweis richtig los.\\Wähle beliebiges $x$, um Allgemeinheit zu wahren\dots\\}
        \onslide<6->{Sei $x$ beliebig}\alert<7>{\onslide<7-|handout:0>{ mit \alert<9>{$x\in A$.}}}
        }
    \item<8->\alert<8-9|handout:0>{
        \only<8|handout:0>{Wir können jetzt unsere Voraussetzungen ausnutzen,\\um $x\in C$ zu folgern.}
        \onslide<9->$\implies x\in B$
        \alert<10|handout:0>{\onslide<10->$\implies x\in C$}
        \onslide<11>\qed
    }
  \end{enumerate}
\end{exampleblock}
\end{frame}
