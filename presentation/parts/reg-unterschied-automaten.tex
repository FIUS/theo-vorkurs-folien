\begin{frame}[fragile]{Verschiedene endliche Automaten}
    \begin{alertblock}{NEA}
        Beim Lesen eines Wortes ist es manchmal unklar, welchen Übergang der Automat nehmen soll.\\
        $\leadsto$ Der Automat ist \emph{\alert{nichtdeterministisch}}.
    \end{alertblock}
    \begin{alertblock}{DEA}
        Wir können unsere Möglichkeiten so einschränken, dass bei jedem Zeichen eindeutig ist, welcher Übergang genutzt wird.\\
        $\leadsto$ Der Automat ist \alert{\emph{deterministisch}}.
    \end{alertblock}
\end{frame}

\begin{frame}[fragile]{Unterschied NEA und DEA}
    Bildlich kann man den Unterschied zwischen NEA und DEA an verschiedenen Merkmalen erkennen:
    \begin{alertblock}{Mehrere Möglichkeiten bei Übergängen:}
        \begin{center}
            \begin{tikzpicture}[->, >=stealth, shorten >=1pt, semithick]
                \node[state]         (1)                            {$q_i$};
                \node[state]         (2) [above right=0.7cm and 2cm of 1]   {$q_j$};
                \node[state]         (3) [below right=0.7cm and 2cm of 1]   {$q_k$};

                \path
                (1) edge                node[above] {$a$}  (2)
                edge                node[below] {$a,b$}(3);
            \end{tikzpicture}
        \end{center}
        Bei einem DEA wäre nur ein Weg mit einem \alert{$a$} von dem Zustand $q_i$ aus erlaubt!
    \end{alertblock}
\end{frame}

\begin{frame}[fragile]{Unterschied NEA und DEA}
    Bildlich kann man den Unterschied zwischen NEA und DEA an verschiedenen Merkmalen erkennen:
    \begin{alertblock}{Nicht definierte Übergänge}
        \begin{center}
            \begin{tikzpicture}[->, >=stealth, shorten >=1pt, semithick]
                \node[state]         (1)                {$q_i$};
                \node[state]         (2) [right = 2cm of 1]   {$q_j$};

                \path
                (1) edge[bend left]  node[above] {$a$}  (2)
                (2) edge[bend left]  node[below] {$a,b$}(1);
            \end{tikzpicture}
        \end{center}
        Bei einem DEA müsste ein Übergang mit \alert{$b$} für den Zustand $q_i$ definiert werden!
    \end{alertblock}
\end{frame}

\begin{frame}[fragile]{Unterschied NEA und DEA}
    Bildlich kann man den Unterschied zwischen NEA und DEA an verschiedenen Merkmalen erkennen:
    \begin{alertblock}{Mehrere Startzustände}
        \begin{center}
            \begin{tikzpicture}[->, >=stealth, shorten >=1pt, semithick]
                \node[state, initial]         (1)                      {$q_0$};
                \node[state, initial]         (2) [below = 1.5cm of 1]   {$\tilde{q}_0$};
                \node[state]                  (3) [below right= 0.75cm and 2cm of 1]   {$q_1$};

                \path
                (1) edge node[above] {$a$}  (3)
                edge[loop above] node[above] {$b$} (1)
                (2) edge node[below] {$a,b$}(3);
            \end{tikzpicture}
        \end{center}
        Ein DEA muss \alert{genau einen} Startzustand besitzen!
    \end{alertblock}
\end{frame}

\begin{frame}{Wann wird ein Wort akzeptiert?}
    Ein Wort $w \in \Sigma^*$ wird von einem NEA akzeptiert, wenn es \alert{mindestens einen} akzeptierenden Pfad gibt.
    \begin{alertblock}{Beispiel}
        Welche Zustände können erreicht werden, wenn $w = aaba$ eingelesen wird?\\
        \begin{tikzpicture}[->, >=stealth, shorten >=1pt, semithick]
        	%normal
            \node<1,11,14,16,18,19->	[state, initial]   (0)                    {$q_0$};
            \node<1-10,12-13,15,17,19->	[state, accepting] (1) [right = 2cm of 0] {$q_1$};
            \node<1-18>	[state]            (2) [right = 2cm of 1] {$q_2$};
            %alerted
            \node<2-10,12-13,15,17> [state, initial,orange]   (0)                    {$q_0$};
            \node<11,14,16,18>	[state, accepting,orange] (1) [right = 2cm of 0] {$q_1$};
            \node<19->		[state,orange]            (2) [right = 2cm of 1] {$q_2$};
            
            %normal
			\path<1,11,16>
			(0) edge[loop above]    node[above] {$a,b$}   (0)
			(0) edge[bend right]    node[below] {$a$}     (1)
			(1) edge[bend right]    node[above] {$b$}     (0)
			(1) edge                node[above] {$a$}     (2)
			(2) edge[loop above]    node[above] {$a,b$}   (2);
			%alerted
            \path<2-9,12>
            (0) edge[loop above,orange] node[above] {$a,b$}   (0)
            (0) edge[bend right]   		node[below] {$a$}     (1)
            (1) edge[bend right]    	node[above] {$b$}     (0)
            (1) edge                	node[above] {$a$}     (2)
            (2) edge[loop above]    	node[above] {$a,b$}   (2);
            \path<10,13,15,17>
            (0) edge[loop above]    	node[above] {$a,b$}   (0)
            (0) edge[bend right,orange] node[below] {$a$}     (1)
            (1) edge[bend right]    	node[above] {$b$}     (0)
            (1) edge                	node[above] {$a$}     (2)
            (2) edge[loop above]    	node[above] {$a,b$}   (2);
            \path<14>
            (0) edge[loop above]    	node[above] {$a,b$}   (0)
            (0) edge[bend right] 		node[below] {$a$}     (1)
            (1) edge[bend right,orange] node[above] {$b$}     (0)
            (1) edge                	node[above] {$a$}     (2)
            (2) edge[loop above]    	node[above] {$a,b$}   (2);
            \path<18>      
            (0) edge[loop above] node[above] {$a,b$}   (0)
            (0) edge[bend right] node[below] {$a$}     (1)
            (1) edge[bend right] node[above] {$b$}     (0)
            (1) edge[orange]     node[above] {$a$}     (2)
            (2) edge[loop above] node[above] {$a,b$}   (2);
            \path<19->
            (0) edge[loop above]    	node[above] {$a,b$}   (0)
            (0) edge[bend right] 		node[below] {$a$}     (1)
            (1) edge[bend right] 		node[above] {$b$}     (0)
            (1) edge                	node[above] {$a$}     (2)
            (2) edge[loop above,orange] node[above] {$a,b$}   (2);
        \end{tikzpicture}
        \vspace*{-5mm}
        \begin{center}
            \begin{enumerate}
                \item[i)]<2-> 
                	\alert<2>{$q_0 \rightarrow$}
                	\alert<3>{$q_0 \rightarrow$} 
                	\alert<4>{$q_0 \rightarrow$} 
                	\alert<5>{$q_0 \rightarrow$} 
                	\alert<6>{$q_0$} 
                \item[ii)]<7-> 
                	\alert<7>{$q_0 \rightarrow$} 
	                \alert<8>{$q_0 \rightarrow$} 
	                \alert<9>{$q_0 \rightarrow$} 
	                \alert<10>{$q_0 \rightarrow$} 
	                \alert<11>{$q_1$} (Endzustand erreicht)
                \item[iii)]<12-> 
                	\alert<12>{$q_0 \rightarrow$} 
	                \alert<13>{$q_0 \rightarrow$} 
	                \alert<14>{$q_1 \rightarrow$} 
	                \alert<15>{$q_0 \rightarrow$} 
	              	\alert<16>{$q_1$} (Endzustand erreicht)
                \item[iv)]<17-> 
                	\alert<17>{$q_0 \rightarrow$} 
	                \alert<18>{$q_1 \rightarrow$} 
	                \alert<19>{$q_2 \rightarrow$} 
	                \alert<20>{$q_2 \rightarrow$} 
	                \alert<21>{$q_2$} 
            \end{enumerate}
        \end{center}
    \end{alertblock}
\end{frame}