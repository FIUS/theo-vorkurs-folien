% Copyright 2018-2022 FIUS
%
% This file is part of theo-vorkurs-folien.
%
% theo-vorkurs-folien is free software: you can redistribute it and/or modify
% it under the terms of the GNU General Public License as published by
% the Free Software Foundation, either version 3 of the License, or
% (at your option) any later version.
%
% theo-vorkurs-folien is distributed in the hope that it will be useful,
% but WITHOUT ANY WARRANTY; without even the implied warranty of
% MERCHANTABILITY or FITNESS FOR A PARTICULAR PURPOSE.  See the
% GNU General Public License for more details.
%
% You should have received a copy of the GNU General Public License
% along with theo-vorkurs-folien.  If not, see <https://www.gnu.org/licenses/>.

\begin{frame}[fragile]{Mengen}
    \begin{itemize}

        \item<1-|handout:1>
              Was ist eine \alert<1,2>{Menge}?
        \item<2->
              \only<2|handout:1>{
                  \vspace*{0.5cm}
                  Eine Menge
                  \begin{itemize}
                      \item ist eine \alert{Sammlung von Zeugs}
                      \item ist unsortiert
                      \item enthält keine Duplikate
                      \item wird mit geschweiften Klammern notiert
                  \end{itemize}

                  \metroset{block=fill}

                  \begin{exampleblock}{Beispiel}
                      $\mathbb{N} = \{0, 1, 2, 3, \dots \}$ = Menge der Natürlichen Zahlen\\
                      Studierende = \{Georg, Tim, Triin, Seb, Babett, $\dots$\}\\
                      $\{1,2\} = \{2,1\} = \{1,1,2,1,1,1\}$\\
                      $\emptyset = \{\} =$ leere Menge
                  \end{exampleblock}}
              \uncover<3-|handout:2>{
                  Was ist ein \alert<3,4>{Element}?}
        \item<4->
              \only<4|handout:2>{
                  \vspace*{0.5cm}
                  Ein Element ist ein \alert{Ding aus einer Menge}.\\

                  \metroset{block=fill}

                  \begin{exampleblock}{Beispiel}
                      $\mathbf{1}$ ist ein Element der \textbf{Natürlichen Zahlen}\\
                      $\mathbf{1} \in \mathbb{N}$\\
                      \vspace*{0.5cm}
                      \textbf{Tim} ist ein Element aus der Menge der \textbf{Studierenden}\\
                      \textbf{Tim} $\in$ \textbf{Studierende}\\
                      \vspace*{0.5cm}
                      $\mathbf{a}$ ist in der Menge $\mathbf{\{u, v, w\}}$ nicht enthalten\\
                      $\mathbf{a} \notin \mathbf{\{u, v, w\}}$
                  \end{exampleblock}
              }
              \uncover<5-|handout:3>{
                  Was ist eine \alert<5,6>{Teilmenge}?
              }
        \item<6|handout:3>
              \vspace*{0.5cm}
              Eine Teilmenge ist eine \alert{spezielle Auswahl} von Elementen einer Menge.\\

              \metroset{block=fill}

              \begin{exampleblock}{Beispiel}
                  $\{1, 2, 3\}$ ist eine Teilmenge der Natürlichen Zahlen\\
                  $\{1,2,3\} \subseteq \mathbb{N}$\\
                  \vspace*{0.5cm}
                  \{\textbf{Julian}\} ist eine Teilmenge der \textbf{Studierenden}\\
                  \{\textbf{Julian}\} $\subseteq$ \textbf{Studierende}
              \end{exampleblock}

    \end{itemize}
\end{frame}
\note[itemize]{
    \item Note Natürliche Zahlen: 0 ist \textbf{in Theo} Teil von $\mathbb{N}$, also auch hier im Vorkurs (in Mathe nicht)
}

\begin{frame}{Echte Teilmengen}
    \begin{itemize}
        \item ist $\{A,B\}$ eine Teilmenge von $\{A,B\}$?
              \pause
        \item \alert{Ja!}
              \pause
        \item Aber keine \alert{echte} Teilmenge
    \end{itemize}
    \pause
    \metroset{block=fill}
    \begin{exampleblock}{Beispiel}
        $N \subseteq M$: $N$ ist eine Teilmenge von $M$ aber darf auch $M$ sein.\\
        $N \subset M$: $N$ ist eine Teilmenge von $M$ und muss mindestens 1 Element weniger enthalten.
        $N \subsetneq M$ bedeutet das selbe wie $N \subset M$, aber ist etwas expliziter.
    \end{exampleblock}
\end{frame}

{\setbeamercolor{palette primary}{bg=ExColor}
\begin{frame}[fragile]{Denkpause}
    \begin{alertblock}{Aufgaben}
        Nenne jeweils 5 Elemente der folgenden Mengen:
    \end{alertblock}

    \metroset{block=fill}
    \begin{block}{Normal}
        \begin{itemize}
            \item $\{a, b, c, d, e, f, g, h, i\}$
            \item $\{0, 2, 4, 8, 16, 32, 64, 128, 256, 512\}$
            \item $\mathbb N$
        \end{itemize}
    \end{block}
    Nenne 5 Teilmengen die keine gegenseitigen Teilmengen sind
    \metroset{block=fill}
    \begin{block}{Etwas schwerer}
        \begin{itemize}
            \item $M = \{\text{\WashCotton}, \text{\NoWash}, \text{\IroningII}, \text{\Tumbler}, \text{\SpecialForty} \}$
            \item $M = \{\text{Lisa}, \text{Tobi}, \text{Fabian}, \text{Linus}\}$
        \end{itemize}
    \end{block}
\end{frame}
}

{\setbeamercolor{palette primary}{bg=ExColor}
\begin{frame}<handout:0>{Lösungen}
    Mögliche Lösungen sind \dots
    \begin{itemize}[<+- | alert@+>]
        \item $a, b, c, d, e \in \{a, b, c, d, e, f, g, h, i\}$
        \item $0, 2, 4, 8, 16,\in \{0, 2, 4, 8, 16, 32, 64, 128, 256, 512\}$
        \item $0,1,2,3,4 \in \mathbb N$
        \item \{\WashCotton\}, \{\NoWash\}, \{\IroningII\}, \{\Tumbler\}, \{\SpecialForty \}
        \item \{Lisa, Tobi\}, \{Lisa, Fabian\}, \{Lisa, Linus\}, \{Tobi, Fabian\}, \{Linus, Tobi\}
    \end{itemize}
\end{frame}
}
