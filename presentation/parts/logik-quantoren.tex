% Copyright 2018-2022 FIUS
%
% This file is part of theo-vorkurs-folien.
%
% theo-vorkurs-folien is free software: you can redistribute it and/or modify
% it under the terms of the GNU General Public License as published by
% the Free Software Foundation, either version 3 of the License, or
% (at your option) any later version.
%
% theo-vorkurs-folien is distributed in the hope that it will be useful,
% but WITHOUT ANY WARRANTY; without even the implied warranty of
% MERCHANTABILITY or FITNESS FOR A PARTICULAR PURPOSE.  See the
% GNU General Public License for more details.
%
% You should have received a copy of the GNU General Public License
% along with theo-vorkurs-folien.  If not, see <https://www.gnu.org/licenses/>.

\begin{frame}[fragile]{Quantoren}
    Oft wollen wir Aussagen nicht nur für ein Element, sondern für viele Elemente treffen.
    \metroset{block=fill}
    \begin{exampleblock}{Beispiel}
        $A_1$: Für die Zahl 5 gilt: Sie hat einen Nachfolger\\
        \emph{Allgemeiner:}\\
        $A_2$: Für jede natürliche Zahl $n$ gilt: $n$ hat einen Nachfolger
    \end{exampleblock}
    \begin{exampleblock}{Beispiel}
        $A_3$: Für die Zahl 5 gilt: Sie ist eine Primzahl\\
        \emph{Allgemeiner:}\\
        $A_4$: Es gibt eine natürliche Zahl $n$, so dass gilt: $n$ ist eine Primzahl
    \end{exampleblock}
\end{frame}

\begin{frame}[fragile]{Quantoren}
    Mithilfe von \textbf{Quantoren} vereinfachen wir uns die Schreibweise dieser Aussagen.\\
    \vspace{0.5cm}
    Quantor \alert{$\forall$}: Die Aussage gilt für alle Elemente.\\
    \metroset{block=fill}
    \begin{exampleblock}{Beispiel}
        $A_1$: $\forall k \in \mathbb{N}:$ $2k$ ist gerade
    \end{exampleblock}
    Quantor \alert{$\exists$}: Die Aussage gilt für mindestens ein Element.\\
    \metroset{block=fill}
    \begin{exampleblock}{Beispiel}
        $A_2$: $\exists k \in \mathbb{N}:$ $k$ ist Primzahl
    \end{exampleblock}
\end{frame}

\begin{frame}[fragile]{Quantoren}
    In einer Aussage können mehrere Quantoren vorkommen.\\
    Wir lesen dann von links nach rechts.
    \metroset{block=fill}
    \begin{exampleblock}{Beispiel}
        $A_1$: $\forall x,y \in \mathbb{N} \exists z \in \mathbb{N}: x+y = z$\\
        Bedeutung: Für zwei beliebige Zahlen $x$ und $y$ aus $\mathbb{N}$ gibt es eine weitere natürliche Zahl $z$, so dass $x+y=z$ gilt.
    \end{exampleblock}
\end{frame}

\begin{frame}[fragile]{Quantoren}
    \alert{Achtung!}\\
    Die Reihenfolge von zwei Quantoren zu vertauschen, kann die Bedeutung einer Aussage deutlich verändern.
    \metroset{block=fill}
    \begin{exampleblock}{Beispiel}
        $x$,$y$ $\in$ Menschen\\
        \textbf{$A_1$: $\forall x \exists y:$ $x$ spricht mit $y$\\
        $A_2$: $\exists x \forall y:$ $x$ spricht mit $y$\\ }
        Was ist der Unterschied zwischen beiden Aussagen?
    \end{exampleblock}
\end{frame}

\begin{frame}{Quantoren}
    \begin{alertblock}{Aufgabe}
      Wir formulieren folgende Aussage mithilfe von Quantoren und den Symbolen der Aussagenlogik (Junktoren).
    \end{alertblock}
    \metroset{block=fill}
    \begin{block}{Beispiel}
    \begin{itemize}
        \item $A_1$: Eine ganze Zahl ist eine natürliche Zahl, wenn sie positiv oder null ist.
    \end{itemize}
    \end{block}
    \begin{block}{Hinführung}
    \begin{itemize}
        \item $A_1$: Für alle ganzen Zahlen $x$ gilt: Wenn $x$ positiv oder null ist, ist $x$ eine natürliche Zahl.
    \end{itemize}
    \end{block}
    \begin{block}{\alert{Lösung}}
    \begin{itemize}
        \item $A_1$: $\forall x \in \mathbb{Z}: x \geq 0 \implies x \in \mathbb{N}$
    \end{itemize}
    \end{block}
\end{frame}

{\setbeamercolor{palette primary}{bg=ExColor}
\begin{frame}{Denkpause}
    \begin{alertblock}{Aufgaben}
      Formuliere folgende Aussagen mithilfe von Quantoren und den Symbolen der Aussagenlogik (Junktoren). 
    \end{alertblock}
    \metroset{block=fill}
    \begin{block}{Normal}
    \begin{itemize}
        \item $A_1$: Die Differenz zweier ganzer Zahlen ist wieder eine ganze Zahl.
    \end{itemize}
    \end{block}
    \begin{block}{Schwer}
    \begin{itemize}
        \item $A_2$: Jede natürliche Zahl lässt sich als Summe von vier Quadratzahlen darstellen.
    \end{itemize}
    \end{block}
    \begin{block}{Da haben selbst wir keinen Bock drauf}
    \begin{itemize}
        \item $A_3$: Eine natürliche Zahl, die von einer von ihr verschiedenen natürlichen Zahl größer als 1 geteilt wird, ist nicht prim.
    \end{itemize}
    \end{block}
\end{frame}
}

{\setbeamercolor{palette primary}{bg=ExColor}
\begin{frame}<handout:0>{Lösungen}
  \begin{itemize}[<+- | alert@+>]
        \item 
            $A_1$: $\forall x,y \in \mathbb{Z}: x-y \in \mathbb{Z}$
        \item
            $A_2$: $\forall x \in \mathbb{N}: \exists a, b, c, d \in \mathbb{N}: x = a^2 + b^2 + c^2 + d^2$
        \item
            $A_3$: $\forall x \in \mathbb{N}: \left(\exists y \in \mathbb{N}: (y>1) \wedge (y \neq x) \wedge (y \mid x)\right) \implies x\ \text{ist keine Primzahl}$.
    \end{itemize}
\end{frame}
}

\begin{frame}[fragile]{Äquivalente Schreibweisen von Mengenoperationen}
	Oft benötigen wir eine Aussagenlogische Äquivalente Bedingung von Mengenoperationen. \emph{Dafür nehmen wir mal die Obermenge $\SigmaStern$}
	\metroset{block=fill}
	\begin{block}{Operationen}
		\begin{itemize}
			\item<1-> \textbf{Teilmenge}: $A$ \alert<1|handout:0>{$\subseteq$} $B$ $\leadsto$ $\forall x \in \SigmaStern : x \in A \implies x \in B$\\
			\item<2-> \textbf{Vereinigung}: $A$ \alert<2|handout:0>{$\cup$} $B$ $\leadsto$ $\forall x \in \SigmaStern : x \in A \alert<2|handout:0>{\cup} B \iff x \in A \vee x \in B$\\
			\item<3-> \textbf{Schnitt}: $A$ \alert<3|handout:0>{$\cap$} $B$ $\leadsto$ $\forall x \in \SigmaStern : x \in A \alert<3|handout:0>{\cap} B \iff x \in A \wedge x \in B$\\
			\item<4-> \textbf{Komplement}: \alert<4|handout:0>{$\overline{A}$} $\leadsto$ $\forall x \in \SigmaStern : x \in \overline{A} \iff x \notin A$
		\end{itemize}
	\end{block}
	
\end{frame}
