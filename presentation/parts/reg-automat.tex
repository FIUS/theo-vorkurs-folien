% Copyright 2018-2022 FIUS
%
% This file is part of theo-vorkurs-folien.
%
% theo-vorkurs-folien is free software: you can redistribute it and/or modify
% it under the terms of the GNU General Public License as published by
% the Free Software Foundation, either version 3 of the License, or
% (at your option) any later version.
%
% theo-vorkurs-folien is distributed in the hope that it will be useful,
% but WITHOUT ANY WARRANTY; without even the implied warranty of
% MERCHANTABILITY or FITNESS FOR A PARTICULAR PURPOSE.  See the
% GNU General Public License for more details.
%
% You should have received a copy of the GNU General Public License
% along with theo-vorkurs-folien.  If not, see <https://www.gnu.org/licenses/>.

\begin{frame}[fragile]{Reguläre Sprachen anders beschreiben}
    Wir können reguläre Sprachen auch graphisch beschreiben.\\
    Dafür nutzen wir \alert{endliche Automaten}.\\
    Ein Automat prüft Wörter und entscheidet, ob sie Teil der Sprache sind oder nicht.\\
    $\leadsto$ Wir nennen das \alert{\emph{akzeptieren}}, bzw. nicht akzeptieren.
    \begin{alertblock}{Funktionsweise}
        \begin{enumerate}
            \item Ein Wort wird in den Automat eingegeben
            \item Wort wird zeichenweise abgearbeitet
            \item Nach jedem Zeichen wird der Automat in einen Zustand überführt, der bestimmt, wie fortgefahren wird
            \item Befindet sich der Automat in einem \emph{Endzustand}, sobald das Wort abgearbeitet wurde, akzeptiert der Automat das Wort.
        \end{enumerate}
    \end{alertblock}
\end{frame}

\subsection{NEA}
\begin{frame}[fragile]{Bestandteile eines endlichen Automaten}
    Der Automat kann als gerichteter Graph notiert werden.\\
    Wir konstruieren ihn aus den folgenden Komponenten:
    \only<1|handout:1>{
        \begin{alertblock}{Startzustand}
            Im Startzustand wird das Wort eingegeben.\\
            \begin{center}
                \begin{tikzpicture}[->,>=stealth',shorten >=1pt,auto,node distance=2cm,semithick]
                    \node[initial,state](q0){$q_0$};
                \end{tikzpicture}
            \end{center}
        \end{alertblock}
    }
    \only<2|handout:1>{
        \begin{alertblock}{Zustandsübergang}
            Wird das Zeichen auf dem Übergang \emph{gelesen}, geht der Automat in den folgenden Zustand über.\\
            \begin{center}
                \begin{tikzpicture}[->,>=stealth',shorten >=1pt,auto,node distance=2cm,semithick]
                    \node[state](qi){$q_i$};
                    \node[state](qj)[right of=qi]{$q_j$};
                    \path (qi) edge node {$a$} (qj);
                \end{tikzpicture}
            \end{center}
        \end{alertblock}
    }
    \only<3|handout:2>{
        \begin{alertblock}{Endzustand}
            Falls sich der Automat in diesem Zustand befindet, und das Wort abgearbeitet ist, wird das Wort akzeptiert.\\
            \begin{center}
                \begin{tikzpicture}[->,>=stealth',shorten >=1pt,auto,node distance=2cm,semithick]
                    \node[accepting,state](qe){$q_E$};
                \end{tikzpicture}
            \end{center}
            \emph{Anmerkung: }Unter Umständen sind mehrere hiervon nötig
        \end{alertblock}
    }
\end{frame}

\begin{frame}[fragile]{Beispiel}
    \begin{exampleblock}{$L=\{axb \mid x\in\{a,b\}^\ast\}$}
        \vspace{0.3cm}
        \begin{tikzpicture}[->,>=stealth',shorten >=1pt,auto,node distance=2cm,
                semithick]
            %\tikzstyle{every state}=[fill=ExColor,draw=none,text=white]

            \node<1,3-> [initial,state]         (1)               {$q_0$};
            \node<2>    [initial,state,orange]  (1)               {$q_0$};
            \node       [state]                 (2) [right of=1]  {$q_1$};
            \node<1-8>  [state,accepting]       (3) [right of=2]  {$q_E$};
            \node<9>    [state,accepting,orange](3) [right of=2]  {$q_E$};

            \path<1,2,9>
            (1) edge                node {$a$}  (2)
            (2) edge [loop above]   node {$a,b$}(2)
            edge                node {$b$}  (3);
            \path<3>
            (1) edge [orange]       node {$a$}  (2)
            (2) edge [loop above]   node {$a,b$}(2)
            edge                node {$b$}  (3);
            \path<4,5,6,7>
            (1) edge                     node {$a$}  (2)
            (2) edge [loop above,orange] node {$a,b$}(2)
            edge                     node {$b$}  (3);
            \path<8>
            (1) edge                node {$a$}  (2)
            (2) edge [loop above]   node {$a,b$}(2)
            edge [orange]       node {$b$}  (3);
        \end{tikzpicture}
    \end{exampleblock}
    \onslide<2->{
        \begin{exampleblock}{Worteingabe:} \alert<3>{$a$}\alert<4>{$a$}\alert<5>{$b$}\alert<6>{$a$}\alert<7>{$b$}\alert<8>{$b$} $\in L$\only<1-8>{?} \only<9>{$\leadsto$ \alert{akzeptiert}
            }
        \end{exampleblock}}
\end{frame}

\begin{frame}[fragile]{Zu Beachten}
Regeln zu Übergängen bei Automaten:
\begin{itemize}
    \item<1-> $\varepsilon$-Übergänge sind \emph{nicht} möglich:\\
    \begin{tikzpicture}[->,>=stealth',shorten >=1pt,auto,node distance=2cm,semithick]
        \node [state] (1) {$q_0$};
        \node [state] (2) [right of= 1] {$q_1$};
        \path
        (1) edge node[draw, strike out] {$\varepsilon$}  (2);
    \end{tikzpicture}
    \item<2-> Mehrere Endzustände sind möglich:\\
    \begin{tikzpicture}[->,>=stealth',shorten >=1pt,auto,node distance=2cm,semithick]
        \node [state] (1) {$q_0$};
        \node [state,accepting] (2) [left of= 1] {$q_1$};
        \node [state,accepting] (3) [right of= 1] {$q_2$};
        \path
        (1) edge node[above] {$a$} (2)
        (1) edge node[above] {$b$} (3);
    \end{tikzpicture}
    \item<3-> Verlassen eines Enzustands ist möglich:
    \begin{tikzpicture}[->,>=stealth',shorten >=1pt,auto,node distance=2cm,semithick]
        \node [state] (1) {$q_0$};
        \node [state,accepting] (2) [right of= 1] {$q_1$};
        \node [state] (3) [right of= 2] {$q_2$};
        \path
        (1) edge node {$a$} (2)
        (2) edge node {$a$} (3);
    \end{tikzpicture}
\end{itemize}
    

\end{frame}

\begin{frame}{Beispiel: Automat}
    Gegeben ist eine Sprache $L$. Gesucht ist ein Automat $M$, der \textbf{genau} die Wörter aus $L$ akzeptiert.
    \vspace{0.5cm}
    % \only<1>{
    % \begin{alertblock}{$L_1 = \{a^{2n} \mid n\in\naturals\}$}
    % \begin{tikzpicture}[->,>=stealth',shorten >=1pt,auto,node distance=2cm,semithick]
    %      \node   [initial,state,accepting]                   (1) {$q_0$};
    %      \node   [state]                     [right of=1]    (2) {$q_1$};
    %      \node   [state, accepting]          [right of=2]    (3) {$q_2$};

    %      \path
    %              (1) edge node {$a$} (2);

    %      \path
    %              (2) edge [bend left] node {$a$} (3);

    %      \path
    %              (3) edge [bend left] node {$a$} (2);
    %  \end{tikzpicture}
    % \end{alertblock}
    % }
    \only<1>{
        \begin{alertblock}{$L_1 = \{uv \mid u\in\{a,b\}^\ast,\;v\in\{c,d\}\}$}
            \begin{tikzpicture}[->,>=stealth',shorten >=1pt,auto,node distance=2cm,semithick]
                \node   [initial,state]                             (1) {$q_0$};
                \node   [state,accepting]           [right of=1]    (2) {$q_1$};

                \path   (1) edge node {$c,d$} (2)
                (1) edge [loop above] node {$a,b$} (1);
            \end{tikzpicture}
        \end{alertblock}
    }

\end{frame}

{\setbeamercolor{palette primary}{bg=ExColor}
\begin{frame}{Denkpause}
    \begin{small}
        \begin{alertblock}{knifflige Aufgabe}
            Wir entwerfen einen Automat zur Aufzugskontrolle.
            Der Aufzug hat folgende Möglichkeiten:\\
            $\Sigma =\{$\begin{footnotesize}
                \framebox{EG$\nearrow$1.OG} , \framebox{EG$\nearrow$2.OG} , \framebox{1.OG$\searrow$EG} , \framebox{1.OG$\nearrow$2.OG} , \framebox{2.OG$\searrow$EG} , \\\qquad\quad\;\framebox{2.OG$\searrow$1.OG} , \framebox{OFF}
            \end{footnotesize}$\}$
            \begin{itemize}
                \item Der Aufzug startet vom Erdgeschoss und darf sich nur aus Stockwerken bewegen, in denen er sich befindet.
                \item Der Aufzug kann nur im Erdgeschoss ausgeschaltet werden. Er kann dann keine Bewegung durchführen.
                \item Der Aufzug muss ausgeschaltet werden.
            \end{itemize}
        \end{alertblock}
        % \begin{footnotesize}
        %  \framebox{EG$\nearrow$2.OG} \framebox{2.OG$\searrow$EG} \framebox{OFF} $\in L$\\
        %  \framebox{2.OG$\searrow$EG} \framebox{OFF} \framebox{EG$\nearrow$1.OG} $\notin L$\\
        % \end{footnotesize}
        \alert{Zeichne einen Automaten an, dessen akzeptierte Sprache genau die Menge der korrekten Abläufe ist.}
    \end{small}
\end{frame}

\begin{frame}<handout:0>{Lösung}
    \begin{center}
        \begin{tikzpicture}[->,>=stealth',shorten >=1pt,auto,node distance=3.2cm,semithick]
            \node [initial,state]   (0)              {$q_0$};
            \node [state]           (1) [above of=0] {$q_1$};
            \node [state]           (2) [above of=1] {$q_2$};
            \node [state,accepting] (E) [right of=0] {$q_E$};

            \path   (0) edge [bend left=10] node [rotate=90,above] {\footnotesize\framebox{EG$\nearrow$1.OG}} (1)
            edge [below]      node {\footnotesize\framebox{OFF}} (E)
            edge [bend left=70] node {\footnotesize\framebox{EG$\nearrow$2.OG}} (2)
            (1) edge [bend left=10] node [rotate=90,above] {\footnotesize\framebox{1.OG$\nearrow$2.OG}} (2)
            edge [bend left=10] node [rotate=90,below] {\footnotesize\framebox{1.OG$\searrow$EG}} (0)
            (2) edge [bend left=10] node [rotate=90,below] {\footnotesize\framebox{2.OG$\searrow$1.OG}}(1)
            edge [bend left=70] node {\footnotesize\framebox{2.OG$\searrow$EG}} (0);
        \end{tikzpicture}
    \end{center}
\end{frame}
}

{\setbeamercolor{palette primary}{bg=ExColor}
\begin{frame}{Denkpause}
    \begin{alertblock}{Aufgaben}
        Finde Automaten, die \textbf{genau} folgende Sprachen erkennen.
    \end{alertblock}
    \metroset{block=fill}
    \begin{block}{Normal}
        \begin{itemize}
            \item $L_1 = \{a^nb^m \mid n, m\in\naturals\}$
            \item $L_2 = \{w \mid |w| = 3, w\in \{a,b,c\}^*\}$
            \item $L_3 = \{uv \mid u\in\{\text{\Rewind, \MoveUp, \Forward, \MoveDown}\}^\ast,\;v\in\{\text{\Stopsign}\}\}$
        \end{itemize}
    \end{block}
    \begin{block}{Etwas Schwerer}
        \begin{itemize}
            \item $L_4 = \{a^n \mid n \equiv 1 \pmod 3\}$
            \item $L_5 = \{w \in \{a,b,c\}^* \mid |w|_a = 3, |w|_b = 1\}$
            \item $L_6=\{w \in \{a,b\}^* \mid |w|_a \equiv |w|_b \pmod 3\}$
        \end{itemize}
    \end{block}
\end{frame}
}



{\setbeamercolor{palette primary}{bg=ExColor}
\begin{frame}<handout:0>{Lösung}
    % \begin{itemize}[<+- | alert@+>]
    %  \item
    \onslide<1->{\alert<1>{
        % \item
        \begin{tikzpicture}[->,>=stealth',shorten >=1pt,auto,node distance=2cm,semithick]
            \node   [initial,state,accepting]                   (1) {$q_0$};
            \node   [state,accepting]           [right of=1]    (2) {$q_1$};

            \path   (1) edge node {$b$} (2)
            (1) edge [loop above] node {$a$} (1);

            \path   (2) edge [loop above] node {$b$} (2);
        \end{tikzpicture}
    }}
    \vspace{0.2cm}
    \onslide<2->{\alert<2>{
        \begin{tikzpicture}[->,>=stealth',shorten >=1pt,auto,node distance=2cm,semithick]
            \node   [initial,state]                             (1) {$q_0$};
            \node   [state]             [right of=1]            (2) {$q_1$};
            \node   [state]             [right of=2]            (3) {$q_2$};
            \node   [state,accepting]   [right of=3]            (4) {$q_3$};

            \path   (1) edge node {$a,b,c$} (2);
            \path   (2) edge node {$a,b,c$} (3);
            \path   (3) edge node {$a,b,c$} (4);
        \end{tikzpicture}
    }}
    \vspace{0.2cm}
    \onslide<3->{\alert<3>{
        \begin{tikzpicture}[->,>=stealth',shorten >=1pt,auto,node distance=2cm,semithick]
            \node   [initial,state]         (1) {$q_0$};
            \node   [state,accepting]   [right of=1]       (2) {$q_1$};

            \path   (1) edge    [loop above]    node    {\text{\Rewind, \MoveUp, \Forward, \MoveDown}} (1)
            (1) edge                    node    {$\text{\Stopsign}$}   (2);
        \end{tikzpicture}}}
    \vspace{0.2cm}
    \onslide<4->{\alert<4>{
        \begin{tikzpicture}[->,>=stealth',shorten >=1pt,auto,node distance=2cm,semithick]
            \node   [initial,state]                             (1) {$q_0$};
            \node   [state,accepting]   [right of=1]            (2) {$q_1$};
            \node   [state]             [right of=2]            (3) {$q_2$};

            \path   (1) edge node {$a$} (2);
            \path   (2) edge node {$a$} (3);
            \path   (3) edge [bend left] node {$a$} (1);
        \end{tikzpicture}
    }}
    % \end{itemize}
\end{frame}
}

{\setbeamercolor{palette primary}{bg=ExColor}
\begin{frame}<handout:0>{Lösung}
    \onslide<1->{\alert<1>{
        \begin{tikzpicture}[->,>=stealth',shorten >=1pt,auto,node distance=1.5cm,semithick]
            \node   [initial,state]                             (1) {$q_0$};
            \node   [state]             [right of=1]            (2) {$q_1$};
            \node   [state]             [right of=2]            (3) {$q_2$};
            \node   [state]             [right of=3]            (4) {$q_3$};
            \node   [state]             [below of=1]             (5) {$q_4$};
            \node   [state]             [below of=2]             (6) {$q_5$};
            \node   [state]             [below of=3]             (7) {$q_6$};
            \node   [state, accepting]  [below of=4]             (8) {$q_7$};

            \path   (1) edge node {$a$} (2)
            (1) edge node {$b$} (5)
            (1) edge [loop above] node {c} (1);

            \path   (2) edge node {$a$} (3)
            (2) edge node {$b$} (6)
            (2) edge [loop above] node {c} (2);

            \path   (3) edge node {$a$} (4)
            (3) edge node {$b$} (7)
            (3) edge [loop above] node {c} (3);

            \path   (4) edge node {$b$} (8)
            (4) edge [loop above] node {c} (4);

            \path   (5) edge node {$a$} (6)
            (5) edge [loop below] node {c} (5);

            \path   (6) edge node {$a$} (7)
            (6) edge [loop below] node {c} (6);

            \path   (7) edge node {$a$} (8)
            (7) edge [loop below] node {c} (7);

            \path   (8) edge [loop below] node {c} (8);
        \end{tikzpicture}
    }}
\end{frame}
}

{\setbeamercolor{palette primary}{bg=ExColor}
\begin{frame}<handout:0>{Lösung}
    \onslide<1->{\alert<1>{
        \begin{tikzpicture}[->,>=stealth',shorten >=1pt,auto,node distance=4cm,semithick]
            \node   [initial,state,accepting]                             (1) {$q_0$};
            \node   [state]             [right of=1]            (2) {$q_1$};
            \node   [state]             [below of=1]            (3) {$q_2$};

            \path   (1) edge [bend right] node {$a$} (2)
            (1) edge [bend right] node {$b$} (3);
            \path   (2) edge node {$a$} (3)
            (2) edge [bend right] node {$b$} (1);
            \path   (3) edge [bend right] node {$a$} (1)
            (3) edge [bend right=60] node {$b$} (2);
        \end{tikzpicture}
    }}
\end{frame}
}
