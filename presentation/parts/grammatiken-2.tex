% Copyright 2018-2024 FIUS
%
% This file is part of theo-vorkurs-folien.
%
% theo-vorkurs-folien is free software: you can redistribute it and/or modify
% it under the terms of the GNU General Public License as published by
% the Free Software Foundation, either version 3 of the License, or
% (at your option) any later version.
%
% theo-vorkurs-folien is distributed in the hope that it will be useful,
% but WITHOUT ANY WARRANTY; without even the implied warranty of
% MERCHANTABILITY or FITNESS FOR A PARTICULAR PURPOSE.  See the
% GNU General Public License for more details.
%
% You should have received a copy of the GNU General Public License
% along with theo-vorkurs-folien.  If not, see <https://www.gnu.org/licenses/>.

\subsubsection{formale Notation}
\begin{frame}[fragile]{Formale Notation}
    Wir beschreiben eine \alert{\emph{Grammatik}} durch ein geordnetes \alert{\emph{Tupel}} $G = (V, \Sigma, P, S)$
    \begin{itemize}
        \item $V$ ist die Menge der verwendeten Nichtterminale
        \item $\Sigma$ die Menge der Terminale bzw. unser Alphabet
        \item $P$ ist die Menge der Produktionsregeln
        \item $S$ ist die Startvariable
    \end{itemize}
    \metroset{block=fill}
    \begin{exampleblock}{Beispiel für  L = \{$ww^R \mid w \in \{a, b\}^n, \; n \geq 1$\}}
        $G = (V,\Sigma,P,S)$ mit\\
        $V = \{S\}$\\
        $\Sigma = \{a,b\}$\\
        $P = \{S \rightarrow aSa, S \rightarrow bSb, S \rightarrow aa, S \rightarrow bb$\}\\
        \qquad bzw. kurz: $P = \{S \rightarrow aSa\ |\ bSb\ |\ aa\ |\ bb$\}
    \end{exampleblock}
\end{frame}

{\setbeamercolor{palette primary}{bg=ExColor}
\begin{frame}{Denkpause}
    \begin{columns}
        \column{0.5\textwidth}
        \begin{alertblock}{Knifflige Aufgabe}
            Totoro will durch das Labyrinth laufen. Er hat folgende Möglichkeiten:\\
            $\Sigma = \{\text{\Rewind}, \text{\MoveUp}, \text{\Forward}, \text{\MoveDown}\}$
            \begin{itemize}
                \item Totoro kann nicht auf ein Feld zurücktreten, von dem er gerade kam
                \item Totoro geht bei jedem Schritt ein Feld in die angegebene Richtung
            \end{itemize}
        \end{alertblock}
        \column{0.5\textwidth}
        \begin{figure}
            \centering
            % Copyright 2018-2024 FIUS
%
% This file is part of theo-vorkurs-folien.
%
% theo-vorkurs-folien is free software: you can redistribute it and/or modify
% it under the terms of the GNU General Public License as published by
% the Free Software Foundation, either version 3 of the License, or
% (at your option) any later version.
%
% theo-vorkurs-folien is distributed in the hope that it will be useful,
% but WITHOUT ANY WARRANTY; without even the implied warranty of
% MERCHANTABILITY or FITNESS FOR A PARTICULAR PURPOSE.  See the
% GNU General Public License for more details.
%
% You should have received a copy of the GNU General Public License
% along with theo-vorkurs-folien.  If not, see <https://www.gnu.org/licenses/>.

\definecolor{labyrinthLine}{RGB}{0,51,102}
\definecolor{labyrinthHead}{RGB}{0,25,50}
\definecolor{labyrinthField}{RGB}{255,230,204}
\definecolor{labyrinthPath}{RGB}{130,179,102}
\definecolor{labyrinthDecision}{RGB}{213,232,212}
\definecolor{labyrinthDecisionText}{RGB}{18,117,181}
\providecommand{\labyrinthSize}{\textwidth}
\providecommand{\labyrinthVariant}{None}
\begin{includetikzpicture}{\labyrinthSize}[x=1mm,y=1mm]
  \tikzset{every path/.style={line width=0.4}}
  % Strichmaennchen
  \draw (15,116.5)--(15,108.5);
  \draw (11,115.5)--(19,115.5);
  \draw[draw=black,fill=labyrinthHead] (15,119) circle[radius=2.5];
  \draw (15,108.5)--(10,103);
  \draw (15,108.5)--(20,103);
  \node[align=center] at (5,112) {\Large Bob};

  % Labyrinth
  \tikzset{every path/.style={line width=5,draw=labyrinthLine}}
  \draw (30,90)--(180,90);
  \draw (0,90)--(0,0)--(180,0)--(180,60)--(150,60);
  \draw (120,0)--(120,30);
  \draw (90,30)--(150,30);
  \draw (30,60)--(120,60);
  \draw (60,60)--(60,30)--(30,30);
  \draw[-{Stealth[inset=0pt, length=12, angle'=60]},line width=7] (15,102)--(15,89);
  \draw[-{Stealth[inset=0pt, length=12, angle'=60]},line width=7] (179,75)--(192,75);

  % Felder
  \tikzset{every path/.style={draw=none,fill=labyrinthField},every circle/.style={radius=11}}
  \foreach \x in {0,...,5}
  \foreach \y in {0,...,2}
    {\fill (15 + \x * 30,15 + \y * 30) circle;}

  % Pfade
  \tikzset{every path/.style={-{Stealth},draw=labyrinthPath,fill=none,rounded corners=10,line width=5}}
  \ifthenelse{\equal{\labyrinthVariant}{Direkt}}
  {
    \draw (15,86)--(15,75)--(176,75); % Direkt
  }{}
  \ifthenelse{\equal{\labyrinthVariant}{Indirekt}}
  {
    \draw (15,86)--(15,15)--(75,15)--(75,45)--(135,45)--(135,75)--(176,75); % Indirekt
  }{}
  \ifthenelse{\equal{\labyrinthVariant}{Uhrzeigersinn}}
  {
    \draw (15,86)--(15,75)--(135,75)--(135,45)--(75,45)--(75,15)--(15,15)--(15,75)--(176,75); % Uhrzeigersinn
    \draw[line width=2,draw=black] (60,82.75)--(90,82.75);
  }{}
  \ifthenelse{\equal{\labyrinthVariant}{GegenUhrzeigersinn}}
  {
    \draw (15,86)--(15,15)--(75,15)--(75,45)--(135,45)--(135,75)--(15,75)--(15,15)--(75,15)--(75,45)--(135,45)--(135,75)--(176,75); % GegenUhrzeigersinn
    \draw[line width=2,draw=black] (90,82.75)--(60,82.75);
  }{}

  \ifthenelse{\equal{\labyrinthVariant}{DecisionPoints}}
  {
  \fill[draw=none,fill=labyrinthDecision] (15,75) circle;
  \fill[draw=none,fill=labyrinthDecision] (135,75) circle;
  \tikzset{every path/.style={-{Stealth},draw=labyrinthDecisionText,line width=2},every node/.style={font=\Huge,text=labyrinthDecisionText,text centered}}
  \node[align=center] at (19,79) {$A_r$};
  \node[align=center] at (11,71) {$A_u$};
  \draw (34,75)--(26,75);
  \draw (15,56)--(15,64);
  \draw[-] (9,81)--(21,69);

  \node[align=center] at (131,79) {$B_l$};
  \node[align=center] at (139,71) {$B_u$};
  \draw (116,75)--(124,75);
  \draw (135,56)--(135,64);
  \draw[-] (129,69)--(141,81);
  }{}
\end{includetikzpicture}
\let\labyrinthVariant\relax
\let\labyrinthSize\relax
            \caption{Totoros Problem}

        \end{figure}
    \end{columns}
    \alert{Gib eine Grammatik an, welche die Sprache beschreibt, die Totoro durch alle ihm möglichen Wege des Labyrinths führt.}
\end{frame}
}

{\setbeamercolor{palette primary}{bg=ExColor}
\begin{frame}{Denkpause}
    \begin{alertblock}{Beispiel}
        \begin{figure}
            \centering
            \def\labyrinthVariant{Direkt}
            \def\labyrinthSize{0.9\textwidth}
            % Copyright 2018-2024 FIUS
%
% This file is part of theo-vorkurs-folien.
%
% theo-vorkurs-folien is free software: you can redistribute it and/or modify
% it under the terms of the GNU General Public License as published by
% the Free Software Foundation, either version 3 of the License, or
% (at your option) any later version.
%
% theo-vorkurs-folien is distributed in the hope that it will be useful,
% but WITHOUT ANY WARRANTY; without even the implied warranty of
% MERCHANTABILITY or FITNESS FOR A PARTICULAR PURPOSE.  See the
% GNU General Public License for more details.
%
% You should have received a copy of the GNU General Public License
% along with theo-vorkurs-folien.  If not, see <https://www.gnu.org/licenses/>.

\definecolor{labyrinthLine}{RGB}{0,51,102}
\definecolor{labyrinthHead}{RGB}{0,25,50}
\definecolor{labyrinthField}{RGB}{255,230,204}
\definecolor{labyrinthPath}{RGB}{130,179,102}
\definecolor{labyrinthDecision}{RGB}{213,232,212}
\definecolor{labyrinthDecisionText}{RGB}{18,117,181}
\providecommand{\labyrinthSize}{\textwidth}
\providecommand{\labyrinthVariant}{None}
\begin{includetikzpicture}{\labyrinthSize}[x=1mm,y=1mm]
  \tikzset{every path/.style={line width=0.4}}
  % Strichmaennchen
  \draw (15,116.5)--(15,108.5);
  \draw (11,115.5)--(19,115.5);
  \draw[draw=black,fill=labyrinthHead] (15,119) circle[radius=2.5];
  \draw (15,108.5)--(10,103);
  \draw (15,108.5)--(20,103);
  \node[align=center] at (5,112) {\Large Bob};

  % Labyrinth
  \tikzset{every path/.style={line width=5,draw=labyrinthLine}}
  \draw (30,90)--(180,90);
  \draw (0,90)--(0,0)--(180,0)--(180,60)--(150,60);
  \draw (120,0)--(120,30);
  \draw (90,30)--(150,30);
  \draw (30,60)--(120,60);
  \draw (60,60)--(60,30)--(30,30);
  \draw[-{Stealth[inset=0pt, length=12, angle'=60]},line width=7] (15,102)--(15,89);
  \draw[-{Stealth[inset=0pt, length=12, angle'=60]},line width=7] (179,75)--(192,75);

  % Felder
  \tikzset{every path/.style={draw=none,fill=labyrinthField},every circle/.style={radius=11}}
  \foreach \x in {0,...,5}
  \foreach \y in {0,...,2}
    {\fill (15 + \x * 30,15 + \y * 30) circle;}

  % Pfade
  \tikzset{every path/.style={-{Stealth},draw=labyrinthPath,fill=none,rounded corners=10,line width=5}}
  \ifthenelse{\equal{\labyrinthVariant}{Direkt}}
  {
    \draw (15,86)--(15,75)--(176,75); % Direkt
  }{}
  \ifthenelse{\equal{\labyrinthVariant}{Indirekt}}
  {
    \draw (15,86)--(15,15)--(75,15)--(75,45)--(135,45)--(135,75)--(176,75); % Indirekt
  }{}
  \ifthenelse{\equal{\labyrinthVariant}{Uhrzeigersinn}}
  {
    \draw (15,86)--(15,75)--(135,75)--(135,45)--(75,45)--(75,15)--(15,15)--(15,75)--(176,75); % Uhrzeigersinn
    \draw[line width=2,draw=black] (60,82.75)--(90,82.75);
  }{}
  \ifthenelse{\equal{\labyrinthVariant}{GegenUhrzeigersinn}}
  {
    \draw (15,86)--(15,15)--(75,15)--(75,45)--(135,45)--(135,75)--(15,75)--(15,15)--(75,15)--(75,45)--(135,45)--(135,75)--(176,75); % GegenUhrzeigersinn
    \draw[line width=2,draw=black] (90,82.75)--(60,82.75);
  }{}

  \ifthenelse{\equal{\labyrinthVariant}{DecisionPoints}}
  {
  \fill[draw=none,fill=labyrinthDecision] (15,75) circle;
  \fill[draw=none,fill=labyrinthDecision] (135,75) circle;
  \tikzset{every path/.style={-{Stealth},draw=labyrinthDecisionText,line width=2},every node/.style={font=\Huge,text=labyrinthDecisionText,text centered}}
  \node[align=center] at (19,79) {$A_r$};
  \node[align=center] at (11,71) {$A_u$};
  \draw (34,75)--(26,75);
  \draw (15,56)--(15,64);
  \draw[-] (9,81)--(21,69);

  \node[align=center] at (131,79) {$B_l$};
  \node[align=center] at (139,71) {$B_u$};
  \draw (116,75)--(124,75);
  \draw (135,56)--(135,64);
  \draw[-] (129,69)--(141,81);
  }{}
\end{includetikzpicture}
\let\labyrinthVariant\relax
\let\labyrinthSize\relax
            \caption{Der direkte Weg ist repräsentiert durch das Wort \alert{\MoveDown\Forward\Forward\Forward\Forward\Forward\Forward}}
        \end{figure}
    \end{alertblock}
\end{frame}
}

{\setbeamercolor{palette primary}{bg=ExColor}
\begin{frame}<handout:0>{Lösung}
    \only<1>{
        \begin{figure}
            \centering
            \def\labyrinthVariant{Indirekt}
            \def\labyrinthSize{0.9\textwidth}
            % Copyright 2018-2024 FIUS
%
% This file is part of theo-vorkurs-folien.
%
% theo-vorkurs-folien is free software: you can redistribute it and/or modify
% it under the terms of the GNU General Public License as published by
% the Free Software Foundation, either version 3 of the License, or
% (at your option) any later version.
%
% theo-vorkurs-folien is distributed in the hope that it will be useful,
% but WITHOUT ANY WARRANTY; without even the implied warranty of
% MERCHANTABILITY or FITNESS FOR A PARTICULAR PURPOSE.  See the
% GNU General Public License for more details.
%
% You should have received a copy of the GNU General Public License
% along with theo-vorkurs-folien.  If not, see <https://www.gnu.org/licenses/>.

\definecolor{labyrinthLine}{RGB}{0,51,102}
\definecolor{labyrinthHead}{RGB}{0,25,50}
\definecolor{labyrinthField}{RGB}{255,230,204}
\definecolor{labyrinthPath}{RGB}{130,179,102}
\definecolor{labyrinthDecision}{RGB}{213,232,212}
\definecolor{labyrinthDecisionText}{RGB}{18,117,181}
\providecommand{\labyrinthSize}{\textwidth}
\providecommand{\labyrinthVariant}{None}
\begin{includetikzpicture}{\labyrinthSize}[x=1mm,y=1mm]
  \tikzset{every path/.style={line width=0.4}}
  % Strichmaennchen
  \draw (15,116.5)--(15,108.5);
  \draw (11,115.5)--(19,115.5);
  \draw[draw=black,fill=labyrinthHead] (15,119) circle[radius=2.5];
  \draw (15,108.5)--(10,103);
  \draw (15,108.5)--(20,103);
  \node[align=center] at (5,112) {\Large Bob};

  % Labyrinth
  \tikzset{every path/.style={line width=5,draw=labyrinthLine}}
  \draw (30,90)--(180,90);
  \draw (0,90)--(0,0)--(180,0)--(180,60)--(150,60);
  \draw (120,0)--(120,30);
  \draw (90,30)--(150,30);
  \draw (30,60)--(120,60);
  \draw (60,60)--(60,30)--(30,30);
  \draw[-{Stealth[inset=0pt, length=12, angle'=60]},line width=7] (15,102)--(15,89);
  \draw[-{Stealth[inset=0pt, length=12, angle'=60]},line width=7] (179,75)--(192,75);

  % Felder
  \tikzset{every path/.style={draw=none,fill=labyrinthField},every circle/.style={radius=11}}
  \foreach \x in {0,...,5}
  \foreach \y in {0,...,2}
    {\fill (15 + \x * 30,15 + \y * 30) circle;}

  % Pfade
  \tikzset{every path/.style={-{Stealth},draw=labyrinthPath,fill=none,rounded corners=10,line width=5}}
  \ifthenelse{\equal{\labyrinthVariant}{Direkt}}
  {
    \draw (15,86)--(15,75)--(176,75); % Direkt
  }{}
  \ifthenelse{\equal{\labyrinthVariant}{Indirekt}}
  {
    \draw (15,86)--(15,15)--(75,15)--(75,45)--(135,45)--(135,75)--(176,75); % Indirekt
  }{}
  \ifthenelse{\equal{\labyrinthVariant}{Uhrzeigersinn}}
  {
    \draw (15,86)--(15,75)--(135,75)--(135,45)--(75,45)--(75,15)--(15,15)--(15,75)--(176,75); % Uhrzeigersinn
    \draw[line width=2,draw=black] (60,82.75)--(90,82.75);
  }{}
  \ifthenelse{\equal{\labyrinthVariant}{GegenUhrzeigersinn}}
  {
    \draw (15,86)--(15,15)--(75,15)--(75,45)--(135,45)--(135,75)--(15,75)--(15,15)--(75,15)--(75,45)--(135,45)--(135,75)--(176,75); % GegenUhrzeigersinn
    \draw[line width=2,draw=black] (90,82.75)--(60,82.75);
  }{}

  \ifthenelse{\equal{\labyrinthVariant}{DecisionPoints}}
  {
  \fill[draw=none,fill=labyrinthDecision] (15,75) circle;
  \fill[draw=none,fill=labyrinthDecision] (135,75) circle;
  \tikzset{every path/.style={-{Stealth},draw=labyrinthDecisionText,line width=2},every node/.style={font=\Huge,text=labyrinthDecisionText,text centered}}
  \node[align=center] at (19,79) {$A_r$};
  \node[align=center] at (11,71) {$A_u$};
  \draw (34,75)--(26,75);
  \draw (15,56)--(15,64);
  \draw[-] (9,81)--(21,69);

  \node[align=center] at (131,79) {$B_l$};
  \node[align=center] at (139,71) {$B_u$};
  \draw (116,75)--(124,75);
  \draw (135,56)--(135,64);
  \draw[-] (129,69)--(141,81);
  }{}
\end{includetikzpicture}
\let\labyrinthVariant\relax
\let\labyrinthSize\relax
            \caption{Indirekter Weg}

        \end{figure}
    }
    \only<2>{
        \begin{figure}
            \centering
            \def\labyrinthVariant{Uhrzeigersinn}
            \def\labyrinthSize{0.9\textwidth}
            % Copyright 2018-2024 FIUS
%
% This file is part of theo-vorkurs-folien.
%
% theo-vorkurs-folien is free software: you can redistribute it and/or modify
% it under the terms of the GNU General Public License as published by
% the Free Software Foundation, either version 3 of the License, or
% (at your option) any later version.
%
% theo-vorkurs-folien is distributed in the hope that it will be useful,
% but WITHOUT ANY WARRANTY; without even the implied warranty of
% MERCHANTABILITY or FITNESS FOR A PARTICULAR PURPOSE.  See the
% GNU General Public License for more details.
%
% You should have received a copy of the GNU General Public License
% along with theo-vorkurs-folien.  If not, see <https://www.gnu.org/licenses/>.

\definecolor{labyrinthLine}{RGB}{0,51,102}
\definecolor{labyrinthHead}{RGB}{0,25,50}
\definecolor{labyrinthField}{RGB}{255,230,204}
\definecolor{labyrinthPath}{RGB}{130,179,102}
\definecolor{labyrinthDecision}{RGB}{213,232,212}
\definecolor{labyrinthDecisionText}{RGB}{18,117,181}
\providecommand{\labyrinthSize}{\textwidth}
\providecommand{\labyrinthVariant}{None}
\begin{includetikzpicture}{\labyrinthSize}[x=1mm,y=1mm]
  \tikzset{every path/.style={line width=0.4}}
  % Strichmaennchen
  \draw (15,116.5)--(15,108.5);
  \draw (11,115.5)--(19,115.5);
  \draw[draw=black,fill=labyrinthHead] (15,119) circle[radius=2.5];
  \draw (15,108.5)--(10,103);
  \draw (15,108.5)--(20,103);
  \node[align=center] at (5,112) {\Large Bob};

  % Labyrinth
  \tikzset{every path/.style={line width=5,draw=labyrinthLine}}
  \draw (30,90)--(180,90);
  \draw (0,90)--(0,0)--(180,0)--(180,60)--(150,60);
  \draw (120,0)--(120,30);
  \draw (90,30)--(150,30);
  \draw (30,60)--(120,60);
  \draw (60,60)--(60,30)--(30,30);
  \draw[-{Stealth[inset=0pt, length=12, angle'=60]},line width=7] (15,102)--(15,89);
  \draw[-{Stealth[inset=0pt, length=12, angle'=60]},line width=7] (179,75)--(192,75);

  % Felder
  \tikzset{every path/.style={draw=none,fill=labyrinthField},every circle/.style={radius=11}}
  \foreach \x in {0,...,5}
  \foreach \y in {0,...,2}
    {\fill (15 + \x * 30,15 + \y * 30) circle;}

  % Pfade
  \tikzset{every path/.style={-{Stealth},draw=labyrinthPath,fill=none,rounded corners=10,line width=5}}
  \ifthenelse{\equal{\labyrinthVariant}{Direkt}}
  {
    \draw (15,86)--(15,75)--(176,75); % Direkt
  }{}
  \ifthenelse{\equal{\labyrinthVariant}{Indirekt}}
  {
    \draw (15,86)--(15,15)--(75,15)--(75,45)--(135,45)--(135,75)--(176,75); % Indirekt
  }{}
  \ifthenelse{\equal{\labyrinthVariant}{Uhrzeigersinn}}
  {
    \draw (15,86)--(15,75)--(135,75)--(135,45)--(75,45)--(75,15)--(15,15)--(15,75)--(176,75); % Uhrzeigersinn
    \draw[line width=2,draw=black] (60,82.75)--(90,82.75);
  }{}
  \ifthenelse{\equal{\labyrinthVariant}{GegenUhrzeigersinn}}
  {
    \draw (15,86)--(15,15)--(75,15)--(75,45)--(135,45)--(135,75)--(15,75)--(15,15)--(75,15)--(75,45)--(135,45)--(135,75)--(176,75); % GegenUhrzeigersinn
    \draw[line width=2,draw=black] (90,82.75)--(60,82.75);
  }{}

  \ifthenelse{\equal{\labyrinthVariant}{DecisionPoints}}
  {
  \fill[draw=none,fill=labyrinthDecision] (15,75) circle;
  \fill[draw=none,fill=labyrinthDecision] (135,75) circle;
  \tikzset{every path/.style={-{Stealth},draw=labyrinthDecisionText,line width=2},every node/.style={font=\Huge,text=labyrinthDecisionText,text centered}}
  \node[align=center] at (19,79) {$A_r$};
  \node[align=center] at (11,71) {$A_u$};
  \draw (34,75)--(26,75);
  \draw (15,56)--(15,64);
  \draw[-] (9,81)--(21,69);

  \node[align=center] at (131,79) {$B_l$};
  \node[align=center] at (139,71) {$B_u$};
  \draw (116,75)--(124,75);
  \draw (135,56)--(135,64);
  \draw[-] (129,69)--(141,81);
  }{}
\end{includetikzpicture}
\let\labyrinthVariant\relax
\let\labyrinthSize\relax
            \caption{Schlaufe Uhrzeigersinn}

        \end{figure}\textbf{}
    }
    \only<3>{
        \begin{figure}
            \centering
            \def\labyrinthVariant{GegenUhrzeigersinn}
            \def\labyrinthSize{0.9\textwidth}
            % Copyright 2018-2024 FIUS
%
% This file is part of theo-vorkurs-folien.
%
% theo-vorkurs-folien is free software: you can redistribute it and/or modify
% it under the terms of the GNU General Public License as published by
% the Free Software Foundation, either version 3 of the License, or
% (at your option) any later version.
%
% theo-vorkurs-folien is distributed in the hope that it will be useful,
% but WITHOUT ANY WARRANTY; without even the implied warranty of
% MERCHANTABILITY or FITNESS FOR A PARTICULAR PURPOSE.  See the
% GNU General Public License for more details.
%
% You should have received a copy of the GNU General Public License
% along with theo-vorkurs-folien.  If not, see <https://www.gnu.org/licenses/>.

\definecolor{labyrinthLine}{RGB}{0,51,102}
\definecolor{labyrinthHead}{RGB}{0,25,50}
\definecolor{labyrinthField}{RGB}{255,230,204}
\definecolor{labyrinthPath}{RGB}{130,179,102}
\definecolor{labyrinthDecision}{RGB}{213,232,212}
\definecolor{labyrinthDecisionText}{RGB}{18,117,181}
\providecommand{\labyrinthSize}{\textwidth}
\providecommand{\labyrinthVariant}{None}
\begin{includetikzpicture}{\labyrinthSize}[x=1mm,y=1mm]
  \tikzset{every path/.style={line width=0.4}}
  % Strichmaennchen
  \draw (15,116.5)--(15,108.5);
  \draw (11,115.5)--(19,115.5);
  \draw[draw=black,fill=labyrinthHead] (15,119) circle[radius=2.5];
  \draw (15,108.5)--(10,103);
  \draw (15,108.5)--(20,103);
  \node[align=center] at (5,112) {\Large Bob};

  % Labyrinth
  \tikzset{every path/.style={line width=5,draw=labyrinthLine}}
  \draw (30,90)--(180,90);
  \draw (0,90)--(0,0)--(180,0)--(180,60)--(150,60);
  \draw (120,0)--(120,30);
  \draw (90,30)--(150,30);
  \draw (30,60)--(120,60);
  \draw (60,60)--(60,30)--(30,30);
  \draw[-{Stealth[inset=0pt, length=12, angle'=60]},line width=7] (15,102)--(15,89);
  \draw[-{Stealth[inset=0pt, length=12, angle'=60]},line width=7] (179,75)--(192,75);

  % Felder
  \tikzset{every path/.style={draw=none,fill=labyrinthField},every circle/.style={radius=11}}
  \foreach \x in {0,...,5}
  \foreach \y in {0,...,2}
    {\fill (15 + \x * 30,15 + \y * 30) circle;}

  % Pfade
  \tikzset{every path/.style={-{Stealth},draw=labyrinthPath,fill=none,rounded corners=10,line width=5}}
  \ifthenelse{\equal{\labyrinthVariant}{Direkt}}
  {
    \draw (15,86)--(15,75)--(176,75); % Direkt
  }{}
  \ifthenelse{\equal{\labyrinthVariant}{Indirekt}}
  {
    \draw (15,86)--(15,15)--(75,15)--(75,45)--(135,45)--(135,75)--(176,75); % Indirekt
  }{}
  \ifthenelse{\equal{\labyrinthVariant}{Uhrzeigersinn}}
  {
    \draw (15,86)--(15,75)--(135,75)--(135,45)--(75,45)--(75,15)--(15,15)--(15,75)--(176,75); % Uhrzeigersinn
    \draw[line width=2,draw=black] (60,82.75)--(90,82.75);
  }{}
  \ifthenelse{\equal{\labyrinthVariant}{GegenUhrzeigersinn}}
  {
    \draw (15,86)--(15,15)--(75,15)--(75,45)--(135,45)--(135,75)--(15,75)--(15,15)--(75,15)--(75,45)--(135,45)--(135,75)--(176,75); % GegenUhrzeigersinn
    \draw[line width=2,draw=black] (90,82.75)--(60,82.75);
  }{}

  \ifthenelse{\equal{\labyrinthVariant}{DecisionPoints}}
  {
  \fill[draw=none,fill=labyrinthDecision] (15,75) circle;
  \fill[draw=none,fill=labyrinthDecision] (135,75) circle;
  \tikzset{every path/.style={-{Stealth},draw=labyrinthDecisionText,line width=2},every node/.style={font=\Huge,text=labyrinthDecisionText,text centered}}
  \node[align=center] at (19,79) {$A_r$};
  \node[align=center] at (11,71) {$A_u$};
  \draw (34,75)--(26,75);
  \draw (15,56)--(15,64);
  \draw[-] (9,81)--(21,69);

  \node[align=center] at (131,79) {$B_l$};
  \node[align=center] at (139,71) {$B_u$};
  \draw (116,75)--(124,75);
  \draw (135,56)--(135,64);
  \draw[-] (129,69)--(141,81);
  }{}
\end{includetikzpicture}
\let\labyrinthVariant\relax
\let\labyrinthSize\relax
            \caption{Schlaufe gegen Uhrzeigersinn}

        \end{figure}
    }
\end{frame}
}

{\setbeamercolor{palette primary}{bg=ExColor}
\begin{frame}<handout:0>{Lösung}
    \begin{columns}
        \column{0.45\textwidth}
        \begin{alertblock}{Eine Möglichkeit:}
            %Wir nehmen uns zwei Variablen um zwischen den Einstiegsrichtungen zu unterscheiden für jeden Entscheidungspunkt und konstruieren damit  unsere Grammatik:\\
            $G = (V, \Sigma, P, S)$, wobei \\
            $V = \{S, A_u, A_r, B_u, B_l\}$ \\
            $\Sigma = \{\text{\Rewind}, \text{\MoveUp}, \text{\Forward}, \text{\MoveDown}\}$ \\
            $P = \{S \rightarrow \text\MoveDown A_u \ |\ \text\MoveDown A_r,$\\
            \qquad\; $A_u \rightarrow \text{\Forward\Forward\Forward\Forward} B_l$\\
            \qquad\; $A_r \rightarrow \text{\MoveDown\MoveDown\Forward\Forward\MoveUp\Forward\Forward\MoveUp} B_u,$\\
            \qquad\; $B_l \rightarrow \text{\MoveDown\Rewind\Rewind\MoveDown\Rewind\Rewind\MoveUp\MoveUp} A_u \ |\ \text{\Forward\Forward},$\\
            \qquad\; $B_u \rightarrow \text{\Rewind\Rewind\Rewind\Rewind} A_r \ |\ \text{\Forward\Forward}\}$
        \end{alertblock}
        \column{0.55\textwidth}
        \begin{figure}
            \centering
            \def\labyrinthVariant{DecisionPoints}
            \def\labyrinthSize{0.9\textwidth}
            % Copyright 2018-2024 FIUS
%
% This file is part of theo-vorkurs-folien.
%
% theo-vorkurs-folien is free software: you can redistribute it and/or modify
% it under the terms of the GNU General Public License as published by
% the Free Software Foundation, either version 3 of the License, or
% (at your option) any later version.
%
% theo-vorkurs-folien is distributed in the hope that it will be useful,
% but WITHOUT ANY WARRANTY; without even the implied warranty of
% MERCHANTABILITY or FITNESS FOR A PARTICULAR PURPOSE.  See the
% GNU General Public License for more details.
%
% You should have received a copy of the GNU General Public License
% along with theo-vorkurs-folien.  If not, see <https://www.gnu.org/licenses/>.

\definecolor{labyrinthLine}{RGB}{0,51,102}
\definecolor{labyrinthHead}{RGB}{0,25,50}
\definecolor{labyrinthField}{RGB}{255,230,204}
\definecolor{labyrinthPath}{RGB}{130,179,102}
\definecolor{labyrinthDecision}{RGB}{213,232,212}
\definecolor{labyrinthDecisionText}{RGB}{18,117,181}
\providecommand{\labyrinthSize}{\textwidth}
\providecommand{\labyrinthVariant}{None}
\begin{includetikzpicture}{\labyrinthSize}[x=1mm,y=1mm]
  \tikzset{every path/.style={line width=0.4}}
  % Strichmaennchen
  \draw (15,116.5)--(15,108.5);
  \draw (11,115.5)--(19,115.5);
  \draw[draw=black,fill=labyrinthHead] (15,119) circle[radius=2.5];
  \draw (15,108.5)--(10,103);
  \draw (15,108.5)--(20,103);
  \node[align=center] at (5,112) {\Large Bob};

  % Labyrinth
  \tikzset{every path/.style={line width=5,draw=labyrinthLine}}
  \draw (30,90)--(180,90);
  \draw (0,90)--(0,0)--(180,0)--(180,60)--(150,60);
  \draw (120,0)--(120,30);
  \draw (90,30)--(150,30);
  \draw (30,60)--(120,60);
  \draw (60,60)--(60,30)--(30,30);
  \draw[-{Stealth[inset=0pt, length=12, angle'=60]},line width=7] (15,102)--(15,89);
  \draw[-{Stealth[inset=0pt, length=12, angle'=60]},line width=7] (179,75)--(192,75);

  % Felder
  \tikzset{every path/.style={draw=none,fill=labyrinthField},every circle/.style={radius=11}}
  \foreach \x in {0,...,5}
  \foreach \y in {0,...,2}
    {\fill (15 + \x * 30,15 + \y * 30) circle;}

  % Pfade
  \tikzset{every path/.style={-{Stealth},draw=labyrinthPath,fill=none,rounded corners=10,line width=5}}
  \ifthenelse{\equal{\labyrinthVariant}{Direkt}}
  {
    \draw (15,86)--(15,75)--(176,75); % Direkt
  }{}
  \ifthenelse{\equal{\labyrinthVariant}{Indirekt}}
  {
    \draw (15,86)--(15,15)--(75,15)--(75,45)--(135,45)--(135,75)--(176,75); % Indirekt
  }{}
  \ifthenelse{\equal{\labyrinthVariant}{Uhrzeigersinn}}
  {
    \draw (15,86)--(15,75)--(135,75)--(135,45)--(75,45)--(75,15)--(15,15)--(15,75)--(176,75); % Uhrzeigersinn
    \draw[line width=2,draw=black] (60,82.75)--(90,82.75);
  }{}
  \ifthenelse{\equal{\labyrinthVariant}{GegenUhrzeigersinn}}
  {
    \draw (15,86)--(15,15)--(75,15)--(75,45)--(135,45)--(135,75)--(15,75)--(15,15)--(75,15)--(75,45)--(135,45)--(135,75)--(176,75); % GegenUhrzeigersinn
    \draw[line width=2,draw=black] (90,82.75)--(60,82.75);
  }{}

  \ifthenelse{\equal{\labyrinthVariant}{DecisionPoints}}
  {
  \fill[draw=none,fill=labyrinthDecision] (15,75) circle;
  \fill[draw=none,fill=labyrinthDecision] (135,75) circle;
  \tikzset{every path/.style={-{Stealth},draw=labyrinthDecisionText,line width=2},every node/.style={font=\Huge,text=labyrinthDecisionText,text centered}}
  \node[align=center] at (19,79) {$A_r$};
  \node[align=center] at (11,71) {$A_u$};
  \draw (34,75)--(26,75);
  \draw (15,56)--(15,64);
  \draw[-] (9,81)--(21,69);

  \node[align=center] at (131,79) {$B_l$};
  \node[align=center] at (139,71) {$B_u$};
  \draw (116,75)--(124,75);
  \draw (135,56)--(135,64);
  \draw[-] (129,69)--(141,81);
  }{}
\end{includetikzpicture}
\let\labyrinthVariant\relax
\let\labyrinthSize\relax
            \caption{Es muss unterschieden werden, ob Totoro von links, rechts oder unten kam}

        \end{figure}
    \end{columns}
    \small\emph{Erinnerung:} Totoro kann nicht auf ein Feld zurücktreten, von dem er gerade kam
\end{frame}
}

\subsubsection{Ableiten}
\begin{frame}[fragile]{Ableiten}
    Wir können durch das Ableiten formal zeigen, dass ein Wort von einer Grammatik erzeugt wird:\\
    \small{Wir betrachten L = \{$ww^R \mid w^R\text{ ist w rückwärts, }w \in \{a, b\}^n, n\in \mathbb{N}\setminus \{0\}$\}\\
        mit der Grammatik $G=(V,\Sigma,P,S)$, wobei\\
        $V=\{S\}$, $\Sigma=\{a,b\}$, $P = \{S \rightarrow aSa \ |\ bSb \ |\ aa \ |\ bb$\}}
    \metroset{block=fill}
    \begin{exampleblock}{Beispiel}
        Wir zeigen $ww^R = ababbbbaba \in$ L.\\
        \small{$S\Rightarrow_G aSa \Rightarrow_G abSba \Rightarrow_G  abaSaba \Rightarrow_G ababSbaba$ \\ $\Rightarrow_G ababbbbaba$}\\\qed
    \end{exampleblock}
\end{frame}

{\setbeamercolor{palette primary}{bg=ExColor}
\begin{frame}{Denkpause}
    \begin{alertblock}{Aufgaben}
        Zeige die folgenden Aussagen
    \end{alertblock}
    \metroset{block=fill}
    \begin{block}{Normal}
        \begin{itemize}
            \item $G_1=(\{S\}, \{a\}, P_1, S)$ erzeugt $aaaa$\\
                  mit $P_1=\{S\rightarrow aaS\ |\ \emptyWord\}$
            \item $G_2=(\{S,A,B\}, \{a,b,c\}, P_2, S)$ erzeugt $aabbc$\\
                  mit $P_2=\{S\rightarrow AB$, $A\rightarrow aAb \ |\ ab\ |\ \emptyWord$, $B\rightarrow cB \ |\  \emptyWord\}$
            \item $G_3=(\{S,U,V\}, \{a,b,c,d\}, P_3, S)$ erzeugt $abac$\\
                  mit $P_3=\{S\rightarrow UV$, $U\rightarrow aU \ |\  bU \ |\  \emptyWord$, $V\rightarrow c \ |\  d\}$
            \item $G_4=(\{S,X\}, \{a,b,c\}, P_4, S)$ erzeugt $aac$\\
                  mit $P_4=\{S\rightarrow XXX$, $X\rightarrow a \ |\  b \ |\  c\}$
        \end{itemize}
    \end{block}
\end{frame}
\begin{frame}{Denkpause}
    \begin{alertblock}{Aufgaben}
        Zeige die folgenden Aussagen
    \end{alertblock}
    \metroset{block=fill}
    \begin{block}{Etwas Schwerer}
        \begin{itemize}
            \item $G_5=(\{S\}, \{a\}, P_5, S)$ erzeugt $aaaa$\\
                  mit $P_5=\{S\rightarrow a \ |\  aaaS\}$
            \item $G_6=(\{S,A,B\}, \{a,b,c\}, P_6, S)$ erzeugt $cabcacca$\\
                  mit $P_6=\{S\rightarrow AAAB$, $AB\rightarrow BA,
                      A\rightarrow cA \ |\  Ac \ |\ a,
                      B\rightarrow cB \ |\  Bc \ |\  b\}$
            \item $G_7=(\{S,U\}, \{\text{\Stopsign},\text{\Rewind},\text{\MoveUp},\text{\Forward},\text{\MoveDown}\}, P_7, S)$  erzeugt \Forward\Stopsign\\
                  mit $P_7=\{S\rightarrow U\text{\Stopsign} \ |\  \text{\Stopsign}$, $U\rightarrow \text{\Rewind} U \ |\  \text{\MoveUp} U \ |\  \text{\Forward} U \ |\  \text{\MoveDown} U \ |\ \emptyWord\}$
        \end{itemize}
    \end{block}
\end{frame}
}

{\setbeamercolor{palette primary}{bg=ExColor}
\begin{frame}<handout:0>{Lösungen}
    Alle Lösungen sind Beispiellösungen, es sind auch andere möglich.
    \begin{itemize}[<+- | alert@+>]
        \item $S\Rightarrow_{G_1} aaS \Rightarrow_{G_1} aaaaS \Rightarrow_{G_1} aaaa$
        \item $S\Rightarrow_{G_2} AB \Rightarrow_{G_2} aAbB \Rightarrow_{G_2} aabbB \Rightarrow_{G_2} aabbcB \Rightarrow_{G_2} aabbc$
        \item $S\Rightarrow_{G_3} UV \Rightarrow_{G_3} aUV \Rightarrow_{G_3} abUV \Rightarrow_{G_3} abaUV \Rightarrow_{G_3} abaV \Rightarrow_{G_3} abac$
        \item $S\Rightarrow_{G_4} XXX \Rightarrow_{G_4} aXX \Rightarrow_{G_4} aaX \Rightarrow_{G_4} aac$
        \item $S\Rightarrow_{G_5} aaaS \Rightarrow_{G_5} aaaa$
        \item $S\Rightarrow_{G_6} AAAB \Rightarrow_{G_6} AABA \Rightarrow_{G_6} ABAA \Rightarrow_{G_6} cABAA \Rightarrow_{G_6} caBAA \Rightarrow_{G_6} cabAA \Rightarrow_{G_6} cabcAA \Rightarrow_{G_6} cabcaA\Rightarrow_{G_6} cabcacA \Rightarrow_{G_6} cabcaccA \Rightarrow_{G_6} cabcacca$
        \item $S\Rightarrow_{G_7} U\text{\Stopsign} \Rightarrow_{G_7} \text{\Forward}U\text{\Stopsign} \Rightarrow_{G_7} \text{\Forward}\text{\Stopsign}$
    \end{itemize}
\end{frame}
}
