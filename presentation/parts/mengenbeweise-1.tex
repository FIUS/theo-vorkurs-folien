% Copyright 2018, 2019, 2020, 2021 FIUS
%
% This file is part of theo-vorkurs-folien.
%
% theo-vorkurs-folien is free software: you can redistribute it and/or modify
% it under the terms of the GNU General Public License as published by
% the Free Software Foundation, either version 3 of the License, or
% (at your option) any later version.
%
% theo-vorkurs-folien is distributed in the hope that it will be useful,
% but WITHOUT ANY WARRANTY; without even the implied warranty of
% MERCHANTABILITY or FITNESS FOR A PARTICULAR PURPOSE.  See the
% GNU General Public License for more details.
%
% You should have received a copy of the GNU General Public License
% along with theo-vorkurs-folien.  If not, see <https://www.gnu.org/licenses/>.

\begin{frame}{einfacher Einstieg}
        \onslide
            Zu zeigen: Schnitt ist Kommutativ, d.h. $A \cap B = B \cap A$
        \begin{columns}
        \column{0.5\textwidth}
        \onslide{
            \begin{align*}
                \text{\quotedblbase}\implies\text{\textquotedblright}:\\
                x\in A \cap B &\implies x\in A \wedge x \in B\\
                &\implies x\in B \wedge x \in A\\
                &\implies x\in B \cap A
            \end{align*}
            }
        \column{0.5\textwidth}
        \onslide{
            \begin{align*}
                \text{\quotedblbase}\impliedby\text{\textquotedblright}:\\
                x\in B \cap A &\implies x\in B \wedge x \in A\\
                &\implies x\in A \wedge x \in B\\
                &\implies x\in A \cap B
            \end{align*}
            }
        \end{columns}
        \qed\\
    \small{\emph{Anmerkung:} $\wedge$ ist kommutativ}
\end{frame}

\begin{frame}{einfacher Einstieg}
        \onslide
            Zu zeigen: $A \setminus (B \cup C) = (A \setminus B) \cap (A \setminus C)$
        \onslide{
            \begin{align*}
                x\in A \setminus (B \cup C) &\iff x\in A \wedge \neg (x \in B \cup C)\\
                &\iff x\in A \wedge \neg (x \in B \vee x \in C)\\
                &\iff x \in A \wedge \neg (x \in B) \wedge \neg (x \in C)\\
                &\iff x \in A \wedge \neg (x \in B) \wedge x \in A \wedge \neg (x \in C)\\
                &\iff (x \in A \wedge \neg (x \in B)) \wedge (x \in A \wedge \neg (x \in C))\\
                &\iff (x \in A \setminus B) \wedge (x \in A \setminus C)\\
                &\iff x \in (A \setminus B) \cap (A \setminus C)
            \end{align*}\qed
            }
        \\
    \small{\emph{Rechenregel:} $\neg (A \wedge B) \iff \neg A \vee \neg B$, \\ \hspace{1.9cm}$\neg (A \vee B) \iff \neg A \wedge \neg B$}
\end{frame}



\subsubsection{Aufgaben}
{\setbeamercolor{palette primary}{bg=ExColor}
\begin{frame}[fragile]{Denkpause}
    \begin{alertblock}{Aufgaben}
    Versuche dich an den folgenden Mengenbeweisen.
    \end{alertblock}
    
    \metroset{block=fill}
    \begin{block}{Normal}
        \begin{itemize}
            \item $\overline{\overline{A}} = A$
        \end{itemize}
    \end{block}
    \metroset{block=fill}
    \begin{block}{Etwas schwerer}
        \begin{itemize}
            \item $A\cap B=\overline{(\overline{A}\cup\overline{B})}$
        \end{itemize}
    \end{block}
\end{frame}
}

{\setbeamercolor{palette primary}{bg=ExColor}
\begin{frame}<handout:0>[fragile]{Lösungen}
\onslide Zu zeigen: $A=\overline{\overline{A}}$
    \onslide{
    \begin{align*}
        x\in\overline{\overline{A}}
        &\iff\neg(x\in\overline{A})\\
        &\iff\neg(\neg (x\in A))\\
        &\iff x\in A
    \end{align*}\qed
    }
\end{frame}
}

{\setbeamercolor{palette primary}{bg=ExColor}
\begin{frame}<handout:0>[fragile]{Lösungen}
    \onslide Zu zeigen: $B\cap A=\overline{(\overline{A}\cup\overline{B})}$
    \onslide{
    \begin{align*}
        x \in \overline{(\overline{A} \cup \overline{B})}
        &\iff \neg(x \in \overline{A} \cup \overline{B})
        \\&\iff \neg(x \in \overline{A} \vee x \in \overline{B})
        \\&\iff \neg(\neg(x \in A) \vee \neg(x \in B))
        \\&\iff \neg(\neg(x\in A))\wedge\neg(\neg(x\in B))
        \\&\iff x \in A \wedge x \in B
        \\&\iff x \in A \cap B
        \\&\iff x \in B \cap A
    \end{align*}\qed
    }
\end{frame}
}
