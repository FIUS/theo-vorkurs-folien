% Copyright 2018-2022 FIUS
%
% This file is part of theo-vorkurs-folien.
%
% theo-vorkurs-folien is free software: you can redistribute it and/or modify
% it under the terms of the GNU General Public License as published by
% the Free Software Foundation, either version 3 of the License, or
% (at your option) any later version.
%
% theo-vorkurs-folien is distributed in the hope that it will be useful,
% but WITHOUT ANY WARRANTY; without even the implied warranty of
% MERCHANTABILITY or FITNESS FOR A PARTICULAR PURPOSE.  See the
% GNU General Public License for more details.
%
% You should have received a copy of the GNU General Public License
% along with theo-vorkurs-folien.  If not, see <https://www.gnu.org/licenses/>.

\subsection{Anwendung}
\begin{frame}[fragile]{Was ist eigentlich Theoretische Informatik?}
    \begin{itemize} 
    \item Theoretische Informatik ist die \textbf{formale} Herangehensweise an Probleme.\\
    \item Diese Probleme befassen sich unter Anderem mit den \textbf{formalen} Sprachen.
    \end{itemize}
\end{frame}

\begin{frame}{Anwendung der theoretischen Informatik}
    \begin{itemize}
        \item Ist ein bestimmtes Problem lösbar, oder \textbf{können} wir gar keine Lösung finden?
        \item IT-Sicherheit / Kryptographie: Die Sicherheit bestimmter Algorithmen \textbf{beweisen}
        \item Reguläre Ausdrücke
        \item Künstliche Intelligenz
        \item Compilerbau
        \item ...und vieles mehr...
    \end{itemize}
\end{frame}

\subsection{Theoretische Informatik in deinem Studium}
\begin{frame}[fragile]{Theoretische Informatik in deinem Studium}
Theoretische Informatik I ist Orientierungsprüfung für Informatik, Medieninformatik, Softwaretechnik und Data Science.
    \begin{itemize} 
    \item Du musst diese Prüfung spätestens zum Ende des dritten Semester bestanden haben.
    \item Du musst spätestens zum Ende des zweiten Semesters eine der beiden Orientierungsprüfungen angetreten haben.
    \item Du kannst die schriftliche Prüfung einmal nachschreiben und hast dann noch einen mündlichen Versuch im selben Semester.
    \end{itemize}
    \alert{Kennt eure \href{https://www.student.uni-stuttgart.de/pruefungsorganisation/pruefungsordnung/}{\underline{Prüfungsordnung}}!}
\end{frame}

\begin{frame}{Theoretische Informatik in deinem Studium}
    \begin{itemize}
        \item Theoretische Informatik I\\
        Formale Sprachen und Automatentheorie (FSuA)\\
        % TODO: Ggf. Auf Kufleitner ändern
        \quad Dozent: Prof. Dr. Ulrich Hertrampf
        \item Theoretische Informatik II\\
        Berechenbarkeit und Komplexität (BuK)\\
        \quad Dozent: Dr. Manfred Kufleitner
        \item Theoretische Informatik III\\
        Algorithmen und Diskrete Strukturen (AuDS)\\
        \quad Dozent: Dr. Manfred Kufleitner
    \end{itemize}
	\alert{Altklausuren helfen bei der Prüfungsvorbereitung. \\Fragt auch nach den Klausuren des alten Fachs.}
\end{frame}

\begin{frame}{Literatur der Vorlesung}
    \small{Uwe Schöning: Theoretische Informatik - kurzgefasst\\
    \phantom{}[\only<1>{\EUR{22,99}}\only<2|handout:0>{\alert{\EUR{0}}}]
    \begin{itemize}
        \item Die Vorlesung von Prof. Hertrampf richtet sich in weiten Teilen nach diesem Buch.
    \end{itemize}
    Boris Hollas: Grundkurs Theoretische Informatik: Mit Aufgaben und Anwendungen \\
    \phantom{}[\only<1>{\EUR{27,99}}\only<2|handout:0>{\alert{\EUR{0}}}]\\
    \begin{itemize}
        \item Weniger formal, dafür intuitiver mit einigen Beispielen und Übungsaufgaben.
    \end{itemize}
    Dirk W. Hoffmann: Theoretische Informatik
    \begin{itemize}
        \item Wird auch gelegentlich empfohlen.
    \end{itemize}}
    
\onslide<2>{\alert{Die Bücher sind alle in der Uni-Bib verfügbar, beim Schöning sollte man sich aber beeilen.}}
\end{frame}
