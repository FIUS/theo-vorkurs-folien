% Copyright 2018, 2019, 2020, 2021 FIUS
%
% This file is part of theo-vorkurs-folien.
%
% theo-vorkurs-folien is free software: you can redistribute it and/or modify
% it under the terms of the GNU General Public License as published by
% the Free Software Foundation, either version 3 of the License, or
% (at your option) any later version.
%
% theo-vorkurs-folien is distributed in the hope that it will be useful,
% but WITHOUT ANY WARRANTY; without even the implied warranty of
% MERCHANTABILITY or FITNESS FOR A PARTICULAR PURPOSE.  See the
% GNU General Public License for more details.
%
% You should have received a copy of the GNU General Public License
% along with theo-vorkurs-folien.  If not, see <https://www.gnu.org/licenses/>.

\begin{frame}{Automaten und Grammatiken}
  \metroset{block=fill}
  \begin{exampleblock}{Satz}
    Jede durch \only<6-|handout:0>{\alert<6>{deterministische}} endliche Automaten erkennbare Sprache ist auch regulär (also Typ 3).
  \end{exampleblock}
  \onslide<2-|handout:1> %
  Sei \alert<2|handout:0>{$A \subseteq \SigmaStern$} eine Sprache und \alert<2|handout:0>{M ein \only<-5|handout:0>{Automat}\only<6-|handout:1>{\alert<6|handout:0>{DEA}}} mit \alert<2-3|handout:0>{$\mathbf{T(M) = A}$}.\\
  \onslide<3-|handout:1> %
  (d.h. $M$ erkennt die Sprache $A$) \\
  \vspace{.3cm} %

  \onslide<4-|handout:1> %
  Wir suchen eine \alert<4|handout:0>{Typ 3-Grammatik G} mit \alert<4-5|handout:0>{$\mathbf{L(G) = A}$}.\\
  \onslide<5-|handout:1> %
  (d.h. die Grammatik $G$ erzeugt die Sprache $A$)
  \vspace{.3cm} %

  \onslide<6-|handout:1> %
  \begin{alertblock}{Anmerkung}
    Wir beschränken uns auf DEAs; in der Vorlesung werdet ihr aber eine allgemeinere Äquivalenz zeigen.
  \end{alertblock}
\end{frame}

\begin{frame}{DEA zu Grammatik: Konstruktion}
  Also: \alert<4-5|handout:0>{Zustände $\hat{=}$ Variablen}; \alert<6|handout:0>{Übergänge $\hat{=}$ Produktionen}

  \onslide<2-|handout:1> %
  Sei also ein \alert<2|handout:0>{DEA $M = (\alert<4|handout:0>{Z}, \alert<3|handout:0>{\Sigma}, \alert<6|handout:0>{\delta}, \alert<5|handout:0>{z_0}, E)$} gegeben. \\
    Wir definieren die \alert<2|handout:0>{Grammatik $G = (\alert<4|handout:0>{V}, \alert<3|handout:0>{\Sigma}, \alert<6|handout:0>{P}, \alert<5|handout:0>{S})$} mit:
  \begin{itemize}
    \item<3- | alert@3|handout:1> Alphabet $\Sigma$
    \item<4- | alert@4|handout:1> Variablenmenge $V = Z$
    \item<5- | alert@5|handout:1> Startsymbol $S = z_0$
    \item<6- | alert@6|handout:1> Produktionsmenge $P$ erzeugen wir aus $\delta$
  \end{itemize}
\end{frame}

\begin{frame}{DEA zu Grammatik: Produktionsregeln}\begin{columns}
    \begin{column}{0.5\textwidth}
      Für die Produktionsmenge $P$ wandeln wir die Übergänge um. \\
      \vspace{.3cm}
      \onslide<2-|handout:1> %
      Jedem $\delta$-Übergang \alert<2-3|handout:0>{$\delta(z_1, a)=z_2$} ordnen wir folgende Regeln zu:
      \begin{itemize}
        \item<2- | alert@2|handout:1> $z_1 \rightarrow a z_2$
        \item<3-|handout:1> Und zusätzlich, falls \alert<3|handout:0>{$z_2 \in E:$ $z_1 \rightarrow a$}
      \end{itemize}
      \onslide<4-|handout:1>{ %
        Falls $z_0 \in E$, brauchen wir außerdem $z_0 \to \emptyWord$.
      }
    \end{column}
    \begin{column}{0.5\textwidth}\centering\alert<handout:0>{\only<2-3|handout:1>{ %
        \begin{tikzpicture}[->,>=stealth',shorten >=1pt,auto,node distance=2cm,semithick]
          \node [state] (1)  {$z_1$};
          \onslide<2|handout:1>{
            \node [state] (2) [right of=1]  {$z_2$};
            \path (1) edge node {a} (2);
          }\onslide<3|handout:1>{
            \node [state, accepting] (2') [right of=1]  {$z_2$};
            \path (1) edge node {a} (2');
          }
        \end{tikzpicture} \\
        \glqq$\delta (z_1,a) = z_2$\grqq \\
        wird zu
        \begin{align*}
           & z_1 \to az_2            \\
           & \onslide<3->{z_1 \to a}
        \end{align*}
      }\only<4|handout:1>{
        \begin{tikzpicture}[->,>=stealth',shorten >=1pt,auto,node distance=2cm,semithick]
          \node [state,initial,accepting] (0)  {$z_0$};
        \end{tikzpicture} \\
        \glqq$z_0 \in E$\grqq \\
        wird zu $$z_0 \to \emptyWord$$
      }}\end{column}
  \end{columns}\end{frame}

\begin{frame}{DEA zu Grammatik: Korrektheit (Beweisidee)}
  \begin{alertblock}{Zu zeigen: $x \in T(M)$ gdw. $x \in L(G)$}
    \onslide<2-|handout:1> %
    Sei $x = a_1 a_2 \ldots a_{n-1} a_n$ ein Wort, das von $M$ akzeptiert wird. \\
    \onslide<3-|handout:1> %
    \begin{center}\centering\begin{tikzpicture}[->,>=stealth',shorten >=1pt,auto,node distance=1.75cm,semithick]
        \alert<5-5|handout:0>{\node [initial,state]   (0)              {$z_0$};}
        \alert<5-6|handout:0>{\node [state]           (1) [right of=0] {$z_1$};}
        \alert<6-8|handout:0>{\node                   (5) [right of=1] {$\ldots$};}
        \alert<8-9|handout:0>{\node [state]           (8) [right of=5] {$z_{n-1}$};}
        \alert<9-9|handout:0>{\node [state,accepting] (9) [right of=8] {$z_n$};}

        \alert<5|handout:0>{\path (0) edge node {$a_1$}     (1);}
        \alert<6|handout:0>{\path (1) edge node {$a_2$}     (5);}
        \alert<8|handout:0>{\path (5) edge node {$a_{n-1}$} (8);}
        \alert<9|handout:0>{\path (8) edge node {$a_n$}     (9);}
      \end{tikzpicture}\end{center}
    \onslide<4-|handout:1> %
    Mit den passenden Regeln \only<5-9|handout:1>{(z.B. %
      \only<5>{$z_0 \to a_1 z_1$}%
      \only<6>{$z_1 \to a_2 z_2$}%
      \only<7>{$z_2 \to a_3 z_3$}%
      \only<8>{$z_{n-2} \to a_{n-1} z_{n-1}$}%
      \only<9>{$z_{n-1} \to a_n$}%
      ) }lässt sich $x$ ableiten:
    $$\alert<5|handout:0>{z_0} \alert<5|handout:0>{\Rightarrow} \alert<5|handout:0>{a_1} \alert<5-6|handout:0>{z_1} \alert<6|handout:0>{\Rightarrow} a_1 \alert<6|handout:0>{a_2} \alert<6-7|handout:0>{z_2} \alert<7|handout:0>{\Rightarrow} \alert<7-8|handout:0>{\ldots} \alert<8|handout:0>{\Rightarrow} a_1 a_2 \ldots \alert<8|handout:0>{a_{n-1}} \alert<8-9|handout:0>{z_{n-1}} \alert<9|handout:0>{\Rightarrow} a_1 a_2 \ldots a_{n-1} \alert<9|handout:0>{a_n} = x$$
    \onslide<10-|handout:1> %
    Also können wir das Wort in der Grammatik ableiten. \\
    \onslide<11-|handout:1> %
    \alert<11|handout:0>{Funktioniert die Argumentation auch andersrum?}
    \onslide<12-|handout:1> %
    \alert<12|handout:0>{Ja!}
  \end{alertblock}
\end{frame}

\begin{frame}{DEA zu Grammatik: Korrektheit}
  \begin{alertblock}{Zu zeigen: $x \in T(M)$ gdw. $x \in L(G)$}
    Dabei gilt immer noch: $x = a_1 a_2 \ldots a_{n-1} a_n$ \\
    \onslide<2-|handout:1> %
    Die folgenden Aussagen sind äquivalent:
    \begin{itemize}
      \item<3- | alert@3|handout:1|handout:alert@0> $x$ wird von Automat $M$ erkannt \emph{$(x\in T(M))$}
      \item<4- | alert@4|handout:1|handout:alert@0> Es gibt eine Folge von Zuständen $z_0, z_1, \ldots, z_{n-1}, z_n$ mit:
            $z_0$ ist Startzustand, $z_n$ ist Endzustand \textbf{und}:
            $\forall i \in \{1, \ldots, n\}: \delta(z_{i-1}, a_i) = z_i$
      \item<5- | alert@5|handout:1|handout:alert@0> Es gibt Folge an Variablen $z_0, z_1, \dots, z_{n-1}$ mit:
            $z_0$ ist Startvariable, $(z_{n-1} \to a_n) \in P$ \textbf{und}:
            $\forall i \in \{1, \ldots, n-1\}: (z_{i-1} \to a_i z_i) \in P$
      \item<6- | alert@6|handout:1|handout:alert@0> Es gibt Folge an Variablen $z_0, z_1, \dots, z_{n-1}$ mit:
            $z_0$ ist Startvariable \textbf{und}: $z_0 \Rightarrow a_1 z_1 \Rightarrow \ldots \Rightarrow a_1 a_2 \ldots a_{n-1} a_n$
      \item<7- | alert@7|handout:1|handout:alert@0> x wird von der Grammatik G produziert \emph{$(x \in L(G))$}
    \end{itemize}
    \onslide<8-|handout:1> %
    Also gilt die Äquivalenz. \qed
  \end{alertblock}
\end{frame}

% \begin{frame}[fragile]{Grammatik zu Automaten}
%  Eine reguläre Grammatik und eine endlicher Automat können die selben Sprachen beschreiben.
%  \begin{exampleblock}{\glqq$\implies$\grqq}
%  \only<1>{Aus $z_1 \to az_2 \in P$ wird:\\
%  \begin{center}
%  \vspace{0.3cm}
%      \begin{tikzpicture}[->,>=stealth',shorten >=1pt,auto,node distance=2cm,
%                      semithick]
%      %\tikzstyle{every state}=[fill=ExColor,draw=none,text=white]

%      \node       [state]                 (1) [right of=1]  {$z_1$};
%      \node       [state]                 (2) [right of=2]  {$z_2$};

%      \path
%              (1) edge                node {a}  (2);
%      \end{tikzpicture}\\
%      \glqq$\delta (z_1,a) = z_2$\grqq
%  \end{center}}

%  \only<2>{
%  Falls $z_2 \in F$ ($z_2$ ein Endzustand)\\
%  Aus $z_1 \to a \in P$ wird:\\
%  \begin{center}
%  \vspace{0.3cm}
%      \begin{tikzpicture}[->,>=stealth',shorten >=1pt,auto,node distance=2cm,
%                      semithick]
%      %\tikzstyle{every state}=[fill=ExColor,draw=none,text=white]

%      \node       [state]                 (1) [right of=1]  {$z_1$};
%      \node       [state, accepting]      (2) [right of=2]  {$z_2$};

%      \path
%              (1) edge                node {a}  (2);
%      \end{tikzpicture}\\
%      \glqq$\delta (z_1,a) = z_2$\grqq
%  \end{center}
%  }
%  \end{exampleblock}
% \end{frame}
