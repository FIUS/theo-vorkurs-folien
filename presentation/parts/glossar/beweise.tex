% Copyright 2018-2022 FIUS
%
% This file is part of theo-vorkurs-folien.
%
% theo-vorkurs-folien is free software: you can redistribute it and/or modify
% it under the terms of the GNU General Public License as published by
% the Free Software Foundation, either version 3 of the License, or
% (at your option) any later version.
%
% theo-vorkurs-folien is distributed in the hope that it will be useful,
% but WITHOUT ANY WARRANTY; without even the implied warranty of
% MERCHANTABILITY or FITNESS FOR A PARTICULAR PURPOSE.  See the
% GNU General Public License for more details.
%
% You should have received a copy of the GNU General Public License
% along with theo-vorkurs-folien.  If not, see <https://www.gnu.org/licenses/>.

\begin{frame}[fragile]{Glossar \textemdash\ Beweise}
	\small
	\begin{tabular}{p{0.2\textwidth} p{0.25\textwidth} p{0.4\textwidth}}
		\toprule
		Abk.            & Bedeutung         & Was?!                                                  \\
		\midrule
		z.z.            & zu zeigen         & Was zu beweisen ist                                    \\
		Sei             &                   & bereits bekannte Objekte werden eingeführt und benannt \\
		$\exists$       & es gibt ein       &                                                        \\
		$\exists !$     & es gibt genau ein &                                                        \\
		x ist genau y   & x = y             & \emph{genau} wird verwendet bei Äquivalenz             \\
		x ist eindeutig & $\exists ! x$     &                                                        \\
		\bottomrule
	\end{tabular}
\end{frame}

\begin{frame}[fragile]{Glossar \textemdash\ Beweise}
	\small
	\begin{tabular}{p{0.1\textwidth} p{0.33\textwidth} p{0.45\textwidth}}
		\toprule
		Abk.     & Bedeutung                           & Was?!                                                                            \\
		\midrule
		\OE      & ohne Einschränkung                  & die Allgemeinheit der Aussage wird nicht durch getroffene Aussagen eingeschränkt \\
		o.B.d.A. & ohne Beschränkung der Allgemeinheit & wie \OE                                                                          \\
		trivial  & offensichtlich                      & Beweisschritte, welche keine weiter Begründung brauchen. (nicht verwenden!)      \\
		$\qed$   & Mic Drop                            & Kommt am Ende eines erfolgreichen Beweises                                       \\
		q.e.d.   & quod erat demonstrandum             & Was zu beweisen war                                                              \\
		\bottomrule
	\end{tabular}
\end{frame}

\begin{frame}[fragile]{Cheatsheet}
	\small
	\begin{tabular}{p{0.2\textwidth} p{0.7\textwidth}}
		\toprule
		Gestalt               & mögliches Vorgehen                                  \\
		\midrule
		nicht F               & Zeige, dass F nicht gilt.                           \\
		F und G               & Zeige F und G in zwei getrennten Beweisen.          \\
		F $\implies$ G        & Füge F in die Menge der Annahmen hinzu und zeige G. \\
		F oder G              & Zeige: nicht F $\implies$ G.                        \\&(Alternativ zeige: nicht G $\implies$ F)\\
		F $\iff$ G            & Zeige: F $\implies$ G und G $\implies$ F.           \\
		$\forall x \in A : F$ & Sei x ein beliebiges Element aus A. Zeige dann F.   \\
		$\exists x \in A : F$ & Sei x ein konkretes Element aus A. Zeige dann F.    \\
		\bottomrule
	\end{tabular}
\end{frame}