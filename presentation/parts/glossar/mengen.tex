% Copyright 2018-2022 FIUS
%
% This file is part of theo-vorkurs-folien.
%
% theo-vorkurs-folien is free software: you can redistribute it and/or modify
% it under the terms of the GNU General Public License as published by
% the Free Software Foundation, either version 3 of the License, or
% (at your option) any later version.
%
% theo-vorkurs-folien is distributed in the hope that it will be useful,
% but WITHOUT ANY WARRANTY; without even the implied warranty of
% MERCHANTABILITY or FITNESS FOR A PARTICULAR PURPOSE.  See the
% GNU General Public License for more details.
%
% You should have received a copy of the GNU General Public License
% along with theo-vorkurs-folien.  If not, see <https://www.gnu.org/licenses/>.
\begin{frame}[fragile]{Glossar \textemdash\ Mengen}
    \small
    \begin{tabular}{p{0.1\textwidth} p{0.25\textwidth} p{0.49\textwidth}}
        \toprule
        Abk.        & Bedeutung         & Was?!                                                       \\
        \midrule
        $\naturals$ & natürliche Zahlen & In der theoretischen Informatik enthält $\naturals$         \\
                    & (mit $0$)         & die $0$: $\naturals=\{0,1,2,3,\dots\}$ (Auch $\naturals_0$) \\
        $\integers$ & ganze Zahlen      & $\integers=\{\dots,-3,-2,-1,\;0,\;1,\;2,\;3,\dots\}$        \\
        $\reals$    & reelle Zahlen     & $\reals=\{$z.B. $ 2,-3,\sqrt{17},\pi,$ usw.$\}$             \\
        $\emptyset$ & \{\}              & leere Menge                                                 \\
        $a|b$       & teilt             & $a$ ist Teiler von $b$, d.h. $a$ teilt $b$ ohne Rest        \\
        $:$         & sodass            & z.B. $\forall a,b\in\integers:$ $a|b$                       \\
        \bottomrule
    \end{tabular}
\end{frame}

\begin{frame}[fragile]{Glossar \textemdash\ Mengenoperationen}
    \small
    \begin{tabular}{p{0.12\textwidth} p{0.23\textwidth} p{0.5\textwidth}}
        \toprule
        Abk.             & Bedeutung                                    & Was?!                                                                                         \\
        \midrule
        $A \subseteq B$  & Teilmenge                                    & Alle Elemente aus $A$ sind auch in $B$ enthalten. Dabei können die Mengen auch gleich sein.   \\
        $A \subsetneq B$ & echte Teilmenge                              & $A\subseteq B$. Und zusätzlich enthält $B$ Elemente, die nicht in $A$ enthalten sind.         \\
                         &                                              & $\implies$ Mengen sind nicht gleich!                                                          \\
        $A \subset B$    & Teilmenge \emph{oder} echte Teilmenge        & Bei manchen Leuten $\subseteq$, bei manchen $\subsetneq$. Mehrdeutig, lieber nicht verwenden! \\
        $A\cap B$        & Schnitt                                      & Enthält alle Elemente, die in $A$ \textit{und} in $B$ enthalten sind                          \\
        $A\cup B$        & Vereinigung                                  & Enthält alle Elemente, die in $A$, $B$ oder beiden enthalten sind                             \\
        $A\setminus B$   & Komplement, gespr. ''$A$ \textit{ohne} $B$'' & Enthält alle Elemente aus $A$, die \textit{nicht} in $B$ enthalten sind                       \\
        $\overline{B}$   & $B$ Komplement                               & Enthält alle Elemente aus einer geg. Obermenge, die \textit{nicht} in $B$ enthalten sind      \\
        \bottomrule
    \end{tabular}
\end{frame}