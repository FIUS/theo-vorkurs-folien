% Copyright 2018, 2019, 2020, 2021 FIUS
%
% This file is part of theo-vorkurs-folien.
%
% theo-vorkurs-folien is free software: you can redistribute it and/or modify
% it under the terms of the GNU General Public License as published by
% the Free Software Foundation, either version 3 of the License, or
% (at your option) any later version.
%
% theo-vorkurs-folien is distributed in the hope that it will be useful,
% but WITHOUT ANY WARRANTY; without even the implied warranty of
% MERCHANTABILITY or FITNESS FOR A PARTICULAR PURPOSE.  See the
% GNU General Public License for more details.
%
% You should have received a copy of the GNU General Public License
% along with theo-vorkurs-folien.  If not, see <https://www.gnu.org/licenses/>.

\begin{frame}[fragile]{Glossar}
	\small
	\begin{tabular}{p{0.2\textwidth} p{0.25\textwidth} p{0.4\textwidth}}
		\toprule
		Abk.                       & Bedeutung        & Was?!                                                                                                              \\
		\midrule
		\begin{tikzpicture}[->,>=stealth',shorten >=1pt,auto,node distance=1cm,semithick,baseline=(q0.base)]
			\node[initial,state](q0){$q_0$};
		\end{tikzpicture}  & Startzustand     & Hier fängt der Automat beim Lesen eines Wortes an                                                                  \\
		\begin{tikzpicture}[->,>=stealth',shorten >=1pt,auto,node distance=1.4cm,semithick,baseline=(qi.base)]
			\node[state](qi){$q_i$};
			\node[state](qj)[right of=qi]{$q_j$};
			\path (qi) edge node {$a$} (qj);
		\end{tikzpicture}  & Zustandsübergang & gibt an, welches Symbol eingelesen werden kann, um in den Folgezustand zu übergehen.                               \\
		\begin{tikzpicture}[->,>=stealth',shorten >=1pt,auto,node distance=1cm,semithick,baseline=(qe.base)]
			\node[accepting,state](qe){$q_E$};
		\end{tikzpicture} & Endzustand       & Hier kann ein fertig gelesenes Wort akzeptiert werden.                                                             \\
		\begin{tikzpicture}[->,>=stealth',shorten >=1pt,auto,node distance=2cm,semithick,baseline=(qi.base)]
			\node[state](qi){$q_f$};
			\path (qi) edge [loop right] node {$x \in \Sigma$} (B);
		\end{tikzpicture} & Fangzustand      & wird benötigt, um Determinismus zu gewährleisten. In Graphiken oft nicht eingezeichnet, ist aber da. Malt den hin. \\
		\bottomrule
	\end{tabular}
\end{frame}