% Copyright 2018, 2019, 2020, 2021 FIUS
%
% This file is part of theo-vorkurs-folien.
%
% theo-vorkurs-folien is free software: you can redistribute it and/or modify
% it under the terms of the GNU General Public License as published by
% the Free Software Foundation, either version 3 of the License, or
% (at your option) any later version.
%
% theo-vorkurs-folien is distributed in the hope that it will be useful,
% but WITHOUT ANY WARRANTY; without even the implied warranty of
% MERCHANTABILITY or FITNESS FOR A PARTICULAR PURPOSE.  See the
% GNU General Public License for more details.
%
% You should have received a copy of the GNU General Public License
% along with theo-vorkurs-folien.  If not, see <https://www.gnu.org/licenses/>.

\begin{frame}[fragile]{Glossar}
	\small
	\begin{tabular}{p{0.1\textwidth} p{0.25\textwidth} p{0.49\textwidth}}
		\toprule
		Abk.&Bedeutung&Was?!\\
		\midrule
		$\naturals$&natürliche Zahlen&In der theoretischen Informatik enthält $\naturals$ \\ 
		&(mit $0$)&die $0$: $\naturals=\{0,1,2,3,\dots\}$ (Auch $\naturals_0$)\\
		$\integers$&ganze Zahlen& $\integers=\{\dots,-3,-2,-1,\;0,\;1,\;2,\;3,\dots\}$\\
		$\rationals$&rationale Zahlen&können als Bruch dargestellt werden\\
		$\Sigma$ & Sigma& mit diesem Zeichen wird oft das Alphabet (die Menge an verwendbaren Symbolen) repräsentiert\\
		$\Sigma^\ast$&Sigma Stern&Menge aller Möglichkeiten Elemente aus $\Sigma$ hintereinander zu schreiben\\
		$\emptyWord$ & leeres Wort & Das Wort (über bel. Alphabet) mit der Länge $0$ ($|\emptyWord|$ = 0)
		$\emptyset$&\{\}&leere Menge\\
		$a|b$&teilt& $a$ ist Teiler von $b$, d.h. $a$ teilt $b$ ohne Rest\\
		$:$&sodass&z.B. $\forall a,b\in\integers:$ $a|b$\\
		$\bmod$&modulo& $a\equiv b \pmod n \iff n|(a-b)$,\\
		&& mit $a,b,n\in\integers$\\
		\bottomrule
	\end{tabular}
\end{frame}