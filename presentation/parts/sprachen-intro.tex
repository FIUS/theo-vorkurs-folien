% Copyright 2018-2022 FIUS
%
% This file is part of theo-vorkurs-folien.
%
% theo-vorkurs-folien is free software: you can redistribute it and/or modify
% it under the terms of the GNU General Public License as published by
% the Free Software Foundation, either version 3 of the License, or
% (at your option) any later version.
%
% theo-vorkurs-folien is distributed in the hope that it will be useful,
% but WITHOUT ANY WARRANTY; without even the implied warranty of
% MERCHANTABILITY or FITNESS FOR A PARTICULAR PURPOSE.  See the
% GNU General Public License for more details.
%
% You should have received a copy of the GNU General Public License
% along with theo-vorkurs-folien.  If not, see <https://www.gnu.org/licenses/>.

\begin{frame}[fragile]{Mengen}
    \begin{itemize}
        
        \item<1-|handout:1>
            Was ist eine \alert<1,2>{Menge}?
        \item<2->
            \only<2|handout:1>{
            \vspace*{0.5cm}
                Eine Menge
                \begin{itemize}
                    \item ist eine \alert{Sammlung von Zeugs}
                    \item ist unsortiert
                    \item enthält keine Duplikate
                    \item wird mit geschweiften Klammern notiert
                \end{itemize}
                
                \metroset{block=fill}
                
                \begin{exampleblock}{Beispiel}
                    $\mathbb{N} = \{0, 1, 2, 3, \dots \}$ = Menge der Natürlichen Zahlen\\
                    Studierende = \{Julian, Joel, Fabian, Noah, $\dots$\}\\
                    $\{1,2\} = \{2,1\} = \{1,1,2,1,1,1\}$\\
                    $\emptyset = \{\} =$ leere Menge
                \end{exampleblock}}
            \uncover<3-|handout:2>{
            Was ist ein \alert<3,4>{Element}?}
        \item<4->
        \only<4|handout:2>{
            \vspace*{0.5cm}
            Ein Element ist ein \alert{Ding aus einer Menge}.\\
            
            \metroset{block=fill}
                
            \begin{exampleblock}{Beispiel}
                $\mathbf{1}$ ist ein Element der \textbf{Natürlichen Zahlen}\\
                $\mathbf{1} \in \mathbb{N}$\\
                \vspace*{0.5cm}
                \textbf{Julian} ist ein Element aus der Menge der \textbf{Studierenden}\\
                \textbf{Julian} $\in$ \textbf{Studierende}\\
                \vspace*{0.5cm}
                $\mathbf{a}$ ist in der Menge $\mathbf{\{u, v, w\}}$ nicht enthalten\\
                $\mathbf{a} \notin \mathbf{\{u, v, w\}}$
            \end{exampleblock}
        }
        \uncover<5-|handout:3>{
            Was ist eine \alert<5,6>{Teilmenge}?
        }
        \item<6|handout:3>
            \vspace*{0.5cm}
            Eine Teilmenge ist eine \alert{spezielle Auswahl} von Elementen einer Menge.\\
            
            \metroset{block=fill}
            
            \begin{exampleblock}{Beispiel}
                $\{1, 2, 3\}$ ist eine Teilmenge der Natürlichen Zahlen\\
                $\{1,2,3\} \subseteq \mathbb{N}$\\
                \vspace*{0.5cm}
                \{\textbf{Julian}\} ist eine Teilmenge der \textbf{Studierenden}\\
                \{\textbf{Julian}\} $\subseteq$ \textbf{Studierende}
            \end{exampleblock}
            
    \end{itemize}
\end{frame}
\note[itemize]{
	\item Note Natürliche Zahlen: 0 ist \textbf{in Theo} Teil von $\mathbb{N}$, also auch hier im Vorkurs (in Mathe nicht)
}

\begin{frame}{Mengen - Mal anders}
    \begin{alertblock}{Ein paar Definitionen}
    Eine nichtleere Menge einstelliger Symbole nennen wir \alert{Alphabet}.
    Es wird oft dargestellt durch den Bezeichner $\Sigma$.\\
    \end{alertblock}
    \metroset{block=fill}
    \begin{exampleblock}{Beispiele}
    \begin{itemize}
        \item $\Sigma = \{a,b\}$
        \item $\Sigma = \{0,1\}$
        \item $\Sigma = \{\text{Rechts, Links, Vorwärts, Rückwärts, Start, Stopp, Pause}\}$
    \end{itemize}
    \end{exampleblock}
\end{frame}

\begin{frame}[fragile]{Mengen - Mal anders}
    \begin{alertblock}{Ein paar Definitionen}
    Auf einem Alphabet können wir die Operation $\cdot$\;, genannt \alert{Konkatenation}, ausüben.\\
    $\rightarrow$ zum Beispiel ist dann $a \cdot b = ab$\\
    Eine beliebig lange Kette an Symbolen aus dem Alphabet nennen wir ein \alert{Wort}.
    \end{alertblock}
    \metroset{block=fill}
    \begin{exampleblock}{Beispiele}
    \begin{itemize}
        \item $abba$ ist ein \emph{Wort} über dem \emph{Alphabet} $\Sigma = \{a,b\}$
        \item $10011101$ ist ein \emph{Wort} über dem \emph{Alphabet} $\Sigma = \{0,1\}$
        \item StartVorwärtsRechtsVorwärtsStopp ist ein Wort über $\Sigma = \{\text{Rechts, Links, Vorwärts, Rückwärts, Start, Stopp, Pause}\}$
    \end{itemize}
    
    \end{exampleblock}
\end{frame}

\begin{frame}{Wortlängen}
    \begin{alertblock}{Wortlänge und das leere Wort}
        Eine endlich lange Kette an Symbolen aus dem Alphabet nennen wir ein Wort.
    \end{alertblock}
    \begin{itemize}
        \item Wort der Länge 3: z.B. $aaa, aba, 110, \text{RechtsPauseStopp} \dots$
        \item Wort der Länge 2: z.B. $aa, ab, 00, \text{StartVorw"arts} \dots$
        \item Wort der Länge 1: z.B. $a, b, 1, \text{Links} \dots$
        \item Wort der Länge \alert<3>{0}: \alert<3>{$\emptyWord$}
    \end{itemize}
    \only<1-2>{Wir schreiben \alert<1>{$\absval{w}$} um \alert<1>{Länge des Wortes $w$} abzukürzen.\\}
    \only<2>{Um \alert<2>{nur ein Symbol} (z.B. $a$) zu zählen verwenden wir \alert<2>{$|w|_a$}.\\}
    \only<3->{$\emptyWord$ \emph{(\glqq Epsilon\grqq)} nennen wir das \glqq leere Wort\grqq.}
    \onslide<3->{\begin{itemize}
        \item \alert{Vergleich:} Es ist vergleichbar mit einem leerem String,\\ \qquad\qquad \,\, also: $\dq\dq=\emptyWord$
        \item \alert{Achtung:} Das leere Wort kann kein Teil eines Alphabets sein,\\\qquad\qquad da es nicht einstellig ist. Es hat die Wortl"ange $|\varepsilon|=0$
    \end{itemize}}
\end{frame}

\begin{frame}{Neutrales Element der Konkatenation}
    \begin{alertblock}{Achtung}
        Wir können bei der Konkatenation auch das leere Wort anhängen. Es verhält sich hierbei als das \alert<1>{neutrale Element}.\\
        d.h. für ein beliebiges Wort $w$, ist $w \cdot \emptyWord = \emptyWord \cdot w = w$
        \begin{exampleblock}{Vergleich}
            \begin{itemize}
                \item Bei der Addition von Zahlen ist die 0 das neutrale Element\\
                $n + 0 = 0 + n = n$
                \item Bei der Multiplikation von Zahlen ist die 1 das neutrale Element\\
                $x * 1 = 1 * x = x$
            \end{itemize}
        \end{exampleblock}
    \end{alertblock}
	\begin{alertblock}{Was heißt das?}
		Bei der Konkatenation $a \cdot a \cdot \emptyWord \cdot a$ entsteht das Wort $aaa$ mit $|aaa| = 3$.
	\end{alertblock}
\end{frame}
