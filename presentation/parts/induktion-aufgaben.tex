% Copyright 2018-2024 FIUS
%
% This file is part of theo-vorkurs-folien.
%
% theo-vorkurs-folien is free software: you can redistribute it and/or modify
% it under the terms of the GNU General Public License as published by
% the Free Software Foundation, either version 3 of the License, or
% (at your option) any later version.
%
% theo-vorkurs-folien is distributed in the hope that it will be useful,
% but WITHOUT ANY WARRANTY; without even the implied warranty of
% MERCHANTABILITY or FITNESS FOR A PARTICULAR PURPOSE.  See the
% GNU General Public License for more details.
%
% You should have received a copy of the GNU General Public License
% along with theo-vorkurs-folien.  If not, see <https://www.gnu.org/licenses/>.

{\setbeamercolor{palette primary}{bg=ExColor}
\begin{frame}[fragile]{Aufgabe}
    \metroset{block=fill}
    \begin{alertblock}{Die folgende Induktion zeigt eine seltsame Aussage.}
        Ist der Beweis korrekt geführt? Was ist passiert?
    \end{alertblock}
    \metroset{block=transparent}
    Sei $A(n)\defeq$ \emph{In einem Wald aus $n$ Bäumen haben alle die selbe Größe.}\\
    Zu zeigen: $A(n)$ gilt für alle $n \in \mathbb{N} \setminus \{0\} $.
    \begin{alertblock}{Induktionsanfang (IA)}
        $A(1)$: Aussage gilt für $n\defeq 1$, da ein Baum nur eine Größe haben kann.
    \end{alertblock}
    \begin{alertblock}{Induktionsvoraussetzung (IV)}
        Ang. $A(n)$ gilt für $n\in\mathbb{N}$ mit $n\geq1$.
    \end{alertblock}
    \begin{alertblock}{Induktionsschritt (IS)}
        Zeige Aussage gilt für alle $n+1$ unter Nutzung der IV:\\
        D.h. wir zeigen $A(n+1)=$ \emph{In einem Wald aus $n+1$ Bäumen haben alle die selbe Größe.}
    \end{alertblock}
\end{frame}
\begin{frame}[fragile]{Aufgabe}
    \footnotesize{
        \begin{alertblock}{Induktionsschritt (IS)}
            Wir betrachten einen Wald aus $n+1$ Bäumen:
            \[\underbrace{\text{\Summertree}\text{\Summertree}\text{\Summertree}\text{\Summertree}\text{\Summertree}\dots\text{\Summertree}\text{\Summertree}}_{n+1}\]
            Wir sondern einen Baum aus und betrachten den Rest. Nach I.V. haben diese alle die selbe Größe.
            \[\underbrace{\text{\Summertree}\text{\Summertree}\text{\Summertree}\text{\Summertree}\text{\Summertree}\dots\text{\Summertree}}_{n}\text{\Wintertree}\]
            Jetzt sondern wir einen anderen Baum aus.
            \[\text{\Summertree}\underbrace{\text{\Summertree}\text{\Summertree}\text{\Summertree}\text{\Summertree}\dots\text{\Summertree}\text{\Wintertree}}_{n}\]
            Die übrigen $n$ Bäume haben nach I.V. wieder die selbe Größe.
            \[\underbrace{\text{\Summertree}\text{\Summertree}\text{\Summertree}\text{\Summertree}\text{\Summertree}\dots\text{\Summertree}\text{\Summertree}}_{n+1}\]
            Also haben alle $n+1$ Bäume die selbe Größe.
            $\leadsto$ $A(n)$ gilt für alle $n$.
        \end{alertblock}
    }
\end{frame}
}

{\setbeamercolor{palette primary}{bg=ExColor}
\begin{frame}<handout:0>[fragile]{Lösungen}
    \small{
        \metroset{block=fill}
        \begin{block}{Das Problem}
            Für $A(n+1)$ wird angenommen, dass beide (Teil-)Mengen an $n$ Bäumen mindestens ein gemeinsames Element haben.
            Sie teilen dann die Größe dieses Elements.

            Das Problem ist, dass $A(1) \Rightarrow A(2)$ nicht zwangsweise erfüllt sein muss! Somit können wir keine weiteren Folgerungen über $A(n+1)$ mit $n \geq 2$ machen.

            \begin{figure}
                \resizebox{.4\textwidth}{!}{
                    \centering%
                    \begin{subfigure}{0.3\textwidth}
                        \centering%
                        {\fontsize{40}{50}\selectfont\Summertree}
                    \end{subfigure}
                    $\qquad$
                    \begin{subfigure}{0.3\textwidth}
                        \centering%
                        {\fontsize{55}{50}\selectfont\Summertree}
                    \end{subfigure}
                }
                \caption{Beide Bäume erfüllen jeweils $A(1)$, zusammen aber nicht $A(2)$}
            \end{figure}

            Denn es gibt ein überlappendes Element erst ab $n+1=3$ Bäumen:
            \[
                A(3):\alert{\rlap{$\overbrace{\phantom{\text{\Summertree\Summertree}}}^{A(2)}$}\text{\Summertree}\underbrace{\text{\Summertree\Summertree}}_{A(2)}}
                \leadsto A(4):\alert{\rlap{$\overbrace{\phantom{\text{\Summertree\Summertree\Summertree}}}^{A(3)}$}\text{\Summertree}\underbrace{\text{\Summertree\Summertree\Summertree}}_{A(3)}}
                \leadsto \dots \leadsto A(n+1): \alert{\rlap{$\overbrace{\phantom{\text{\Summertree\Summertree} ... \text{\Summertree}}}^{A(n)}$}\text{\Summertree}\underbrace{\text{\Summertree} ... \text{\Summertree\Summertree}}_{A(n)}}
            \]

        \end{block}
    }
\end{frame}
\note{
    Einordnung der Grafiken:
    \begin{itemize}
        \item In der ersten Grafik wird gezeigt, dass das Argument für $A(k)$ mit $k \geq 3$ funktioniert falls man $A(2)$ annimmt.
        \item Die zweite Grafik zeigt nun aber auf, dass man nicht einfach $A(2)$ annehmen kann!
    \end{itemize}

    \alert{Optional:} Wieso ist das aus formaler Sicht ein Problem?

    Betrachtet man die Definition von Induktion (angewandt auf unseren Beweis)
    \begin{align*}
        \underbrace{(\forall n \in \mathbb{N}_1: A(n))}_{(*)} \iff (A(1) \wedge \forall n \in \mathbb{N}_1: \underbrace{(A(n) \implies A(n+1))}_{(**)}),
    \end{align*}
    so sieht man, dass $(**)$ für $n=1$ nicht funktioniert (siehe zweite Grafik). Damit ist $(*)$ falsch!
}
}
