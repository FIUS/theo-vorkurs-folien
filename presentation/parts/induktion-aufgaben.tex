% Copyright 2018-2022 FIUS
%
% This file is part of theo-vorkurs-folien.
%
% theo-vorkurs-folien is free software: you can redistribute it and/or modify
% it under the terms of the GNU General Public License as published by
% the Free Software Foundation, either version 3 of the License, or
% (at your option) any later version.
%
% theo-vorkurs-folien is distributed in the hope that it will be useful,
% but WITHOUT ANY WARRANTY; without even the implied warranty of
% MERCHANTABILITY or FITNESS FOR A PARTICULAR PURPOSE.  See the
% GNU General Public License for more details.
%
% You should have received a copy of the GNU General Public License
% along with theo-vorkurs-folien.  If not, see <https://www.gnu.org/licenses/>.

{\setbeamercolor{palette primary}{bg=ExColor}
	\begin{frame}[fragile]{Aufgabe}
		\metroset{block=fill}
		\begin{alertblock}{Die folgende Induktion zeigt eine seltsame Aussage.}
			Ist der Beweis korrekt geführt? Was ist passiert?
		\end{alertblock}
		\metroset{block=transparent}
		Sei $A(n)\defeq$ \emph{In einer Herde aus $n$ Telefonen haben alle die selbe Farbe.}\\
		Zu zeigen: $A(n)$ gilt für alle $n \in \mathbb{N} \setminus \{0\} $.
		\begin{alertblock}{Induktionsanfang (IA)}
			$A(1)$: Aussage gilt für $n\defeq 1$, da ein Telefon nur eine Farbe haben kann.
		\end{alertblock}
		\begin{alertblock}{Induktionsvorraussetzung (IV)}
			Ang. $A(n)$ gilt für $n\geq1$.
		\end{alertblock}
		\begin{alertblock}{Induktionsschritt (IS)}
			Zeige Aussage gilt für alle $n+1$ unter Nutzung der IV:\\
			D.h. wir zeigen $A(n+1)=$ \emph{In einer Herde aus $n+1$ Telefonen haben alle die selbe Farbe.}
		\end{alertblock}
	\end{frame}
	\begin{frame}[fragile]{Aufgabe}
		\footnotesize{
			\begin{alertblock}{Induktionsschritt (IS)}
				Wir betrachten eine Herde aus $n+1$ Telefonen:
				\[\underbrace{\text{\Telefon}\text{\Telefon}\text{\Telefon}\text{\Telefon}\text{\Telefon}\dots\text{\Telefon}\text{\Telefon}}_{n+1}\]
				Wir sondern ein Telefon aus und betrachten den Rest. Nach I.V. haben diese alle die selbe Farbe.
				\[\underbrace{\alert{\text{\Telefon}\text{\Telefon}\text{\Telefon}\text{\Telefon}\text{\Telefon}\dots\text{\Telefon}}}_{n}\text{\Telefon}\]
				Jetzt sondern wir ein anderes Telefon aus.
				\[\alert{\text{\Telefon}}\underbrace{\alert{\text{\Telefon}\text{\Telefon}\text{\Telefon}\text{\Telefon}\dots\text{\Telefon}}\text{\Telefon}}_{n}\]
				Die übrigen $n$ Telefone haben nach I.V. wieder die selbe Farbe.
				\[\underbrace{\alert{\text{\Telefon}\text{\Telefon}\text{\Telefon}\text{\Telefon}\text{\Telefon}\dots\text{\Telefon}\text{\Telefon}}}_{n+1}\]
				Also haben alle $n+1$ Telefone die selbe Farbe.
				$\leadsto$ $A(n)$ gilt für alle $n$.
			\end{alertblock}
		}
	\end{frame}
}

{\setbeamercolor{palette primary}{bg=ExColor}
	\begin{frame}<handout:0>[fragile]{Lösungen}
		\small{
			\metroset{block=fill}
			\begin{block}{Das Problem}
				Die Vorangehensweise erfordert, dass die betrachteten Mengen an $n-1$ Telefonen mindestens ein gemeinsames Element haben. Sie teilen dann die Farbe dieses Elements. Allerdings gibt es ein überlappendes Element erst ab $n=3$ Telefonen:
				\[
					A(2):\alert{\text{\Telefon}\text{\Telefon}} \color{black} \leadsto 
					A(3):\alert{\rlap{$\overbrace{\phantom{\text{\Telefon\Telefon}}}^{A(2)}$}\text{\Telefon}\underbrace{\text{\Telefon\Telefon}}_{A(2)}}
					\leadsto \dots \leadsto A(n): \alert{\rlap{$\overbrace{\phantom{\text{\Telefon\Telefon} ... \text{\Telefon}}}^{A(n-1)}$}\text{\Telefon}\underbrace{\text{\Telefon} ... \text{\Telefon\Telefon}}_{A(n-1)}}
				\]
				Das Problem ist nun, dass $A(2)$ nicht zwangsweise erfüllt sein muss! Somit ist die Voraussetzung $A(2)$ verletzt und wir können keine weiteren Folgerungen über $A(n)$ machen.

				\begin{figure}
				\resizebox{.4\textwidth}{!}{
					\centering%
					\begin{subfigure}{0.3\textwidth}
						\centering%
						\includegraphics[height=0.5in]{../figures/telephoneGreen.png}
					\end{subfigure}
					$\qquad$
					\begin{subfigure}{0.3\textwidth}
						\centering%
						\includegraphics[height=0.5in]{../figures/telephoneWhite.png}
					\end{subfigure}
				}
					\caption{Beide Telefone erfüllen jeweils $A(1)$, zusammen aber nicht $A(2)$}
				\end{figure}

			\end{block}
		}
	\end{frame}
}
