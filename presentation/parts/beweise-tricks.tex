% Copyright 2018-2022 FIUS
%
% This file is part of theo-vorkurs-folien.
%
% theo-vorkurs-folien is free software: you can redistribute it and/or modify
% it under the terms of the GNU General Public License as published by
% the Free Software Foundation, either version 3 of the License, or
% (at your option) any later version.
%
% theo-vorkurs-folien is distributed in the hope that it will be useful,
% but WITHOUT ANY WARRANTY; without even the implied warranty of
% MERCHANTABILITY or FITNESS FOR A PARTICULAR PURPOSE.  See the
% GNU General Public License for more details.
%
% You should have received a copy of the GNU General Public License
% along with theo-vorkurs-folien.  If not, see <https://www.gnu.org/licenses/>.

%\subsection{Tricks}
\begin{frame}[fragile]{Tricks: Fallunterscheidung}
	\begin{alertblock}{Hilfe! Der Beweis ist zu komplex! Was nun?}
		Manchmal lässt sich ein Beweis in kleinere Aussagen zerlegen. Wenn wir alle Teilaussagen beweisen, haben wir die Gesamtaussage gezeigt.
	\end{alertblock}
	\metroset{block=fill}
	\small\begin{exampleblock}{Beispiel}
		Z.z. für alle $n\in\mathbb{N}$ gilt, dass der Rest von $\frac{n^2}{4}$ entweder 0 oder 1 ist.
		\footnotesize\begin{itemize}
			\item
			      \alert{Fall 1:} n ist gerade\\
			      $n^2=n\cdot n\overset{\text{n gerade}}{=\joinrel=\joinrel=\joinrel=\joinrel=}(2k)\cdot(2k) = 4k^2$,  mit $k\in\mathbb{Z}$\\
			      $\implies \frac{4k^2}{4} = k^2$ Rest: 0
			\item \alert{Fall 2:} n ist ungerade\\
			      $n^2=n\cdot n\overset{\text{n ungerade}}{=\joinrel=\joinrel=\joinrel=\joinrel=\joinrel=}(2k+1)\cdot(2k+1)=(2k)^2+2(2k)+1=4(k^2+k)+1$, \\mit $k\in\mathbb{Z}$\\
			      $\implies \frac{4(k^2+k)+1}{4}= k^2+k$ Rest: 1
		\end{itemize}
		Da $n$ nur gerade oder ungerade sein kann, ist der Rest von $\frac{n^2}{4}$ \\entweder 0 oder 1. \qed\;
	\end{exampleblock}
\end{frame}

\begin{frame}[fragile]{Tricks: Beispiele und Gegenbeispiele}
	\begin{alertblock}{Reicht nicht auch ein Beispiel als Beweis?}
		% Manchmal\dots
	\end{alertblock}
	\metroset{block=fill}
	\begin{block}{Wann ein Beispiel \emph{nicht} ausreicht:}
		Zeige allgemeine Aussagen, also Aussagen der Form:\\$\forall n\in\mathbb{N}$ gilt \dots, $\neg\exists n\in\mathbb{N}$\dots, $\exists!n\in\mathbb{N}$\dots, etc.\\
		\alert{Warum nicht?}\\
		Beispiele zeigen uns nur endlich viele Möglichkeiten.\\
		\glqq für alle gilt\dots\grqq, \glqq es existiert kein\dots\grqq, \glqq es existiert genau ein\dots\grqq, etc. \\sind meist zu allgemeine Aussagen um sie mit endlich vielen Beispielen lückenlos zu beweisen.
	\end{block}
\end{frame}

\begin{frame}[fragile]{Tricks: Beispiele und Gegenbeispiele}
	\begin{alertblock}{Reicht nicht auch ein Beispiel als Beweis?}
		% Manchmal\dots
	\end{alertblock}
	\metroset{block=fill}
	\begin{block}{Wann ein Beispiel ausreichen kann:}
		Zeige nicht allgemeine Aussagen der Form:\\
		$\exists n\in\mathbb{N}$, $\neg\forall n\in\mathbb{N}$ gilt, \dots\\
		\alert{Warum?}\\
		\glqq es gibt ein Element, sodass\dots\grqq, \glqq für nicht alle Element gilt\dots\grqq\\wären durch Angabe eines solchen Elements gezeigt.
	\end{block}
	$\leadsto$ will man zeigen, dass eine Aussage falsch ist, sind die Formen entsprechend negiert.
\end{frame}
