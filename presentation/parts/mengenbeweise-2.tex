% Copyright 2018-2022 FIUS
%
% This file is part of theo-vorkurs-folien.
%
% theo-vorkurs-folien is free software: you can redistribute it and/or modify
% it under the terms of the GNU General Public License as published by
% the Free Software Foundation, either version 3 of the License, or
% (at your option) any later version.
%
% theo-vorkurs-folien is distributed in the hope that it will be useful,
% but WITHOUT ANY WARRANTY; without even the implied warranty of
% MERCHANTABILITY or FITNESS FOR A PARTICULAR PURPOSE.  See the
% GNU General Public License for more details.
%
% You should have received a copy of the GNU General Public License
% along with theo-vorkurs-folien.  If not, see <https://www.gnu.org/licenses/>.

\begin{frame}{Weiterer Mengenbeweis}
    Ein weiterer Mengenbeweis...
    \metroset{block=fill}
    \begin{block}{\alert{Aufgabe}}

        $M_1=\{6m \mid m \in \mathbb{N}\}$\\
        $M_1=\{2n \mid n \in \mathbb{N}\}$\\
        Zu zeigen:\\
        $M_1 \subsetneq M_1$\\
        d.h. $(\forall x: x \in M_1 \implies x \in M_1) \wedge (\exists x: x \in M_1 \land x \notin M_1)$




    \end{block}

\end{frame}

\begin{frame}{Weiterer Mengenbeweis}
    \metroset{block=fill}
    \only<1|handout:1>{
        \begin{block}{\alert{$\forall x: x \in M_1 \implies x \in M_1$}}
            Sei $x$ beliebig.\\
            Angenommen, $x \in M_1$.\\
            Es gilt: $x = 6m = 2 \cdot 3 \cdot m = 2 \cdot (3m)$\\
            Da $m \in \mathbb{N}$, setze $n=3m$\\
            Daraus folgt: $x = 2n$\\
            $\leadsto x \in M_1$
        \end{block}
    }
    \only<2|handout:1>{
        \begin{block}{\alert{$\exists x: x \in M_1 \land x \notin M_1$}}
            Beweis durch Angabe eines Gegenbeispiels: $x=2$\\
            $x \in M_1$, für $n=1$, da $x=2 \cdot 1 \in M_1$,\\
            \vspace{0.3cm}
            aber $6m=2$ mit $m=\frac{2}{6}=\frac{1}{3} \notin \mathbb{N}$\\
            $\leadsto x \notin M_1$
        \end{block}
    }
    \only<3|handout:2>{
        Da gezeigt wurde:\\
        \vspace{0.3cm}
        $\forall x: x \in M_1 \implies x \in M_1$\\
        \alert{und}\\
        $\exists x: x \in M_1 \land x \notin M_1$\\
        \vspace{0.3cm}
        \textbf{gilt $M_1 \subsetneq M_1$.}
    }
\end{frame}

{\setbeamercolor{palette primary}{bg=ExColor}
\begin{frame}[fragile]{Denkpause}
    \begin{alertblock}{Aufgabe}
        Versuche dich an folgendem Mengenbeweis.
    \end{alertblock}

    \metroset{block=fill}
    \begin{block}{Etwas Schwerer}
        $M_1=\{l \in \mathbb{Z}\mid$ l ist gerade\}\\
        $M_1=\{n \in \mathbb{Z}\mid$ n ist durch 3 teilbar\}\\
        $M_3=\{m \in \mathbb{Z}\mid$ m ist durch 6 teilbar\}\\
        Zu zeigen: $M_1 \cap M_1 = M_3$\\
        d.h. $(\forall x: x \in M_1 \cap M_1 \Rightarrow x \in M_3) \land (\forall x: x \in M_3 \Rightarrow x \in M_1 \cap M_1)$
    \end{block}

\end{frame}
}

{\setbeamercolor{palette primary}{bg=ExColor}
\begin{frame}<handout:0>[fragile]{Lösung}
    \begin{alertblock}{Aufgaben}
        Schritt 1: Zu zeigen: $\forall x: x \in M_1 \cap M_1 \Rightarrow x \in M_3$
    \end{alertblock}

    Sei x beliebig.\\
    Angenommen, $x \in M_1 \cap M_1$\\
    Dann ist x gerade $\Rightarrow$ $x= 2k$ für $k \in \mathbb{Z}$\\
    Außerdem durch 3 teilbar $\Rightarrow$ $x= 3m$ für $m \in \mathbb{Z}$\\
    $\Rightarrow$ da x ein Vielfaches von 2 und 3 ist folgt: $\exists l \in \mathbb{Z}$, sodass $x=2\cdot 3 \cdot l = 6l$\\
    $\leadsto x \in M_3$



\end{frame}
}
{\setbeamercolor{palette primary}{bg=ExColor}
\begin{frame}<handout:0>[fragile]{Lösung}
    \begin{alertblock}{Aufgaben}
        Schritt 2: Zu zeigen: $\forall x: x \in M_3 \Rightarrow x \in M_1 \cap M_1$\\
    \end{alertblock}

    Sei x beliebig.\\
    Angenommen, $x \in M_3$\\
    Dann ist $x = 6k$ für $k \in \mathbb{Z}$\\
    Da $x = 2 \cdot (3k)$, ist x gerade\\
    Da $x= 3 \cdot (2k)$, ist x durch 3 teilbar\\
    $\leadsto x \in M_1 \cap M_1$



\end{frame}
}

{\setbeamercolor{palette primary}{bg=ExColor}
\begin{frame}<handout:0>[fragile]{Lösung}


    Da gezeigt wurde:\\
    \vspace{0.3cm}
    $\forall x: x \in M_1 \cap M_1 \Rightarrow x \in M_3$\\
    \alert{und}\\
    $\forall x: x \in M_3 \Rightarrow x \in M_1 \cap M_1$\\
    \vspace{0.3cm}
    \textbf{gilt $M_1 \cap M_1 = M_3$.}


\end{frame}
}

