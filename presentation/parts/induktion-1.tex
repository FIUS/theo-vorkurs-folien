% Copyright 2018-2022 FIUS
%
% This file is part of theo-vorkurs-folien.
%
% theo-vorkurs-folien is free software: you can redistribute it and/or modify
% it under the terms of the GNU General Public License as published by
% the Free Software Foundation, either version 3 of the License, or
% (at your option) any later version.
%
% theo-vorkurs-folien is distributed in the hope that it will be useful,
% but WITHOUT ANY WARRANTY; without even the implied warranty of
% MERCHANTABILITY or FITNESS FOR A PARTICULAR PURPOSE.  See the
% GNU General Public License for more details.
%
% You should have received a copy of the GNU General Public License
% along with theo-vorkurs-folien.  If not, see <https://www.gnu.org/licenses/>.

\subsubsection{Idee}
\begin{frame}[fragile]{Idee}
\begin{columns}
\column{0.5\textwidth}
    \begin{alertblock}{Zeige Aussagen der Form:\\\emph{Für alle $n\in\mathbb{N}$ gilt\ldots}}
    \begin{enumerate}
        \item Zeige Aussage für das kleinste Element
        \item<1-> \only<7,8|handout:0>{\alert<7>{Zeige, wenn Aussage für beliebiges $n$ gilt, gilt sie auch für dessen Nachfolger, also $n+1$.}}\onslide<1-6>{Zeige, dass Aussage auch für das folgende Element gilt.}
        \item<2-6,8> \only<8|handout:0>{\alert<8>{$\leadsto$ Aussage gilt für alle $n$.}}\onslide<2-6>{\small Zeige, dass Aussage auch für das folgende Element gilt.}
        \item<3-6> \footnotesize Zeige, dass Aussage auch für das folgende Element gilt.
        \item<4-6> \scriptsize Zeige, dass Aussage auch für das folgende Element gilt.
        \item<5-6> \tiny Zeige, dass Aussage auch für das folgende Element gilt.
        \item<6> \dots
    \end{enumerate}
    \end{alertblock}
\column{0.5\textwidth}
    \begin{figure}
        \centering
        \includegraphics[width=0.7\textwidth]{../figures/induction.png}
        %\caption{Idee}
        %
    \end{figure}
\end{columns}
\end{frame}

\subsubsection{Funktionsweise}
\begin{frame}[fragile]{Struktur}
    \begin{alertblock}{Zeige Aussagen der Form:\\\emph{Für alle $n\in\mathbb{N}$ gilt\ldots}}
    \begin{enumerate}
        \item \alert{Induktionsanfang}\\Zeige Aussage für das kleinste Element
        \item \alert{Induktionsvoraussetzung}\\Zeige, unter der Voraussetzung: \\\emph{die Aussage gelte für beliebiges $n$},\dots
        \item \alert{Induktionsschritt}\\\dots dann gilt die Aussage auch für dessen Nachfolger $n+1$.
        \item $\leadsto$ Aussage gilt für alle $n \in \mathbb{N}$.
    \end{enumerate}
    \end{alertblock}
\end{frame}

\begin{frame}[fragile]{Beispiel}
\center $\displaystyle\sum_{i = 0}^{n} (2i+1) = (n+1)^2,\quad\forall n \in\mathbb{N}$.
    \begin{figure}
        \centering
        \includegraphics[width=0.5\textheight]{../figures/Summe.png}\qquad \dots
        %\caption{Idee}
        %
    \end{figure}
\end{frame}
\note[itemize]{
    \item Idee: Es werden 1, 3, 5, \ldots Felder hinzugenommen
    \item Lässt sich jeweils zu einem Quadrat zusammensetzen
    \item Mit jedem Schritt wird Kantenlänge des Quadrats um eins größer
}

\begin{frame}[fragile]{Beispiel}
Zeigen Sie $\displaystyle\sum_{i = 0}^{n} (2i+1) = (n+1)^2,\quad\forall n \in\mathbb{N}$.
\begin{alertblock}{Induktionsanfang (IA)}
    Zeige Aussage gilt für $n\defeq0$:\\
    \begin{align*}
        &\sum_{i = 0}^{0} (2i+1)& &\overset{!}{=}& &(0+1)^2&\\
        \iffspace &2 \cdot 0 + 1& &\overset{!}{=}& &1^2&\\
        \iffspace &1& &=& &1& \qquad\checkmark
    \end{align*}
\end{alertblock}
\end{frame}

\begin{frame}[fragile]{Beispiel}
Zeigen Sie $\displaystyle\sum_{i = 0}^{n} (2i+1) = (n+1)^2,\quad\forall n \in\mathbb{N}$.
\begin{alertblock}{Induktionsanfang (IA)}
    Aussage gilt für $n\defeq0$, da $\displaystyle\sum_{i = 0}^{0} (2i+1) = (0+1)^2$.
\end{alertblock}
\begin{alertblock}{Induktionsvoraussetzung (IV)}
    Ang. Aussage gilt für $n \in\mathbb{N}$.
\end{alertblock}
\begin{alertblock}{Induktionsschritt (IS)}
    Zeige Aussage gilt für alle $n+1$ unter Nutzung der (IV):\par
    $\displaystyle\sum_{i = 0}^{\alert{n+1}} (2i+1) \overset{!}{=} (\alert{(n+1)}+1)^2$
\end{alertblock}
\end{frame}

\begin{frame}[fragile]{Beispiel}
\small\begin{alertblock}{Induktionsschritt}
    Zeige Aussage gilt für alle $n+1$ unter Nutzung der IV:\@
    \begin{align*}
        \onslide<1->{&\sum_{i = 0}^{n+1} (2i+1)& &\overset{!}{=}& &((n+1)+1)^2&}\\
        \onslide<2->{\iffspace &\sum_{i = 0}^{\alert<2>{n}} (2i+1) + \sum_{i = \alert<2>{n+1}}^{n+1} (2i+1)& &\overset{!}{=}& &(n+2)^2&}\\
        \onslide<3->{\iffspace &\sum_{i = 0}^{n} (2i+1) + ( 2(n+1)+1 )& &\overset{!}{=}& &n^2 + 2 \cdot 2n + 2^2&}\\
        \onslide<4->{\overset{\alert<4>{IV}}\iffspace &\alert<4>{(n+1)^2} + ( 2(n+1)+1 )& &\overset{!}{=}& &n^2+4n+4&}\\
        \onslide<5->{\iffspace &n^2+2n+1^2+2n+2+1& &\overset{!}{=}& &n^2+4n+4&}\\
        \onslide<6>{\iffspace &n^2+4n+4& &\alert<6>{=}& &n^2+4n+4&}
    \end{align*}
\end{alertblock}
\end{frame}

\begin{frame}[fragile]{Beispiel}
Zeigen Sie $\displaystyle\sum_{i = 0}^{n} (2i+1) = (n+1)^2, \quad\forall n \in\mathbb{N}$.
\begin{alertblock}{Induktionsanfang (IA)}
    Aussage gilt für $n\defeq0$, da $\displaystyle\sum_{i = 0}^{0} (2i+1) = 1^2$.
\end{alertblock}
\begin{alertblock}{Induktionsvoraussetzung (IV)}
    Ang. Aussage gilt für ein (beliebiges aber festes) $n \in\mathbb{N}$.
\end{alertblock}
\begin{alertblock}{Induktionsschritt (IS)}
    Aussage gilt für alle $n+1$ unter Nutzung der IV, da\par
    $\displaystyle\sum_{i = 0}^{n+1} (2i+1) = ((n+1)+1)^2$
\end{alertblock}
\alert{$\leadsto$ Aussage gilt für alle n.}\qed
\end{frame}


{\setbeamercolor{palette primary}{bg=ExColor}
\begin{frame}[fragile]{Denkpause}
    \begin{alertblock}{Aufgaben}
    Versuche dich an den folgenden Induktionsbeweisen.
    \end{alertblock}

    \metroset{block=fill}
    \begin{block}{Normal}
        $\displaystyle\sum_{i=0}^{n} i = \frac{n(n+1)}{2}, \quad \forall n \in \mathbb{N}$
    \end{block}
    \begin{block}{Schwerer}
        $\displaystyle\prod_{i=1}^{n} 4^i = 2^{n(n+1)}, \quad \forall n \in \mathbb{N}\setminus \{0\}$
    \end{block}
\end{frame}
}

%\subsubsection{Lösungen normal}
{\setbeamercolor{palette primary}{bg=ExColor}
\begin{frame}<handout:0>[fragile]{Lösungen: normale Aufgabe}
    Zu zeigen: $\displaystyle\sum_{i=0}^{n} i = \frac{n(n+1)}{2}$ gilt für alle $n \in \mathbb{N}$.
    \begin{alertblock}{Induktionsanfang (IA)}
        Aussage gilt für $n\defeq 0$, da $\displaystyle\sum_{i=0}^{0} i = 0 = \frac{0(0+1)}{2}$.
    \end{alertblock}
    \begin{alertblock}{Induktionsvoraussetzung (IV)}
        Ang. Aussage gilt für $n \in\mathbb{N}$.
    \end{alertblock}
    \begin{alertblock}{Induktionsschritt (IS)}
        Zeige Aussage gilt für alle $n+1$ unter Nutzung der IV:\par
        $\displaystyle\sum_{i=0}^{\alert{n+1}} i \overset{!}{=} \frac{(\alert{n+1})\left((\alert{n+1})+1\right)}{2}$
    \end{alertblock}
\end{frame}


\begin{frame}<handout:0>[fragile]{Lösungen: normale Aufgabe}
\small\begin{alertblock}{Induktionsschritt}
    Zeige Aussage gilt für $n+1$ unter Nutzung der I.V.:
    \begin{align*}
        \onslide<1->{&\displaystyle\sum_{i=0}^{\alert<1>{n+1}} i& &\overset{!}{=}& &\frac{(\alert<1>{n+1})\left((\alert<1>{n+1})+1\right)}{2}&}\\
        \onslide<2->{\iffspace &\left(\displaystyle\sum_{i=0}^{n} i\right)+(n+1)& &\overset{!}{=}& &\frac{(n+1)(n+2)}{2}&}\\
        \onslide<3->{\iffspace &\left(\displaystyle\sum_{i=0}^{n} i\right)+(n+1)& &\overset{!}{=}& &\frac{n^2+3n+2}{2}&}\\
        \onslide<4->{\overset{\alert<4>{IV}}\iffspace &\alert<4>{\frac{n(n+1)}{2}}+(n+1)& &\overset{!}{=}& &\frac{n^2+3n+2}{2}&}\\
        \onslide<5->{\iffspace &\frac{n^2+n}{2}+\frac{2n+2}{2}& &\overset{!}{=}& &\frac{n^2+3n+2}{2}&}\\
        \onslide<6->{\iffspace &\frac{n^2+3n+2}{2}& &\alert{=}& &\frac{n^2+3n+2}{2}&}\\
    \end{align*}
\end{alertblock}
\end{frame}


\begin{frame}<handout:0>[fragile]{Lösungen: normale Aufgabe}
    Zu zeigen: $\displaystyle\sum_{i=0}^{n} i = \frac{n(n+1)}{2}$ gilt für alle $n \in \mathbb{N}$.
    \begin{alertblock}{Induktionsanfang (IA)}
        Aussage gilt für $n\defeq 0$, da $\displaystyle\sum_{i=0}^{0} i = 0 = \frac{0(0+1)}{2}$.
    \end{alertblock}
    \begin{alertblock}{Induktionsvoraussetzung (IV)}
        Ang. Aussage gilt für $n \in\mathbb{N}$.
    \end{alertblock}
    \begin{alertblock}{Induktionsschritt (IS)}
        Zeige Aussage gilt für $n+1$ unter Nutzung der IV:\par
        $\displaystyle\sum_{i=0}^{\alert{n+1}} i \overset{!}{=} \frac{(\alert{n+1})\left((\alert{n+1})+1\right)}{2}$ gilt für alle $n \in \mathbb{N}$
    \end{alertblock}
    \alert{$\leadsto$ Aussage gilt für alle $n$.}\qed
\end{frame}
}

% \begin{frame}[standout]
%   Fragen dazu?
% \end{frame}

%\subsubsection{Lösungen schwerer}
{\setbeamercolor{palette primary}{bg=ExColor}
\begin{frame}<handout:0>[fragile]{Lösungen: schwerere Aufgabe}
    Zu zeigen: $\displaystyle\prod_{i=1}^{n} 4^i = 2^{n(n+1)}$ gilt für alle $n \in \mathbb{N}\setminus \{0\}$.
    \begin{alertblock}{Induktionsanfang (IA)}
        Aussage gilt für $n\defeq 1$, da $\displaystyle\prod_{i=1}^{1} 4^i = 4^1 = 4 = 2^2 = 2^{1(1+1)}$.
    \end{alertblock}
    \begin{alertblock}{Induktionsvoraussetzung (IV)}
        Ang. Aussage gilt für $n \in\mathbb{N}\setminus \{0\}$.
    \end{alertblock}
    \begin{alertblock}{Induktionsschritt (IS)}
        Zeige Aussage gilt für $n+1$ unter Nutzung der IV:\par
        $\displaystyle\prod_{i=1}^{\alert{n+1}} 4^i \overset{!}{=} 2^{(\alert{n+1})((\alert{n+1})+1)}$
    \end{alertblock}
\end{frame}

\begin{frame}<handout:0>[fragile]{Lösungen: schwerere Aufgabe}
\small\begin{alertblock}{Induktionsschritt}
    Zeige Aussage gilt für $n+1$ unter Nutzung der I.V.:
    \begin{align*}
        \onslide<1->{&\displaystyle\prod_{i=1}^{\alert<1>{n+1}} 4^i& &\overset{!}{=}& &2^{(\alert<1>{n+1})((\alert<1>{n+1})+1)}&}\\
        \onslide<2->{\iffspace &\left(\displaystyle\prod_{i=1}^{n} 4^i\right) \cdot 4^{(n+1)}& &\overset{!}{=}& &2^{(n+1)(n+2)}&}\\
        \onslide<3->{\overset{\alert<3>{IV}}\iffspace &\alert<3>{\left(2^{n(n+1)}\right)} \cdot 4^{(n+1)}& &\overset{!}{=}& &2^{n^2+3n+2}&}\\
        \onslide<4->{\iffspace &2^{n^2+n} \cdot 2^{2(n+1)}& &\overset{!}{=}& &2^{n^2+3n+2}&}\\
        \onslide<5->{\iffspace &2^{n^2+n} \cdot 2^{2n+2}& &\overset{!}{=}& &2^{n^2+3n+2}&}\\
        \onslide<6->{\iffspace &2^{(n^2+n)+(2n+2)}& &\overset{!}{=}& &2^{n^2+3n+2}&}\\
        \onslide<7->{\iffspace &2^{n^2+3n+2}& &\alert{=}& &2^{n^2+3n+2}&}
    \end{align*}
\end{alertblock}
\end{frame}


\begin{frame}<handout:0>[fragile]{Lösungen: schwerere Aufgabe}
     Zu zeigen: $\displaystyle\prod_{i=1}^{n} 4^i = 2^{n(n+1)}$ gilt für alle $n \in \mathbb{N}\setminus \{0\}$.
    \begin{alertblock}{Induktionsanfang (IA)}
        Aussage gilt für $n\defeq 1$, da $\displaystyle\prod_{i=1}^{1} 4^i = 4^1 = 4 = 2^2 = 2^{1(1+1)}$.
    \end{alertblock}
    \begin{alertblock}{Induktionsvoraussetzung (IV)}
        Ang. Aussage gilt für $n \in\mathbb{N}\setminus \{0\}$.
    \end{alertblock}
    \begin{alertblock}{Induktionsschritt (IS)}
        Zeige Aussage gilt für $n+1$ unter Nutzung der IV:\par
        $\displaystyle\prod_{i=1}^{\alert{n+1}} 4^i \overset{!}{=} 2^{(\alert{n+1})((\alert{n+1})+1)}$ gilt für alle $n \in \mathbb{N}\setminus \{0\}$
    \end{alertblock}
    \alert{$\leadsto$ Aussage gilt für alle $n$.}\qed
\end{frame}
}

% \begin{frame}[standout]
%   Fragen dazu?
% \end{frame}
