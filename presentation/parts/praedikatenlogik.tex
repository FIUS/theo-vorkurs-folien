\subsection{Motivierendes Beispiel}
\begin{frame}[fragile]{Prädikatenlogik among us}
    \begin{itemize}
        \item<1-> Ihr seid auf einem Raumschiff
        \item<2-> Mit eurer Crew 
        $$
        U = \{\sus{red}, \sus{green}, \sus{blue}, \sus{white}, \sus{yellow}, \sus{magenta}\}
        $$
        \item<3-> Aber manche von euch sind Imposter
        
    \end{itemize}
    \only<2>{\hspace{3cm}\includegraphics[width=4.5cm]{../figures/amongus_crew.png}} 
    \only<3>{\hspace{3cm}\includegraphics[width=4.5cm]{../figures/amongus_imposter.jpg}} 
\end{frame}

\begin{frame}[fragile]{Prädikatenlogik among us}
    \begin{itemize}
        \item<1-> Wir brauchen eine Funktion um Imposter zu erkennen
        $$C: U \to \{wahr, falsch\}$$
        \item<2-> Und eine Funktion um zu beschreiben ob ein Crewmitglied noch lebt
        $$A: U \to \{wahr, falsch\}$$
        \item<3-> Und eine Funktion um zu beschreiben was ein Imposter macht wenn er ein Crewmitglied findet
        $$K: U\times U \to \{wahr, falsch\}$$
    \end{itemize}
\end{frame}


\begin{frame}[fragile]{Eine kurze Geschichte}
$$
\overline{C(\sus{red})}\wedge C(\sus{white})\wedge K(\sus{red},\sus{white})\implies \overline{A(\sus{white})}
$$
\end{frame}

\begin{frame}[fragile]{Aber wer ist den Imposter?}
\begin{itemize}
    \item<1-> \alert{Problem:} Wer ist den Imposter?
    \item<2-> Wir wissen nicht welche Crewmitglieder wir für was einsetzen können 
    \item<3-> Also müssen wir allgemeiner werden:
    $$
    \exists x,z \in U: \overline{C(x)}\wedge C(y)\wedge K(x,y)\implies \overline{A(y)}
    $$
\end{itemize}
\begin{center}
\only<4>{Das ist eine prädikatenlogische Formel}
\end{center}
\end{frame}

\subsection{Definitionen}

\begin{frame}{Prädikatenlogik erster Ordnung}
\begin{itemize}
    \item<1-> \alert{Variablen} sind wie ihr sie aus der Aussagenlogik bereits kennt.
    \item<2-> $U$ ist unser \alert{Universum}. Es enthält alle Elemente die wir in unsere Variablen einsetzen dürfen 
    \item<3-> \alert{Prädikatensymbole} sind Mappings von Elementen aus $U$ auf $wahr$ oder $falsch$.
    Sie werden normal groß geschrieben.
    \item<4-> \alert{Funktionssymbole} sind Mappings von Elementen aus $U$ auf andere Elemente aus $U$.
    Sie werden normal klein geschrieben.
    \item<5> Eine \alert{Interpretation} $I$ ist ein Lösungsvorschlag für die prädikatenlogische Formel.
    Es ist ein Mapping von Variablen auf Elemente des Universums, der Prädikatensymbole auf Prädikate und der Funktionsformeln auf Funktionen.
\end{itemize}
\end{frame}

\begin{frame}{Unser Beispiel}
    \begin{itemize}
        \item $U = \{\sus{red}, \sus{green}, \sus{blue}, \sus{white}, \sus{yellow}, \sus{magenta}\}$
        \item Prädikate:
        \begin{itemize}
            \item $P_{sus}(x,y)$: $x$ verdächtigt $y$
            \item $P_{kill}(x,y)$: $x$ tötet $y$
            \item $P_{seen}(x,y)$: $x$ und $y$ wurden zusammen gesehen
            \item $P_{alive}(x)$: $x$ lebt (noch)
            \item $P_{crew}(x)$: $x$ ist ein echtes Crewmitglied
        \end{itemize}
    \end{itemize}
\end{frame}

{\setbeamercolor{palette primary}{bg=ExColor}
\begin{frame}[fragile]{Denkpause}
    \footnotesize
        \begin{alertblock}{Aufgaben}
            Findet Interpreatationen für die folgenden Prädikatenlogischen Formeln in unserem Beispiel:
        \end{alertblock}
        \metroset{block=fill}
        \begin{block}{Normal}
            \begin{itemize}
                \item $P_1(x,y) \Leftrightarrow P_2(y,x)$
                \item $(P_1(x,y) \wedge \overline{P_2(y)}) \implies P_3(y)$
            \end{itemize}
        \end{block}
        \begin{block}{Etwas Schwerer}
            \begin{itemize}
                \item $(P_1(x,y) \wedge P_2(y)) \implies P_3(x)$
            \end{itemize}
        \end{block}
        \alert{Hinweis:} In dem Spiel Among Us gibt es viele Gründe etwas zu tun oder nicht zu tun.
        Verschwendet nicht zu viel Zeit die Aktionen zu überdenken. Wenn ihr argumentieren könnt warum jemand agiert wird es schon passen.
\end{frame}

\begin{frame}{Lösung}
    $$(P_1(x,y) \wedge \overline{P_2(y)}) \implies P_3(y)$$
        \begin{align*}
            I(P_1) &= P_{seen}\\
            I(P_2) &= P_{seen}\\
            I(x) &= \sus{blue}\\
            I(y) &= \sus{yellow}
        \end{align*}
        Wir haben \sus{red} und \sus{green} zusammen gesehen und \sus{green} ist tot. Dann muss \sus{red} ein Imposter sein.
\end{frame}

\begin{frame}{Lösung}
    $$(P_1(x,y) \wedge \overline{P_2(y)}) \implies P_3(y)$$
        \begin{align*}
            I(P_1) &= P_{seen}\\
            I(P_2) &= P_{alive}\\
            I(P_3) &= P_{crew}\\
            I(x) &= \sus{red}\\
            I(y) &= \sus{green}
        \end{align*}
        Wir haben \sus{red} und \sus{green} zusammen gesehen und \sus{green} ist tot. Dann muss \sus{red} ein Imposter sein.
\end{frame}

\begin{frame}{Lösung}
    $$(P_1(x,y) \wedge P_2(y)) \implies P_3(x)$$
        \begin{align*}
            I(P_1) &= P_{sus}\\
            I(P_2) &= P_{alive}\\
            I(P_3) &= P_{crew}\\
            I(x) &= \sus{white}\\
            I(y) &= \sus{magenta}
        \end{align*}
        \sus{white} verdächtigt \sus{magenta}, und \sus{white} ist noch am Leben. Also ist \sus{magenta} kein Imposter.
\end{frame}
}



\begin{frame}{Quantoren und Funktionen}
    \begin{itemize}
        \item TODO : Quantoren und Funktionen erklären, dann auf Mathematische Beispiele wechseln (funktionen +, - , * , / und U = N...)
    \end{itemize}
\end{frame}

\begin{frame}{Ein weiteres Beispiel}
    \begin{itemize}
        \item $U = \mathbb Z$
        \item Prädikate:
        \begin{itemize}
            \item $P_{ood}(x)$: $x$ is ood
            \item $P_{prim}(x)$: $x$ is prim
            \item $P_{+}(x)$: $x$ is positive
            \item $P_>(x,y): x>y$ 
            \item $P_=(x,y): x=y$ 
        \end{itemize}
        \item Funktionen: $+, -, \cdot$ 
    \end{itemize}
\end{frame}

{\setbeamercolor{palette primary}{bg=ExColor}
\begin{frame}[fragile]{Denkpause}
    \footnotesize
        \begin{alertblock}{Aufgaben}
            Findet Interpreatationen für die folgenden Prädikatenlogischen Formeln in unserem Beispiel:
        \end{alertblock}
        \metroset{block=fill}
        \begin{block}{Normal}
            \begin{itemize}
                \item $\forall x,y : P_1(x) \wedge P_1(y) \implies \overline{P_1(f_1(x,y))}$
                \item $\forall x\exists y: P_1(f_1(x,y),x)$
                \item $\forall x\exists y: P_1(f_1(x,y),y)$
            \end{itemize}
        \end{block}
        \begin{block}{Etwas Schwerer}
            \begin{itemize}
                \item TODO
            \end{itemize}
        \end{block}
\end{frame}

\begin{frame}{Lösung}
    $$\forall x,y : P_1(x) \wedge P_1(y) \implies \overline{P_1(f_1(x,y))}$$
        \begin{align*}
            I(P_1) &= P_{ood}\\
            I(f_1) &= +
        \end{align*}
        Für alle $x,y \in \mathbb N$ gilt: wenn $x$ und $x$ ungerade sind, dann ist $x+y$ ungerade
\end{frame}

\begin{frame}{Lösung}
    $$\forall x\exists y: P_1(f_1(x,y),x)$$
        \begin{align*}
            I(P_1) &= P_{=}\\
            I(f_1) &= +
        \end{align*}
        Für alle $x \in \mathbb N$ existiert ein $y$ so dass $x+y = x$. (Dieses $y$ ist $0$).
\end{frame}

\begin{frame}{Lösung}
    $$\forall x\exists y: P_1(f_1(x,y),y)$$
        \begin{align*}
            I(P_1) &= P_{=}\\
            I(f_1) &= \cdot
        \end{align*}
        Für alle $x \in \mathbb N$ existiert ein $y$ so dass $x\cdot y = y$. (Dieses $y$ ist $0$).
\end{frame}
}