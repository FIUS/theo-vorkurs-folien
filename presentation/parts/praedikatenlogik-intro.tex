\begin{frame}[fragile]{Beispiel}
	7-Segment-Anzeigen:
	\metroset{block=fill}
	\begin{block}{Kann Zahlen zeigen}
		\begin{center}
			\sevensegnum[size=15mm]{2}
			\hspace{1em}
			\sevensegnum[size=15mm]{0}
			\hspace{1em}
			\sevensegnum[size=15mm]{2}
			\hspace{1em}
			\sevensegnum[size=15mm]{5}
		\end{center}
	\end{block}
	\begin{block}{Zeigt manchmal wirres Zeugs}
		\begin{center}
			\sevenseg[size=15mm]{{0,0,0,1,0,0,0,}}
			\hspace{1em}
			\sevenseg[size=15mm]{{0,1,0,0,1,0,0,}}
			\hspace{1em}
			\sevenseg[size=15mm]{{1,1,0,1,1,1,1,}}
			\hspace{1em}
			\sevenseg[size=15mm]{{1,0,1,1,0,0,0,}}
		\end{center}
	\end{block}
\end{frame}

\begin{frame}{Beispiel}
	Wir brauchen einen \textit{Checker}:
	\Huge
	\begin{center}
		$$
			\only<1>{P_{\text{Zahl}}\left(\sevensegnum{6}\right)}
			\only<2->{P_{\text{Zahl}}\left(\sevenseg{{1,0,1,1,0,0,0,}}\right)}
		$$
	\end{center}
	\normalsize
	% todo
	Wenn wir in diesen \textit{Checker} eine Anzeige einsetzen, haben wir eine logische Aussage
\end{frame}

{\setbeamercolor{palette primary}{bg=ExColor}
\begin{frame}{Denkpause}
	Sind die folgenden Aussagen wahr oder falsch
	\metroset{block=fill}
	\begin{block}{Normal}
		\begin{itemize}
			\item $A_1$: $P_{\text{Zahl}}\left(\right)$
		\end{itemize}
	\end{block}
\end{frame}
}

\begin{frame}{Beispiel}
	Wie reparieren wir die Anzeige?
	\begin{center}
		\sevenseg[size=15mm]{{0,0,0,1,0,0,0,}}
		\hspace{1em}
		\sevenseg[size=15mm]{{0,1,0,0,1,0,0,}}
		\hspace{1em}
		\sevenseg[size=15mm]{{1,1,0,1,1,1,1,}}
		\hspace{1em}
		\sevenseg[size=15mm]{{1,0,1,1,0,0,0,}}
	\end{center}
\end{frame}

\begin{frame}{Beispiel}
	Wir brauchen eine Funktion, die Segmente einschaltet
	\Large
	\begin{center}
		$
			f_{\text{On}}\left(
			\sevenseg[size=10mm]{{1,0,0,1,0,0,1,}},
			\sevenseg[size=10mm]{{0,1,1,0,1,1,0,}}
			\right) =
			\sevenseg[size=10mm]{{1,1,1,1,1,1,1,}}
		$
	\end{center}
	\normalsize
	\only<3->{Wir brauchen eine weitere Funktion:
		\Large
		\begin{center}
			$
				f_{\text{On}}\left(
				\sevenseg[size=10mm]{{1,0,0,1,0,0,1,}},
				\sevenseg[size=10mm]{{0,1,1,0,1,1,0,}}
				\right) =
				\sevenseg[size=10mm]{{1,1,1,1,1,1,1,}}
			$
		\end{center}
		\normalsize
	}
	Reicht das?
	\begin{itemize}
		\item<2-> $f_{\text{On}}\left(\sevenseg{{0,0,0,1,0,0,0,}}, \sevenseg{{0,1,1,0,1,1,0,}}\right) = \sevensegnum{2}$
		\item<2-> $f_{\text{On}}\left(\sevenseg{{0,1,0,0,1,0,0,}}, \sevenseg{{0,1,1,0,1,1,0,}}\right) = \sevensegnum{2}$
		\item<2-> $f_{\text{On}}\left(\sevenseg{{1,1,0,1,1,1,1,}}, \sevenseg{{0,1,1,0,1,1,0,}}\right) = \sevensegnum{2}$
		\item<2-> $f_{\text{On}}\left(\sevenseg{{1,0,1,1,0,0,0,}}, \sevenseg{{0,1,1,0,1,1,0,}}\right) = \sevensegnum{2}$
	\end{itemize}
\end{frame}

\begin{frame}{Beispiel}
	Kombiniert mit unserem \textit{Checker}
	\Large
	\begin{center}
		$
			P_{\text{Zahl}}\left(f_{\text{On}}\left(
			\sevenseg[size=10mm]{{1,0,0,0,0,0,0,}},
			\sevenseg[size=10mm]{{0,1,1,0,0,0,0,}}
			\right)\right)
		$
	\end{center}
	\normalsize
	\pause
	Diese Aussage ist wahr
	\par
	\pause
	Allgemeiner:
\end{frame}

{\setbeamercolor{palette primary}{bg=ExColor}
\begin{frame}{Denkpause}
	Finde Anzeigen $x, y$, für die die Aussage stimmt
	\metroset{block=fill}
	\begin{block}{Normal}
		\begin{itemize}
			\item $P_{\text{Zahl}}\left(f_{\text{On}}\left(\sevenseg{{0,0,0,0,0,0,0,}}, x_1\right)\right)$
			\item $P_{\text{Zahl}}\left(f_{\text{Off}}\left(x_2, y_2\right)\right)$
			\item $P_{\text{Zahl}}\left(f_{\text{Off}}\left(x_3, \sevenseg{{0,1,1,0,0,0,0,0,}}\right)\right) \land P_{\text{Zahl}}\left(f_{\text{On}}\left(x_3, \sevenseg{{1,0,0,1,0,0,0,}}\right)\right)$
			\item $\lnot P_{\text{Zahl}}\left(f_{\text{On}}\left(f_{\text{On}}\left(\sevenseg{{0,0,0,0,0,0,0,}}, x_4\right), \sevenseg{{1,1,1,1,0,0,1,}}\right)\right)$
		\end{itemize}
	\end{block}
	Beweise ob die folgenden Aussagen wahr oder falsch sind
	\metroset{block=fill}
	\begin{block}{Etwas schwerer}
		\begin{itemize}
			\item $A_5$: $\forall x : P_{\text{Zahl}}\left(f_{\text{On}}\left(x, \sevenseg{{1,1,1,1,1,1,1,1,}}\right)\right)$
			\item $A_6$: $\forall x : P_{\text{Zahl}}\left(f_{\text{On}}\left(\right)\right)$
		\end{itemize}
	\end{block}
\end{frame}
}