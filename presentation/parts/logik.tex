% Copyright 2018, 2019, 2020, 2021 FIUS
%
% This file is part of theo-vorkurs-folien.
%
% theo-vorkurs-folien is free software: you can redistribute it and/or modify
% it under the terms of the GNU General Public License as published by
% the Free Software Foundation, either version 3 of the License, or
% (at your option) any later version.
%
% theo-vorkurs-folien is distributed in the hope that it will be useful,
% but WITHOUT ANY WARRANTY; without even the implied warranty of
% MERCHANTABILITY or FITNESS FOR A PARTICULAR PURPOSE.  See the
% GNU General Public License for more details.
%
% You should have received a copy of the GNU General Public License
% along with theo-vorkurs-folien.  If not, see <https://www.gnu.org/licenses/>.

\begin{frame}[fragile]{Was ist Aussagenlogik?}
    \begin{alertblock}{Aussagen}
    \begin{itemize}
        \item Paris ist die Hauptstadt von Frankreich
        \item Mäuse jagen Elefanten
        \item $5 \in \mathbb{N}$
        \item $5 = 8$
        \item $u \in \{u, v, w\}$
    \end{itemize}
    \end{alertblock}
    Eine Aussage $A$ ist entweder \textbf{wahr} oder \textbf{falsch}.
\end{frame}

\begin{frame}[fragile]{Was ist Aussagenlogik?}
    \begin{alertblock}{Das sind keine Aussagen}
    \begin{itemize}
        \item Macht theoretische Informatik Spaß?
        \item Geh dein Zimmer aufräumen!
        \item Wie viele Tiere wohnen in der Uni?
        \item $(x+y)^2+1$
        \item $\{a,b,c\}$
        \item ...
    \end{itemize}
    \end{alertblock}
    Diesen Sätzen können wir keinen eindeutigen Wahrheitswert \textbf{wahr} oder \textbf{falsch} zuordnen.
\end{frame}

\begin{frame}[fragile]{Was ist Aussagenlogik?}
    \begin{alertblock}{Wozu brauchen wir das?}
    \begin{itemize}
        \item Wir untersuchen, wie wir Aussagen verknüpfen können.
        \item Damit ziehen wir formale Schlüsse und führen Beweise.
    \end{itemize}
    \end{alertblock}
\end{frame}

\begin{frame}[fragile]{Logische Operationen}
Wir können Aussagen verändern oder durch Operationen zu neuen Aussagen verbinden.
\begin{itemize}
    \item $A$: Fred möchte Schokolade.
    \item $B$: Fred möchte Gummibärchen.
\end{itemize}
\metroset{block=fill}
\begin{alertblock}{Grundoperationen}
\begin{itemize}
    \item<1-> \textbf{Und}: $A$ \alert<1>{$\wedge$} $B$ $\leadsto$ Fred möchte Schokolade \alert<1>{und} Gummibärchen.\\
    \only<1>{\emph{Analog}: $M$: $M_1 \cap M_2$, Jedes Element aus $M$ liegt in $M_1$ \textbf{und} in $M_2$}
    \item<2-> \textbf{Oder}: $A$ \alert<2>{$\vee$} $B$ $\leadsto$ Fred möchte Schokolade \alert<2>{oder} Gummibärchen.\\
    \only<2>{\emph{Anmerkung}: Inklusives \glqq oder\grqq, kein \glqq entweder oder\grqq \\
    Das heißt, es können auch beide Aussagen wahr sein.\\}
	\only<2>{\emph{Analog}: $M$: $M_1 \cup M_2$, Jedes Element aus $M$ liegt in $M_1$ \textbf{oder} in $M_2$}
    \item<3> \textbf{Nicht}: \alert<3>{$\neg$}$A$ $\leadsto$ Fred möchte \alert<3>{keine} Schokolade.\\
    \only<3>{\emph{Analog}: $M$: $\overline{M_1}$, Jedes Element aus $M$ liegt \textbf{nicht} in $M_1$}
\end{itemize}
\end{alertblock}
\end{frame}


\begin{frame}{Überblick: Mengenoperationen}

	Auf \alert{Mengen} $A$, $B$ lassen sich verschiedene Mengenoperationen ausführen.

	\metroset{block=fill}
	\begin{exampleblock}{Mengenoperationen}
		\begin{itemize}
			\item Schnitt: $A \cap B$
			\item Vereinigung: $A \cup B$
			\item Komplement: $\overline{A}$
		\end{itemize}
	\end{exampleblock}
\end{frame}

\begin{frame}{Überblick: Logische Operationen}

	Auf \alert{Aussagen} $A$, $B$ lassen sich verschiedene logische Operationen ausführen.

	\metroset{block=fill}
	\begin{exampleblock}{Logische Operationen}
		\begin{itemize}
			\item Logisches Und: $A \wedge B$
			\item Logisches Oder: $A \vee B$
			\item Logisches Nicht: $\neg A$
		\end{itemize}
	\end{exampleblock}
\end{frame}

{\setbeamercolor{palette primary}{bg=ExColor}
	\begin{frame}{Logische Operationen vs. Mengenoperationen}
		\alert{Mengenoperationen und logische Operationen dürfen nicht verwechselt werden.}
		\begin{table}[]
			\begin{tabular}{l l}
				$A$: $5 \in \mathbb{N}$ & $B$: Es regnet\\
				$C$: \{$w \mid |w|=2$ \} \ & $D$: \{a,b,c,x,y\}\\
			\end{tabular}
		\end{table}
		\metroset{block=fill}
		\metroset{block=fill}
		\begin{block}{Welche dieser Verknüpfungen sind zulässig?}
			\begin{enumerate}
				\item $A \wedge B$
				\item $A \vee C$
				\item $C \cap D$
				\item $A \wedge B \cup C$
			\end{enumerate}
		\end{block}
	\end{frame}
}

{\setbeamercolor{palette primary}{bg=ExColor}
	\begin{frame}<handout:0>[fragile]{Logische Operationen vs. Mengenoperationen}
		\begin{enumerate}[<+- | alert@+>]
			\item Zulässig
			\item Nicht zulässig
			\item Zulässig
			\item Nicht zulässig
		\end{enumerate}
	\end{frame}
}



\begin{frame}{Logische Operationen: Implikation}
\begin{alertblock}{$A\implies B$}
\begin{itemize}
    \item \glqq Wenn $A$ wahr ist, dann muss auch $B$ wahr sein.\grqq
    \item kurz: \glqq\textbf{Wenn} $A$, \textbf{dann} $B$.\grqq
    \item Wenn $A$ falsch ist können wir keine Aussage über $B$ treffen.
    \item $A\implies B$ ist dieselbe Aussage wie $\neg A \vee B$
\end{itemize}
\end{alertblock}
\end{frame}

{\setbeamercolor{palette primary}{bg=ExColor}
	\begin{frame}{Denkpause}
		\begin{alertblock}{Aufgaben}
			Die folgenden Teilaufgaben bestehen aus einer Aussage, einem Geschehen und aus einer Folgerung. Welche der Folgerungen sind richtig, unter der Annahme, 
			dass die Aussagen wahr sind?
		\end{alertblock}
		\metroset{block=fill}
		\begin{block}{Normal bis Schwer}
			\begin{itemize}
				\item Aussage: "Lukas ist im Vorkurs oder schläft noch"\\
				Geschehen: Lukas ist nicht im Vorkurs.\\
				Folgerung: Also schläft er noch.
				\item Aussage: "Wenn Anne nicht rennt, bekommt sie die Bahn nicht"\\
				Geschehen: Anne bekommt die Bahn.\\
				Folgerung: Also ist sie gerannt.
				\item Aussage: "Wenn Tobi auf die Prüfung nicht lernt, besteht er nicht"\\
				Geschehen: Tobi hat auf die Prüfung gelernt.\\
				Folgerung: Also besteht er.
			\end{itemize}
		\end{block}
	\end{frame}
}
	
	
{\setbeamercolor{palette primary}{bg=ExColor}
	\begin{frame}<handout:0>{Lösungen}
		\only<1>{
			\metroset{block=fill}
			\begin{block}{Normal bis Schwer}
				\begin{itemize}
					\item Aussage: "Lukas ist im Vorkurs oder schläft noch"\\
					Geschehen: Lukas ist nicht im Vorkurs.\\
					Folgerung: Also schläft er noch.
				\end{itemize}
			\end{block}
		}
		\only<2>{
			\metroset{block=fill}
			\begin{block}{Normal bis Schwer}
				\begin{itemize}
					\item<2> Aussage: "Wenn Anne nicht rennt, bekommt sie die Bahn nicht"\\
					Geschehen: Anne bekommt die Bahn.\\
					Folgerung: Also ist sie gerannt.
				\end{itemize}
			\end{block}
		}
		\only<3>{
			\metroset{block=fill}
			\begin{block}{Normal bis Schwer}
				\begin{itemize}
					\item<3> Aussage: "Wenn Tobi auf die Prüfung nicht lernt, besteht er nicht"\\
					Geschehen: Tobi hat auf die Prüfung gelernt.\\
					Folgerung: Also besteht er.
				\end{itemize}
			\end{block}
		}
		
		\only<1>{Die Folgerung ist richtig. Laut Aussage ist Lukas im Vorkurs oder schläft noch (oder beides). Wenn er also nicht im Vorkurs ist, bleibt nur noch die Option übrig, dass er noch schläft.}
		\only<2>{Die Folgerung ist richtig. Die zweite Aussage ist die Kontraposition der ersten Aussage. Wie gezeigt wurde, ist die Kontraposition einer Aussage wahr gdw. die Aussage selbst wahr ist. }
		\only<3>{Die Folgerung ist falsch. Es ist auch möglich, dass Tobi lernt und nicht besteht!}
	\end{frame}
}

\begin{frame}<handout:0>{Logische Operationen: Äquivalenz}
	\begin{alertblock}{$A\iff B$}
		\begin{itemize}
			\item \glqq $A$ ist wahr, \textbf{genau dann wenn} $B$ wahr ist.\grqq
			\item kurz: \glqq $A$ gdw. $B$\grqq
			\item $A$ und $B$ müssen den selben Wahrheitswert haben.
			\item $A\iff B$ ist dieselbe Aussage wie $(A \implies B) \wedge (B \implies A)$
		\end{itemize}
	\end{alertblock}
\end{frame}

{\setbeamercolor{palette primary}{bg=ExColor}
\begin{frame}[fragile]{Denkpause}
    \begin{alertblock}{Aufgaben}
      Berechne den Wahrheitswert folgender Aussagen.
    \end{alertblock}
    \metroset{block=fill}
    \begin{block}{Normal}
    \begin{itemize}
        \item $A_1$: $5 \in \mathbb{N} \wedge a \in \{a, b, c\}$
        \item $A_2$: $0 \in \mathbb{N} \vee a \in \{a, b, c, d\}$
        \item $A_3$: $A_1 \iff A_2$
    \end{itemize}
    \end{block}
    \begin{block}{Etwas Schwerer}
    \begin{itemize}
        \item $A_4$: $(\emptyset=\emptyset^{*}) \implies (a \in \emptyset)$
        \item $A_5$: $(a \notin \emptyset) \implies (\emptyset = \emptyset^{*})$
        \item $A_6$: $A_4 \iff A_5$
        \item $A_7$: $(7 \in \{1, 2, 7, 9\}) \cap (2 = 7-5)$
        \item $A_8$: Wenn mein Auto fliegt, dann hast du auch ein fliegendes Auto.
    \end{itemize}
    \end{block}
\end{frame}
}

% {\setbeamercolor{palette primary}{bg=ExColor}
% \begin{frame}[fragile]{Denkpause}
%  \begin{alertblock}{Aufgaben}
%    Löse folgende Zusatzaufgabe.
%  \end{alertblock}
%  \metroset{block=fill}
%  \begin{block}{Zusatz}
%  \begin{itemize}
%      \item $A_8$: Gegeben zwei Aussagen A, B.\\
%      Formuliere die Aussage $A \iff B$ nur unter Verwendung der Junktoren $\wedge, \vee, \neg$
%  \end{itemize}
%  \end{block}
% \end{frame}
% }

{\setbeamercolor{palette primary}{bg=ExColor}
\begin{frame}<handout:0>{Lösungen}
  \begin{itemize}[<+- | alert@+>]
        \item
            $A_1$: wahr
        \item
            $A_2$: wahr
        \item
            $A_3$: wahr
        \item
            $A_4$: wahr
        \item
            $A_5$: falsch
       	\item
       		$A_6$: falsch
        \item
            $A_7$: Das ist keine Aussage, da der Schnitt ($\cap$) verwendet wurde um zwei Aussagen miteinander zu verknüpfen
       	\item
       		$A_8$: wahr
    \end{itemize}
\end{frame}
}

% {\setbeamercolor{palette primary}{bg=ExColor}
% \begin{frame}{Lösungen}
%  \metroset{block=fill}
%  \begin{block}{Zusatz}
%  \begin{itemize}
%      \item $A_8$: Gegeben zwei Aussagen A, B.\\
%      Formuliere die Aussage $A \iff B$ nur unter Verwendung der Junktoren $\wedge, \vee, \neg$
%  \end{itemize}
%  \end{block}
%   \begin{alertblock}{Lösung}
%      \begin{itemize}[<+- | alert@+>]
%          \item Äquivalenz bedeutet intuitiv: Beide wahr oder beide falsch
%          \item $A_8$: $(A \wedge B) \vee (\neg A \wedge \neg B)$
%      \end{itemize}
%   \end{alertblock}
% \end{frame}
% }

\begin{frame}{Anwendung der Implikation}
    Wir haben zwei Aussagen $A$ und $B$. Wir nehmen nun an, $A$ sei wahr. Wenn wir zeigen, dass dann auch $B$ wahr ist, wissen wir, dass $A \implies B$ gilt.
\metroset{block=fill}
\begin{exampleblock}{Beispiel}
\begin{enumerate}
    \item<1-> Wir wollen zeigen, dass für eine ganze Zahl $x$ \\
    die Implikation $3 = x - 2 \implies x = 5$ gilt.
    \item<2-> Wir nehmen an, dass die linke Aussage wahr ist\dots
    \item<3-> \dots und zeigen, dass dann die rechte Aussage gilt.
    \item<4-> $(3 = x - 2) \implies (3 + 2 = x) \implies (5 = x) \implies (x = 5)$
    \item<5-> Also folgt die rechte Aussage aus der Linken.
    \item<6-> Somit gilt die Implikation. \qed\;
\end{enumerate}
\end{exampleblock}
\end{frame}
