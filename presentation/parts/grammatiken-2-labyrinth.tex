% Copyright 2018-2024 FIUS
%
% This file is part of theo-vorkurs-folien.
%
% theo-vorkurs-folien is free software: you can redistribute it and/or modify
% it under the terms of the GNU General Public License as published by
% the Free Software Foundation, either version 3 of the License, or
% (at your option) any later version.
%
% theo-vorkurs-folien is distributed in the hope that it will be useful,
% but WITHOUT ANY WARRANTY; without even the implied warranty of
% MERCHANTABILITY or FITNESS FOR A PARTICULAR PURPOSE.  See the
% GNU General Public License for more details.
%
% You should have received a copy of the GNU General Public License
% along with theo-vorkurs-folien.  If not, see <https://www.gnu.org/licenses/>.

\definecolor{labyrinthLine}{RGB}{0,51,102}
\definecolor{labyrinthHead}{RGB}{0,25,50}
\definecolor{labyrinthField}{RGB}{255,230,204}
\definecolor{labyrinthPath}{RGB}{130,179,102}
\definecolor{labyrinthDecision}{RGB}{213,232,212}
\definecolor{labyrinthDecisionText}{RGB}{18,117,181}
\providecommand{\labyrinthSize}{\textwidth}
\providecommand{\labyrinthVariant}{None}
\begin{includetikzpicture}{\labyrinthSize}[x=1mm,y=1mm]
    \tikzset{every path/.style={line width=0.4}}
    % Strichmaennchen
    \draw (15,116.5)--(15,108.5);
    \draw (11,115.5)--(19,115.5);
    \draw[draw=black,fill=labyrinthHead] (15,119) circle[radius=2.5];
    \draw (15,108.5)--(10,103);
    \draw (15,108.5)--(20,103);
    \node[align=center] at (5,112) {\Large Mei};

    % Labyrinth
    \tikzset{every path/.style={line width=5,draw=labyrinthLine}}
    \draw (30,90)--(180,90);
    \draw (0,90)--(0,0)--(180,0)--(180,60)--(150,60);
    \draw (120,0)--(120,30);
    \draw (90,30)--(150,30);
    \draw (30,60)--(120,60);
    \draw (60,60)--(60,30)--(30,30);
    \draw[-{Stealth[inset=0pt, length=12, angle'=60]},line width=7] (15,102)--(15,89);
    \draw[-{Stealth[inset=0pt, length=12, angle'=60]},line width=7] (179,75)--(192,75);

    % Felder
    \tikzset{every path/.style={draw=none,fill=labyrinthField},every circle/.style={radius=11}}
    \foreach \x in {0,...,5}
    \foreach \y in {0,...,2}
        {\fill (15 + \x * 30,15 + \y * 30) circle;}

    % Pfade
    \tikzset{every path/.style={-{Stealth},draw=labyrinthPath,fill=none,rounded corners=10,line width=5}}
    \ifthenelse{\equal{\labyrinthVariant}{Direkt}}
    {
        \draw (15,86)--(15,75)--(176,75); % Direkt
    }{}
    \ifthenelse{\equal{\labyrinthVariant}{Indirekt}}
    {
        \draw (15,86)--(15,15)--(75,15)--(75,45)--(135,45)--(135,75)--(176,75); % Indirekt
    }{}
    \ifthenelse{\equal{\labyrinthVariant}{Uhrzeigersinn}}
    {
        \draw (15,86)--(15,75)--(135,75)--(135,45)--(75,45)--(75,15)--(15,15)--(15,75)--(176,75); % Uhrzeigersinn
        \draw[line width=2,draw=black] (60,82.75)--(90,82.75);
    }{}
    \ifthenelse{\equal{\labyrinthVariant}{GegenUhrzeigersinn}}
    {
        \draw (15,86)--(15,15)--(75,15)--(75,45)--(135,45)--(135,75)--(15,75)--(15,15)--(75,15)--(75,45)--(135,45)--(135,75)--(176,75); % GegenUhrzeigersinn
        \draw[line width=2,draw=black] (90,82.75)--(60,82.75);
    }{}

    \ifthenelse{\equal{\labyrinthVariant}{DecisionPoints}}
    {
    \fill[draw=none,fill=labyrinthDecision] (15,75) circle;
    \fill[draw=none,fill=labyrinthDecision] (135,75) circle;
    \tikzset{every path/.style={-{Stealth},draw=labyrinthDecisionText,line width=2},every node/.style={font=\Huge,text=labyrinthDecisionText,text centered}}
    \node[align=center] at (19,79) {$A_r$};
    \node[align=center] at (11,71) {$A_u$};
    \draw (34,75)--(26,75);
    \draw (15,56)--(15,64);
    \draw[-] (9,81)--(21,69);

    \node[align=center] at (131,79) {$B_l$};
    \node[align=center] at (139,71) {$B_u$};
    \draw (116,75)--(124,75);
    \draw (135,56)--(135,64);
    \draw[-] (129,69)--(141,81);
    }{}
\end{includetikzpicture}
\let\labyrinthVariant\relax
\let\labyrinthSize\relax