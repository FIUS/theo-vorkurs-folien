% Copyright 2018-2022 FIUS
%
% This file is part of theo-vorkurs-folien.
%
% theo-vorkurs-folien is free software: you can redistribute it and/or modify
% it under the terms of the GNU General Public License as published by
% the Free Software Foundation, either version 3 of the License, or
% (at your option) any later version.
%
% theo-vorkurs-folien is distributed in the hope that it will be useful,
% but WITHOUT ANY WARRANTY; without even the implied warranty of
% MERCHANTABILITY or FITNESS FOR A PARTICULAR PURPOSE.  See the
% GNU General Public License for more details.
%
% You should have received a copy of the GNU General Public License
% along with theo-vorkurs-folien.  If not, see <https://www.gnu.org/licenses/>.

\subsection{Organisatorisches}
\begin{frame}[fragile]{Wer sind wir?}
    \begin{itemize}
        \item
            Fachgruppe Informatik
            \begin{itemize}
                \item Unser Ziel: \\
                Das Leben von uns Studis während des Studiums angenehmer zu gestalten
                \item organisieren Veranstaltungen (Grillen, Spieleabende, Vorkurse, ...)
                \item verleihen Prüfungen aus den früheren Semestern
                \item vertreten die studentische Sicht in offiziellen Gremien
                \item ...und vieles mehr (es gibt z.B. einen 3D-Drucker)
            \end{itemize}
        \item Arbeitskreis Theoretische Informatik
        \begin{itemize}
            \item Teilmenge der Fachgruppe Informatik
            \item haben diesen Vorkurs organisiert
        \end{itemize}
    \end{itemize}
\end{frame}
\note[itemize]{
	\item test
	\item test
}

\subsection{Tipps zum Studium}
\begin{frame}[fragile]{Tipps zum Studium}
    \begin{itemize}
        \item Nützliche Links:\\
            \begin{itemize}
                \item Fachgruppe Informatik:\\
                \url{https://fius.de/}
                \item Handouts und Foliensätze:\\ \url{https://fius.de/index.php/studien-interessierte/vorkurs-theoretische-informatik/}
                \item Materialien Ergänzung Theoretische Informatik 1 (Wintersemester 19/20): \\
                \url{https://fmi.uni-stuttgart.de/ti/teaching/w19/eti1/}
                \item Ersti Telegram-Gruppe:\\
                \qrcode[hyperlink]{\telegramurl}
                 \url{\telegramurl}
        	\end{itemize}
        \item E-Mail der Fachgruppe: fius@informatik.uni-stuttgart.de

    \end{itemize}
\end{frame}

% \begin{frame}[fragile]{Hygieneregeln}
% 	\begin{alertblock}{Die wichtigsten Regeln...}
% 		\begin{itemize}
% 			\item Eine Masken\emph{pflicht} besteht aktuell nicht.
%             \item Speziell wenn Abstände nicht eingehalten werden können \emph{empfehlen} wir euch eine Maske zu tragen
% 			\item Sagt bei Krankheitssymptomen oder einem positiven Test Bescheid und bleibt zu Hause; ihr könnt dann online weiterhin teilnehmen.
% 			\item Regeln unter Vorbehalt, dass die Uni nichts ändert
% 		\end{itemize}
% 	\end{alertblock}
% \end{frame}

%\begin{frame}[fragile]{Hygieneregeln}
%	\begin{alertblock}{Warum das alles?}
%		\begin{itemize}
%			\item Stellt euch vor, jede*r hier - bis auf eine*r - hätte Corona. Wir wollen uns so verhalten, dass selbst in dem Fall diese eine Person gesund nach Hause gehen kann.
%			\item Ihr könnt euch trotzdem untereinander kennen lernen und Aufgaben miteinander bearbeiten. Lasst euch von den Maßnamen nicht abschrecken - wir können sie einhalten und den Vorkurs damit gut meistern.
%		\end{itemize}
%	\alert{Inzwischen kennen wir es doch alle: Abstand, Hygiene, Maske auf.}
%	\end{alertblock}
%\end{frame}

\begin{frame}[fragile]{Infos zum Online-Ablauf}
	\begin{alertblock}{Ablauf und Notfallplan}
		\begin{itemize}
            \item Der Online-Vorkurs wird eine Übertragung aus/in einem Hörsaal sein
			\item Es wird zwischen Vorlesungs- und Aufgabephasen abgewechselt.
			\item Wir benutzen BigBlueButton - wenn ihr hier seid, wisst ihr das schon. 
            \item Bei technischen Problemen, die sich nicht zügig beheben lassen, wechseln wir ggf. auf eine andere Plattform. Den Joinlink verschicken wir dann per Mail und stellen ihn auf \url{https://fius.de/index.php/studien-interessierte/vorkurs-theoretische-informatik/}.
		\end{itemize}
		\alert{Traut euch, Fragen zu stellen und mitzumachen.}
	\end{alertblock}
\end{frame}
