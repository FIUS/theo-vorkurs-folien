% Copyright 2018-2022 FIUS
%
% This file is part of theo-vorkurs-folien.
%
% theo-vorkurs-folien is free software: you can redistribute it and/or modify
% it under the terms of the GNU General Public License as published by
% the Free Software Foundation, either version 3 of the License, or
% (at your option) any later version.
%
% theo-vorkurs-folien is distributed in the hope that it will be useful,
% but WITHOUT ANY WARRANTY; without even the implied warranty of
% MERCHANTABILITY or FITNESS FOR A PARTICULAR PURPOSE.  See the
% GNU General Public License for more details.
%
% You should have received a copy of the GNU General Public License
% along with theo-vorkurs-folien.  If not, see <https://www.gnu.org/licenses/>.

% !TeX program = pdflatex
% !TeX spellcheck = de
% Copyright 2018-2024 FIUS
%
% This file is part of theo-vorkurs-folien.
%
% theo-vorkurs-folien is free software: you can redistribute it and/or modify
% it under the terms of the GNU General Public License as published by
% the Free Software Foundation, either version 3 of the License, or
% (at your option) any later version.
%
% theo-vorkurs-folien is distributed in the hope that it will be useful,
% but WITHOUT ANY WARRANTY; without even the implied warranty of
% MERCHANTABILITY or FITNESS FOR A PARTICULAR PURPOSE.  See the
% GNU General Public License for more details.
%
% You should have received a copy of the GNU General Public License
% along with theo-vorkurs-folien.  If not, see <https://www.gnu.org/licenses/>.

\documentclass[aspectratio=43,10pt]{beamer}

\usetheme[progressbar=frametitle,subsectionpage=progressbar]{metropolis}
\usepackage{appendixnumberbeamer}
\usepackage[ngerman]{babel}
\usepackage[utf8]{inputenc}
%\usepackage{t1enc}
\usepackage{iftex}

\ifLuaTeX
    \usepackage{fontspec}
\else
    \ifxetex
        \usepackage{fontspec}
    \else
        \usepackage[T1]{fontenc}
    \fi
\fi

\usepackage[sfdefault,scaled=.85,lf]{FiraSans}
\usepackage{newtxsf}

\usepackage{booktabs}
\usepackage[scale=2]{ccicons}
\usepackage{hyperref}

\usepackage{pgf}
\makeatletter
\@ifclasswith{beamer}{notes}{
    \usepackage{pgfpages}
    \setbeameroption{show notes on second screen}
}{}
\makeatother
\usepackage{tikz}
\usetikzlibrary{arrows,automata,positioning,shapes,arrows.meta,shapes.geometric, through, calc}
\usepackage{pgfplots}
\usepgfplotslibrary{dateplot}

\usepackage{xspace}
\newcommand{\themename}{\textbf{\textsc{metropolis}}\xspace}

\usepackage{blindtext}
\usepackage{graphicx}
\usepackage{subcaption}
\usepackage{comment}
\usepackage{mathtools}
\usepackage{amsmath}
\usepackage{centernot}
\usepackage{amssymb}
\usepackage{proof}
\usepackage{tabularx}
\renewcommand{\figurename}{Abb.}
\usepackage{tikzsymbols}
\let\Coffeecup\relax
\let\Heart\relax
\let\Smiley\relax
\usepackage{marvosym}
\usepackage{mathtools}
\usepackage{qrcode}
\usepackage{advdate}
\usepackage{ifthen}
\usepackage{tikz-among-us}
\usepackage{multicol}
\usepackage{outlines}
\usepackage{ulem}

% \IfFontExistsTF{Segoe UI}{%
%     \usefonttheme{professionalfonts}
%     \defaultfontfeatures{Scale = MatchLowercase}
%     \setsansfont{Consolas}[
%         Scale = 1.0,
%         BoldFont = Consolas ,
%         BoldItalicFont = Consolas ]
%     \ifLuaTeX
% }{
% }
% \else
% \ifxelatex
%     \IfFontExistsTF{Segoe UI}{%
%         \usefonttheme{professionalfonts}
%         \defaultfontfeatures{Scale = MatchLowercase}
%         \setsansfont{Consolas}[
%             Scale = 1.0,
%             BoldFont = Segoe UI Semibold ,
%             BoldItalicFont = Segoe UI Semibold Italic ]
%     }{
%     }
% \fi

\definecolor{Bluecreen}{HTML}{0873aa}

\makeatletter
\setlength{\metropolis@progressonsectionpage@linewidth}{1.1em}
\setlength{\metropolis@progressinheadfoot@linewidth}{2pt}
\makeatother

\setbeamercolor{progress bar}{%
    fg=Bluecreen,
    bg=Bluecreen!70!black!45
}

\makeatletter
\newlength{\metropolis@progressonsectionpage@blockwidth}%
\newlength{\metropolis@progressonsectionpage@blockborder}%
\setlength{\metropolis@progressonsectionpage@blockborder}{1pt}%
\setbeamertemplate{progress bar in section page}{
    \vspace{0.5\metropolis@progressonsectionpage@linewidth}
    \setlength{\metropolis@progressonsectionpage}{%
        \textwidth * \ratio{\insertframenumber pt}{\inserttotalframenumber pt}%
    }%
    \setlength{\metropolis@progressonsectionpage@blockwidth}{%
        0.05\textwidth - 0.05\metropolis@progressonsectionpage@blockborder
    }%
    \begin{tikzpicture}
        \fill[bg] (0,-\metropolis@progressonsectionpage@blockborder) rectangle (\textwidth, \metropolis@progressonsectionpage@linewidth + \metropolis@progressonsectionpage@blockborder);
        %\fill[fg] (0,0) rectangle (\metropolis@progressonsectionpage, \metropolis@progressonsectionpage@linewidth);

        \foreach \i in {1,...,20} {%
                \pgfmathparse{\insertframenumber*100/\inserttotalframenumber >= \i*100/20 ? 1 : 0}
                \ifthenelse {\pgfmathresult>0}
                {%
                    \pgfmathparse{\i*0.05-0.005}
                    \fill[fg] (\i*\metropolis@progressonsectionpage@blockwidth, 0) rectangle ++ (-\metropolis@progressonsectionpage@blockwidth + \metropolis@progressonsectionpage@blockborder, \metropolis@progressonsectionpage@linewidth);
                }
                {\node at (\i,0) {\i};}% otherwise do nothing
            }

        \node[color=white] at (0.5\textwidth, 0.5\metropolis@progressonsectionpage@linewidth) {\textnormal{%
                \fontsize{0.85\metropolis@progressonsectionpage@linewidth}{\metropolis@progressonsectionpage@linewidth}\selectfont loading
                \pgfmathparse{\insertframenumber*100/\inserttotalframenumber}%
                \pgfmathprintnumber[fixed,precision=2]{\pgfmathresult}\,\% complete...%
            }%
        };

    \end{tikzpicture}%
}
\makeatother

\newcommand\daynr{0}


\definecolor{ExColor}{HTML}{17819b}

\newcommand{\emptyWord}{\varepsilon}
\let \emptyset\varnothing
\newcommand{\SigmaStern}{\Sigma^{*}}
\newcommand{\absval}[1]{|#1|}
\newcommand{\defeq}{\vcentcolon=}
\newcommand{\eqdef}{=\vcentcolon}
\newcommand{\nimplies}{\centernot\implies}

\newcommand{\naturals}{\ensuremath{\mathbb{N}}}
\newcommand{\integers}{\ensuremath{\mathbb{Z}}}
\newcommand{\rationals}{\ensuremath{\mathbb{Q}}}
\newcommand{\reals}{\ensuremath{\mathbb{R}}}
\newcommand{\iffspace}{\ensuremath{\iff\ }}

\newcommand{\sus}[1]{%
    \tikz{\node[scale=0.05] at (0,0) {\amongUsOriginal{#1}{cyan}}}
}

\setbeamertemplate{footline}[text line]
{\parbox{\linewidth}{Fachgruppe Informatik\hfill\insertpagenumber\hfill Vorkurs Theoretische Informatik\vspace{0.2in}}}

\newcommand{\Center}[1]{
    \begin{frame}<handout:0>[standout]
        #1
    \end{frame}
}

\newcommand{\cleft}[2][.]{%
    \begingroup\colorlet{savedleftcolor}{.}%
    \color{#1}\left#2\color{savedleftcolor}%
}
\newcommand{\cright}[2][.]{%
    \color{#1}\right#2\endgroup
}

% Make subsections in toc small
\makeatletter
\setbeamertemplate{subsection in toc}{\small\leftskip=2em\inserttocsubsection\par}
\makeatother

\AtBeginDocument{%

    % Fix section pages in appendix
    \apptocmd{\appendix}{%
        \setbeamertemplate{section page}[simple]%
    }{}{}
}

\addtobeamertemplate{block begin}{}{\vskip 0em}
\addtobeamertemplate{block alerted begin}{}{\vskip 0em}
\addtobeamertemplate{block example begin}{}{\vskip 0em}

\newsavebox\tikzBox
\newenvironment{includetikzpicture}[1]{%
    \def\sizeArgument{#1}\begin{lrbox}{\tikzBox}\begin{tikzpicture}
            }{%
        \end{tikzpicture}\end{lrbox}\resizebox{\sizeArgument}{!}{\usebox\tikzBox}%
}

\pgfkeys{
    /sevenseg/.is family, /sevenseg,
    shrink/.estore in     = \sevensegShrink,    % avoids overlapping of segments
    oncolor/.estore in    = \sevensegOncolor,   % color of an ON segment
    offcolor/.estore in   = \sevensegOffcolor,  % color of an OFF segment
    size/.estore in       = \sevensegSize,      % height
    onSize/.estore in     = \sevensegOnsize,     % line width if ON
    offSize/.estore in    = \sevensegOffsize    % line width if OFF
}

\pgfkeys{
    /sevenseg,
    default/.style = {
        shrink = 0.1, 
        size = 1em, 
        oncolor = red, 
        offcolor = blue!70!black!40,
        onSize = 2,
        offSize =1.5
        }
}

\newcommand{\sevenseg}[2][]% options, values
{%
  \pgfkeys{/sevenseg, default, #1}%
    \begin{tikzpicture}[x=\sevensegSize,y=\sevensegSize,baseline=.25*\sevensegSize]%
        % unten li
        \path (0,0) ++(0,0) coordinate (P1);
        % unten re
        \path (0,0) ++(0.5,0) coordinate (P2);
        % mitte li
        \path (0,0) ++(90:0.5) coordinate (P3);
        % mitte re
        \path (P2)  ++(90:0.5) coordinate (P4);
        % oben li
        \path (P3)  ++(90:0.5) coordinate (P5);
        % oben re
        \path (P4)  ++(90:0.5) coordinate (P6);
        % then step through the 1/0 values in the segment array


    \foreach \val [count=\i from 0] in {#2} {%
      \ifthenelse{\equal{\val}{1}}%
        {\let\mycolor=\sevensegOncolor \let\mysize=\sevensegOnsize}%
        {\let\mycolor=\sevensegOffcolor \let\mysize=\sevensegOffsize}%

      % then draw segment depending on \i
      \ifthenelse{\equal{\i}{0}}{\path[draw=\mycolor, line width=\mysize] (P5) -- (P6);}{}%
      \ifthenelse{\equal{\i}{1}}{\path[draw=\mycolor, line width=\mysize] (P6) -- (P4);}{}%
      \ifthenelse{\equal{\i}{2}}{\path[draw=\mycolor, line width=\mysize] (P4) -- (P2);}{}%
      \ifthenelse{\equal{\i}{3}}{\path[draw=\mycolor, line width=\mysize] (P1) -- (P2);}{}%
      \ifthenelse{\equal{\i}{4}}{\path[draw=\mycolor, line width=\mysize] (P1) -- (P3);}{}%
      \ifthenelse{\equal{\i}{5}}{\path[draw=\mycolor, line width=\mysize] (P3) -- (P5);}{}%
      \ifthenelse{\equal{\i}{6}}{\path[draw=\mycolor, line width=\mysize] (P3) -- (P4);}{}%
    }
  \end{tikzpicture}%
}

\newcommand{\sevensegnum}[2][]%
{%                                          
    \ifthenelse{\equal{#2}{0}}{\sevenseg[#1]{1,1,1,1,1,1,0,}}{%
        \ifthenelse{\equal{#2}{1}}{\sevenseg[#1]{0,1,1,0,0,0,0,}}{%
            \ifthenelse{\equal{#2}{2}}{\sevenseg[#1]{1,1,0,1,1,0,1,}}{%
                \ifthenelse{\equal{#2}{3}}{\sevenseg[#1]{1,1,1,1,0,0,1,}}{%
                    \ifthenelse{\equal{#2}{4}}{\sevenseg[#1]{0,1,1,0,0,1,1,}}{%
                        \ifthenelse{\equal{#2}{5}}{\sevenseg[#1]{1,0,1,1,0,1,1,}}{%
                            \ifthenelse{\equal{#2}{6}}{\sevenseg[#1]{1,0,1,1,1,1,1,}}{%
                                \ifthenelse{\equal{#2}{7}}{\sevenseg[#1]{1,1,1,0,0,0,0,}}{%
                                    \ifthenelse{\equal{#2}{8}}{\sevenseg[#1]{1,1,1,1,1,1,1,}}{%
                                        \ifthenelse{\equal{#2}{9}}{\sevenseg[#1]{1,1,1,1,0,1,1,}}{%
                                            {\sevenseg[#1]{0,0,0,0,0,0,0,}}
                                        }
                                    }
                                }
                            }
                        }
                    }
                }
            }
        }
    }%
}

% Choose engine primitive
\makeatletter
\@ifundefined{pdfuniformdeviate}{%
  \let\RandDeviate\uniformdeviate   % XeLaTeX / LuaLaTeX
}{%
  \let\RandDeviate\pdfuniformdeviate % pdfLaTeX
}
\makeatother

% Fully expandable random bit (0 or 1)
\newcommand{\randbit}{\the\numexpr\RandDeviate 2\relax}

% Convert one random draw to a clean literal 0/1
\newcommand{\randbitZ}{%
  \ifnum\numexpr\RandDeviate 2\relax>0 1\else 0\fi
}

% Use it to build the seven-seg list
\newcommand{\randsevenseg}{%
  \sevenseg[]{\randbitZ,\randbitZ,\randbitZ,\randbitZ,\randbitZ,\randbitZ,\randbitZ}%
}


% Copyright 2018-2022 FIUS
%
% This file is part of theo-vorkurs-folien.
%
% theo-vorkurs-folien is free software: you can redistribute it and/or modify
% it under the terms of the GNU General Public License as published by
% the Free Software Foundation, either version 3 of the License, or
% (at your option) any later version.
%
% theo-vorkurs-folien is distributed in the hope that it will be useful,
% but WITHOUT ANY WARRANTY; without even the implied warranty of
% MERCHANTABILITY or FITNESS FOR A PARTICULAR PURPOSE.  See the
% GNU General Public License for more details.
%
% You should have received a copy of the GNU General Public License
% along with theo-vorkurs-folien.  If not, see <https://www.gnu.org/licenses/>.



% Configuration for slides

% The date of the first day of the Theo-Vorkurs in Format dd/mm/yyyy
\SetDate[10/10/2022]

% Invite URL to the Ersti-Telegram-Group. Used for text on slide as well as QR-Code
\newcommand\telegramurl{https://t.me/+Q92w5biyY903NjEy}

\renewcommand\daynr{3}
% Copyright 2018-2022 FIUS
%
% This file is part of theo-vorkurs-folien.
%
% theo-vorkurs-folien is free software: you can redistribute it and/or modify
% it under the terms of the GNU General Public License as published by
% the Free Software Foundation, either version 3 of the License, or
% (at your option) any later version.
%
% theo-vorkurs-folien is distributed in the hope that it will be useful,
% but WITHOUT ANY WARRANTY; without even the implied warranty of
% MERCHANTABILITY or FITNESS FOR A PARTICULAR PURPOSE.  See the
% GNU General Public License for more details.
%
% You should have received a copy of the GNU General Public License
% along with theo-vorkurs-folien.  If not, see <https://www.gnu.org/licenses/>.

% This sets the template for the titlepage. 
% Only change to the default is that the titlegraphic is not in the left upper but in the right lower corner
\setbeamertemplate{title page}{
    \begin{minipage}[b][\paperheight]{\textwidth}
    \vfill%
    \ifx\inserttitle\@empty
    \else\usebeamertemplate*{title}
    \fi
    \ifx\insertsubtitle\@empty
    \else\usebeamertemplate*{subtitle}
    \fi
    \usebeamertemplate*{title separator}
    \ifx\beamer@shortauthor\@empty
    \else\usebeamertemplate*{author}
    \fi
    \ifx\insertdate\@empty
    \else\usebeamertemplate*{date}
    \fi
    \ifx\insertinstitute\@empty
    \else\usebeamertemplate*{institute}
    \fi
    \ifx\inserttitlegraphic\@empty
    \else{\hfill\inserttitlegraphic\hspace{.1\textwidth}}
    \fi
    \vfill
    \vspace*{1mm}
    \end{minipage}
}


\title{Vorkurs Theoretische Informatik}

\if\daynr1
    \subtitle{Einführung in die Grundideen, Mengenlehre und Aussagenlogik}
    \newcommand\daynamestr{Montag}
\fi
\if\daynr2
    \subtitle{Grundlagen der Beweise}
    \newcommand\daynamestr{Dienstag}
    \AdvanceDate
\fi
\if\daynr3
    \subtitle{Induktion und Einführung in die Grammatik}
    \newcommand\daynamestr{Mittwoch}
    \AdvanceDate\AdvanceDate
\fi
\if\daynr4
    \subtitle{Einführung in reguläre Sprachen}
    \newcommand\daynamestr{Donnerstag}
    \AdvanceDate\AdvanceDate\AdvanceDate
\fi
\if\daynr5
    \subtitle{Einführung in reguläre Sprachen}
    \newcommand\daynamestr{Freitag}
    \AdvanceDate\AdvanceDate\AdvanceDate
\fi


\date{\daynamestr, \today}

\author{Arbeitskreis Theo-Vorkurs}
\institute{\href{https://fius.de}{Fachgruppe Informatik Universität Stuttgart}}
% \titlegraphic{\hfill\includegraphics[height=1.5cm]{logo.pdf}}

% sets the qr-code to the current handout slides on the title page. 
% This can be changed to let the qr-code appear on every page by exchanging \titlegraphic with \logo.
\titlegraphic{
    \only<1|handout:0>{\fbox{\parbox{2cm+.5em}{\centering
        Aktuelle Folien:\par \vspace{.5ex}
        \qrcode{\handouturl{\daynr}}
    }}}
}


\begin{document}

\maketitle

\begin{frame}[fragile]{Übersicht}
  \setbeamertemplate{section in toc}[sections numbered]
  \tableofcontents%[hideallsubsections]
\end{frame}

\section{Vollständige Induktion}

\subsubsection{Idee}
\begin{frame}[fragile]{Idee}
\begin{columns}
\column{0.5\textwidth}
    \begin{alertblock}{Zeige Aussagen der Form:\\\emph{Für alle $n\in\mathbb{N}$ gilt...}}
    \begin{enumerate}
        \item Zeige Aussage für das kleinste Element
        \item<1-> \only<7,8>{\alert<7>{Zeige, wenn Aussage für beliebiges n gilt, gilt sie auch für dessen Nachfolger, also n+1.}}\onslide<1-6>{Zeige, dass Aussage auch für das folgende Element gilt.}
        \item<2-6,8> \only<8>{\alert<8>{$\leadsto$ Aussage gilt für alle n.}}\onslide<2-6>{\small Zeige, dass Aussage auch für das folgende Element gilt.}
        \item<3-6> \footnotesize Zeige, dass Aussage auch für das folgende Element gilt.
        \item<4-6> \scriptsize Zeige, dass Aussage auch für das folgende Element gilt.
        \item<5-6> \tiny Zeige, dass Aussage auch für das folgende Element gilt.
        \item<6> \dots
    \end{enumerate}
    \end{alertblock}
\column{0.5\textwidth}
    \begin{figure}
        \centering
        \includegraphics[width=0.7\textwidth]{../figures/induction.png}
        %\caption{Idee}
        %
    \end{figure}
\end{columns}
\end{frame}

\subsubsection{Funktionsweise}
\begin{frame}[fragile]{Struktur}
    \begin{alertblock}{Zeige Aussagen der Form:\\\emph{Für alle $n\in\mathbb{N}$ gilt...}}
    \begin{enumerate}
        \item \alert{Induktionsanfang}\\Zeige Aussage für das kleinste Element
        \item \alert{Induktionsvorraussetzung}\\Zeige, unter der Vorraussetzung: \\\emph{die Aussage gelte für beliebiges n},\dots
        \item \alert{Induktionsschritt}\\\dots dann gilt die Aussage auch für dessen Nachfolger n+1.
        \item $\leadsto$ Aussage gilt für alle $n \in \mathbb{N}$.
    \end{enumerate}
    \end{alertblock}
\end{frame}

\begin{frame}[fragile]{Beispiel}
\center $\displaystyle\sum_{i = 0}^{n} (2i+1) = (n+1)^2,\quad\forall n \in\mathbb{N}$.
    \begin{figure}
        \centering
        \includegraphics[width=0.5\textheight]{../figures/Summe.png}\qquad \dots
        %\caption{Idee}
        %
    \end{figure}
\end{frame}

\begin{frame}[fragile]{Beispiel}
Zeigen Sie $\displaystyle\sum_{i = 0}^{n} (2i+1) = (n+1)^2,\quad\forall n \in\mathbb{N}$.
\begin{alertblock}{Induktionsanfang IA}
    Zeige Aussage gilt für $n\defeq0$:\\
    \begin{align*}
        \sum_{i = 0}^{0} (2i+1) &\overset{!}{=} (0+1)^2\\
        \iff 2 * 0 + 1 &\overset{!}{=}1^2\\
        \iff 1 &= 1 \qquad\checkmark
    \end{align*}
\end{alertblock}
\end{frame}

\begin{frame}[fragile]{Beispiel}
Zeigen Sie $\displaystyle\sum_{i = 0}^{n} (2i+1) = (n+1)^2,\quad\forall n \in\mathbb{N}$.
\begin{alertblock}{Induktionsanfang IA}
    Aussage gilt für $n\defeq0$, da $\sum_{i = 0}^{0} (2i+1) = 0^2$
\end{alertblock}
\begin{alertblock}{Induktionsvorraussetzung IV}
    Ang. Aussage gilt für $n \in\mathbb{N}$.
\end{alertblock}
\begin{alertblock}{Induktionsschritt IS}
    Zeige Aussage gilt für alle n+1 unter Nutzung der I.V.:\\
    $\sum_{i = 0}^{\alert{n+1}} (2i+1) \overset{!}{=} (\alert{(n+1)}+1)^2$
\end{alertblock}
\end{frame}

\begin{frame}[fragile]{Beispiel}
\small\begin{alertblock}{Induktionsschritt}
    Zeige Aussage gilt für alle n+1 unter Nutzung der I.V.:
    \begin{align*}
        \onslide<1->{&\sum_{i = 0}^{n+1} (2i+1)&\overset{!}{=} ((n+1)+1)^2}\\
        \onslide<2->{\iff&\sum_{i = 0}^{\alert<2>{n}} (2i+1) + \sum_{i = \alert<2>{n+1}}^{n+1} (2i+1)&\overset{!}{=} (n+2)^2}\\
        \onslide<3->{\iff&\sum_{i = 0}^{n} (2i+1) + ( 2(n+1)+1 )&\overset{!}{=} n^2 + 2 * 2n + 2^2}\\
        \onslide<4->{\overset{\alert<4>{IV}}\iff&\alert<4>{(n+1)^2} + ( 2(n+1)+1 )&\overset{!}{=} n^2+4n+4}\\
        \onslide<5->{\iff&n^2+2n+1^2+2n+2+1&\overset{!}{=} n^2+4n+4}\\
        \onslide<6>{\iff&n^2+4n+4&\alert<6>{=} n^2+4n+4}
    \end{align*}
\end{alertblock}
\end{frame}

\begin{frame}[fragile]{Beispiel}
Zeigen Sie $\displaystyle\sum_{i = 0}^{n} (2i+1) = (n+1)^2,\quad\forall n \in\mathbb{N}$.
\begin{alertblock}{Induktionsanfang IA}
    Aussage gilt für $n\defeq0$, da $\sum_{i = 0}^{0} (2i+1) = 1^2$
\end{alertblock}
\begin{alertblock}{Induktionsvorraussetzung IV}
    Ang. Aussage gilt für alle $n \in\mathbb{N}$.
\end{alertblock}
\begin{alertblock}{Induktionsschritt IS}
    Aussage gilt für alle n+1 unter Nutzung der I.V., da\\
    $\sum_{i = 0}^{n+1} (2i+1) = ((n+1)+1)^2$
\end{alertblock}
\alert{$\leadsto$ Aussage gilt für alle n.}\qed
\end{frame}


{\setbeamercolor{palette primary}{bg=ExColor}
\begin{frame}[fragile]{Denkpause}
    \begin{alertblock}{Aufgaben}
    Versuche dich an den folgenden Induktionsbeweisen.
    \end{alertblock}
    
    \metroset{block=fill}
    \begin{block}{Normal}
        $\displaystyle\sum_{i=0}^{n} i = \frac{n(n+1)}{2}$, für alle $n \in \mathbb{N}$
    \end{block}
    \begin{block}{Schwerer}
        $\displaystyle\prod_{i=1}^{n} 4^i = 2^{n(n+1)}$, für alle $n \in \mathbb{N}\setminus \{0\}$
    \end{block}
\end{frame}
}

%\subsubsection{Lösungen normal}
{\setbeamercolor{palette primary}{bg=ExColor}
\begin{frame}[fragile]{Lösungen: normale Aufgabe}
    Zu zeigen: $\displaystyle\sum_{i=0}^{n} i = \frac{n(n+1)}{2}$ gilt für alle $n \in \mathbb{N}$.
    \begin{alertblock}{Induktionsanfang IA}
        Aussage gilt für $n\defeq 0$, da $\displaystyle\sum_{i=0}^{1} i = 0 = \frac{0(0+1)}{2}$.
    \end{alertblock}
    \begin{alertblock}{Induktionsvorraussetzung IV}
        Ang. Aussage gilt für $n \in\mathbb{N}$.
    \end{alertblock}
    \begin{alertblock}{Induktionsschritt IS}
        Zeige Aussage gilt für alle n+1 unter Nutzung der I.V.:\\
        $\displaystyle\sum_{i=0}^{\alert{n+1}} i \overset{!}{=} \frac{(\alert{n+1})((\alert{n+1})+1)}{2}$
    \end{alertblock}
\end{frame}


\begin{frame}[fragile]{Lösungen: normale Aufgabe}
\small\begin{alertblock}{Induktionsschritt}
    Zeige Aussage gilt für alle n+1 unter Nutzung der I.V.:
    \begin{align*}
        \onslide<1->{&\displaystyle\sum_{i=0}^{\alert<1>{n+1}} i &\overset{!}{=} \frac{(\alert<1>{n+1})((\alert<1>{n+1})+1)}{2}}\\
        \onslide<2->{\iff&(\displaystyle\sum_{i=0}^{n} i)+(n+1) &\overset{!}{=} \frac{(n+1)(n+2)}{2}}\\
        \onslide<3->{\iff&(\displaystyle\sum_{i=0}^{n} i)+(n+1) &\overset{!}{=} \frac{n^2+3n+2}{2}}\\
        \onslide<4->{\overset{\alert<4>{IV}}\iff&\alert<4>{\frac{n(n+1)}{2}}+(n+1) &\overset{!}{=} \frac{n^2+3n+2}{2}}\\
        \onslide<5->{\iff&\frac{n^2+n}{2}+\frac{2n+2}{2} &\overset{!}{=} \frac{n^2+3n+2}{2}}\\
        \onslide<6->{\iff&\frac{n^2+3n+2}{2} &\alert{=} \frac{n^2+3n+2}{2}}\\
    \end{align*}
\end{alertblock}
\end{frame}


\begin{frame}[fragile]{Lösungen: normale Aufgabe}
    Zu zeigen: $\displaystyle\sum_{i=0}^{n} i = \frac{n(n+1)}{2}$ gilt für alle $n \in \mathbb{N}$.
    \begin{alertblock}{Induktionsanfang IA}
        Aussage gilt für $n\defeq 0$, da $\displaystyle\sum_{i=0}^{1} i = 0 = \frac{0(0+1)}{2}$.
    \end{alertblock}
    \begin{alertblock}{Induktionsvorraussetzung IV}
        Ang. Aussage gilt für $n \in\mathbb{N}$.
    \end{alertblock}
    \begin{alertblock}{Induktionsschritt IS}
        Zeige Aussage gilt für alle n+1 unter Nutzung der I.V.:\\
        $\displaystyle\sum_{i=0}^{\alert{n+1}} i \overset{!}{=} \frac{(\alert{n+1})((\alert{n+1})+1)}{2}$ gilt für alle $n \in \mathbb{N}$
    \end{alertblock}
    \alert{$\leadsto$ Aussage gilt für alle n.}\qed
\end{frame}
}

% \begin{frame}[standout]
%   Fragen dazu?
% \end{frame}

%\subsubsection{Lösungen schwerer}
{\setbeamercolor{palette primary}{bg=ExColor}
\begin{frame}[fragile]{Lösungen: schwerere Aufgabe}
    Zu zeigen: $\displaystyle\prod_{i=1}^{n} 4^i = 2^{n(n+1)}$, für alle $n \in \mathbb{N}\setminus \{0\}$.
    \begin{alertblock}{Induktionsanfang IA}
        Aussage gilt für $n\defeq 1$, da $\displaystyle\prod_{i=1}^{1} 4^i = 4^1 = 4 = 2^2 = 2^{1(1+1)}$.
    \end{alertblock}
    \begin{alertblock}{Induktionsvorraussetzung IV}
        Ang. Aussage gilt für $n \in\mathbb{N}\setminus \{0\}$.
    \end{alertblock}
    \begin{alertblock}{Induktionsschritt IS}
        Zeige Aussage gilt für alle n+1 unter Nutzung der I.V.:\\
        $\displaystyle\prod_{i=1}^{\alert{n+1}} 4^i \overset{!}{=} 2^{(\alert{n+1})((\alert{n+1})+1)}$
    \end{alertblock}
\end{frame}

\begin{frame}[fragile]{Lösungen: schwerere Aufgabe}
\small\begin{alertblock}{Induktionsschritt}
    Zeige Aussage gilt für alle n+1 unter Nutzung der I.V.:
    \begin{align*}
        \onslide<1->{&\displaystyle\prod_{i=1}^{\alert<1>{n+1}} 4^i &\overset{!}{=} 2^{(\alert<1>{n+1})((\alert<1>{n+1})+1)}}\\
        \onslide<2->{\iff&(\displaystyle\prod_{i=1}^{n} 4^i) * 4^{(n+1)} &\overset{!}{=} 2^{(n+1)(n+2)}}\\
        \onslide<3->{\overset{\alert<3>{IV}}\iff&\alert<3>{(2^{n(n+1)})} * 4^{(n+1)} &\overset{!}{=} 2^{n^2+3n+2}}\\
        \onslide<4->{\iff&2^{n^2+n} * 2^{2(n+1)} &\overset{!}{=} 2^{n^2+3n+2}}\\
        \onslide<5->{\iff&2^{n^2+n} * 2^{2n+2} &\overset{!}{=} 2^{n^2+3n+2}}\\
        \onslide<6->{\iff&2^{(n^2+n)+(2n+2)} &\overset{!}{=} 2^{n^2+3n+2}}\\
        \onslide<7->{\iff&2^{n^2+3n+3} &\alert{=} 2^{n^2+3n+2}}
    \end{align*}
\end{alertblock}
\end{frame}


\begin{frame}[fragile]{Lösungen: schwerere Aufgabe}
     Zu zeigen: $\displaystyle\prod_{i=1}^{n} 4^i = 2^{n(n+1)}$, für alle $n \in \mathbb{N}\setminus \{0\}$.
    \begin{alertblock}{Induktionsanfang IA}
        Aussage gilt für $n\defeq 1$, da $\displaystyle\prod_{i=1}^{1} 4^i = 4^1 = 4 = 2^2 = 2^{1(1+1)}$.
    \end{alertblock}
    \begin{alertblock}{Induktionsvorraussetzung IV}
        Ang. Aussage gilt für $n \in\mathbb{N}\setminus \{0\}$.
    \end{alertblock}
    \begin{alertblock}{Induktionsschritt IS}
        Zeige Aussage gilt für alle n+1 unter Nutzung der I.V.:\\
        $\displaystyle\prod_{i=1}^{\alert{n+1}} 4^i \overset{!}{=} 2^{(\alert{n+1})((\alert{n+1})+1)}$ gilt für alle $n \in \mathbb{N}\setminus \{0\}$
    \end{alertblock}
    \alert{$\leadsto$ Aussage gilt für alle n.}\qed
\end{frame}
}

% \begin{frame}[standout]
%   Fragen dazu?
% \end{frame}


\Center{Murmelpause}

%Vollständige Induktion aus Tag 2
% \section{Wiederholung: Vollständige Induktion}

% \subsection{Definition}

% \begin{frame}[fragile]{Definition}
%     \begin{enumerate}
%         \item \textbf{Induktionsanfang} (Gilt die Aussage für ein $n_0$) \\
%         \item \textbf{Induktionsvoraussetzung} (wir nehmen dann an, die Aussage gilt tatsächlich)
%         \item \textbf{Induktionsschritt} (Hier zeigen wir, dass für alle $n \in \mathbb{N}$ die Aussage gilt, unter Verwendung der IV)
%     \end{enumerate}
% \end{frame}

\subsubsection{formalere Definition}
\begin{frame}{Definition nochmal formaler}
    \begin{equation*}
        (\forall n \in \mathbb{N}_{n_0}: P(n)) \iff (P(n_0) \wedge \forall n \in \mathbb{N}_{n_0}: (P(n) \implies P(n+1)))
    \end{equation*}    
\end{frame}

\begin{frame}{Definition nochmal formaler}
    \onslide<1->$(\forall n \in \mathbb{N}_{n_0}: P(n)) \iff (\alert<2>{\underbrace{P(n_0)}_{\text{IA}}}\wedge \overbrace{\alert<3>{\forall n \in \mathbb{N}_{n_0}:} (\alert<4>{\underbrace{P(n)}_{\text{IV}}} \alert<5>{\implies P(n+1)})}^{\text{IS}})$
    \begin{enumerate}
        \item<2->\alert<2>{\textbf{IA:} $n = n_0$}
        \item<3->\onslide<3->{\alert<3>{\textbf{IS:} Sei $n\in\mathbb{N}_{n_0}$ beliebig.}}
        \onslide<4->{\alert<4>{Ang. es gilt P(n). \tiny{\textbf{(IV)}}}}    
        \item<5->\alert{$\leadsto$ Zeigen, dass P(n+1) gilt, unter Verwendung von P(n) \tiny{\textbf{(IV)}}}
    \end{enumerate}
\end{frame}

% \begin{frame}{Aufgabe zur Wiederholung}
%     Warum das ganze nur für Summen.\\
%     Angenommen $n^3-n$ ist durch 3 teilbar für alle natürlichen Zahlen.\\
%     Wie gehen wir dann hier vor?
% \end{frame}

% \begin{frame}{Aufgabe zur Wiederholung}
%     \begin{itemize}
%         \item<1->
%             Schreiben wir das ganze erst mal etwas Mathematischer.
%         \item<2->
%             $3 \mid n^3-n$, also 3 teilt $n^3-n$
%         \item<3->
%             Jetzt vollständige Induktion
%     \end{itemize}
% \end{frame}

% \begin{frame}{Vollständige Induktion}
%     \begin{enumerate}
%         \item<1->
%             \textbf{IA:} n = 1
%                 \begin{equation*}
%                     3 \mid 1^3 - 1 \iff 3 \mid 1 - 1 \iff 3 \mid 0 \qquad \checkmark
%                 \end{equation*}
%         \item<2->
%             % \textbf{IV:}\\
%             % Da $3 \mid n^3-n$ für 1 gilt, existiert also eine Zahl $n\in \mathbb{N}$ (beliebig aus den natürlichen Zahlen, hier 1 da wir es bereits dafür gezeigt haben), für welche die Aussage $3 \mid n^3 -n$ gilt.
%              \textbf{IS:} Sei $n \in \mathbb{N}$ beliebig. Ang., es gilt $3\mid n^3-n$ (IV)
%             \begin{align*}
%                 3 \mid (n+1)^3-(n+1) &\iff 3\mid(n+1)^3 - n - 1\\
%                 &\iff 3\mid n^3 + 3n^2 + 3n + 1 - n - 1\\
%                 &\iff 3\mid \underbrace{n^3 - n}_{\text{Induktionsvoraussetzung}} + \underbrace{3n^2 + 3n}_{\text{vielfache von 3}} + \underbrace{1 - 1}_{= 0}\\
%                 &\iff 3\mid(n^3 - n) + 3(n^2 + n)
%             \end{align*}
%         \item<3->
%         \textbf{Fazit:}\\
%             Nach Voraussetzung ist der erste Summand durch 3 teilbar, und der zweite Summand ist ein vielfaches von 3. Somit ist auch die Summe durch 3 teilbar.
%     \end{enumerate}    
% \end{frame}

% \begin{frame}[fragile]{Ein Aufgabe zur Übung}
%     \begin{itemize}
%         \item Zeigen Sie, dass für alle natürlichen Zahlen $n \geq 4$ gilt: \\
%         \begin{center}
%             $n! > 2^n$\\
%         \end{center}
%     \end{itemize}
% \end{frame}

% {\setbeamercolor{palette primary}{bg=ExColor}
% \begin{frame}{Lösung}
%     \begin{enumerate}
%         \item 
%             \textbf{IA:} n = 4
%             \begin{equation*}
%                 4! = 4 \cdot 3 \cdot 2 \cdot 1 = 24 > 16 = 2^4
%             \end{equation*}
%         \item    
%             \textbf{IV:}
%             Die Aussage $n! > 2^n$ gilt für n=4, also existiert ein $x\in \mathbb{N}$, sodass diese Aussage gilt.
%         \item    
%             \textbf{IS:} Also gilt die Aussage für n+1
%             \begin{align*}
%                 (n+1)! &= (n+1) \cdot n!\\
%                 &\overset{\text{nach IV.}}{>} \underbrace{(n+1)}_{\text{da n min. 4}} \cdot 2^n\\
%                 &\overset{(n+1)>2}{>} 2 \cdot 2^n = 2^{n+1}
%             \end{align*}
%         \item
%             \textbf{Fazit:}\\
%             Somit ist für alle $n\in \mathbb{N}$(beliebige natürliche Zahl) gezeigt, dass $n! > 2^n$ für $n \geq 4$.
%     \end{enumerate}
% \end{frame}
% }

% %An der Tafel die Lösung besprechen 


% Copyright 2018-2022 FIUS
%
% This file is part of theo-vorkurs-folien.
%
% theo-vorkurs-folien is free software: you can redistribute it and/or modify
% it under the terms of the GNU General Public License as published by
% the Free Software Foundation, either version 3 of the License, or
% (at your option) any later version.
%
% theo-vorkurs-folien is distributed in the hope that it will be useful,
% but WITHOUT ANY WARRANTY; without even the implied warranty of
% MERCHANTABILITY or FITNESS FOR A PARTICULAR PURPOSE.  See the
% GNU General Public License for more details.
%
% You should have received a copy of the GNU General Public License
% along with theo-vorkurs-folien.  If not, see <https://www.gnu.org/licenses/>.

{\setbeamercolor{palette primary}{bg=ExColor}
	\begin{frame}[fragile]{Aufgabe}
		\metroset{block=fill}
		\begin{alertblock}{Die folgende Induktion zeigt eine seltsame Aussage.}
			Ist der Beweis korrekt geführt? Was ist passiert?
		\end{alertblock}
		\metroset{block=transparent}
		Sei $A(n)\defeq$ \emph{In einer Herde aus $n$ Telefonen haben alle die selbe Farbe.}\\
		Zu zeigen: $A(n)$ gilt für alle $n \in \mathbb{N} \setminus \{0\} $.
		\begin{alertblock}{Induktionsanfang (IA)}
			$A(1)$: Aussage gilt für $n\defeq 1$, da ein Telefon nur eine Farbe haben kann.
		\end{alertblock}
		\begin{alertblock}{Induktionsvorraussetzung (IV)}
			Ang. $A(n)$ gilt für $n\geq1$.
		\end{alertblock}
		\begin{alertblock}{Induktionsschritt (IS)}
			Zeige Aussage gilt für alle $n+1$ unter Nutzung der IV:\\
			D.h. wir zeigen $A(n+1)=$ \emph{In einer Herde aus $n+1$ Telefonen haben alle die selbe Farbe.}
		\end{alertblock}
	\end{frame}
	\begin{frame}[fragile]{Aufgabe}
		\footnotesize{
			\begin{alertblock}{Induktionsschritt (IS)}
				Wir betrachten eine Herde aus $n+1$ Telefonen:
				\[\underbrace{\text{\Telefon}\text{\Telefon}\text{\Telefon}\text{\Telefon}\text{\Telefon}\dots\text{\Telefon}\text{\Telefon}}_{n+1}\]
				Wir sondern ein Telefon aus und betrachten den Rest. Nach I.V. haben diese alle die selbe Farbe.
				\[\underbrace{\alert{\text{\Telefon}\text{\Telefon}\text{\Telefon}\text{\Telefon}\text{\Telefon}\dots\text{\Telefon}}}_{n}\text{\Telefon}\]
				Jetzt sondern wir ein anderes Telefon aus.
				\[\alert{\text{\Telefon}}\underbrace{\alert{\text{\Telefon}\text{\Telefon}\text{\Telefon}\text{\Telefon}\dots\text{\Telefon}}\text{\Telefon}}_{n}\]
				Die übrigen $n$ Telefone haben nach I.V. wieder die selbe Farbe.
				\[\underbrace{\alert{\text{\Telefon}\text{\Telefon}\text{\Telefon}\text{\Telefon}\text{\Telefon}\dots\text{\Telefon}\text{\Telefon}}}_{n+1}\]
				Also haben alle $n+1$ Telefone die selbe Farbe.
				$\leadsto$ $A(n)$ gilt für alle $n$.
			\end{alertblock}
		}
	\end{frame}
}

{\setbeamercolor{palette primary}{bg=ExColor}
	\begin{frame}<handout:0>[fragile]{Lösungen}
		\small{
			\metroset{block=fill}
			\begin{block}{Das Problem}
				Die Vorangehensweise erfordert, dass die betrachteten Mengen an $n-1$ Telefonen mindestens ein gemeinsames Element haben. Sie teilen dann die Farbe dieses Elements. Allerdings gibt es ein überlappendes Element erst ab $n=3$ Telefonen:
				\[
					A(2):\alert{\text{\Telefon}\text{\Telefon}} \color{black} \leadsto 
					A(3):\alert{\rlap{$\overbrace{\phantom{\text{\Telefon\Telefon}}}^{A(2)}$}\text{\Telefon}\underbrace{\text{\Telefon\Telefon}}_{A(2)}}
					\leadsto \dots \leadsto A(n): \alert{\rlap{$\overbrace{\phantom{\text{\Telefon\Telefon} ... \text{\Telefon}}}^{A(n-1)}$}\text{\Telefon}\underbrace{\text{\Telefon} ... \text{\Telefon\Telefon}}_{A(n-1)}}
				\]
				Das Problem ist nun, dass $A(2)$ nicht zwangsweise erfüllt sein muss! Somit ist die Voraussetzung $A(2)$ verletzt und wir können keine weiteren Folgerungen über $A(n)$ machen.

				\begin{figure}
				\resizebox{.4\textwidth}{!}{
					\centering%
					\begin{subfigure}{0.3\textwidth}
						\centering%
						\includegraphics[height=0.5in]{../figures/telephoneGreen.png}
					\end{subfigure}
					$\qquad$
					\begin{subfigure}{0.3\textwidth}
						\centering%
						\includegraphics[height=0.5in]{../figures/telephoneWhite.png}
					\end{subfigure}
				}
					\caption{Beide Telefone erfüllen jeweils $A(1)$, zusammen aber nicht $A(2)$}
				\end{figure}

			\end{block}
		}
	\end{frame}
}


\Center{Murmelpause}

\section{Grammatiken}

\begin{frame}[fragile]{Wörter in Sprachen}
Wir können inzwischen Sprachen in Mengenschreibweise darstellen.\\Aber welche Wörter sind enthalten?\\
\vspace{0.3cm}
Wir können weitere Regeln formulieren, mit denen wir von einem gegebenen Startpunkt aus alle Wörter einer Sprache erzeugen können.
\end{frame}

\begin{frame}[fragile]{Beispiel Worterzeugung}
    \small{Wir betrachten L = \{$ww^R\;|\;w^R\text{ ist w rückwärts, }w \in \{a, b\}^n, n>0, n\in \mathbb{N}$\}\\
    Hier ist z.B. $\alert<1>{w}w^R$ = \alert<1>{ababb}bbaba $\in$ L.}\\
    \begin{enumerate}
    \item <2-> 
            \alert<2,5>{Wir beginnen mit einer Variablen S}
    \item <3-> 
            \alert<3>{Wir formulieren Regeln um S umzuwandeln}
            \alert<4>{\onslide<4->{
            \begin{align*}\alert<6,8>{S \rightarrow aSa}&\text{ oder }\alert<7,9>{S \rightarrow bSb}\\\text{oder }S \rightarrow aa &\text{ oder }\alert<10>{S \rightarrow bb}\end{align*}}}\vspace{-0.3in}
    \item <5->
            \alert<5>{Damit können wir jetzt Wörter aus der Sprache beschreiben.}\\
            z.B.: \alert<6>{a}\alert<7>{b}\alert<8>{a}\alert<9>{b}\alert<10>{bb}\alert<9>{b}\alert<8>{a}\alert<7>{b}\alert<6>{a} $\leadsto$ \only<5>{\alert<5>{S}}\only<6>{\alert<6>{aSa}}\only<7>{a\alert<7>{bSb}a}\only<8>{ab\alert<8>{aSa}ba}\only<9>{aba\alert<9>{bSb}aba}\only<10->{abab\alert<10>{bb}baba}
    \item <11> \alert<11>{Wir nennen diese Umformungsregeln Produktionsregeln.}
    \end{enumerate}
\end{frame}

\subsubsection{Produktionsregeln}
\begin{frame}{Produktionsregeln}
    \begin{alertblock}{Einschränkungen}
    \begin{itemize}
        \item \alert{\emph{Nichtterminale}} werden meist durch Großbuchstaben repräsentiert und müssen durch Produktionsregeln abgeändert werden
        \item \alert{\emph{Terminale}} werden meist durch Kleinbuchstaben repräsentiert und sollten \emph{nicht} durch weitere Produktionsregeln abgeändert werden
        \item Mehrere Symbole können auf einen Schlag überführt werden. Dabei sollten die Terminale nicht entfernt oder umsortiert werden.\\
        z.B. $AB \rightarrow CD$ ist erlaubt.\\
        Auch $abAB \rightarrow BbAa$, aber das gehört sich nicht.
    \end{itemize}
    \end{alertblock}
\end{frame}

\begin{frame}{Weitere Beispiele für Produktionen}
    \begin{alertblock}{Aufgaben}
    Gesucht: Produktionsregeln für die folgenden Sprachen.
    \end{alertblock}
    \metroset{block=fill}
    \begin{exampleblock}{$L_1 = \{a\}^*$}
    $\onslide<2->{P=\{S \rightarrow aS \only<3->{\mid \emptyWord}\}}$
    \end{exampleblock}
    \only<4->{
    \begin{exampleblock}{$L_2 = \{a, b\}^*$}
    $\onslide<5->{P=\{S \rightarrow  aS \only<6->{\mid bS}\only<7->{\mid \emptyWord}\}}$
    \end{exampleblock}
    }
\end{frame}

{\setbeamercolor{palette primary}{bg=ExColor}
\begin{frame}{Denkpause}
    \begin{alertblock}{Aufgaben}
    Findet Produktionsregeln für die folgenden Sprachen.
    \end{alertblock}
    \metroset{block=fill}
    \begin{block}{Normal}
    \begin{itemize}
        \item $L_1 = \{a^{2n}\;|\;n\in\mathbb{N}\}$
        \item $L_2 = \{a^nb^nc^m\;|\;n, m\in\mathbb{N}\}$
        \item $L_3 = \{uv\;|\;u\in\{a,b\}^\ast,\;v\in\{c,d\}\}$
        \item $L_4 = \{w\;|\;|w| = 3, w\in \{a,b,c\}^*\}$
    \end{itemize}
    \end{block}
    \begin{block}{Etwas Schwerer}
    \begin{itemize}
        \item $L_5 = \{a^n\;|\;n \equiv 1 \bmod 3\}$
        \item $L_6 = \{w\;|\;|w|_a = 3, |w|_b = 1, w\in \{a,b,c\}^*\}$
        \item $L_7 = \{uv\;|\;u\in\{\text{\Rewind, \MoveUp, \Forward, \MoveDown}\}^\ast,\;v\in\{\text{\Stopsign}\}\}$
        \item $L_8 = \{w\mid |w|=2, w \in \{a, b\}\}$
    \end{itemize}
    \end{block}
\end{frame}
}

{\setbeamercolor{palette primary}{bg=ExColor}
\begin{frame}{Lösungen}
Alle Lösungen sind Beispiellösungen, es sind auch andere möglich.
    \begin{itemize}
        \item<1-> \alert<1>{$P_1 = \{S\rightarrow aaS\;|\;\emptyWord$\}}
        \item<2-> \alert<2>{$P_2 = \{S\rightarrow AB$, $A\rightarrow aAb \;|\; ab\; |\;\emptyWord$, $B\rightarrow cB \;|\; \emptyWord$\}}
        \item<3-> \alert<3>{$P_3 = \{S\rightarrow UV$, $U\rightarrow aU \;|\; bU \; |\; \emptyWord$, $V\rightarrow c \;|\; d$\}}
        \item<4-> \alert<4>{$P_4 = \{S\rightarrow XXX$, $X\rightarrow a \;|\; b \;|\; c$\}}
        \item<5-> \alert<5>{$P_5 = \{S\rightarrow a \;|\; aaaS\}$}
        \item<6-> \alert<6>{$P_6 = \{S\rightarrow AAAB$, $AB\rightarrow BA$, 
        $A\rightarrow cA \;|\; Ac \;|\; a$, 
        $B\rightarrow cB \;|\; Bc \;|\; b$\}}
        \item<7-> \alert<7>{$P_7 = \{S\rightarrow U\text{\Stopsign} \;|\; \text{\Stopsign}$, $U\rightarrow \text{\Rewind} U \;|\; \text{\MoveUp} U \;|\; \text{\Forward} U \;|\; \text{\MoveDown} U \;|\;\emptyWord$\}}
        \item<8-> \alert<8>{$P_8 = \{\} \leadsto$ Es gibt keine Produktionsregeln!}
    \end{itemize}
\end{frame}
}  


\Center{Murmelpause}

% Copyright 2018-2024 FIUS
%
% This file is part of theo-vorkurs-folien.
%
% theo-vorkurs-folien is free software: you can redistribute it and/or modify
% it under the terms of the GNU General Public License as published by
% the Free Software Foundation, either version 3 of the License, or
% (at your option) any later version.
%
% theo-vorkurs-folien is distributed in the hope that it will be useful,
% but WITHOUT ANY WARRANTY; without even the implied warranty of
% MERCHANTABILITY or FITNESS FOR A PARTICULAR PURPOSE.  See the
% GNU General Public License for more details.
%
% You should have received a copy of the GNU General Public License
% along with theo-vorkurs-folien.  If not, see <https://www.gnu.org/licenses/>.

\subsubsection{formale Notation}
\begin{frame}[fragile]{Formale Notation}
  Wir beschreiben eine \alert{\emph{Grammatik}} durch ein geordnetes \alert{\emph{Tupel}} $G = (V, \Sigma, P, S)$
  \begin{itemize}
    \item $V$ ist die Menge der verwendeten Nichtterminale
    \item $\Sigma$ die Menge der Terminale bzw. unser Alphabet
    \item $P$ ist die Menge der Produktionsregeln
    \item $S$ ist die Startvariable
  \end{itemize}
  \metroset{block=fill}
  \begin{exampleblock}{Beispiel für  L = \{$ww^R \mid w \in \{a, b\}^n, \; n \geq 1$\}}
    $G = (V,\Sigma,P,S)$ mit\\
    $V = \{S\}$\\
    $\Sigma = \{a,b\}$\\
    $P = \{S \rightarrow aSa, S \rightarrow bSb, S \rightarrow aa, S \rightarrow bb$\}\\
    \qquad bzw. kurz: $P = \{S \rightarrow aSa\ |\ bSb\ |\ aa\ |\ bb$\}
  \end{exampleblock}
\end{frame}

{\setbeamercolor{palette primary}{bg=ExColor}
\begin{frame}{Denkpause}
  \begin{columns}
    \column{0.5\textwidth}
    \begin{alertblock}{Knifflige Aufgabe}
      Bob will durch das Labyrinth laufen. Er hat folgende Möglichkeiten:\\
      $\Sigma = \{\text{\Rewind}, \text{\MoveUp}, \text{\Forward}, \text{\MoveDown}\}$
      \begin{itemize}
        \item Bob kann nicht auf ein Feld zurücktreten, von dem er gerade kam
        \item Bob geht bei jedem Schritt ein Feld in die angegebene Richtung
      \end{itemize}
    \end{alertblock}
    \column{0.5\textwidth}
    \begin{figure}
      \centering
      % Copyright 2018-2024 FIUS
%
% This file is part of theo-vorkurs-folien.
%
% theo-vorkurs-folien is free software: you can redistribute it and/or modify
% it under the terms of the GNU General Public License as published by
% the Free Software Foundation, either version 3 of the License, or
% (at your option) any later version.
%
% theo-vorkurs-folien is distributed in the hope that it will be useful,
% but WITHOUT ANY WARRANTY; without even the implied warranty of
% MERCHANTABILITY or FITNESS FOR A PARTICULAR PURPOSE.  See the
% GNU General Public License for more details.
%
% You should have received a copy of the GNU General Public License
% along with theo-vorkurs-folien.  If not, see <https://www.gnu.org/licenses/>.

\definecolor{labyrinthLine}{RGB}{0,51,102}
\definecolor{labyrinthHead}{RGB}{0,25,50}
\definecolor{labyrinthField}{RGB}{255,230,204}
\definecolor{labyrinthPath}{RGB}{130,179,102}
\definecolor{labyrinthDecision}{RGB}{213,232,212}
\definecolor{labyrinthDecisionText}{RGB}{18,117,181}
\providecommand{\labyrinthSize}{\textwidth}
\providecommand{\labyrinthVariant}{None}
\begin{includetikzpicture}{\labyrinthSize}[x=1mm,y=1mm]
    \tikzset{every path/.style={line width=0.4}}
    % Strichmaennchen
    \draw (15,116.5)--(15,108.5);
    \draw (11,115.5)--(19,115.5);
    \draw[draw=black,fill=labyrinthHead] (15,119) circle[radius=2.5];
    \draw (15,108.5)--(10,103);
    \draw (15,108.5)--(20,103);
    \node[align=center] at (5,112) {\Large Mei};

    % Labyrinth
    \tikzset{every path/.style={line width=5,draw=labyrinthLine}}
    \draw (30,90)--(180,90);
    \draw (0,90)--(0,0)--(180,0)--(180,60)--(150,60);
    \draw (120,0)--(120,30);
    \draw (90,30)--(150,30);
    \draw (30,60)--(120,60);
    \draw (60,60)--(60,30)--(30,30);
    \draw[-{Stealth[inset=0pt, length=12, angle'=60]},line width=7] (15,102)--(15,89);
    \draw[-{Stealth[inset=0pt, length=12, angle'=60]},line width=7] (179,75)--(192,75);

    % Felder
    \tikzset{every path/.style={draw=none,fill=labyrinthField},every circle/.style={radius=11}}
    \foreach \x in {0,...,5}
    \foreach \y in {0,...,2}
        {\fill (15 + \x * 30,15 + \y * 30) circle;}

    % Pfade
    \tikzset{every path/.style={-{Stealth},draw=labyrinthPath,fill=none,rounded corners=10,line width=5}}
    \ifthenelse{\equal{\labyrinthVariant}{Direkt}}
    {
        \draw (15,86)--(15,75)--(176,75); % Direkt
    }{}
    \ifthenelse{\equal{\labyrinthVariant}{Indirekt}}
    {
        \draw (15,86)--(15,15)--(75,15)--(75,45)--(135,45)--(135,75)--(176,75); % Indirekt
    }{}
    \ifthenelse{\equal{\labyrinthVariant}{Uhrzeigersinn}}
    {
        \draw (15,86)--(15,75)--(135,75)--(135,45)--(75,45)--(75,15)--(15,15)--(15,75)--(176,75); % Uhrzeigersinn
        \draw[line width=2,draw=black] (60,82.75)--(90,82.75);
    }{}
    \ifthenelse{\equal{\labyrinthVariant}{GegenUhrzeigersinn}}
    {
        \draw (15,86)--(15,15)--(75,15)--(75,45)--(135,45)--(135,75)--(15,75)--(15,15)--(75,15)--(75,45)--(135,45)--(135,75)--(176,75); % GegenUhrzeigersinn
        \draw[line width=2,draw=black] (90,82.75)--(60,82.75);
    }{}

    \ifthenelse{\equal{\labyrinthVariant}{DecisionPoints}}
    {
    \fill[draw=none,fill=labyrinthDecision] (15,75) circle;
    \fill[draw=none,fill=labyrinthDecision] (135,75) circle;
    \tikzset{every path/.style={-{Stealth},draw=labyrinthDecisionText,line width=2},every node/.style={font=\Huge,text=labyrinthDecisionText,text centered}}
    \node[align=center] at (19,79) {$A_r$};
    \node[align=center] at (11,71) {$A_u$};
    \draw (34,75)--(26,75);
    \draw (15,56)--(15,64);
    \draw[-] (9,81)--(21,69);

    \node[align=center] at (131,79) {$B_l$};
    \node[align=center] at (139,71) {$B_u$};
    \draw (116,75)--(124,75);
    \draw (135,56)--(135,64);
    \draw[-] (129,69)--(141,81);
    }{}
\end{includetikzpicture}
\let\labyrinthVariant\relax
\let\labyrinthSize\relax
      \caption{Bobs Problem}

    \end{figure}
  \end{columns}
  \alert{Geben Sie eine Grammatik an, welche die Sprache beschreibt, die Bob durch alle ihm möglichen Wege des Labyrinths führt.}
\end{frame}
}

{\setbeamercolor{palette primary}{bg=ExColor}
\begin{frame}{Denkpause}
  \begin{alertblock}{Beispiel}
    \begin{figure}
      \centering
      \def\labyrinthVariant{Direkt}
      \def\labyrinthSize{0.9\textwidth}
      % Copyright 2018-2024 FIUS
%
% This file is part of theo-vorkurs-folien.
%
% theo-vorkurs-folien is free software: you can redistribute it and/or modify
% it under the terms of the GNU General Public License as published by
% the Free Software Foundation, either version 3 of the License, or
% (at your option) any later version.
%
% theo-vorkurs-folien is distributed in the hope that it will be useful,
% but WITHOUT ANY WARRANTY; without even the implied warranty of
% MERCHANTABILITY or FITNESS FOR A PARTICULAR PURPOSE.  See the
% GNU General Public License for more details.
%
% You should have received a copy of the GNU General Public License
% along with theo-vorkurs-folien.  If not, see <https://www.gnu.org/licenses/>.

\definecolor{labyrinthLine}{RGB}{0,51,102}
\definecolor{labyrinthHead}{RGB}{0,25,50}
\definecolor{labyrinthField}{RGB}{255,230,204}
\definecolor{labyrinthPath}{RGB}{130,179,102}
\definecolor{labyrinthDecision}{RGB}{213,232,212}
\definecolor{labyrinthDecisionText}{RGB}{18,117,181}
\providecommand{\labyrinthSize}{\textwidth}
\providecommand{\labyrinthVariant}{None}
\begin{includetikzpicture}{\labyrinthSize}[x=1mm,y=1mm]
    \tikzset{every path/.style={line width=0.4}}
    % Strichmaennchen
    \draw (15,116.5)--(15,108.5);
    \draw (11,115.5)--(19,115.5);
    \draw[draw=black,fill=labyrinthHead] (15,119) circle[radius=2.5];
    \draw (15,108.5)--(10,103);
    \draw (15,108.5)--(20,103);
    \node[align=center] at (5,112) {\Large Mei};

    % Labyrinth
    \tikzset{every path/.style={line width=5,draw=labyrinthLine}}
    \draw (30,90)--(180,90);
    \draw (0,90)--(0,0)--(180,0)--(180,60)--(150,60);
    \draw (120,0)--(120,30);
    \draw (90,30)--(150,30);
    \draw (30,60)--(120,60);
    \draw (60,60)--(60,30)--(30,30);
    \draw[-{Stealth[inset=0pt, length=12, angle'=60]},line width=7] (15,102)--(15,89);
    \draw[-{Stealth[inset=0pt, length=12, angle'=60]},line width=7] (179,75)--(192,75);

    % Felder
    \tikzset{every path/.style={draw=none,fill=labyrinthField},every circle/.style={radius=11}}
    \foreach \x in {0,...,5}
    \foreach \y in {0,...,2}
        {\fill (15 + \x * 30,15 + \y * 30) circle;}

    % Pfade
    \tikzset{every path/.style={-{Stealth},draw=labyrinthPath,fill=none,rounded corners=10,line width=5}}
    \ifthenelse{\equal{\labyrinthVariant}{Direkt}}
    {
        \draw (15,86)--(15,75)--(176,75); % Direkt
    }{}
    \ifthenelse{\equal{\labyrinthVariant}{Indirekt}}
    {
        \draw (15,86)--(15,15)--(75,15)--(75,45)--(135,45)--(135,75)--(176,75); % Indirekt
    }{}
    \ifthenelse{\equal{\labyrinthVariant}{Uhrzeigersinn}}
    {
        \draw (15,86)--(15,75)--(135,75)--(135,45)--(75,45)--(75,15)--(15,15)--(15,75)--(176,75); % Uhrzeigersinn
        \draw[line width=2,draw=black] (60,82.75)--(90,82.75);
    }{}
    \ifthenelse{\equal{\labyrinthVariant}{GegenUhrzeigersinn}}
    {
        \draw (15,86)--(15,15)--(75,15)--(75,45)--(135,45)--(135,75)--(15,75)--(15,15)--(75,15)--(75,45)--(135,45)--(135,75)--(176,75); % GegenUhrzeigersinn
        \draw[line width=2,draw=black] (90,82.75)--(60,82.75);
    }{}

    \ifthenelse{\equal{\labyrinthVariant}{DecisionPoints}}
    {
    \fill[draw=none,fill=labyrinthDecision] (15,75) circle;
    \fill[draw=none,fill=labyrinthDecision] (135,75) circle;
    \tikzset{every path/.style={-{Stealth},draw=labyrinthDecisionText,line width=2},every node/.style={font=\Huge,text=labyrinthDecisionText,text centered}}
    \node[align=center] at (19,79) {$A_r$};
    \node[align=center] at (11,71) {$A_u$};
    \draw (34,75)--(26,75);
    \draw (15,56)--(15,64);
    \draw[-] (9,81)--(21,69);

    \node[align=center] at (131,79) {$B_l$};
    \node[align=center] at (139,71) {$B_u$};
    \draw (116,75)--(124,75);
    \draw (135,56)--(135,64);
    \draw[-] (129,69)--(141,81);
    }{}
\end{includetikzpicture}
\let\labyrinthVariant\relax
\let\labyrinthSize\relax
      \caption{Der direkte Weg ist repräsentiert durch das Wort \alert{\MoveDown\Forward\Forward\Forward\Forward\Forward\Forward}}
    \end{figure}
  \end{alertblock}
\end{frame}
}

{\setbeamercolor{palette primary}{bg=ExColor}
\begin{frame}<handout:0>{Lösung}
  \only<1>{
    \begin{figure}
      \centering
      \def\labyrinthVariant{Indirekt}
      \def\labyrinthSize{0.9\textwidth}
      % Copyright 2018-2024 FIUS
%
% This file is part of theo-vorkurs-folien.
%
% theo-vorkurs-folien is free software: you can redistribute it and/or modify
% it under the terms of the GNU General Public License as published by
% the Free Software Foundation, either version 3 of the License, or
% (at your option) any later version.
%
% theo-vorkurs-folien is distributed in the hope that it will be useful,
% but WITHOUT ANY WARRANTY; without even the implied warranty of
% MERCHANTABILITY or FITNESS FOR A PARTICULAR PURPOSE.  See the
% GNU General Public License for more details.
%
% You should have received a copy of the GNU General Public License
% along with theo-vorkurs-folien.  If not, see <https://www.gnu.org/licenses/>.

\definecolor{labyrinthLine}{RGB}{0,51,102}
\definecolor{labyrinthHead}{RGB}{0,25,50}
\definecolor{labyrinthField}{RGB}{255,230,204}
\definecolor{labyrinthPath}{RGB}{130,179,102}
\definecolor{labyrinthDecision}{RGB}{213,232,212}
\definecolor{labyrinthDecisionText}{RGB}{18,117,181}
\providecommand{\labyrinthSize}{\textwidth}
\providecommand{\labyrinthVariant}{None}
\begin{includetikzpicture}{\labyrinthSize}[x=1mm,y=1mm]
    \tikzset{every path/.style={line width=0.4}}
    % Strichmaennchen
    \draw (15,116.5)--(15,108.5);
    \draw (11,115.5)--(19,115.5);
    \draw[draw=black,fill=labyrinthHead] (15,119) circle[radius=2.5];
    \draw (15,108.5)--(10,103);
    \draw (15,108.5)--(20,103);
    \node[align=center] at (5,112) {\Large Mei};

    % Labyrinth
    \tikzset{every path/.style={line width=5,draw=labyrinthLine}}
    \draw (30,90)--(180,90);
    \draw (0,90)--(0,0)--(180,0)--(180,60)--(150,60);
    \draw (120,0)--(120,30);
    \draw (90,30)--(150,30);
    \draw (30,60)--(120,60);
    \draw (60,60)--(60,30)--(30,30);
    \draw[-{Stealth[inset=0pt, length=12, angle'=60]},line width=7] (15,102)--(15,89);
    \draw[-{Stealth[inset=0pt, length=12, angle'=60]},line width=7] (179,75)--(192,75);

    % Felder
    \tikzset{every path/.style={draw=none,fill=labyrinthField},every circle/.style={radius=11}}
    \foreach \x in {0,...,5}
    \foreach \y in {0,...,2}
        {\fill (15 + \x * 30,15 + \y * 30) circle;}

    % Pfade
    \tikzset{every path/.style={-{Stealth},draw=labyrinthPath,fill=none,rounded corners=10,line width=5}}
    \ifthenelse{\equal{\labyrinthVariant}{Direkt}}
    {
        \draw (15,86)--(15,75)--(176,75); % Direkt
    }{}
    \ifthenelse{\equal{\labyrinthVariant}{Indirekt}}
    {
        \draw (15,86)--(15,15)--(75,15)--(75,45)--(135,45)--(135,75)--(176,75); % Indirekt
    }{}
    \ifthenelse{\equal{\labyrinthVariant}{Uhrzeigersinn}}
    {
        \draw (15,86)--(15,75)--(135,75)--(135,45)--(75,45)--(75,15)--(15,15)--(15,75)--(176,75); % Uhrzeigersinn
        \draw[line width=2,draw=black] (60,82.75)--(90,82.75);
    }{}
    \ifthenelse{\equal{\labyrinthVariant}{GegenUhrzeigersinn}}
    {
        \draw (15,86)--(15,15)--(75,15)--(75,45)--(135,45)--(135,75)--(15,75)--(15,15)--(75,15)--(75,45)--(135,45)--(135,75)--(176,75); % GegenUhrzeigersinn
        \draw[line width=2,draw=black] (90,82.75)--(60,82.75);
    }{}

    \ifthenelse{\equal{\labyrinthVariant}{DecisionPoints}}
    {
    \fill[draw=none,fill=labyrinthDecision] (15,75) circle;
    \fill[draw=none,fill=labyrinthDecision] (135,75) circle;
    \tikzset{every path/.style={-{Stealth},draw=labyrinthDecisionText,line width=2},every node/.style={font=\Huge,text=labyrinthDecisionText,text centered}}
    \node[align=center] at (19,79) {$A_r$};
    \node[align=center] at (11,71) {$A_u$};
    \draw (34,75)--(26,75);
    \draw (15,56)--(15,64);
    \draw[-] (9,81)--(21,69);

    \node[align=center] at (131,79) {$B_l$};
    \node[align=center] at (139,71) {$B_u$};
    \draw (116,75)--(124,75);
    \draw (135,56)--(135,64);
    \draw[-] (129,69)--(141,81);
    }{}
\end{includetikzpicture}
\let\labyrinthVariant\relax
\let\labyrinthSize\relax
      \caption{Indirekter Weg}

    \end{figure}
  }
  \only<2>{
    \begin{figure}
      \centering
      \def\labyrinthVariant{Uhrzeigersinn}
      \def\labyrinthSize{0.9\textwidth}
      % Copyright 2018-2024 FIUS
%
% This file is part of theo-vorkurs-folien.
%
% theo-vorkurs-folien is free software: you can redistribute it and/or modify
% it under the terms of the GNU General Public License as published by
% the Free Software Foundation, either version 3 of the License, or
% (at your option) any later version.
%
% theo-vorkurs-folien is distributed in the hope that it will be useful,
% but WITHOUT ANY WARRANTY; without even the implied warranty of
% MERCHANTABILITY or FITNESS FOR A PARTICULAR PURPOSE.  See the
% GNU General Public License for more details.
%
% You should have received a copy of the GNU General Public License
% along with theo-vorkurs-folien.  If not, see <https://www.gnu.org/licenses/>.

\definecolor{labyrinthLine}{RGB}{0,51,102}
\definecolor{labyrinthHead}{RGB}{0,25,50}
\definecolor{labyrinthField}{RGB}{255,230,204}
\definecolor{labyrinthPath}{RGB}{130,179,102}
\definecolor{labyrinthDecision}{RGB}{213,232,212}
\definecolor{labyrinthDecisionText}{RGB}{18,117,181}
\providecommand{\labyrinthSize}{\textwidth}
\providecommand{\labyrinthVariant}{None}
\begin{includetikzpicture}{\labyrinthSize}[x=1mm,y=1mm]
    \tikzset{every path/.style={line width=0.4}}
    % Strichmaennchen
    \draw (15,116.5)--(15,108.5);
    \draw (11,115.5)--(19,115.5);
    \draw[draw=black,fill=labyrinthHead] (15,119) circle[radius=2.5];
    \draw (15,108.5)--(10,103);
    \draw (15,108.5)--(20,103);
    \node[align=center] at (5,112) {\Large Mei};

    % Labyrinth
    \tikzset{every path/.style={line width=5,draw=labyrinthLine}}
    \draw (30,90)--(180,90);
    \draw (0,90)--(0,0)--(180,0)--(180,60)--(150,60);
    \draw (120,0)--(120,30);
    \draw (90,30)--(150,30);
    \draw (30,60)--(120,60);
    \draw (60,60)--(60,30)--(30,30);
    \draw[-{Stealth[inset=0pt, length=12, angle'=60]},line width=7] (15,102)--(15,89);
    \draw[-{Stealth[inset=0pt, length=12, angle'=60]},line width=7] (179,75)--(192,75);

    % Felder
    \tikzset{every path/.style={draw=none,fill=labyrinthField},every circle/.style={radius=11}}
    \foreach \x in {0,...,5}
    \foreach \y in {0,...,2}
        {\fill (15 + \x * 30,15 + \y * 30) circle;}

    % Pfade
    \tikzset{every path/.style={-{Stealth},draw=labyrinthPath,fill=none,rounded corners=10,line width=5}}
    \ifthenelse{\equal{\labyrinthVariant}{Direkt}}
    {
        \draw (15,86)--(15,75)--(176,75); % Direkt
    }{}
    \ifthenelse{\equal{\labyrinthVariant}{Indirekt}}
    {
        \draw (15,86)--(15,15)--(75,15)--(75,45)--(135,45)--(135,75)--(176,75); % Indirekt
    }{}
    \ifthenelse{\equal{\labyrinthVariant}{Uhrzeigersinn}}
    {
        \draw (15,86)--(15,75)--(135,75)--(135,45)--(75,45)--(75,15)--(15,15)--(15,75)--(176,75); % Uhrzeigersinn
        \draw[line width=2,draw=black] (60,82.75)--(90,82.75);
    }{}
    \ifthenelse{\equal{\labyrinthVariant}{GegenUhrzeigersinn}}
    {
        \draw (15,86)--(15,15)--(75,15)--(75,45)--(135,45)--(135,75)--(15,75)--(15,15)--(75,15)--(75,45)--(135,45)--(135,75)--(176,75); % GegenUhrzeigersinn
        \draw[line width=2,draw=black] (90,82.75)--(60,82.75);
    }{}

    \ifthenelse{\equal{\labyrinthVariant}{DecisionPoints}}
    {
    \fill[draw=none,fill=labyrinthDecision] (15,75) circle;
    \fill[draw=none,fill=labyrinthDecision] (135,75) circle;
    \tikzset{every path/.style={-{Stealth},draw=labyrinthDecisionText,line width=2},every node/.style={font=\Huge,text=labyrinthDecisionText,text centered}}
    \node[align=center] at (19,79) {$A_r$};
    \node[align=center] at (11,71) {$A_u$};
    \draw (34,75)--(26,75);
    \draw (15,56)--(15,64);
    \draw[-] (9,81)--(21,69);

    \node[align=center] at (131,79) {$B_l$};
    \node[align=center] at (139,71) {$B_u$};
    \draw (116,75)--(124,75);
    \draw (135,56)--(135,64);
    \draw[-] (129,69)--(141,81);
    }{}
\end{includetikzpicture}
\let\labyrinthVariant\relax
\let\labyrinthSize\relax
      \caption{Schlaufe Uhrzeigersinn}

    \end{figure}\textbf{}
  }
  \only<3>{
    \begin{figure}
      \centering
      \def\labyrinthVariant{GegenUhrzeigersinn}
      \def\labyrinthSize{0.9\textwidth}
      % Copyright 2018-2024 FIUS
%
% This file is part of theo-vorkurs-folien.
%
% theo-vorkurs-folien is free software: you can redistribute it and/or modify
% it under the terms of the GNU General Public License as published by
% the Free Software Foundation, either version 3 of the License, or
% (at your option) any later version.
%
% theo-vorkurs-folien is distributed in the hope that it will be useful,
% but WITHOUT ANY WARRANTY; without even the implied warranty of
% MERCHANTABILITY or FITNESS FOR A PARTICULAR PURPOSE.  See the
% GNU General Public License for more details.
%
% You should have received a copy of the GNU General Public License
% along with theo-vorkurs-folien.  If not, see <https://www.gnu.org/licenses/>.

\definecolor{labyrinthLine}{RGB}{0,51,102}
\definecolor{labyrinthHead}{RGB}{0,25,50}
\definecolor{labyrinthField}{RGB}{255,230,204}
\definecolor{labyrinthPath}{RGB}{130,179,102}
\definecolor{labyrinthDecision}{RGB}{213,232,212}
\definecolor{labyrinthDecisionText}{RGB}{18,117,181}
\providecommand{\labyrinthSize}{\textwidth}
\providecommand{\labyrinthVariant}{None}
\begin{includetikzpicture}{\labyrinthSize}[x=1mm,y=1mm]
    \tikzset{every path/.style={line width=0.4}}
    % Strichmaennchen
    \draw (15,116.5)--(15,108.5);
    \draw (11,115.5)--(19,115.5);
    \draw[draw=black,fill=labyrinthHead] (15,119) circle[radius=2.5];
    \draw (15,108.5)--(10,103);
    \draw (15,108.5)--(20,103);
    \node[align=center] at (5,112) {\Large Mei};

    % Labyrinth
    \tikzset{every path/.style={line width=5,draw=labyrinthLine}}
    \draw (30,90)--(180,90);
    \draw (0,90)--(0,0)--(180,0)--(180,60)--(150,60);
    \draw (120,0)--(120,30);
    \draw (90,30)--(150,30);
    \draw (30,60)--(120,60);
    \draw (60,60)--(60,30)--(30,30);
    \draw[-{Stealth[inset=0pt, length=12, angle'=60]},line width=7] (15,102)--(15,89);
    \draw[-{Stealth[inset=0pt, length=12, angle'=60]},line width=7] (179,75)--(192,75);

    % Felder
    \tikzset{every path/.style={draw=none,fill=labyrinthField},every circle/.style={radius=11}}
    \foreach \x in {0,...,5}
    \foreach \y in {0,...,2}
        {\fill (15 + \x * 30,15 + \y * 30) circle;}

    % Pfade
    \tikzset{every path/.style={-{Stealth},draw=labyrinthPath,fill=none,rounded corners=10,line width=5}}
    \ifthenelse{\equal{\labyrinthVariant}{Direkt}}
    {
        \draw (15,86)--(15,75)--(176,75); % Direkt
    }{}
    \ifthenelse{\equal{\labyrinthVariant}{Indirekt}}
    {
        \draw (15,86)--(15,15)--(75,15)--(75,45)--(135,45)--(135,75)--(176,75); % Indirekt
    }{}
    \ifthenelse{\equal{\labyrinthVariant}{Uhrzeigersinn}}
    {
        \draw (15,86)--(15,75)--(135,75)--(135,45)--(75,45)--(75,15)--(15,15)--(15,75)--(176,75); % Uhrzeigersinn
        \draw[line width=2,draw=black] (60,82.75)--(90,82.75);
    }{}
    \ifthenelse{\equal{\labyrinthVariant}{GegenUhrzeigersinn}}
    {
        \draw (15,86)--(15,15)--(75,15)--(75,45)--(135,45)--(135,75)--(15,75)--(15,15)--(75,15)--(75,45)--(135,45)--(135,75)--(176,75); % GegenUhrzeigersinn
        \draw[line width=2,draw=black] (90,82.75)--(60,82.75);
    }{}

    \ifthenelse{\equal{\labyrinthVariant}{DecisionPoints}}
    {
    \fill[draw=none,fill=labyrinthDecision] (15,75) circle;
    \fill[draw=none,fill=labyrinthDecision] (135,75) circle;
    \tikzset{every path/.style={-{Stealth},draw=labyrinthDecisionText,line width=2},every node/.style={font=\Huge,text=labyrinthDecisionText,text centered}}
    \node[align=center] at (19,79) {$A_r$};
    \node[align=center] at (11,71) {$A_u$};
    \draw (34,75)--(26,75);
    \draw (15,56)--(15,64);
    \draw[-] (9,81)--(21,69);

    \node[align=center] at (131,79) {$B_l$};
    \node[align=center] at (139,71) {$B_u$};
    \draw (116,75)--(124,75);
    \draw (135,56)--(135,64);
    \draw[-] (129,69)--(141,81);
    }{}
\end{includetikzpicture}
\let\labyrinthVariant\relax
\let\labyrinthSize\relax
      \caption{Schlaufe gegen Uhrzeigersinn}

    \end{figure}
  }
\end{frame}
}

{\setbeamercolor{palette primary}{bg=ExColor}
\begin{frame}<handout:0>{Lösung}
  \begin{columns}
    \column{0.45\textwidth}
    \begin{alertblock}{Eine Möglichkeit:}
      %Wir nehmen uns zwei Variablen um zwischen den Einstiegsrichtungen zu unterscheiden für jeden Entscheidungspunkt und konstruieren damit  unsere Grammatik:\\
      $G = (V, \Sigma, P, S)$, wobei \\
      $V = \{S, A_u, A_r, B_u, B_l\}$ \\
      $\Sigma = \{\text{\Rewind}, \text{\MoveUp}, \text{\Forward}, \text{\MoveDown}\}$ \\
      $P = \{S \rightarrow \text\MoveDown A_u \ |\ \text\MoveDown A_r,$\\
      \qquad\; $A_u \rightarrow \text{\Forward\Forward\Forward\Forward} B_l$\\
      \qquad\; $A_r \rightarrow \text{\MoveDown\MoveDown\Forward\Forward\MoveUp\Forward\Forward\MoveUp} B_u,$\\
      \qquad\; $B_l \rightarrow \text{\MoveDown\Rewind\Rewind\MoveDown\Rewind\Rewind\MoveUp\MoveUp} A_u \ |\ \text{\Forward\Forward},$\\
      \qquad\; $B_u \rightarrow \text{\Rewind\Rewind\Rewind\Rewind} A_r \ |\ \text{\Forward\Forward}\}$
    \end{alertblock}
    \column{0.55\textwidth}
    \begin{figure}
      \centering
      \def\labyrinthVariant{DecisionPoints}
      \def\labyrinthSize{0.9\textwidth}
      % Copyright 2018-2024 FIUS
%
% This file is part of theo-vorkurs-folien.
%
% theo-vorkurs-folien is free software: you can redistribute it and/or modify
% it under the terms of the GNU General Public License as published by
% the Free Software Foundation, either version 3 of the License, or
% (at your option) any later version.
%
% theo-vorkurs-folien is distributed in the hope that it will be useful,
% but WITHOUT ANY WARRANTY; without even the implied warranty of
% MERCHANTABILITY or FITNESS FOR A PARTICULAR PURPOSE.  See the
% GNU General Public License for more details.
%
% You should have received a copy of the GNU General Public License
% along with theo-vorkurs-folien.  If not, see <https://www.gnu.org/licenses/>.

\definecolor{labyrinthLine}{RGB}{0,51,102}
\definecolor{labyrinthHead}{RGB}{0,25,50}
\definecolor{labyrinthField}{RGB}{255,230,204}
\definecolor{labyrinthPath}{RGB}{130,179,102}
\definecolor{labyrinthDecision}{RGB}{213,232,212}
\definecolor{labyrinthDecisionText}{RGB}{18,117,181}
\providecommand{\labyrinthSize}{\textwidth}
\providecommand{\labyrinthVariant}{None}
\begin{includetikzpicture}{\labyrinthSize}[x=1mm,y=1mm]
    \tikzset{every path/.style={line width=0.4}}
    % Strichmaennchen
    \draw (15,116.5)--(15,108.5);
    \draw (11,115.5)--(19,115.5);
    \draw[draw=black,fill=labyrinthHead] (15,119) circle[radius=2.5];
    \draw (15,108.5)--(10,103);
    \draw (15,108.5)--(20,103);
    \node[align=center] at (5,112) {\Large Mei};

    % Labyrinth
    \tikzset{every path/.style={line width=5,draw=labyrinthLine}}
    \draw (30,90)--(180,90);
    \draw (0,90)--(0,0)--(180,0)--(180,60)--(150,60);
    \draw (120,0)--(120,30);
    \draw (90,30)--(150,30);
    \draw (30,60)--(120,60);
    \draw (60,60)--(60,30)--(30,30);
    \draw[-{Stealth[inset=0pt, length=12, angle'=60]},line width=7] (15,102)--(15,89);
    \draw[-{Stealth[inset=0pt, length=12, angle'=60]},line width=7] (179,75)--(192,75);

    % Felder
    \tikzset{every path/.style={draw=none,fill=labyrinthField},every circle/.style={radius=11}}
    \foreach \x in {0,...,5}
    \foreach \y in {0,...,2}
        {\fill (15 + \x * 30,15 + \y * 30) circle;}

    % Pfade
    \tikzset{every path/.style={-{Stealth},draw=labyrinthPath,fill=none,rounded corners=10,line width=5}}
    \ifthenelse{\equal{\labyrinthVariant}{Direkt}}
    {
        \draw (15,86)--(15,75)--(176,75); % Direkt
    }{}
    \ifthenelse{\equal{\labyrinthVariant}{Indirekt}}
    {
        \draw (15,86)--(15,15)--(75,15)--(75,45)--(135,45)--(135,75)--(176,75); % Indirekt
    }{}
    \ifthenelse{\equal{\labyrinthVariant}{Uhrzeigersinn}}
    {
        \draw (15,86)--(15,75)--(135,75)--(135,45)--(75,45)--(75,15)--(15,15)--(15,75)--(176,75); % Uhrzeigersinn
        \draw[line width=2,draw=black] (60,82.75)--(90,82.75);
    }{}
    \ifthenelse{\equal{\labyrinthVariant}{GegenUhrzeigersinn}}
    {
        \draw (15,86)--(15,15)--(75,15)--(75,45)--(135,45)--(135,75)--(15,75)--(15,15)--(75,15)--(75,45)--(135,45)--(135,75)--(176,75); % GegenUhrzeigersinn
        \draw[line width=2,draw=black] (90,82.75)--(60,82.75);
    }{}

    \ifthenelse{\equal{\labyrinthVariant}{DecisionPoints}}
    {
    \fill[draw=none,fill=labyrinthDecision] (15,75) circle;
    \fill[draw=none,fill=labyrinthDecision] (135,75) circle;
    \tikzset{every path/.style={-{Stealth},draw=labyrinthDecisionText,line width=2},every node/.style={font=\Huge,text=labyrinthDecisionText,text centered}}
    \node[align=center] at (19,79) {$A_r$};
    \node[align=center] at (11,71) {$A_u$};
    \draw (34,75)--(26,75);
    \draw (15,56)--(15,64);
    \draw[-] (9,81)--(21,69);

    \node[align=center] at (131,79) {$B_l$};
    \node[align=center] at (139,71) {$B_u$};
    \draw (116,75)--(124,75);
    \draw (135,56)--(135,64);
    \draw[-] (129,69)--(141,81);
    }{}
\end{includetikzpicture}
\let\labyrinthVariant\relax
\let\labyrinthSize\relax
      \caption{Es muss unterschieden werden, ob Bob von links, rechts oder unten kam}

    \end{figure}
  \end{columns}
  \small\emph{Erinnerung:} Bob kann nicht auf ein Feld zurücktreten, von dem er gerade kam
\end{frame}
}

\subsubsection{Ableiten}
\begin{frame}[fragile]{Ableiten}
  Wir können durch das Ableiten formal zeigen, dass ein Wort von einer Grammatik erzeugt wird:\\
  \small{Wir betrachten L = \{$ww^R \mid w^R\text{ ist w rückwärts, }w \in \{a, b\}^n, n\in \mathbb{N}\setminus \{0\}$\}\\
    mit der Grammatik $G=(V,\Sigma,P,S)$, wobei\\
    $V=\{S\}$, $\Sigma=\{a,b\}$, $P = \{S \rightarrow aSa \ |\ bSb \ |\ aa \ |\ bb$\}}
  \metroset{block=fill}
  \begin{exampleblock}{Beispiel}
    Wir zeigen $ww^R = ababbbbaba \in$ L.\\
    \small{$S\Rightarrow_G aSa \Rightarrow_G abSba \Rightarrow_G  abaSaba \Rightarrow_G ababSbaba$ \\ $\Rightarrow_G ababbbbaba$}\\\qed
  \end{exampleblock}
\end{frame}

{\setbeamercolor{palette primary}{bg=ExColor}
\begin{frame}{Denkpause}
  \begin{alertblock}{Aufgaben}
    Zeige die folgenden Aussagen
  \end{alertblock}
  \metroset{block=fill}
  \begin{block}{Normal}
    \begin{itemize}
      \item $G_1=(\{S\}, \{a\}, P_1, S)$ erzeugt $aaaa$\\
            mit $P_1=\{S\rightarrow aaS\ |\ \emptyWord\}$
      \item $G_2=(\{S,A,B\}, \{a,b,c\}, P_2, S)$ erzeugt $aabbc$\\
            mit $P_2=\{S\rightarrow AB$, $A\rightarrow aAb \ |\ ab\ |\ \emptyWord$, $B\rightarrow cB \ |\  \emptyWord\}$
      \item $G_3=(\{S,U,V\}, \{a,b,c,d\}, P_3, S)$ erzeugt $abac$\\
            mit $P_3=\{S\rightarrow UV$, $U\rightarrow aU \ |\  bU \ |\  \emptyWord$, $V\rightarrow c \ |\  d\}$
      \item $G_4=(\{S,X\}, \{a,b,c\}, P_4, S)$ erzeugt $aac$\\
            mit $P_4=\{S\rightarrow XXX$, $X\rightarrow a \ |\  b \ |\  c\}$
    \end{itemize}
  \end{block}
\end{frame}
\begin{frame}{Denkpause}
  \begin{alertblock}{Aufgaben}
    Zeige die folgenden Aussagen
  \end{alertblock}
  \metroset{block=fill}
  \begin{block}{Etwas Schwerer}
    \begin{itemize}
      \item $G_5=(\{S\}, \{a\}, P_5, S)$ erzeugt $aaaa$\\
            mit $P_5=\{S\rightarrow a \ |\  aaaS\}$
      \item $G_6=(\{S,A,B\}, \{a,b,c\}, P_6, S)$ erzeugt $cabcacca$\\
            mit $P_6=\{S\rightarrow AAAB$, $AB\rightarrow BA,
              A\rightarrow cA \ |\  Ac \ |\ a,
              B\rightarrow cB \ |\  Bc \ |\  b\}$
      \item $G_7=(\{S,U\}, \{\text{\Stopsign},\text{\Rewind},\text{\MoveUp},\text{\Forward},\text{\MoveDown}\}, P_7, S)$  erzeugt \Forward\Stopsign\\
            mit $P_7=\{S\rightarrow U\text{\Stopsign} \ |\  \text{\Stopsign}$, $U\rightarrow \text{\Rewind} U \ |\  \text{\MoveUp} U \ |\  \text{\Forward} U \ |\  \text{\MoveDown} U \ |\ \emptyWord\}$
    \end{itemize}
  \end{block}
\end{frame}
}

{\setbeamercolor{palette primary}{bg=ExColor}
\begin{frame}<handout:0>{Lösungen}
  Alle Lösungen sind Beispiellösungen, es sind auch andere möglich.
  \begin{itemize}[<+- | alert@+>]
    \item $S\Rightarrow_{G_1} aaS \Rightarrow_{G_1} aaaaS \Rightarrow_{G_1} aaaa$
    \item $S\Rightarrow_{G_2} AB \Rightarrow_{G_2} aAbB \Rightarrow_{G_2} aabbB \Rightarrow_{G_2} aabbcB \Rightarrow_{G_2} aabbc$
    \item $S\Rightarrow_{G_3} UV \Rightarrow_{G_3} aUV \Rightarrow_{G_3} abUV \Rightarrow_{G_3} abaUV \Rightarrow_{G_3} abaV \Rightarrow_{G_3} abac$
    \item $S\Rightarrow_{G_4} XXX \Rightarrow_{G_4} aXX \Rightarrow_{G_4} aaX \Rightarrow_{G_4} aac$
    \item $S\Rightarrow_{G_5} aaaS \Rightarrow_{G_5} aaaa$
    \item $S\Rightarrow_{G_6} AAAB \Rightarrow_{G_6} AABA \Rightarrow_{G_6} ABAA \Rightarrow_{G_6} cABAA \Rightarrow_{G_6} caBAA \Rightarrow_{G_6} cabAA \Rightarrow_{G_6} cabcAA \Rightarrow_{G_6} cabcaA\Rightarrow_{G_6} cabcacA \Rightarrow_{G_6} cabcaccA \Rightarrow_{G_6} cabcacca$
    \item $S\Rightarrow_{G_7} U\text{\Stopsign} \Rightarrow_{G_7} \text{\Forward}U\text{\Stopsign} \Rightarrow_{G_7} \text{\Forward}\text{\Stopsign}$
  \end{itemize}
\end{frame}
}


\section{Wiederholung}
\begin{frame}[fragile]{Das können wir jetzt beantworten}
	\begin{alertblock}{Vollständige Induktion}
		\begin{itemize}
			\item Was ist die Idee der Induktion?
			\item Welche Schritte hat die Induktion? %IA,IV,IS
			\item Für welche Aussagen ist die Induktion geeignet?
		\end{itemize}
	\end{alertblock}
\end{frame}

\begin{frame}[fragile]{Das können wir jetzt beantworten}
	\begin{alertblock}{Grammatiken}
		\begin{itemize}
        	\item Was sind Grammatiken?
			\item Was ist der Zusammenhang zwischen Grammatiken und Sprachen?
			\item Was sind Nichtterminale?
			\item Was sind Terminale?
			\item Bilden einer Grammatik für gegebene Sprache
        	\item Wie finde ich raus, ob ein Wort von einer Grammatik erzeugt wird?
		\end{itemize}
	\end{alertblock}
\end{frame}

\Center{Noch Fragen?}

% Copyright 2018-2022 FIUS
%
% This file is part of theo-vorkurs-folien.
%
% theo-vorkurs-folien is free software: you can redistribute it and/or modify
% it under the terms of the GNU General Public License as published by
% the Free Software Foundation, either version 3 of the License, or
% (at your option) any later version.
%
% theo-vorkurs-folien is distributed in the hope that it will be useful,
% but WITHOUT ANY WARRANTY; without even the implied warranty of
% MERCHANTABILITY or FITNESS FOR A PARTICULAR PURPOSE.  See the
% GNU General Public License for more details.
%
% You should have received a copy of the GNU General Public License
% along with theo-vorkurs-folien.  If not, see <https://www.gnu.org/licenses/>.

\begin{frame}[fragile]{Glossar}
	\small
	\begin{tabular}{p{0.12\textwidth} p{0.23\textwidth} p{0.5\textwidth}}
		\toprule
		Abk.&Bedeutung&Was?!\\
		\midrule
		$A \subseteq B$ & Teilmenge & Alle Elemente aus $A$ sind auch in $B$ enthalten. Dabei können die Mengen auch gleich sein.\\
		$A \subsetneq B$ & echte Teilmenge & Alle Elemente aus $A$ sind auch in $B$ enthalten. Jedoch enthält $B$ noch Elemente, die nicht in $A$ enthalten sind.
		\\&&$\implies$ Mengen sind nicht gleich!\\
		$A \subset B$ & Teilmenge \emph{oder} echte Teilmenge & Bei manchen Leuten $\subseteq$, bei manchen $\subsetneq$. Mehrdeutig, lieber nicht verwenden!\\
		\bottomrule
	\end{tabular}
\end{frame}

\appendix
% Copyright 2018, 2019, 2020, 2021 FIUS
%
% This file is part of theo-vorkurs-folien.
%
% theo-vorkurs-folien is free software: you can redistribute it and/or modify
% it under the terms of the GNU General Public License as published by
% the Free Software Foundation, either version 3 of the License, or
% (at your option) any later version.
%
% theo-vorkurs-folien is distributed in the hope that it will be useful,
% but WITHOUT ANY WARRANTY; without even the implied warranty of
% MERCHANTABILITY or FITNESS FOR A PARTICULAR PURPOSE.  See the
% GNU General Public License for more details.
%
% You should have received a copy of the GNU General Public License
% along with theo-vorkurs-folien.  If not, see <https://www.gnu.org/licenses/>.

\subsection{Lizenz}
\begin{frame}[fragile]{Lizenz}
    \begin{itemize}
    \item Unsere Folien sind frei!\\
    \item Jeder darf die Folien unter den Bedingungen der \textbf{GNU General Public License v3} (oder jeder späteren Version) weiterverwenden.\\
    \item Ihr findet den Quelltext unter \url{https://www.github.com/FIUS/theo-vorkurs-folien}
    \end{itemize}
\end{frame}
\begin{frame}<handout:0>[fragile]{Online-Whiteboard}
	\phantom{text}
\end{frame}

\end{document}
