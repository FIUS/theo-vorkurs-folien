% Copyright 2018-2022 FIUS
%
% This file is part of theo-vorkurs-folien.
%
% theo-vorkurs-folien is free software: you can redistribute it and/or modify
% it under the terms of the GNU General Public License as published by
% the Free Software Foundation, either version 3 of the License, or
% (at your option) any later version.
%
% theo-vorkurs-folien is distributed in the hope that it will be useful,
% but WITHOUT ANY WARRANTY; without even the implied warranty of
% MERCHANTABILITY or FITNESS FOR A PARTICULAR PURPOSE.  See the
% GNU General Public License for more details.
%
% You should have received a copy of the GNU General Public License
% along with theo-vorkurs-folien.  If not, see <https://www.gnu.org/licenses/>.

% !TeX program = pdflatex
% !TeX spellcheck = de
% Copyright 2018-2024 FIUS
%
% This file is part of theo-vorkurs-folien.
%
% theo-vorkurs-folien is free software: you can redistribute it and/or modify
% it under the terms of the GNU General Public License as published by
% the Free Software Foundation, either version 3 of the License, or
% (at your option) any later version.
%
% theo-vorkurs-folien is distributed in the hope that it will be useful,
% but WITHOUT ANY WARRANTY; without even the implied warranty of
% MERCHANTABILITY or FITNESS FOR A PARTICULAR PURPOSE.  See the
% GNU General Public License for more details.
%
% You should have received a copy of the GNU General Public License
% along with theo-vorkurs-folien.  If not, see <https://www.gnu.org/licenses/>.

\documentclass[aspectratio=43,10pt]{beamer}

\usetheme[progressbar=frametitle,subsectionpage=progressbar]{metropolis}
\usepackage{appendixnumberbeamer}
\usepackage[ngerman]{babel}
\usepackage[utf8]{inputenc}
%\usepackage{t1enc}
\usepackage{iftex}

\ifLuaTeX
    \usepackage{fontspec}
\else
    \ifxetex
        \usepackage{fontspec}
    \else
        \usepackage[T1]{fontenc}
    \fi
\fi

\usepackage[sfdefault,scaled=.85,lf]{FiraSans}
\usepackage{newtxsf}

\usepackage{booktabs}
\usepackage[scale=2]{ccicons}
\usepackage{hyperref}

\usepackage{pgf}
\makeatletter
\@ifclasswith{beamer}{notes}{
    \usepackage{pgfpages}
    \setbeameroption{show notes on second screen}
}{}
\makeatother
\usepackage{tikz}
\usetikzlibrary{arrows,automata,positioning,shapes,arrows.meta,shapes.geometric, through, calc}
\usepackage{pgfplots}
\usepgfplotslibrary{dateplot}

\usepackage{xspace}
\newcommand{\themename}{\textbf{\textsc{metropolis}}\xspace}

\usepackage{blindtext}
\usepackage{graphicx}
\usepackage{subcaption}
\usepackage{comment}
\usepackage{mathtools}
\usepackage{amsmath}
\usepackage{centernot}
\usepackage{amssymb}
\usepackage{proof}
\usepackage{tabularx}
\renewcommand{\figurename}{Abb.}
\usepackage{tikzsymbols}
\let\Coffeecup\relax
\let\Heart\relax
\let\Smiley\relax
\usepackage{marvosym}
\usepackage{mathtools}
\usepackage{qrcode}
\usepackage{advdate}
\usepackage{ifthen}
\usepackage{tikz-among-us}
\usepackage{multicol}
\usepackage{outlines}
\usepackage{ulem}

% \IfFontExistsTF{Segoe UI}{%
%     \usefonttheme{professionalfonts}
%     \defaultfontfeatures{Scale = MatchLowercase}
%     \setsansfont{Consolas}[
%         Scale = 1.0,
%         BoldFont = Consolas ,
%         BoldItalicFont = Consolas ]
%     \ifLuaTeX
% }{
% }
% \else
% \ifxelatex
%     \IfFontExistsTF{Segoe UI}{%
%         \usefonttheme{professionalfonts}
%         \defaultfontfeatures{Scale = MatchLowercase}
%         \setsansfont{Consolas}[
%             Scale = 1.0,
%             BoldFont = Segoe UI Semibold ,
%             BoldItalicFont = Segoe UI Semibold Italic ]
%     }{
%     }
% \fi

\definecolor{Bluecreen}{HTML}{0873aa}

\makeatletter
\setlength{\metropolis@progressonsectionpage@linewidth}{1.1em}
\setlength{\metropolis@progressinheadfoot@linewidth}{2pt}
\makeatother

\setbeamercolor{progress bar}{%
    fg=Bluecreen,
    bg=Bluecreen!70!black!45
}

\makeatletter
\newlength{\metropolis@progressonsectionpage@blockwidth}%
\newlength{\metropolis@progressonsectionpage@blockborder}%
\setlength{\metropolis@progressonsectionpage@blockborder}{1pt}%
\setbeamertemplate{progress bar in section page}{
    \vspace{0.5\metropolis@progressonsectionpage@linewidth}
    \setlength{\metropolis@progressonsectionpage}{%
        \textwidth * \ratio{\insertframenumber pt}{\inserttotalframenumber pt}%
    }%
    \setlength{\metropolis@progressonsectionpage@blockwidth}{%
        0.05\textwidth - 0.05\metropolis@progressonsectionpage@blockborder
    }%
    \begin{tikzpicture}
        \fill[bg] (0,-\metropolis@progressonsectionpage@blockborder) rectangle (\textwidth, \metropolis@progressonsectionpage@linewidth + \metropolis@progressonsectionpage@blockborder);
        %\fill[fg] (0,0) rectangle (\metropolis@progressonsectionpage, \metropolis@progressonsectionpage@linewidth);

        \foreach \i in {1,...,20} {%
                \pgfmathparse{\insertframenumber*100/\inserttotalframenumber >= \i*100/20 ? 1 : 0}
                \ifthenelse {\pgfmathresult>0}
                {%
                    \pgfmathparse{\i*0.05-0.005}
                    \fill[fg] (\i*\metropolis@progressonsectionpage@blockwidth, 0) rectangle ++ (-\metropolis@progressonsectionpage@blockwidth + \metropolis@progressonsectionpage@blockborder, \metropolis@progressonsectionpage@linewidth);
                }
                {\node at (\i,0) {\i};}% otherwise do nothing
            }

        \node[color=white] at (0.5\textwidth, 0.5\metropolis@progressonsectionpage@linewidth) {\textnormal{%
                \fontsize{0.85\metropolis@progressonsectionpage@linewidth}{\metropolis@progressonsectionpage@linewidth}\selectfont loading
                \pgfmathparse{\insertframenumber*100/\inserttotalframenumber}%
                \pgfmathprintnumber[fixed,precision=2]{\pgfmathresult}\,\% complete...%
            }%
        };

    \end{tikzpicture}%
}
\makeatother

\newcommand\daynr{0}


\definecolor{ExColor}{HTML}{17819b}

\newcommand{\emptyWord}{\varepsilon}
\let \emptyset\varnothing
\newcommand{\SigmaStern}{\Sigma^{*}}
\newcommand{\absval}[1]{|#1|}
\newcommand{\defeq}{\vcentcolon=}
\newcommand{\eqdef}{=\vcentcolon}
\newcommand{\nimplies}{\centernot\implies}

\newcommand{\naturals}{\ensuremath{\mathbb{N}}}
\newcommand{\integers}{\ensuremath{\mathbb{Z}}}
\newcommand{\rationals}{\ensuremath{\mathbb{Q}}}
\newcommand{\reals}{\ensuremath{\mathbb{R}}}
\newcommand{\iffspace}{\ensuremath{\iff\ }}

\newcommand{\sus}[1]{%
    \tikz{\node[scale=0.05] at (0,0) {\amongUsOriginal{#1}{cyan}}}
}

\setbeamertemplate{footline}[text line]
{\parbox{\linewidth}{Fachgruppe Informatik\hfill\insertpagenumber\hfill Vorkurs Theoretische Informatik\vspace{0.2in}}}

\newcommand{\Center}[1]{
    \begin{frame}<handout:0>[standout]
        #1
    \end{frame}
}

\newcommand{\cleft}[2][.]{%
    \begingroup\colorlet{savedleftcolor}{.}%
    \color{#1}\left#2\color{savedleftcolor}%
}
\newcommand{\cright}[2][.]{%
    \color{#1}\right#2\endgroup
}

% Make subsections in toc small
\makeatletter
\setbeamertemplate{subsection in toc}{\small\leftskip=2em\inserttocsubsection\par}
\makeatother

\AtBeginDocument{%

    % Fix section pages in appendix
    \apptocmd{\appendix}{%
        \setbeamertemplate{section page}[simple]%
    }{}{}
}

\addtobeamertemplate{block begin}{}{\vskip 0em}
\addtobeamertemplate{block alerted begin}{}{\vskip 0em}
\addtobeamertemplate{block example begin}{}{\vskip 0em}

\newsavebox\tikzBox
\newenvironment{includetikzpicture}[1]{%
    \def\sizeArgument{#1}\begin{lrbox}{\tikzBox}\begin{tikzpicture}
            }{%
        \end{tikzpicture}\end{lrbox}\resizebox{\sizeArgument}{!}{\usebox\tikzBox}%
}

\pgfkeys{
    /sevenseg/.is family, /sevenseg,
    shrink/.estore in     = \sevensegShrink,    % avoids overlapping of segments
    oncolor/.estore in    = \sevensegOncolor,   % color of an ON segment
    offcolor/.estore in   = \sevensegOffcolor,  % color of an OFF segment
    size/.estore in       = \sevensegSize,      % height
    onSize/.estore in     = \sevensegOnsize,     % line width if ON
    offSize/.estore in    = \sevensegOffsize    % line width if OFF
}

\pgfkeys{
    /sevenseg,
    default/.style = {
        shrink = 0.1, 
        size = 1em, 
        oncolor = red, 
        offcolor = blue!70!black!40,
        onSize = 2,
        offSize =1.5
        }
}

\newcommand{\sevenseg}[2][]% options, values
{%
  \pgfkeys{/sevenseg, default, #1}%
    \begin{tikzpicture}[x=\sevensegSize,y=\sevensegSize,baseline=.25*\sevensegSize]%
        % unten li
        \path (0,0) ++(0,0) coordinate (P1);
        % unten re
        \path (0,0) ++(0.5,0) coordinate (P2);
        % mitte li
        \path (0,0) ++(90:0.5) coordinate (P3);
        % mitte re
        \path (P2)  ++(90:0.5) coordinate (P4);
        % oben li
        \path (P3)  ++(90:0.5) coordinate (P5);
        % oben re
        \path (P4)  ++(90:0.5) coordinate (P6);
        % then step through the 1/0 values in the segment array


    \foreach \val [count=\i from 0] in {#2} {%
      \ifthenelse{\equal{\val}{1}}%
        {\let\mycolor=\sevensegOncolor \let\mysize=\sevensegOnsize}%
        {\let\mycolor=\sevensegOffcolor \let\mysize=\sevensegOffsize}%

      % then draw segment depending on \i
      \ifthenelse{\equal{\i}{0}}{\path[draw=\mycolor, line width=\mysize] (P5) -- (P6);}{}%
      \ifthenelse{\equal{\i}{1}}{\path[draw=\mycolor, line width=\mysize] (P6) -- (P4);}{}%
      \ifthenelse{\equal{\i}{2}}{\path[draw=\mycolor, line width=\mysize] (P4) -- (P2);}{}%
      \ifthenelse{\equal{\i}{3}}{\path[draw=\mycolor, line width=\mysize] (P1) -- (P2);}{}%
      \ifthenelse{\equal{\i}{4}}{\path[draw=\mycolor, line width=\mysize] (P1) -- (P3);}{}%
      \ifthenelse{\equal{\i}{5}}{\path[draw=\mycolor, line width=\mysize] (P3) -- (P5);}{}%
      \ifthenelse{\equal{\i}{6}}{\path[draw=\mycolor, line width=\mysize] (P3) -- (P4);}{}%
    }
  \end{tikzpicture}%
}

\newcommand{\sevensegnum}[2][]%
{%                                          
    \ifthenelse{\equal{#2}{0}}{\sevenseg[#1]{1,1,1,1,1,1,0,}}{%
        \ifthenelse{\equal{#2}{1}}{\sevenseg[#1]{0,1,1,0,0,0,0,}}{%
            \ifthenelse{\equal{#2}{2}}{\sevenseg[#1]{1,1,0,1,1,0,1,}}{%
                \ifthenelse{\equal{#2}{3}}{\sevenseg[#1]{1,1,1,1,0,0,1,}}{%
                    \ifthenelse{\equal{#2}{4}}{\sevenseg[#1]{0,1,1,0,0,1,1,}}{%
                        \ifthenelse{\equal{#2}{5}}{\sevenseg[#1]{1,0,1,1,0,1,1,}}{%
                            \ifthenelse{\equal{#2}{6}}{\sevenseg[#1]{1,0,1,1,1,1,1,}}{%
                                \ifthenelse{\equal{#2}{7}}{\sevenseg[#1]{1,1,1,0,0,0,0,}}{%
                                    \ifthenelse{\equal{#2}{8}}{\sevenseg[#1]{1,1,1,1,1,1,1,}}{%
                                        \ifthenelse{\equal{#2}{9}}{\sevenseg[#1]{1,1,1,1,0,1,1,}}{%
                                            {\sevenseg[#1]{0,0,0,0,0,0,0,}}
                                        }
                                    }
                                }
                            }
                        }
                    }
                }
            }
        }
    }%
}

% Choose engine primitive
\makeatletter
\@ifundefined{pdfuniformdeviate}{%
  \let\RandDeviate\uniformdeviate   % XeLaTeX / LuaLaTeX
}{%
  \let\RandDeviate\pdfuniformdeviate % pdfLaTeX
}
\makeatother

% Fully expandable random bit (0 or 1)
\newcommand{\randbit}{\the\numexpr\RandDeviate 2\relax}

% Convert one random draw to a clean literal 0/1
\newcommand{\randbitZ}{%
  \ifnum\numexpr\RandDeviate 2\relax>0 1\else 0\fi
}

% Use it to build the seven-seg list
\newcommand{\randsevenseg}{%
  \sevenseg[]{\randbitZ,\randbitZ,\randbitZ,\randbitZ,\randbitZ,\randbitZ,\randbitZ}%
}


% Copyright 2018-2022 FIUS
%
% This file is part of theo-vorkurs-folien.
%
% theo-vorkurs-folien is free software: you can redistribute it and/or modify
% it under the terms of the GNU General Public License as published by
% the Free Software Foundation, either version 3 of the License, or
% (at your option) any later version.
%
% theo-vorkurs-folien is distributed in the hope that it will be useful,
% but WITHOUT ANY WARRANTY; without even the implied warranty of
% MERCHANTABILITY or FITNESS FOR A PARTICULAR PURPOSE.  See the
% GNU General Public License for more details.
%
% You should have received a copy of the GNU General Public License
% along with theo-vorkurs-folien.  If not, see <https://www.gnu.org/licenses/>.



% Configuration for slides

% The date of the first day of the Theo-Vorkurs in Format dd/mm/yyyy
\SetDate[10/10/2022]

% Invite URL to the Ersti-Telegram-Group. Used for text on slide as well as QR-Code
\newcommand\telegramurl{https://t.me/+Q92w5biyY903NjEy}

\renewcommand\daynr{1}
% Copyright 2018-2022 FIUS
%
% This file is part of theo-vorkurs-folien.
%
% theo-vorkurs-folien is free software: you can redistribute it and/or modify
% it under the terms of the GNU General Public License as published by
% the Free Software Foundation, either version 3 of the License, or
% (at your option) any later version.
%
% theo-vorkurs-folien is distributed in the hope that it will be useful,
% but WITHOUT ANY WARRANTY; without even the implied warranty of
% MERCHANTABILITY or FITNESS FOR A PARTICULAR PURPOSE.  See the
% GNU General Public License for more details.
%
% You should have received a copy of the GNU General Public License
% along with theo-vorkurs-folien.  If not, see <https://www.gnu.org/licenses/>.

% This sets the template for the titlepage. 
% Only change to the default is that the titlegraphic is not in the left upper but in the right lower corner
\setbeamertemplate{title page}{
    \begin{minipage}[b][\paperheight]{\textwidth}
    \vfill%
    \ifx\inserttitle\@empty
    \else\usebeamertemplate*{title}
    \fi
    \ifx\insertsubtitle\@empty
    \else\usebeamertemplate*{subtitle}
    \fi
    \usebeamertemplate*{title separator}
    \ifx\beamer@shortauthor\@empty
    \else\usebeamertemplate*{author}
    \fi
    \ifx\insertdate\@empty
    \else\usebeamertemplate*{date}
    \fi
    \ifx\insertinstitute\@empty
    \else\usebeamertemplate*{institute}
    \fi
    \ifx\inserttitlegraphic\@empty
    \else{\hfill\inserttitlegraphic\hspace{.1\textwidth}}
    \fi
    \vfill
    \vspace*{1mm}
    \end{minipage}
}


\title{Vorkurs Theoretische Informatik}

\if\daynr1
    \subtitle{Einführung in die Grundideen, Mengenlehre und Aussagenlogik}
    \newcommand\daynamestr{Montag}
\fi
\if\daynr2
    \subtitle{Grundlagen der Beweise}
    \newcommand\daynamestr{Dienstag}
    \AdvanceDate
\fi
\if\daynr3
    \subtitle{Induktion und Einführung in die Grammatik}
    \newcommand\daynamestr{Mittwoch}
    \AdvanceDate\AdvanceDate
\fi
\if\daynr4
    \subtitle{Einführung in reguläre Sprachen}
    \newcommand\daynamestr{Donnerstag}
    \AdvanceDate\AdvanceDate\AdvanceDate
\fi
\if\daynr5
    \subtitle{Einführung in reguläre Sprachen}
    \newcommand\daynamestr{Freitag}
    \AdvanceDate\AdvanceDate\AdvanceDate
\fi


\date{\daynamestr, \today}

\author{Arbeitskreis Theo-Vorkurs}
\institute{\href{https://fius.de}{Fachgruppe Informatik Universität Stuttgart}}
% \titlegraphic{\hfill\includegraphics[height=1.5cm]{logo.pdf}}

% sets the qr-code to the current handout slides on the title page. 
% This can be changed to let the qr-code appear on every page by exchanging \titlegraphic with \logo.
\titlegraphic{
    \only<1|handout:0>{\fbox{\parbox{2cm+.5em}{\centering
        Aktuelle Folien:\par \vspace{.5ex}
        \qrcode{\handouturl{\daynr}}
    }}}
}


% \titlegraphic{\hfill\includegraphics[height=1.5cm]{logo.pdf}}

\begin{document}

\maketitle

\only<1|handout:0>{
	\setbeamercolor{background canvas}{bg=Bluecreen}
	\setbeamertemplate{title page}{
    \begin{minipage}[b][\paperheight]{\textwidth}
        \vspace{2em}
        \IfFontExistsTF{Segoe UI}{\fontspec{Segoe UI}}{}
        {
            \makeatletter
            \color{white}
            {\fontsize{40}{50}\selectfont
                :)
            }
            \par
            \vspace{1em}
            Your Studiengang ran into a problem and needs to be restructured.
            We're working on a \textbf{\inserttitle} for you.
            \par
            \vspace{.5em}
            \the\day .\the\month .\the\year\% complete
            % \ifx\inserttitle\@empty
            % \else\usebeamertemplate*{title}
            % \fi
            % \ifx\insertsubtitle\@empty
            % \else\usebeamertemplate*{subtitle}
            % \fi
            % \usebeamertemplate*{title separator}
            % \ifx\beamer@shortauthor\@empty
            % \else\usebeamertemplate*{author}
            % \fi
            % \ifx\insertdate\@empty
            % \else\usebeamertemplate*{date}
            % \fi
            % \ifx\insertinstitute\@empty
            % \else\usebeamertemplate*{institute}
            % \fi
            \par
            \vspace{2.5cm}
            \insertsubtitle
            \vfill
            {\fontsize{5pt}{5pt}\selectfont
                \insertauthor\ \textemdash\ \insertinstitute
            }
            \vspace{1.5em}
            \makeatother
        }
        \ifx\inserttitlegraphic\@empty
            \else{%
                \begin{tikzpicture}[remember picture,overlay]
                    \node[xshift=-10.5cm,yshift=-5cm] at (current page.north east){%
                        \colorbox{white}{
                            \only<1|handout:0>{
                                \qrcode[height=4em,padding]{\handouturl{\daynr}}
                            }
                        }};
                    \node[xshift=-8cm,yshift=-4.2cm] at (current page.north east){%
                        \small
                        \color{white}
                        Aktuelle Folien
                    };
                \end{tikzpicture}

            }
        \fi
    \end{minipage}
}
	\maketitle
	\setbeamercolor{background canvas}{
		use=palette primary,
		bg=palette primary.fg
	}
}

\metroset{sectionpage=progressbar,subsectionpage=none}

\section{Allgemeines}
% Copyright 2018-2022 FIUS
%
% This file is part of theo-vorkurs-folien.
%
% theo-vorkurs-folien is free software: you can redistribute it and/or modify
% it under the terms of the GNU General Public License as published by
% the Free Software Foundation, either version 3 of the License, or
% (at your option) any later version.
%
% theo-vorkurs-folien is distributed in the hope that it will be useful,
% but WITHOUT ANY WARRANTY; without even the implied warranty of
% MERCHANTABILITY or FITNESS FOR A PARTICULAR PURPOSE.  See the
% GNU General Public License for more details.
%
% You should have received a copy of the GNU General Public License
% along with theo-vorkurs-folien.  If not, see <https://www.gnu.org/licenses/>.

\subsection{Organisatorisches}
\begin{frame}[fragile]{Wer sind wir?}
  \begin{itemize}
    \item
          Fachgruppe Informatik
          \begin{itemize}
            \item Unser Ziel: \\
                  Das Leben von uns Studis während des Studiums angenehmer zu gestalten
            \item organisieren Veranstaltungen (Grillen, Spieleabende, Vorkurse, ...)
            \item verleihen Prüfungen aus den früheren Semestern
            \item vertreten die studentische Sicht in offiziellen Gremien
            \item ...und vieles mehr (es gibt z.B. einen 3D-Drucker)
          \end{itemize}
    \item Arbeitskreis Theoretische Informatik
          \begin{itemize}
            \item Teilmenge der Fachgruppe Informatik
            \item haben diesen Vorkurs organisiert
          \end{itemize}
  \end{itemize}
\end{frame}
\note[itemize]{
  \item test
  \item test
}

\subsection{Tipps zum Studium}
\begin{frame}[fragile]{Tipps zum Studium}
  \begin{itemize}
    \item Nützliche Links:\\
          \begin{itemize}
            \item Fachgruppe Informatik:\\
                  \url{https://fius.de/}
            \item Handouts und Foliensätze:\\ \url{https://fius.de/index.php/studien-interessierte/vorkurs-theoretische-informatik/}
            \item Materialien Ergänzung Theoretische Informatik 1 (Wintersemester 19/20): \\
                  \url{https://fmi.uni-stuttgart.de/ti/teaching/w19/eti1/}
            \item Ersti Telegram-Gruppe:\\
                  \qrcode[hyperlink]{\telegramurl}
                  \url{\telegramurl}
          \end{itemize}
    \item E-Mail der Fachgruppe: fius@informatik.uni-stuttgart.de

  \end{itemize}
\end{frame}

\begin{frame}[fragile]{Infos zum Online-Ablauf}
  \begin{alertblock}{Ablauf und Notfallplan}
    \begin{itemize}
      \item Der Online-Vorkurs wird eine Übertragung aus/in einem Hörsaal sein
      \item Es wird zwischen Vorlesungs- und Aufgabephasen abgewechselt.
      \item Wir benutzen BigBlueButton - wenn ihr hier seid, wisst ihr das schon.
      \item Bei technischen Problemen, die sich nicht zügig beheben lassen, wechseln wir ggf. auf eine andere Plattform. Den Joinlink verschicken wir dann per Mail und stellen ihn auf \url{https://fius.de/index.php/studien-interessierte/vorkurs-theoretische-informatik/}.
    \end{itemize}
    \alert{Traut euch, Fragen zu stellen und mitzumachen.}
  \end{alertblock}
\end{frame}


\begin{frame}[fragile]{Übersicht}
	\setbeamertemplate{section in toc}[sections numbered]
	\tableofcontents%[hideallsubsections]
\end{frame}

\section{Theoretische Informatik}

% Copyright 2018-2022 FIUS
%
% This file is part of theo-vorkurs-folien.
%
% theo-vorkurs-folien is free software: you can redistribute it and/or modify
% it under the terms of the GNU General Public License as published by
% the Free Software Foundation, either version 3 of the License, or
% (at your option) any later version.
%
% theo-vorkurs-folien is distributed in the hope that it will be useful,
% but WITHOUT ANY WARRANTY; without even the implied warranty of
% MERCHANTABILITY or FITNESS FOR A PARTICULAR PURPOSE.  See the
% GNU General Public License for more details.
%
% You should have received a copy of the GNU General Public License
% along with theo-vorkurs-folien.  If not, see <https://www.gnu.org/licenses/>.

\subsection{Anwendung}
\begin{frame}[fragile]{Was ist eigentlich Theoretische Informatik?}
    \begin{itemize}
        \item Theoretische Informatik ist die \textbf{formale} Herangehensweise an Probleme.\\
        \item Grundlage für viele andere Gebiete der Informatik
    \end{itemize}
\end{frame}

\begin{frame}{Anwendung der theoretischen Informatik}
    \begin{itemize}
        \item Ist ein bestimmtes Problem lösbar, oder \textbf{können} wir gar keine Lösung finden?
        \item IT-Sicherheit / Kryptographie: Die Sicherheit bestimmter Algorithmen \textbf{beweisen}
        \item Reguläre Ausdrücke
        \item Künstliche Intelligenz
        \item Compilerbau
        \item ...und vieles mehr...
    \end{itemize}
\end{frame}

\subsection{Theoretische Informatik in deinem Studium}
\begin{frame}[fragile]{Theoretische Informatik in deinem Studium}
    Logik und diskrete Strukturen ist Orientierungsprüfung für Informatik, Medieninformatik, Softwaretechnik und Data Science.
    \begin{itemize}
        \item Du musst diese Prüfung spätestens zum Ende des dritten Semester bestanden haben.
        \item Du musst spätestens zum Ende des zweiten Semesters eine der beiden Orientierungsprüfungen angetreten haben.
        \item Du kannst die schriftliche Prüfung einmal nachschreiben und hast dann noch einen mündlichen Versuch im selben Semester.
    \end{itemize}
    \alert{Kennt eure \href{https://www.student.uni-stuttgart.de/pruefungsorganisation/pruefungsordnung/}{\underline{Prüfungsordnung}}!}
\end{frame}

\begin{frame}{Theoretische Informatik in deinem Studium}
    \begin{itemize}
        \item Theoretische Informatik I\\
              Logik und diskrete Strukturen\\
              \quad Dozent: Dr. Manfred Kufleitner
        \item Theoretische Informatik II\\
              Formale Sprachen und Berechnbarkeit\\
              \quad Dozent: Dr. Manfred Kufleitner
        \item Theoretische Informatik III\\
              Algorithmik und Komplexitätstheorie\\
              \quad Dozent: Prof. Stefan Funke
    \end{itemize}
    \alert{Altklausuren helfen bei der Prüfungsvorbereitung. \\Fragt auch nach den Klausuren des alten Fachs.}\\
    \textit{Ihr seid die ersten die die umstrukturierten Theo Fächer hören, die vohandenen Alklausuren passen vmtl. nicht auf die neue Strukturen.}
\end{frame}

%\begin{frame}{Literatur der Vorlesung}
%TODO: Rausfinden, welche Literatur in der VL empfohlen wird
%TODO: Ggf. gute YT-Kanäle finden
%\end{frame}


\metroset{sectionpage=none,subsectionpage=progressbar}

\section{Mengen}
\subsection{Grundlagen}
% Copyright 2018-2022 FIUS
%
% This file is part of theo-vorkurs-folien.
%
% theo-vorkurs-folien is free software: you can redistribute it and/or modify
% it under the terms of the GNU General Public License as published by
% the Free Software Foundation, either version 3 of the License, or
% (at your option) any later version.
%
% theo-vorkurs-folien is distributed in the hope that it will be useful,
% but WITHOUT ANY WARRANTY; without even the implied warranty of
% MERCHANTABILITY or FITNESS FOR A PARTICULAR PURPOSE.  See the
% GNU General Public License for more details.
%
% You should have received a copy of the GNU General Public License
% along with theo-vorkurs-folien.  If not, see <https://www.gnu.org/licenses/>.

\begin{frame}[fragile]{Mengen}
    \begin{itemize}

        \item<1-|handout:1>
              Was ist eine \alert<1,2>{Menge}?
        \item<2->
              \only<2|handout:1>{
                  \vspace*{0.5cm}
                  Eine Menge
                  \begin{itemize}
                      \item ist eine \alert{Sammlung von Zeugs}
                      \item ist unsortiert
                      \item enthält keine Duplikate
                      \item wird mit geschweiften Klammern notiert
                  \end{itemize}

                  \metroset{block=fill}

                  \begin{exampleblock}{Beispiel}
                      $\mathbb{N} = \{0, 1, 2, 3, \dots \}$ = Menge der Natürlichen Zahlen\\
                      Studierende = \{Georg, Tim, Triin, Seb, Babett, $\dots$\}\\
                      $\{1,2\} = \{2,1\} = \{1,1,2,1,1,1\}$\\
                      $\emptyset = \{\} =$ leere Menge
                  \end{exampleblock}}
              \uncover<3-|handout:2>{
                  Was ist ein \alert<3,4>{Element}?}
        \item<4->
              \only<4|handout:2>{
                  \vspace*{0.5cm}
                  Ein Element ist ein \alert{Ding aus einer Menge}.\\

                  \metroset{block=fill}

                  \begin{exampleblock}{Beispiel}
                      $\mathbf{1}$ ist ein Element der \textbf{Natürlichen Zahlen}\\
                      $\mathbf{1} \in \mathbb{N}$\\
                      \vspace*{0.5cm}
                      \textbf{Tim} ist ein Element aus der Menge der \textbf{Studierenden}\\
                      \textbf{Tim} $\in$ \textbf{Studierende}\\
                      \vspace*{0.5cm}
                      $\mathbf{a}$ ist in der Menge $\mathbf{\{u, v, w\}}$ nicht enthalten\\
                      $\mathbf{a} \notin \mathbf{\{u, v, w\}}$
                  \end{exampleblock}
              }
              \uncover<5-|handout:3>{
                  Was ist eine \alert<5,6>{Teilmenge}?
              }
        \item<6|handout:3>
              \vspace*{0.5cm}
              Eine Teilmenge ist eine \alert{spezielle Auswahl} von Elementen einer Menge.\\

              \metroset{block=fill}

              \begin{exampleblock}{Beispiel}
                  $\{1, 2, 3\}$ ist eine Teilmenge der Natürlichen Zahlen\\
                  $\{1,2,3\} \subseteq \mathbb{N}$\\
                  \vspace*{0.5cm}
                  \{\textbf{Julian}\} ist eine Teilmenge der \textbf{Studierenden}\\
                  \{\textbf{Julian}\} $\subseteq$ \textbf{Studierende}
              \end{exampleblock}

    \end{itemize}
\end{frame}
\note[itemize]{
    \item Note Natürliche Zahlen: 0 ist \textbf{in Theo} Teil von $\mathbb{N}$, also auch hier im Vorkurs (in Mathe nicht)
}

\begin{frame}{Echte Teilmengen}
    \begin{itemize}
        \item ist $\{A,B\}$ eine Teilmenge von $\{A,B\}$?
              \pause
        \item \alert{Ja!}
              \pause
        \item Aber keine \alert{echte} Teilmenge
    \end{itemize}
    \pause
    \metroset{block=fill}
    \begin{exampleblock}{Beispiel}
        $N \subseteq M$: $N$ ist eine Teilmenge von $M$ aber darf auch $M$ sein.\\
        $N \subset M$: $N$ ist eine Teilmenge von $M$ und muss mindestens 1 Element weniger enthalten.
        $N \subsetneq M$ bedeutet das selbe wie $N \subset M$, aber ist etwas expliziter.
    \end{exampleblock}
\end{frame}

{\setbeamercolor{palette primary}{bg=ExColor}
\begin{frame}[fragile]{Denkpause}
    \begin{alertblock}{Aufgaben}
        Nenne jeweils 5 Elemente der folgenden Mengen:
    \end{alertblock}

    \metroset{block=fill}
    \begin{block}{Normal}
        \begin{itemize}
            \item $\{a, b, c, d, e, f, g, h, i\}$
            \item $\{0, 2, 4, 8, 16, 32, 64, 128, 256, 512\}$
            \item $\mathbb N$
        \end{itemize}
    \end{block}
    Nenne 5 Teilmengen die keine gegenseitigen Teilmengen sind
    \metroset{block=fill}
    \begin{block}{Etwas schwerer}
        \begin{itemize}
            \item $M = \{\text{\WashCotton}, \text{\NoWash}, \text{\IroningII}, \text{\Tumbler}, \text{\SpecialForty} \}$
            \item $M = \{\text{Lisa}, \text{Tobi}, \text{Fabian}, \text{Linus}\}$
        \end{itemize}
    \end{block}
\end{frame}
}

{\setbeamercolor{palette primary}{bg=ExColor}
\begin{frame}<handout:0>{Lösungen}
    Mögliche Lösungen sind \dots
    \begin{itemize}[<+- | alert@+>]
        \item $a, b, c, d, e \in \{a, b, c, d, e, f, g, h, i\}$
        \item $0, 2, 4, 8, 16,\in \{0, 2, 4, 8, 16, 32, 64, 128, 256, 512\}$
        \item $0,1,2,3,4 \in \mathbb N$
        \item \{\WashCotton\}, \{\NoWash\}, \{\IroningII\}, \{\Tumbler\}, \{\SpecialForty \}
        \item \{Lisa, Tobi\}, \{Lisa, Fabian\}, \{Lisa, Linus\}, \{Tobi, Fabian\}, \{Linus, Tobi\}
    \end{itemize}
\end{frame}
}


\subsection{Mengenschreibweise}
\input{../parts/mengen-mengenschreibweise.tex}

\subsection{Mengenoperationen}
% Copyright 2018-2022 FIUS
%
% This file is part of theo-vorkurs-folien.
%
% theo-vorkurs-folien is free software: you can redistribute it and/or modify
% it under the terms of the GNU General Public License as published by
% the Free Software Foundation, either version 3 of the License, or
% (at your option) any later version.
%
% theo-vorkurs-folien is distributed in the hope that it will be useful,
% but WITHOUT ANY WARRANTY; without even the implied warranty of
% MERCHANTABILITY or FITNESS FOR A PARTICULAR PURPOSE.  See the
% GNU General Public License for more details.
%
% You should have received a copy of the GNU General Public License
% along with theo-vorkurs-folien.  If not, see <https://www.gnu.org/licenses/>.

\begin{frame}{Mengenoperationen - Schnitt}
    \begin{columns}
        \column{0.5\textwidth}
        \begin{alertblock}{Schnitt - $A\cap B$}
            Gegeben zwei Mengen A und B.\\
            In der Schnittmenge liegt alles, das in Menge A \textbf{und} in Menge B ist.
        \end{alertblock}
        \column{0.5\textwidth}
        \begin{figure}
            \centering
            \includegraphics[width=0.7\textwidth]{../figures/AundB.png}
            \caption{Veranschaulichung der Schnittmenge}
            \label{fig:my_label}
        \end{figure}
    \end{columns}
\end{frame}

\begin{frame}{Mengenoperationen - Vereinigung}
    \begin{columns}
        \column{0.5\textwidth}
        \begin{alertblock}{Vereinigung - $A\cup B$}
            Gegeben zwei Mengen A und B.\\
            In der Vereinigung liegt alles, das nur in A, nur in B \textbf{oder} in beiden Mengen liegt.
        \end{alertblock}
        \column{0.5\textwidth}
        \begin{figure}
            \centering
            \includegraphics[width=0.7\textwidth]{../figures/AoderB.png}
            \caption{Veranschaulichung der Vereinigung}
            \label{fig:my_label}
        \end{figure}
    \end{columns}
\end{frame}

\begin{frame}{Mengenoperationen - Komplement}
    \begin{columns}
        \column{0.5\textwidth}
        \begin{alertblock}{Komplement - $\overline{A}$}
            Gegeben sei eine Menge A.\\
            Im Komplement der Menge A liegen alle Elemente, die in der Obermenge (z.B. $\mathbb{N}$), aber nicht in der Menge A selbst liegen.
        \end{alertblock}
        \column{0.5\textwidth}
        \begin{figure}
            \centering
            \begin{tikzpicture}[align=center]
    \node[
        ellipse, draw, minimum height = 2.5cm, minimum width = 4cm, fill = orange!45!white, line width = 0.25mm
    ] at (0,0) {$\overline{A}$};
    \node[
        ellipse, draw, minimum height = 1.2 cm, minimum width = 1.2cm, fill = white, line width = 0.25mm
    ] at (1cm,0) {$A$};
    \node[] at (2cm, -1cm) {$\mathbb{N}$};
\end{tikzpicture}
            \caption{Veranschaulichung des Komplements}
            \label{fig:komplement}
        \end{figure}
    \end{columns}
    \onslide<2>{\alert{\emph{Anmerkung:}} Kann auch geschrieben werden als $\mathbb{N}\setminus A$. \\
        \hspace{2cm}(gesprochen $\mathbb{N}$ \emph{\glqq ohne\grqq} $A$)}
\end{frame}

{\setbeamercolor{palette primary}{bg=ExColor}
\begin{frame}{Mengenoperationen}
    Berechne folgende Mengen
    \metroset{block=fill}
    \begin{alertblock}{Normal}
        \begin{itemize}
            \item $M_1 = \{1\}\cup \{2\}$
            \item $M_2 = \{\} \cap \{-1, 0, 1\}$
            \item $M_3 = \mathbb{N} \cup \mathbb{Z}$
            \item $M_4 = \overline{\{3^{n}\mid n \ \text{ist gerade}\} }$ über $\{3^{n}\mid n \in \mathbb{N}\}$
        \end{itemize}
    \end{alertblock}
    \begin{alertblock}{Schwer bis sehr schwer}
        \begin{itemize}
            \item $M_5 = \{1, 2, 3\} \cap  \{1, \{2, 3\}\}$
            \item $M_6 = \{u \mid |u| \equiv 0 \pmod 2, u \in \mathbb{N}\}$\\\hspace{0.65cm}$\cup$ $\{v \mid |v| \equiv 0 \pmod 4, v \in \mathbb{N}\}$
            \item $M_7 = \overline{\{a^{n} \mid n \ \text{ist gerade}, a \in \{3,4\}\}}$ über $\mathbb{N}$
        \end{itemize}
    \end{alertblock}
\end{frame}

\begin{frame}<handout:0>{Lösungen}
    \begin{itemize}[<+- | alert@+>]
        \item
              $M_1 = \{1, 2\}$
        \item
              $M_2 = \emptyset$
        \item
              $M_3 = \mathbb{Z}$
        \item
              $M_4 = \{3^{n} \mid$ n ist ungerade$\}$
        \item
              $M_5 = \{1\}$
        \item
              $M_6 = \{u \mid |u| \equiv 0 \pmod 2, u \in \mathbb{N}\}$
        \item
              $M_7 = \mathbb{N} \setminus \{9^n, 16^n \mid n \in \mathbb N\}$\\
              \hspace{0.44cm}$ = \mathbb{N} \setminus \{3^{2n}, 4^{2n} \mid n \in \mathbb N\}$
    \end{itemize}
\end{frame}
}


\metroset{sectionpage=progressbar,subsectionpage=none}

\section{Aussagenlogik}
\begin{frame}[fragile]{Was ist Aussagenlogik?}
    \begin{alertblock}{Aussagen}
    \begin{itemize}
        \item Paris ist die Hauptstadt von Frankreich
        \item Mäuse jagen Elefanten
        \item $5 \in \mathbb{N}$
        \item 5 = 8
        \item $u \in \{u, v, w\}$
    \end{itemize}
    \end{alertblock}
    Eine Aussage A ist entweder \textbf{wahr} oder \textbf{falsch}.
\end{frame}

\begin{frame}[fragile]{Was ist Aussagenlogik?}
    \begin{alertblock}{Das sind keine Aussagen}
    \begin{itemize}
        \item Macht theoretische Informatik Spaß?
        \item Geh dein Zimmer aufräumen!
        \item Wie viele Tiere wohnen in der Uni?
        \item $(x+y)^2+1$
        \item \{a,b,c\}
        \item ...
    \end{itemize}
    \end{alertblock}
    Diesen Sätzen können wir keinen eindeutigen Wahrheitswert \textbf{wahr} oder \textbf{falsch} zuordnen.
\end{frame}

\begin{frame}[fragile]{Was ist Aussagenlogik?}
    \begin{alertblock}{Wozu brauchen wir das?}
    \begin{itemize}
        \item Wir untersuchen, wie wir Aussagen verknüpfen können.
        \item Damit ziehen wir formale Schlüsse und führen Beweise.
    \end{itemize}
    \end{alertblock}
\end{frame}

\begin{frame}[fragile]{Logische Operationen}
Wir können Aussagen verändern oder durch Operationen zu neuen Aussagen verbinden.
\begin{itemize}
    \item A: Fred möchte Schokolade.
    \item B: Fred möchte Gummibärchen.
\end{itemize}
\metroset{block=fill}
\begin{alertblock}{Grundoperationen}
\begin{itemize}
    \item<1-> \textbf{Und}: A \alert<1>{$\wedge$} B $\leadsto$ Fred möchte Schokolade \alert<1>{und} Gummibärchen.\\
    \only<1>{\emph{Analog}: M: $M_1 \cap M_2$, Jedes Element aus M liegt in $M_1$ \textbf{und} in $M_2$}
    \item<2-> \textbf{Oder}: A \alert<2>{$\vee$} B $\leadsto$ Fred möchte Schokolade \alert<2>{oder} Gummibärchen.\\
    \only<2>{\emph{Anmerkung}: Inklusives \glqq oder\grqq, kein \glqq entweder oder\grqq \\
    Das heißt, es können auch beide Aussagen wahr sein.\\}
	\only<2>{\emph{Analog}: M: $M_1 \cup M_2$, Jedes Element aus M liegt in $M_1$ \textbf{oder} in $M_2$}
    \item<3> \textbf{Nicht}: \alert<3>{$\neg$}A $\leadsto$ Fred möchte \alert<3>{keine} Schokolade.\\
    \only<3>{\emph{Analog}: M: $\overline{M_1}$, Jedes Element aus M liegt \textbf{nicht} in $M_1$}
\end{itemize}
\end{alertblock}
\end{frame}


\begin{frame}{Überblick: Mengenoperationen}
	
	Auf \alert{Mengen} A, B lassen sich verschiedene Mengenoperationen ausführen.
	
	\metroset{block=fill}
	\begin{exampleblock}{Mengenoperationen}
		\begin{itemize}
			\item Schnitt: $A \cap B$
			\item Vereinigung: $A \cup B$
			\item Komplement: $\overline{A}$
		\end{itemize}
	\end{exampleblock}
\end{frame}

\begin{frame}{Überblick: Logische Operationen}
	
	Auf \alert{Aussagen} A, B lassen sich verschiedene logische Operationen ausführen.
	
	\metroset{block=fill}
	\begin{exampleblock}{Logische Operationen}
		\begin{itemize}
			\item Logisches Und: $A \wedge B$
			\item Logisches Oder: $A \vee B$
			\item Logisches Nicht: $\neg A$
		\end{itemize}
	\end{exampleblock}
\end{frame}

{\setbeamercolor{palette primary}{bg=ExColor}
	\begin{frame}{Logische Operationen vs. Mengenoperationen}
		\alert{Mengenoperationen und logische Operationen dürfen nicht verwechselt werden.}
		\begin{table}[]
			\begin{tabular}{l l}
				A: $5 \in \mathbb{N}$ & B: Es regnet\\
				C: \{$w \mid |w|=2$ \} \ & D: \{a,b,c,x,y\}\\
			\end{tabular}
		\end{table}
		\metroset{block=fill}
		\metroset{block=fill}
		\begin{block}{Welche dieser Verknüpfungen sind zulässig?}
			\begin{itemize}
				\item $L_1: A \wedge B$
				\item $L_2: A \vee C$
				\item $L_3: C \cap D$
				\item $L_4: A \wedge B \cup C$
			\end{itemize}
		\end{block}
	\end{frame}
}

{\setbeamercolor{palette primary}{bg=ExColor}
	\begin{frame}[fragile]{Logische Operationen vs. Mengenoperationen}
		\begin{itemize}[<+- | alert@+>]
			\item $L_1$: Zulässig
			\item $L_2$: Nicht zulässig
			\item $L_3$: Zulässig
			\item $L_4$: Nicht zulässig
		\end{itemize}
	\end{frame}
}



\begin{frame}{Logische Operationen: Implikation}
\begin{alertblock}{A$\implies$B}
\begin{itemize}
    \item \glqq Wenn A wahr ist, dann muss auch B wahr sein.\grqq
    \item kurz: \glqq\textbf{Wenn} A, \textbf{dann} B.\grqq
    \item Wenn A falsch ist können wir keine Aussage über B treffen.
    \item A$\implies$B ist dieselbe Aussage wie $\neg A \vee B$
\end{itemize}
\end{alertblock}
\end{frame}

\begin{frame}{Logische Operationen: Äquivalenz}
\begin{alertblock}{A$\iff$B}
\begin{itemize}
    \item \glqq A ist wahr, \textbf{genau dann wenn} B wahr ist.\grqq
    \item kurz: \glqq A gdw. B\grqq
    \item A und B müssen den selben Wahrheitswert haben.
    \item A$\iff$B ist dieselbe Aussage wie $(A \implies B) \wedge (B \implies A)$
\end{itemize}
\end{alertblock}
\end{frame}

{\setbeamercolor{palette primary}{bg=ExColor}
\begin{frame}[fragile]{Denkpause}
    \begin{alertblock}{Aufgaben}
      Berechne den Wahrheitswert folgender Aussagen. 
    \end{alertblock}
    \metroset{block=fill}
    \begin{block}{Normal}
    \begin{itemize}
        \item $A_1$: $5 \in \mathbb{N} \wedge a \in \{a, b, c\}$
        \item $A_2$: $0 \in \mathbb{N} \vee a \in \{a, b, c, d\}$
        \item $A_3$: $A_1 \iff A_2$
    \end{itemize}
    \end{block}
    \begin{block}{Etwas Schwerer}
    \begin{itemize}
        \item $A_4$: $(\emptyset=\emptyset^{*}) \implies (a \in \emptyset)$
        \item $A_5$: $(a \notin \emptyset) \implies (\emptyset = \emptyset^{*})$
        \item $A_6$: $A_4 \iff A_5$
        \item $A_7$: $(7 \in \{1, 2, 7, 9\}) \cap (2 = 7-5)$
        \item $A_8$: Wenn mein Auto fliegen kann, hast du auch ein fliegendes Auto.
    \end{itemize}
    \end{block}
\end{frame}
}

% {\setbeamercolor{palette primary}{bg=ExColor}
% \begin{frame}[fragile]{Denkpause}
%     \begin{alertblock}{Aufgaben}
%       Löse folgende Zusatzaufgabe. 
%     \end{alertblock}
%     \metroset{block=fill}
%     \begin{block}{Zusatz}
%     \begin{itemize}
%         \item $A_8$: Gegeben zwei Aussagen A, B.\\
%         Formuliere die Aussage $A \iff B$ nur unter Verwendung der Junktoren $\wedge, \vee, \neg$
%     \end{itemize}
%     \end{block}
% \end{frame}
% }
\subsubsection{Aufgaben}
{\setbeamercolor{palette primary}{bg=ExColor}
\begin{frame}{Lösungen}
  \begin{itemize}[<+- | alert@+>]
        \item 
            $A_1$: wahr
        \item
            $A_2$: wahr
        \item
            $A_3$: wahr
        \item
            $A_4$: wahr 
        \item
            $A_5$: falsch
       	\item
       		$A_6$: falsch
        \item
            $A_7$: Das ist keine Aussage, da der Schnitt($\cap$) verwendet wurde um zwei Aussagen miteinander zu verknüpfen
       	\item
       		$A_8$: wahr
    \end{itemize}
\end{frame}
}

\begin{frame}[standout]
    Murmelpause
\end{frame}

% {\setbeamercolor{palette primary}{bg=ExColor}
% \begin{frame}{Lösungen}
%     \metroset{block=fill}
%     \begin{block}{Zusatz}
%     \begin{itemize}
%         \item $A_8$: Gegeben zwei Aussagen A, B.\\
%         Formuliere die Aussage $A \iff B$ nur unter Verwendung der Junktoren $\wedge, \vee, \neg$
%     \end{itemize}
%     \end{block}
%   \begin{alertblock}{Lösung}
%         \begin{itemize}[<+- | alert@+>]
%             \item Äquivalenz bedeutet intuitiv: Beide wahr oder beide falsch
%             \item $A_8$: $(A \wedge B) \vee (\neg A \wedge \neg B)$
%         \end{itemize}
%   \end{alertblock}
% \end{frame}
% }




\begin{frame}{Anwendung der Implikation}
    Wir haben zwei Aussagen $A$ und $B$. Wir nehmen nun an, $A$ sei wahr. Wenn wir zeigen, dass dann auch $B$ wahr ist, wissen wir, dass $A \implies B$ gilt.
\metroset{block=fill}
\begin{exampleblock}{Beispiel}
\begin{enumerate}
    \item<1-> Wir wollen zeigen, dass für eine ganze Zahl $x$ \\
    die Implikation $3 = x - 2 \implies x = 5$ gilt.
    \item<2-> Wir nehmen an, dass die linke Aussage wahr ist\dots
    \item<3-> \dots und zeigen, dass dann die rechte Aussage gilt.
    \item<4-> $(3 = x - 2) \implies (3 + 2 = x) \implies (5 = x) \implies (x = 5)$
    \item<5-> Also folgt die rechte Aussage aus der linken
    \item<6-> Somit gilt die Implikation \qed\;
\end{enumerate}
\end{exampleblock}
\end{frame}

% {\setbeamercolor{palette primary}{bg=ExColor}
% \begin{frame}{Denkpause}
%     \begin{alertblock}{Aufgaben}
%       Welche der folgenden Schlüsse sind richtig? 
%     \end{alertblock}
%     \metroset{block=fill}
%     \begin{block}{Normal bis Schwer}
%     \begin{itemize}
%         \item $S_1$: Lukas ist im Vorkurs oder schläft noch. Im Vorkurs ist er nicht. Also schläft er noch.
%         \item $S_2$: Wenn Anne nicht rennt, bekommt sie die Bahn nicht. Sie bekommt die Bahn. Also ist sie gerannt.
%         \item $S_3$: Wenn Tobi nicht lernt, besteht er nicht. Tobi hat gelernt. Also besteht er.
%     \end{itemize}
%     \end{block}
% \end{frame}
% }

% {\setbeamercolor{palette primary}{bg=ExColor}
% \begin{frame}{Lösungen}
%   \begin{itemize}[<+- | alert@+>]
%         \item 
%             $S_1$: Richtig.
%         \item
%             $S_2$: Richtig. Begründung: Die zweite Aussage ist die Kontraposition der ersten Aussage. Wie gezeigt wurde, ist die Kontraposition einer Aussage wahr gdw. die Aussage selbst wahr ist.
%         \item
%             $S_3$: Falsch. Beispiel: Tobi lernt, und besteht trotzdem nicht. \Frowny
%     \end{itemize}
% \end{frame}
% }


\section{Quantoren}
\begin{frame}[fragile]{Quantoren}
    Oft wollen wir Aussagen nicht nur für ein Element, sondern für viele Elemente treffen.
    \metroset{block=fill}
    \begin{exampleblock}{Beispiel}
        $A_1$: Für die Zahl 5 gilt: Sie hat einen Nachfolger\\
        \emph{Allgemeiner:}\\
        $A_2$: Für jede natürliche Zahl n gilt: n hat einen Nachfolger
    \end{exampleblock}
    \begin{exampleblock}{Beispiel}
        $A_3$: Für die Zahl 5 gilt: Sie ist eine Primzahl\\
        \emph{Allgemeiner:}\\
        $A_4$: Es gibt eine natürliche Zahl n, so dass gilt: n ist eine Primzahl
    \end{exampleblock}
\end{frame}

\begin{frame}[fragile]{Quantoren}
    Mithilfe von \textbf{Quantoren} vereinfachen wir uns die Schreibweise dieser Aussagen.\\
    \vspace{0.5cm}
    Quantor \alert{$\forall$}: Die Aussage gilt für alle Elemente.\\
    \metroset{block=fill}
    \begin{exampleblock}{Beispiel}
        $A_1$: $\forall k \in \mathbb{N}:$ 2k ist gerade
    \end{exampleblock}
    Quantor \alert{$\exists$}: Die Aussage gilt für mindestens ein Element.\\
    \metroset{block=fill}
    \begin{exampleblock}{Beispiel}
        $A_2$: $\exists k \in \mathbb{N}:$ k ist Primzahl
    \end{exampleblock}
\end{frame}

\begin{frame}[fragile]{Quantoren}
    In einer Aussage können mehrere Quantoren vorkommen.\\
    Wir lesen dann von links nach rechts.
    \metroset{block=fill}
    \begin{exampleblock}{Beispiel}
        $A_1$: $\forall x,y \in \mathbb{N}: \exists z \in \mathbb{N}: x+y = z$\\
        Bedeutung: Für zwei beliebige Zahlen x und y aus $\mathbb{N}$ gibt es eine weitere natürliche Zahl z, so dass $x+y=z$ gilt.
    \end{exampleblock}
\end{frame}

\begin{frame}[fragile]{Quantoren}
    \alert{Achtung!}\\
    Die Reihenfolge von zwei Quantoren zu vertauschen, kann die Bedeutung einer Aussage deutlich verändern.
    \metroset{block=fill}
    \begin{exampleblock}{Beispiel}
        x,y $\in$ Studenten\\
        \textbf{$A_1$: $\forall x \exists y:$ x schlägt y\\
        $A_2$: $\exists x \forall y:$ x schlägt y\\ }
        Was ist der Unterschied zwischen beiden Aussagen?
    \end{exampleblock}
\end{frame}

\begin{frame}{Quantoren}
    \begin{alertblock}{Aufgabe}
      Wir formulieren folgende Aussage mithilfe von Quantoren und den Symbolen der Aussagenlogik (Junktoren).
    \end{alertblock}
    \metroset{block=fill}
    \begin{block}{Beispiel}
    \begin{itemize}
        \item $A_1$: Eine ganze Zahl ist eine natürliche Zahl, wenn sie positiv oder null ist.
    \end{itemize}
    \end{block}
    \begin{block}{Hinführung}
    \begin{itemize}
        \item $A_1$: Für alle ganzen Zahlen x gilt: Wenn x positiv oder null ist, ist x eine natürliche Zahl.
    \end{itemize}
    \end{block}
    \begin{block}{\alert{Lösung}}
    \begin{itemize}
        \item $A_1$: $\forall x \in \mathbb{Z}: x \geq 0 \implies x \in \mathbb{N}$
    \end{itemize}
    \end{block}
\end{frame}

{\setbeamercolor{palette primary}{bg=ExColor}
\begin{frame}{Denkpause}
    \begin{alertblock}{Aufgaben}
      Formuliere folgende Aussagen mithilfe von Quantoren und den Symbolen der Aussagenlogik (Junktoren). 
    \end{alertblock}
    \metroset{block=fill}
    \begin{block}{Normal}
    \begin{itemize}
        \item $A_1$: Die Differenz zweier ganzer Zahlen ist wieder eine ganze Zahl.
    \end{itemize}
    \end{block}
    \begin{block}{Schwer}
    \begin{itemize}
        \item $A_2$: Jede natürliche Zahl lässt sich als Summe von vier Quadratzahlen darstellen.
    \end{itemize}
    \end{block}
    \begin{block}{Da haben selbst wir keinen Bock drauf}
    \begin{itemize}
        \item $A_3$: Eine natürliche Zahl, die von einer von ihr verschiedenen natürlichen Zahl größer als 1 geteilt wird, ist nicht prim.
    \end{itemize}
    \end{block}
\end{frame}
}

{\setbeamercolor{palette primary}{bg=ExColor}
\begin{frame}{Lösungen}
  \begin{itemize}[<+- | alert@+>]
        \item 
            $A_1$: $\forall x,y \in \mathbb{Z}: x-y \in \mathbb{Z}$
        \item
            $A_2$: $\forall x \in \mathbb{N}: \exists a, b, c, d \in \mathbb{N}: x = a^2 + b^2 + c^2 + d^2$
        \item
            $A_3$: $\forall x \in \mathbb{N}: \exists y \in \mathbb{N}: (y>1) \wedge (y \neq x) \wedge (y \mid x) \implies x\ \text{ist keine Primzahl}$.
    \end{itemize}
\end{frame}
}


\section{Beweisen}
% Copyright 2018-2022 FIUS
%
% This file is part of theo-vorkurs-folien.
%
% theo-vorkurs-folien is free software: you can redistribute it and/or modify
% it under the terms of the GNU General Public License as published by
% the Free Software Foundation, either version 3 of the License, or
% (at your option) any later version.
%
% theo-vorkurs-folien is distributed in the hope that it will be useful,
% but WITHOUT ANY WARRANTY; without even the implied warranty of
% MERCHANTABILITY or FITNESS FOR A PARTICULAR PURPOSE.  See the
% GNU General Public License for more details.
%
% You should have received a copy of the GNU General Public License
% along with theo-vorkurs-folien.  If not, see <https://www.gnu.org/licenses/>.

\begin{frame}{Einführung}
    \begin{alertblock}{Was ist ein Beweis?}
        \begin{itemize}
            \item lückenlose Folge von logischen Schlüssen,\\welche zur zu beweisenden Behauptung führen
            \item nicht nur einleuchtend, sondern zweifelsfrei korrekt
        \end{itemize}
    \end{alertblock}
    \onslide<2|handout:1>{
        \begin{alertblock}{Warum beweisen?}
            \begin{itemize}
                \item Aussage basierend auf Fakten und nicht subjektiv belegen
                \item Bestätigung von Aussagen für weitere Nutzung
                \item Zeigen der absoluten Wahrheit
            \end{itemize}
        \end{alertblock}
    }
\end{frame}

\subsection{Beweisbeispiel: Transitivität der Teilmenge}
\begin{frame}[fragile]{Beispielbeweis}
    \begin{exampleblock}{Zu zeigen: Teilmengen sind transitiv.}
        \begin{enumerate}
            \item<1->\alert<1|handout:0>{
                      \only<1|handout:0>{zu zeigen: }\onslide<2->{z.z. }$A\subseteq B\wedge B\subseteq C \alert<3|handout:0>{\implies\text{}}A\subseteq C$
                  }
            \item<2->\alert<2|handout:0>{
                      \only<2|handout:0>{Umschreiben:\\}
                      $\iff $\alert<4,5|handout:0>{$($\alert<9|handout:0>{$($\alert<6|handout:0>{$\forall x$}$\ : x \in A \implies x \in B)$}$ \wedge $\alert<10|handout:0>{$($\alert<6|handout:0>{$\forall x$}$\ : x \in B \implies x \in C)$}$)$}\\
                  \qquad\alert<3|handout:0>{$\implies$}$\;($\alert<6|handout:0>{$\forall x$}$\ :\ $\alert<7|handout:0>{$x \in A$}$ \implies x \in C)$
                  }
            \item<3->\alert<3|handout:0>{
                      \only<3|handout:0>{\emph{Implikation}\\
                          linke Seite wahr $\implies$ rechte Seite muss wahr sein.\\
                          linke Seite falsch $\implies$ beliebiges kann folgen\\
                          $\implies$ uns interessiert also nur der Fall \emph{links ist wahr}}
                      \alert<4>{\only<4,5|handout:0>{Wir machen uns also \emph{\textquotedbl die linke Seite ist wahr\textquotedbl} zur Voraussetzung}\alert<5>{\only<5|handout:0>{:\\Angenommen, $A \subseteq B \wedge B \subseteq C$ gilt.}}}
                  \onslide<6->{Ang., $A \subseteq B \wedge B \subseteq C$.}
                  }
            \item<6->\alert<6|handout:0>{
                      \only<6|handout:0>{Jetzt geht der Beweis richtig los.\\Wähle beliebiges $x$, um Allgemeinheit zu wahren\dots\\}
                      \onslide<6->{Sei $x$ beliebig}\alert<7>{\onslide<7-|handout:0>{ mit \alert<9>{$x\in A$.}}}
                  }
            \item<8->\alert<8-9|handout:0>{
                      \only<8|handout:0>{Wir können jetzt unsere Voraussetzungen ausnutzen,\\um $x\in C$ zu folgern.}
                      \onslide<9->$\implies x\in B$
                      \alert<10|handout:0>{\onslide<10->$\implies x\in C$}
                      \onslide<11>\qed
                  }
        \end{enumerate}
    \end{exampleblock}
\end{frame}


\section{Wiederholung}
\begin{frame}[fragile]{Das können wir jetzt beantworten}
	\begin{alertblock}{Einführung}
		\begin{itemize}
			\item Theoretische Informatik ist ganz schön wichtig...
			\item ...für mein Studium.
		\end{itemize}
	\end{alertblock}
\end{frame}

% Copyright 2018-2022 FIUS
%
% This file is part of theo-vorkurs-folien.
%
% theo-vorkurs-folien is free software: you can redistribute it and/or modify
% it under the terms of the GNU General Public License as published by
% the Free Software Foundation, either version 3 of the License, or
% (at your option) any later version.
%
% theo-vorkurs-folien is distributed in the hope that it will be useful,
% but WITHOUT ANY WARRANTY; without even the implied warranty of
% MERCHANTABILITY or FITNESS FOR A PARTICULAR PURPOSE.  See the
% GNU General Public License for more details.
%
% You should have received a copy of the GNU General Public License
% along with theo-vorkurs-folien.  If not, see <https://www.gnu.org/licenses/>.
\begin{frame}[fragile]{Glossar \textemdash\ Mengen}
    \small
    \begin{tabular}{p{0.1\textwidth} p{0.25\textwidth} p{0.49\textwidth}}
        \toprule
        Abk.        & Bedeutung         & Was?!                                                       \\
        \midrule
        $\naturals$ & natürliche Zahlen & In der theoretischen Informatik enthält $\naturals$         \\
                    & (mit $0$)         & die $0$: $\naturals=\{0,1,2,3,\dots\}$ (Auch $\naturals_0$) \\
        $\integers$ & ganze Zahlen      & $\integers=\{\dots,-3,-2,-1,\;0,\;1,\;2,\;3,\dots\}$        \\
        $\reals$    & reelle Zahlen     & $\reals=\{$z.B. $ 2,-3,\sqrt{17},\pi,$ usw.$\}$             \\
        $\emptyset$ & \{\}              & leere Menge                                                 \\
        $a|b$       & teilt             & $a$ ist Teiler von $b$, d.h. $a$ teilt $b$ ohne Rest        \\
        $:$         & sodass            & z.B. $\forall a,b\in\integers:$ $a|b$                       \\
        \bottomrule
    \end{tabular}
\end{frame}

\begin{frame}[fragile]{Glossar \textemdash\ Mengenoperationen}
    \small
    \begin{tabular}{p{0.12\textwidth} p{0.23\textwidth} p{0.5\textwidth}}
        \toprule
        Abk.             & Bedeutung                                    & Was?!                                                                                         \\
        \midrule
        $A \subseteq B$  & Teilmenge                                    & Alle Elemente aus $A$ sind auch in $B$ enthalten. Dabei können die Mengen auch gleich sein.   \\
        $A \subsetneq B$ & echte Teilmenge                              & $A\subseteq B$. Und zusätzlich enthält $B$ Elemente, die nicht in $A$ enthalten sind.         \\
                         &                                              & $\implies$ Mengen sind nicht gleich!                                                          \\
        $A \subset B$    & Teilmenge \emph{oder} echte Teilmenge        & Bei manchen Leuten $\subseteq$, bei manchen $\subsetneq$. Mehrdeutig, lieber nicht verwenden! \\
        $A\cap B$        & Schnitt                                      & Enthält alle Elemente, die in $A$ \textit{und} in $B$ enthalten sind                          \\
        $A\cup B$        & Vereinigung                                  & Enthält alle Elemente, die in $A$, $B$ oder beiden enthalten sind                             \\
        $A\setminus B$   & Komplement, gespr. ''$A$ \textit{ohne} $B$'' & Enthält alle Elemente aus $A$, die \textit{nicht} in $B$ enthalten sind                       \\
        $\overline{B}$   & $B$ Komplement                               & Enthält alle Elemente aus einer geg. Obermenge, die \textit{nicht} in $B$ enthalten sind      \\
        \bottomrule
    \end{tabular}
\end{frame}
\begin{frame}[fragile]{Das können wir jetzt beantworten}
	\begin{alertblock}{Logik}
		\begin{itemize}
			\item Was ist eine Aussage?
			\item Was bedeuten $\wedge$, $\vee$ und $\neg$?
			\item Wie funktioniert die Implikation?
			\item Wie funktioniert die Äquivalenz?
		\end{itemize}
	\end{alertblock}
\end{frame}
\begin{frame}[fragile]{Glossar - Quantoren}
    \small
    \begin{tabular}{p{0.33\textwidth} p{0.2\textwidth} p{0.45\textwidth}}
        \toprule
        Abk.                         & Bedeutung                                        & Was?!                                                              \\
        \midrule
        $\forall x \in M : A(x)$     & Für alle $x$ aus $M$ gilt $A(x)$                 & Die Aussage $A$ muss für alle Elemente der Menge $M$ wahr sein     \\
        $\exists x \in M : A(x)$     & Es existiert ein $x$ in $M$ für das $A(x)$ gilt  & Die Aussage $A(x)$ muss für mind. 1 (oder mehr) Elemente wahr sein \\
        $\forall x,y \in M : A(x,y)$ & Für alle $x$ \textit{und} alle $y$ gilt $A(x,y)$ & Äquivalent zu $\forall x \forall y : A(x,y)$                       \\
        \bottomrule
    \end{tabular}
\end{frame}
% Copyright 2018-2022 FIUS
%
% This file is part of theo-vorkurs-folien.
%
% theo-vorkurs-folien is free software: you can redistribute it and/or modify
% it under the terms of the GNU General Public License as published by
% the Free Software Foundation, either version 3 of the License, or
% (at your option) any later version.
%
% theo-vorkurs-folien is distributed in the hope that it will be useful,
% but WITHOUT ANY WARRANTY; without even the implied warranty of
% MERCHANTABILITY or FITNESS FOR A PARTICULAR PURPOSE.  See the
% GNU General Public License for more details.
%
% You should have received a copy of the GNU General Public License
% along with theo-vorkurs-folien.  If not, see <https://www.gnu.org/licenses/>.

\begin{frame}{Einführung}
    \begin{alertblock}{Was ist ein Beweis?}
        \begin{itemize}
            \item lückenlose Folge von logischen Schlüssen,\\welche zur zu beweisenden Behauptung führen
            \item nicht nur einleuchtend, sondern zweifelsfrei korrekt
        \end{itemize}
    \end{alertblock}
    \onslide<2|handout:1>{
        \begin{alertblock}{Warum beweisen?}
            \begin{itemize}
                \item Aussage basierend auf Fakten und nicht subjektiv belegen
                \item Bestätigung von Aussagen für weitere Nutzung
                \item Zeigen der absoluten Wahrheit
            \end{itemize}
        \end{alertblock}
    }
\end{frame}

\subsection{Beweisbeispiel: Transitivität der Teilmenge}
\begin{frame}[fragile]{Beispielbeweis}
    \begin{exampleblock}{Zu zeigen: Teilmengen sind transitiv.}
        \begin{enumerate}
            \item<1->\alert<1|handout:0>{
                      \only<1|handout:0>{zu zeigen: }\onslide<2->{z.z. }$A\subseteq B\wedge B\subseteq C \alert<3|handout:0>{\implies\text{}}A\subseteq C$
                  }
            \item<2->\alert<2|handout:0>{
                      \only<2|handout:0>{Umschreiben:\\}
                      $\iff $\alert<4,5|handout:0>{$($\alert<9|handout:0>{$($\alert<6|handout:0>{$\forall x$}$\ : x \in A \implies x \in B)$}$ \wedge $\alert<10|handout:0>{$($\alert<6|handout:0>{$\forall x$}$\ : x \in B \implies x \in C)$}$)$}\\
                  \qquad\alert<3|handout:0>{$\implies$}$\;($\alert<6|handout:0>{$\forall x$}$\ :\ $\alert<7|handout:0>{$x \in A$}$ \implies x \in C)$
                  }
            \item<3->\alert<3|handout:0>{
                      \only<3|handout:0>{\emph{Implikation}\\
                          linke Seite wahr $\implies$ rechte Seite muss wahr sein.\\
                          linke Seite falsch $\implies$ beliebiges kann folgen\\
                          $\implies$ uns interessiert also nur der Fall \emph{links ist wahr}}
                      \alert<4>{\only<4,5|handout:0>{Wir machen uns also \emph{\textquotedbl die linke Seite ist wahr\textquotedbl} zur Voraussetzung}\alert<5>{\only<5|handout:0>{:\\Angenommen, $A \subseteq B \wedge B \subseteq C$ gilt.}}}
                  \onslide<6->{Ang., $A \subseteq B \wedge B \subseteq C$.}
                  }
            \item<6->\alert<6|handout:0>{
                      \only<6|handout:0>{Jetzt geht der Beweis richtig los.\\Wähle beliebiges $x$, um Allgemeinheit zu wahren\dots\\}
                      \onslide<6->{Sei $x$ beliebig}\alert<7>{\onslide<7-|handout:0>{ mit \alert<9>{$x\in A$.}}}
                  }
            \item<8->\alert<8-9|handout:0>{
                      \only<8|handout:0>{Wir können jetzt unsere Voraussetzungen ausnutzen,\\um $x\in C$ zu folgern.}
                      \onslide<9->$\implies x\in B$
                      \alert<10|handout:0>{\onslide<10->$\implies x\in C$}
                      \onslide<11>\qed
                  }
        \end{enumerate}
    \end{exampleblock}
\end{frame}


\Center{Noch Fragen?}
\appendix

% Copyright 2018-2022 FIUS
%
% This file is part of theo-vorkurs-folien.
%
% theo-vorkurs-folien is free software: you can redistribute it and/or modify
% it under the terms of the GNU General Public License as published by
% the Free Software Foundation, either version 3 of the License, or
% (at your option) any later version.
%
% theo-vorkurs-folien is distributed in the hope that it will be useful,
% but WITHOUT ANY WARRANTY; without even the implied warranty of
% MERCHANTABILITY or FITNESS FOR A PARTICULAR PURPOSE.  See the
% GNU General Public License for more details.
%
% You should have received a copy of the GNU General Public License
% along with theo-vorkurs-folien.  If not, see <https://www.gnu.org/licenses/>.
\begin{frame}[fragile]{Glossar \textemdash\ Mengen}
    \small
    \begin{tabular}{p{0.1\textwidth} p{0.25\textwidth} p{0.49\textwidth}}
        \toprule
        Abk.        & Bedeutung         & Was?!                                                       \\
        \midrule
        $\naturals$ & natürliche Zahlen & In der theoretischen Informatik enthält $\naturals$         \\
                    & (mit $0$)         & die $0$: $\naturals=\{0,1,2,3,\dots\}$ (Auch $\naturals_0$) \\
        $\integers$ & ganze Zahlen      & $\integers=\{\dots,-3,-2,-1,\;0,\;1,\;2,\;3,\dots\}$        \\
        $\reals$    & reelle Zahlen     & $\reals=\{$z.B. $ 2,-3,\sqrt{17},\pi,$ usw.$\}$             \\
        $\emptyset$ & \{\}              & leere Menge                                                 \\
        $a|b$       & teilt             & $a$ ist Teiler von $b$, d.h. $a$ teilt $b$ ohne Rest        \\
        $:$         & sodass            & z.B. $\forall a,b\in\integers:$ $a|b$                       \\
        \bottomrule
    \end{tabular}
\end{frame}

\begin{frame}[fragile]{Glossar \textemdash\ Mengenoperationen}
    \small
    \begin{tabular}{p{0.12\textwidth} p{0.23\textwidth} p{0.5\textwidth}}
        \toprule
        Abk.             & Bedeutung                                    & Was?!                                                                                         \\
        \midrule
        $A \subseteq B$  & Teilmenge                                    & Alle Elemente aus $A$ sind auch in $B$ enthalten. Dabei können die Mengen auch gleich sein.   \\
        $A \subsetneq B$ & echte Teilmenge                              & $A\subseteq B$. Und zusätzlich enthält $B$ Elemente, die nicht in $A$ enthalten sind.         \\
                         &                                              & $\implies$ Mengen sind nicht gleich!                                                          \\
        $A \subset B$    & Teilmenge \emph{oder} echte Teilmenge        & Bei manchen Leuten $\subseteq$, bei manchen $\subsetneq$. Mehrdeutig, lieber nicht verwenden! \\
        $A\cap B$        & Schnitt                                      & Enthält alle Elemente, die in $A$ \textit{und} in $B$ enthalten sind                          \\
        $A\cup B$        & Vereinigung                                  & Enthält alle Elemente, die in $A$, $B$ oder beiden enthalten sind                             \\
        $A\setminus B$   & Komplement, gespr. ''$A$ \textit{ohne} $B$'' & Enthält alle Elemente aus $A$, die \textit{nicht} in $B$ enthalten sind                       \\
        $\overline{B}$   & $B$ Komplement                               & Enthält alle Elemente aus einer geg. Obermenge, die \textit{nicht} in $B$ enthalten sind      \\
        \bottomrule
    \end{tabular}
\end{frame}
\begin{frame}[fragile]{Glossar - Aussagenlogik}
    \small
    \begin{tabular}{p{0.33\textwidth} p{0.2\textwidth} p{0.45\textwidth}}
        \toprule
        Abk.           & Bedeutung                            & Was?!                                                      \\
        \midrule
        $A\land B$     & A \textit{und} B                     & A und B müssen wahr sein, damit die Gesamtaussage wahr ist \\
        $A\lor B$      & A \textit{oder} B                    & Es müssen mindestend A, B oder beide wahr sein             \\
        $\lnot A$      & \textit{nicht} A                     & A muss falsch sein, damit $\lnot A$ wahr ist               \\
        $B \implies B$ & A ist notwendig für B                & A muss wahr sein, wenn B wahr ist                          \\
        $A \implies B$ & A ist hinreichend für / impliziert B & B muss wahr sein, wenn A wahr ist                          \\
        $A \iff B$     & notwendig und hinreichend            & genau dann, wenn                                           \\
        \bottomrule
    \end{tabular}
\end{frame}
\begin{frame}[fragile]{Glossar - Quantoren}
    \small
    \begin{tabular}{p{0.33\textwidth} p{0.2\textwidth} p{0.45\textwidth}}
        \toprule
        Abk.                         & Bedeutung                                        & Was?!                                                              \\
        \midrule
        $\forall x \in M : A(x)$     & Für alle $x$ aus $M$ gilt $A(x)$                 & Die Aussage $A$ muss für alle Elemente der Menge $M$ wahr sein     \\
        $\exists x \in M : A(x)$     & Es existiert ein $x$ in $M$ für das $A(x)$ gilt  & Die Aussage $A(x)$ muss für mind. 1 (oder mehr) Elemente wahr sein \\
        $\forall x,y \in M : A(x,y)$ & Für alle $x$ \textit{und} alle $y$ gilt $A(x,y)$ & Äquivalent zu $\forall x \forall y : A(x,y)$                       \\
        \bottomrule
    \end{tabular}
\end{frame}

% Copyright 2018, 2019, 2020, 2021 FIUS
%
% This file is part of theo-vorkurs-folien.
%
% theo-vorkurs-folien is free software: you can redistribute it and/or modify
% it under the terms of the GNU General Public License as published by
% the Free Software Foundation, either version 3 of the License, or
% (at your option) any later version.
%
% theo-vorkurs-folien is distributed in the hope that it will be useful,
% but WITHOUT ANY WARRANTY; without even the implied warranty of
% MERCHANTABILITY or FITNESS FOR A PARTICULAR PURPOSE.  See the
% GNU General Public License for more details.
%
% You should have received a copy of the GNU General Public License
% along with theo-vorkurs-folien.  If not, see <https://www.gnu.org/licenses/>.

\subsection{Lizenz}
\begin{frame}[fragile]{Lizenz}
    \begin{itemize}
    \item Unsere Folien sind frei!\\
    \item Jeder darf die Folien unter den Bedingungen der \textbf{GNU General Public License v3} (oder jeder späteren Version) weiterverwenden.\\
    \item Ihr findet den Quelltext unter \url{https://www.github.com/FIUS/theo-vorkurs-folien}
    \end{itemize}
\end{frame}
\begin{frame}<handout:0>[fragile]{Online-Whiteboard}
	\phantom{text}
\end{frame}

\end{document}
