% Copyright 2018, 2019, 2020, 2021 FIUS
%
% This file is part of theo-vorkurs-folien.
%
% theo-vorkurs-folien is free software: you can redistribute it and/or modify
% it under the terms of the GNU General Public License as published by
% the Free Software Foundation, either version 3 of the License, or
% (at your option) any later version.
%
% theo-vorkurs-folien is distributed in the hope that it will be useful,
% but WITHOUT ANY WARRANTY; without even the implied warranty of
% MERCHANTABILITY or FITNESS FOR A PARTICULAR PURPOSE.  See the
% GNU General Public License for more details.
%
% You should have received a copy of the GNU General Public License
% along with theo-vorkurs-folien.  If not, see <https://www.gnu.org/licenses/>.

\documentclass[aspectratio=43,10pt]{beamer}

\usetheme[progressbar=frametitle]{metropolis}
\usepackage{appendixnumberbeamer}
\usepackage[ngerman]{babel}
\usepackage[utf8]{inputenc}
%\usepackage{t1enc}
\usepackage[T1]{fontenc}
\usepackage[sfdefault,scaled=.85]{FiraSans}
\usepackage{newtxsf}

\usepackage{booktabs}
\usepackage[scale=2]{ccicons}
\usepackage{hyperref}

\usepackage{pgf}
\usepackage{tikz}
\usetikzlibrary{arrows,automata,positioning}
\usepackage{pgfplots}
\usepgfplotslibrary{dateplot}

\usepackage{xspace}
\newcommand{\themename}{\textbf{\textsc{metropolis}}\xspace}

\usepackage{blindtext}
\usepackage{graphicx}
\usepackage{subcaption}
\usepackage{comment}
\usepackage{mathtools}
\usepackage{amsmath}
\usepackage{centernot}
\usepackage{amssymb}
\usepackage{proof}
\usepackage{tabularx}
\renewcommand{\figurename}{Abb.}
\usepackage{marvosym}
\usepackage{mathtools}
\usepackage{qrcode}

\definecolor{ExColor}{HTML}{17819b}

\newcommand{\emptyWord}{\varepsilon}
\let \emptyset\varnothing
\newcommand{\SigmaStern}{\Sigma^{*}}
\newcommand{\absval}[1]{|#1|}
\newcommand{\defeq}{\vcentcolon=}
\newcommand{\eqdef}{=\vcentcolon}
\newcommand{\nimplies}{\centernot\implies}

\newcommand{\naturals}{\ensuremath{\mathbb{N}}}
\newcommand{\integers}{\ensuremath{\mathbb{Z}}}
\newcommand{\rationals}{\ensuremath{\mathbb{Q}}}
\newcommand{\reals}{\ensuremath{\mathbb{R}}}

\setbeamertemplate{footline}[text line]
{\parbox{\linewidth}{Fachgruppe Informatik\hfill\insertpagenumber\hfill Vorkurs Theoretische Informatik\vspace{0.2in}}}

\newcommand{\Center}[1]{
	\begin{frame}<handout:0>[standout]
		#1
	\end{frame}
}

% Fix section pages in appendix
\AtBeginDocument{%
  \apptocmd{\appendix}{%
    \setbeamertemplate{section page}[simple]%
  }{}{}
}